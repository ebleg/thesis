% Introduction
\chapter{Introduction}
\label{chap:intro}
\index{emiel}

Original Liouville ideas:
\begin{itemize}
    \item Showcase complex behaviour using the van der Pol oscillator
    \item (Optimal) control of the distributions using the Brockett approach
    \item (Stochastic) inputs, link with Langevin equations
    \item Liouville thing (in continuity form, not incompressibility) can be applied to any diff. eq.
    \item Bayesian inversion of chaotic systems; guess the initial state by sampling after a certain time
    \item Define as streamtube, continuity equation asserts that streamlines cannot cross; i.e. streamtubes are conserves. To reduce computational complexity, define level sets (curves in 2-D) and check how they deform through the evolution of the phase space fluid; should always contain the same amount of probability troughout the evolution of the system.
\end{itemize}

\subsection*{Notation check}
\begin{table}[h]
    \centering
    \begin{tabular}{lcccc}
    \toprule
        \textbf{Object} & \textbf{Roman lower} & \textbf{Roman upper} & \textbf{Greek lower} & \textbf{Greek upper} \\
    \midrule
        Standard & $a b c d e$ & $A B C D E$ & $\alpha \beta \gamma \delta \epsilon $ & $ \Gamma \Delta \Upsilon \Omega \Theta $\\
        Vector & $\vec{a} \vec{b} \vec{c} \vec{d} \vec{e}$ & $\vec{A} \vec{B} \vec{C} \vec{D} \vec{E}$ & $\vec{\alpha} \vec{\beta} \vec{\gamma} \vec{\delta} \vec{\epsilon} $ & $ \vec{\Gamma} \vec{\Delta} \vec{\Upsilon} \vec{\Omega} \vec{\Theta} $\\
        Tensor & $\tens{a} \tens{b} \tens{c} \tens{d} \tens{e}$ & $\tens{A} \tens{B} \tens{C} \tens{D} \tens{E}$ & $\tens{\alpha} \tens{\beta} \tens{\gamma} \tens{\delta} \tens{\epsilon} $ & $ \tens{\Gamma} \tens{\Delta} \tens{\Upsilon} \tens{\Omega} \tens{\Theta} $\\
    \bottomrule
    \end{tabular}
    \caption{Caption}
    \label{tab:my_label}
\end{table}

%Christoffel symbol: $\christ$\\
Math constants: $\ii \ec \pic$\\
Variation: $\var S$\\
Musical isomorphism\\
Flat: $\lowerIndex{X}$\\
Sharp: $\raiseIndex{\omega}$\\
Lie derivative: $\lied{X}{H}$\\
Interior product: $\intpr{X}{\omega}$\\
Lowercase mathcal: $\mathcal{i}$\\
Kinematic momentum: $\mathfrak{p}p$

\subsection*{About mathematical notation and sign conventions}
For symplectic geometry, the sign convention used by \citet{Abraham1978} and \citet{Cannas2001} is observed --- not the one used by Arnol'd in his \emph{Mathematical methods of classical mechanics}, nonetheless often referred to in this text.

\begin{itemize}
    \item Matrices, vectors and tensors are bold upper case.
    \item Differential forms are typically denoted by Greek letters, with their rank as a superscript (cf. Arnol'd). 
\end{itemize}
