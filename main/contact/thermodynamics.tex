\section{Contact structure from thermodynamics}
It has been known for some time that contact geometry is the proper mathematical framework for the theory of thermodynamics; dating from the original work of Gibbs, to \citet{Arnold1989b} and \citet{Hermann1973}. The contact structure arises as a consequence of the first law of the thermodynamics, that is
\begin{equation}
    \dd{U} = \eta - \beta,
\end{equation}
where $U$ is the internal energy of the system, $\eta$ the heat added to the system and $\beta$ the work done by the system on its environment. Both $\eta$ and $\beta$ are 1-forms that are not exact; which is why it makes no sense (in the context of exterior systems) to denote them by $\dd{Q}$ and $\dd{W}$. The essence of the first law really is that the difference of these forms is \emph{closed}. Locally, it can then be written as the gradient of a function, called the \emph{internal energy} $U$ of the system. As a result, the forms 
$$ \dd{\chi} = \dd{U} - \eta + \beta $$
should pull back to zero over the `allowable' states of the systems. This form defines a contact structure, which means that the allowable trajectories live on Legendre submanifolds of the overall contact manifold. \cite{Frankel2012,Bamberg1988} In literature, such submanifolds are called \emph{thermodynamic manifolds}.

For the damped harmonic oscillator, we consider the \emph{overall} system to be completely isolated; that is, there is no energy in the form of work or heat added to the system (for we consider the damper part of the system itself). As a result, the first law simply states that
$$ \dd{U} = 0. $$
Let uw know decompose the system into two subsystems: first, the mass-spring system storing the mechanical energy, and the damper, to which we may attach the conceptual picture of a heat bath (this may just as well be the damper fluid, or the surrounding air). The total internal energy then becomes:
$$ U = U_1 + U_2 = \frac{p^2}{2m} + \frac{1}{2}kq^2 + U_2. $$
If specific assumptions are made about the nature of the heat bath, an explicit expression for $U_2$ may be found as well, but we will leave this possibility open for now. The first law can also be applied to these two subsystems as well. We know that the mechanical system performs work on the damper fluid, which manifests itself as heat added to that system. We therefore have
\begin{equation}
    \begin{split}
        \dd{U}_1 = 
    \end{split}
\end{equation}








