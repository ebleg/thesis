\chapter{Liouville and Contact Geometry for Dissipative Systems}
\label{chap:contact_mechanics} 
%The contact-geometric counterpart of Hamiltonian and Lagrangian mechanics has been the subject of increasing academic interest in recent years, see for example \citet{VanderSchaft2021a,VanderSchaft2018,Maschke2018,Bravetti2017,DeLeon2020}, etc. The conception of the idea arguably traces back to the work of Herglotz \cite{Guenther1996}, who derived it using the variational principle, and the developments in differential geometry, by e.g. \citet{Arnold1989,Arnold1989a,Arnold1989b,Arnold1991} and \citet{Libermann1987}.
%
%In this chapter, the direct connection is made between the Caldirola-Kanai Hamiltonian given by \cref{eq:ham_CK} and the contact Hamiltonian described by \citet{Bravetti2017}, using Liouville geometry\footnote{It is interesting to note that Bravetti gives the Caldirola-Kanai method as an example of dissipative Hamiltonians in his paper, but fails to make the connection with his own method.}. We then proceed to \emph{contact Lagrangian mechanics}, strongly related to the Herglotz' work. Finally, the whole theory is explained from a thermodynamic perspective as well. While it was already known for some time (dating back to Arnol'd) that contact geometry from fitting geometrical framework for thermodynamics, its equivalence to contact geometry in (dissipative) classical mechanics has not been desribed in past literature. This underpins a famous statement by Vladimir Arnol'd that `contact geometry is all geometry', in the sense that conservative mechanical systems can be considered as part of a larger class of systems for which energy dissipation \emph{is} allowed. \cite{Geiges2008}
%
%The traditional picture is that Hamiltonian mechanics takes place in the space of generalized positions and momenta, colloquialy denoted by $q$'s and $p$'s. The generalized positions form coordinates of the configuration manifold, which encodes all the possible positions that the system can find itself in. The momenta, on the other hand, are \emph{cotangent variables}: they live in the cotangent space of linear functions acting on tangent (velocity) vectors to the configuration manifold $Q$. We say that the Hamiltonian is a function on the \emph{cotangent bundle} to the configuration manifold $\ctbundle{Q}$. This cotangent bundle has a canonical `symplectic' structure, given by its symplectic form $\omega$, that pairs every position direction with its corresponding momenta:
%$$ \omega = \sum \wedgep{\dd{q^i}}{\dd{p_i}}. $$
%A vital property of symplectic manifolds (which include the cotangent bundle) is the fact that they are always even-dimensional: every position coordinate has a corresponding momentum and vice versa. Likewise, in an economic context this asserts that every product has its own price. It is precisely this symmetry that is broken by the introduction of contact manifolds.
%
%The symplectic structure of Hamiltonian mechanics is related to the conservation of energy principle. The Hamiltonian function is conserved along the integral curves of the Hamiltonian vector field that it generates. In contrast, for dissipative mechanics, this strict reciprocity between position and momentum is broken. Either, one constructs an explicit time-dependence and acknowledges the special nature of time, or one can introduce another coordinate that acts as `reservoir' to facilitate the dissipation in the system. These Hamiltonian systems are known as n\emph{contact Hamiltonian systems}. 

\section{Introduction}
\label{sec:contact_intro}
\subsection{Historical perspective} 
The traditional symplectic structure of Hamiltonian systems does not intrinsically support energy dissipation. Some efforts have been made to make this possible; the celebrated report by \citet{Dekker1981} provides an excellent overview of the advances in this field. Roughly speaking, four approaches exist:
\begin{itemize}
    \item The \emph{Bateman approach}, doubles the number of degrees of freedom in the system to create a `mirror system' that runs in the opposite time direction. \cite{Bateman1931}. In the overall system the irreversible effects cancel out due to the two time directions, which is why it behaves in a symplectic fashion.
    \item The family of \emph{complex dissipative Hamiltonians} proposed in various forms by \citet{Bopp1974}, \citet{Dekker1975}, \citet{Dedene1980}, \citet{Rajeev2007}, and more recently, \citet{Hutters2020} in the research group of the author. Originally, this method was developed to facilitate the quantization of dissipative mechanics for applications in quantum mechanics. The relation of the work in this thesis with the complex dissipative Hamiltonians is given in \cref{ssec:complex_ham}.
    \item Some purely mathematical constructs have been devised as well, e.g. by \citet{Havas1957} solving the so-called Helmholtz conditions. These \emph{mathematical Hamiltonians} do not offer insights from a physical standpoint (i.e. the Hamiltonian representing the energy in the system); and they rely on singularities of the Hamiltonian function to get around the inevitable limitations imposed by de Rham's theorem (this point is discussed in greater detail later in \cref{sec:liouville}).
    \item Finally, the time-dependent Hamiltonian (and Lagrangian) functions include the dissipative action through an explicit time-dependence that takes care of the `energy shrinkage' of the system. Although reinvented by several authors, this approach is commonly attributed to \citet{Caldirola1941} and \citet{Kanai1948}, which it is why it is usually referred to as the Caldirola-Kanai Hamiltonian \cite{Schuch1997,Tokieda2021}.
\end{itemize}
In addition, for practical (engineering) applications with dissipation the \emph{Rayleigh damping function} is often used in the Lagrangian setting \cite{Goldstein2011}. However, this damping function does not conform to the mathematical structure of classical mechanics, in contrast to the methods mentioned above. Hence, we consider this a purely ad hoc \emph{ad hoc} method, and we we will not be concerned with it in this thesis.

\subsection{Contact Hamiltonian systems} 
In this chapter, a different approach from the described methods is taken: instead of trying to force the dissipation into the symplectic framework, we acknowledge that symplectic manifolds are \emph{not} the natural setting for dissipative mechanics. We turn towards a different, but closely related, type of geometry called \emph{contact geometry}. Contact geometry is often called the odd-dimensional counterpart of symplectic geometry, and for good reason. Indeed, one can see a contact manifold as a symplectic manifold, with one extra dimension which can be identified through the contact structure as `special'. Furthermore, the symplectic part of the contact structure does the usual job of `pairing' the other coordinate directions. In the language of mechanics, these pairings are the generalized positions and their conjugate momenta. As demonstrated by \citet{Arnold1989}, the notion of Hamiltonian systems on symplectic manifolds can be extended to contact Hamiltonian systems on contact manifolds. It is this \emph{contact Hamiltonian formalism} that will be used to describe the mechanics of dissipative mechanical systems. A concise summary of contact geometry and contact Hamiltonian systems is given in \cref{app:contact_geometry}.

There are generally two ways to study contact manifolds: one is directly through the contact structure, another is to use the so-called symplectization. The symplectization of a contact manifold adds another dimension to the contact manifold and assigns it a symplectic structure (instead of a contact structure) in a canonical fashion. Due to the canonical nature of this conversion, the symplectified manifold can serve as a fine representative of the underlying contact manifold \cite{Arnold1989}.
This chapter follows each of these two methods to arive at the same result:
\begin{itemize}
    \item In \cref{sec:thermodynamics}, the basic principles from thermodynamics (and the associated contact structure) are used to provide a constructive argument for the correct contact Hamiltonian system for the damped harmonic oscillator. The corresponding Hamiltonian has already been shown by \citet{Bravetti2017} to generate the correct equations of motion, but his argument is neither constructive, nor does it provide the correct interpretation of the mathematical structure.
    \item In \cref{sec:liouville} the symplectification of the contact Hamiltonian system is used to show that the time-dependent Caldirola-Kanai Hamiltonian is actually a time-explicit version of the symplectified contact Hamiltonian system derived in \cref{sec:thermodynamics}.
\end{itemize}
Both viewpoints have their advantages: the contact structure corresponds directly to the physical states of the system, but the mathematics of contact Hamiltonian systems has little of the mathematical elegance and simplicity that a purely symplectic system offers. Also from a computational viewpoint, the symplectification is a lot more approachable, since we can use traditional tools such as Poisson brackets to perform calculations.

[Lagrangian formalism]

[Generalizations]

Position coordinates are denoted by $q$ (with optional subscript), while $p$ refers to the associated (generalized) momentum. We denote symplectic 2-forms by $\omega$, Liouville 1-forms by $\theta$ and contact forms by $\alpha$.

\subsection{The damped harmonic oscillator} 
This chapter is primarily concerned with the prototypical dissipative mechanical system: the linearly damped harmonic oscillator depicted in \cref{fig:dho}, with the governing second-order differential equation being
\begin{equation}  
    m\dv[2]{q}{t} + b\dv{q}{t} + kq = 0,
    \label{eq:dho_eom}
\end{equation}
where $m$ is the mass of the object, $b$ the damping coefficient, and $k$ the spring constant. However, we prefer to express the dynamics in terms of some other parameters which are derived from these quantities; they are defined in \cref{tab:dho_params}.

The choice of the damped harmonic oscillator as examplary system is rather perspicuous, since it is arguably the `easiest' dissipative system that also exhibits second-order dynamics and is linear in all terms. Furthermore, as discussed below, it serves as the test case of choice in the overwhelming majority of research into dissipative Lagrangian and Hamiltonian mechanics \cite{Dekker1981,Hutters2020}.

%<symbol: \theta> <expl: Liouville 1-form> <tags: greek, math>
%<symbol: \omega> <expl: Symplectic 2-form> <tags: greek, math>

%However, the method described in this section can be generalized directly to a general (possibly time-dependent) potential function $V = V(q, t)$. To make calculations and notation easier, some special parameters are frequently used throughout this chapter, they are summarized in \cref{tab:dho_params}.
\begin{figure}[ht!]
    \centering
    \begin{tikzpicture}[every node/.style={outer sep=0pt,thick}]
    \tikzstyle{spring}=[thick,decorate,decoration={zigzag,pre length=0.3cm,post length=0.3cm,segment length=6}]
    \tikzstyle{damper}=[thick,decoration={markings,  
      mark connection node=dmp,
      mark=at position 0.5 with 
      {
        \node (dmp) [thick,inner sep=0pt,transform shape,rotate=-90,minimum width=15pt,minimum height=3pt,draw=none] {};
        \draw [thick] ($(dmp.north east)+(2pt,0)$) -- (dmp.south east) -- (dmp.south west) -- ($(dmp.north west)+(2pt,0)$);
        \draw [thick] ($(dmp.north)+(0,-5pt)$) -- ($(dmp.north)+(0,5pt)$);
      }
    }, decorate]
    \tikzstyle{ground}=[fill,pattern=north east lines,draw=none,minimum width=0.75cm,minimum height=0.3cm]

    \node (M) [draw,minimum width=1cm, minimum height=1.5cm] {$m$};

    \node (ground) [ground,anchor=north,yshift=-0.25cm,minimum width=1.5cm] at (M.south) {};
    \draw (ground.north east) -- (ground.north west);
    \draw [thick] (M.south west) ++ (0.2cm,-0.125cm) circle (0.125cm)  (M.south east) ++ (-0.2cm,-0.125cm) circle (0.125cm);

    \node (wall) [ground, rotate=-90, minimum width=2cm,yshift=-3cm] {};
    \draw (wall.north east) -- (wall.north west);

    \draw [spring] (wall.160) -- ($(M.north west)!(wall.160)!(M.south west)$) node[pos=0.5,anchor=south, outer sep=4pt] {$k$};
    \draw [damper] (wall.20) -- ($(M.north west)!(wall.20)!(M.south west)$) node[pos=0.5,anchor=north, outer sep=10pt] {$b$};

    \path (wall) ++(0.1cm, 1.2cm) -| node (q) {} (M);
    \draw[|->] (wall) ++(0.2cm, 1.2cm) -- (q.center) node[pos=0.5, anchor=south] {$q$};
    \draw (q) ++(0, 0.1cm) -- ++(0, -0.5cm);

    \node[bgelement] (J1) at (4.5, -1) {1};
    \node[bgelement, label=north:$k$] (C) at (4.5, 0.5) {C};
    \node[bgelement, label=east:$m$]  (I) at (6, -1) {I};
    \node[bgelement, label=west:$b$]  (R) at (3, 0.5) {R};

    % test
    \draw[bonds] 
        (J1) edge[e_out] (I)
        (J1) edge[f_out] (R)
        (J1) edge[f_out] (C);


\end{tikzpicture}

    \caption{Schematic of the mass-spring-damper system.}
    \label{fig:dho}
\end{figure}

\begin{table}[ht!]
    \caption{Parameter conventions of the damped harmonic oscillator. To avoid confusion with the symplectic form $\omega$, angular frequencies are denoted by $\Omega$ instead of the conventional lower case Greek letter.}
    \label{tab:dho_params}
    \centering
    \begin{tabular}{llll}
        \toprule
        \textbf{Name} & \textbf{Symbol} & \textbf{Value} & \textbf{Units} \\
        \midrule
        Damping coefficient & $\gamma$ & $b/m$ & \si{\per \second }\\[0.4cm]
        Undamped frequency & $\Omega_o$ & $\sqrt{k/m}$ & \si{\per \second }\\[0.4cm]
        Damped frequency & $\Omega_d$ & $\sqrt{\Omega_0^2 - \qty(\frac{\gamma}{2})^2}$ & \si{\per \second }\\[0.4cm]  
        Damping ratio & $\zeta$ & $\frac{b}{2\sqrt{mk}}$ & -- \\[0.2cm]
        \bottomrule
    \end{tabular}
\end{table}
%<symbol: \gamma> <expl: Damping coefficient> <tags: greek, mech>

%% THERMODYNAMIC REASONING
\section{Contact structure from thermodynamics}
It has been known for some time that contact geometry is the proper mathematical framework for the theory of thermodynamics; dating from the original work of Gibbs, to \citet{Arnold1989b} and \citet{Hermann1973}. The contact structure arises as a consequence of the first law of the thermodynamics, that is
\begin{equation}
    \dd{U} = \eta - \beta,
\end{equation}
where $U$ is the internal energy of the system, $\eta$ the heat added to the system and $\beta$ the work done by the system on its environment. Both $\eta$ and $\beta$ are 1-forms that are not exact; which is why it makes no sense (in the context of exterior systems) to denote them by $\dd{Q}$ and $\dd{W}$. The essence of the first law really is that the difference of these forms is \emph{closed}. Locally, it can then be written as the gradient of a function, called the \emph{internal energy} $U$ of the system. As a result, the forms 
$$ \dd{\chi} = \dd{U} - \eta + \beta $$
should pull back to zero over the `allowable' states of the systems. This form defines a contact structure, which means that the allowable trajectories live on Legendre submanifolds of the overall contact manifold. \cite{Frankel2012,Bamberg1988} In literature, such submanifolds are called \emph{thermodynamic manifolds}.

For the damped harmonic oscillator, we consider the \emph{overall} system to be completely isolated; that is, there is no energy in the form of work or heat added to the system (for we consider the damper part of the system itself). As a result, the first law simply states that
$$ \dd{U} = 0. $$
Let uw know decompose the system into two subsystems: first, the mass-spring system storing the mechanical energy, and the damper, to which we may attach the conceptual picture of a heat bath (this may just as well be the damper fluid, or the surrounding air). The total internal energy then becomes:
$$ U = U_1 + U_2 = \frac{p^2}{2m} + \frac{1}{2}kq^2 + U_2. $$
If specific assumptions are made about the nature of the heat bath, an explicit expression for $U_2$ may be found as well, but we will leave this possibility open for now. The first law can also be applied to these two subsystems as well. We know that the mechanical system performs work on the damper fluid, which manifests itself as heat added to that system. We therefore have
\begin{equation}
    \begin{split}
        \dd{U}_1 = 
    \end{split}
\end{equation}










%% LIOUVILLE GEOMETRY 
\chapter{The Liouville theorem}

\section{Harmonic oscillator}
Although the Liouville theorem is usually expressed directly in terms of Poisson brackets (which, in turn, have a trivial form if expressed in Darboux coordinates), a slightly more insightful approach will be taken here. More specifically, instead of applying the Poisson brackts directly, they are formulated like so:
$$ \poisson{f}{g} = X_g(f) $$
where $X_g$ is the Hamiltonian vector field associated to $g$. The defintion of Poisson brackets in terms of Hamiltonian vector fields makes it easy to draw connection between fluid mechanics and the classical mechanics.

For the simple, undamped harmonic oscillator, the configuration manifold $M$ is simply $\real$. As such, the cotangent bundle $T^*M = \real^2$. The Hamiltonian, being a smooth function on $T^*M$, is simply a 0-form given in Darboux coordinates $(p, q)$ by:
\begin{equation}
    \ham:\quad T^*M \to \real:\quad \ham(p, q) = \frac{m}{2}p^2 + \frac{k}{2}q^2.
\end{equation}
To apply Liouville's theorem, the Hamiltonian vector field $X_\ham$ associated with $\ham$ must be found. By definition, one can do this by virtue of the natural isomorphism induced by the symplectic 2-form:
$$ \dd{\ham}(\cdot) = \omega^2(X_{\ham}, \cdot), $$
this isomorphism is sometimes called $\raiseIndex{\omega}$, or the `musical isomorphism' \cite{Abraham1978}. When applied as a simple transformation from $\real^{2n} \to \real^{2n}$, this isomorphism can be identified with the transformation matrix \cite{Arnold1989}
$$ \mqty(0_n & -I_n \\ I_n & 0_n). $$
The differential 1-form $\dd{\ham}$ is
$$ \dd{\ham} = \frac{p}{m}\dd{p} + kq\dd{q}, $$ 
such that the Hamiltonian vector field becomes (in the chart-induced basis)
$$ X_{\ham} = kq\pdv{}{p} - \frac{p}{m}\pdv{}{q}. $$
Having found the Hamiltonian vector field, Liouville's theorem can be applied to to an arbitrary distribution $\rho$ over the phase space:
\begin{equation}
    \pdv{\rho}{t} = -\poisson{H}{\rho} = \poisson{\rho}{H} = X_H(\rho) = kq\pdv{\rho}{p} - \frac{p}{m}\pdv{\rho}{q}.
    \label{eq:pde_ho}
\end{equation}
This is a simple transport equation without diffusion; hence, the initial probability distribution will simply `drift' along the streamlines of the Hamiltonian flow. As such, this problem is analogous to a flow that is purely characterized by convection. The convection equation may be readily solved using the method of characteristics.

\begin{aside}{The method of characteristics}
    \Cref{eq:pde_ho} is part of a larger class of linear first-order PDE's of the form\footnote{If the functions $a$ and $c$ depend on $\rho$, the equation is called \emph{semilinear}. This is, however, never the case for a PDE arising from the Liouville equation.} \cite[p. 207]{Farlow1989}.
    \begin{equation}
        \sum_{i=1}^n a_i(x_1, \ldots, x_n, \rho) \pdv{\rho}{x_i} = c(x_1, \ldots, x_n, \rho),
    \end{equation}
    which are traditionally solved using the \emph{method of characteristics}. This method attempts to find characteristic lines along which the solution is constant, as to convert the PDE problem into an ODE problem. More specifically, one whishes to find a parameterization of $x_i$ and $\rho$ such that:
    \begin{equation}
        \begin{split}
            \dv{x_i}{s} &= a_i\\
            \dv{\rho}{s} &= c.
        \end{split}
    \end{equation}
    Given this parameterization, the PDE can be easily rewritten as follows: \cite{Farlow1989}
    $$ \dv{\rho}{s} = \sum_{i = 1}^n \pdv{\rho}{x_i}\dv{x_i}{s}. $$
    The solution of the ODE problem then produces the trajectories for the characteristics. The reparameterization in terms of $s$ must be accompagnied by another reparameterization of the initial conditions in terms of the variable(s) $r_i$; essentially, $s$ provides the parameterization along the characteristic curves while $r_i$ is the parameterization of the initial curves. The expressions for $r_i$ are found by asserting that $x_i(0) = r_i$, and then solving for the integration constants that are still present in the found ODE solutions. Then, finally, one solves the ODE in terms of the characteristic parameterization $\qty(s, r_1, \ldots, r_n)$
    $$ \dv{\rho}{s} + c\qty(x_1(s, r), \ldots, x_n(s, r))\rho = 0, $$
    after which that solution can be written in terms of the old coordinates to obtain the solution of the PDE.

\end{aside}
As it turns out, the method of characteristics takes a particularly simple form for the harmonic oscillator (and Hamiltonian systems in general). The reparameterization in terms of $s$ is
\begin{equation}
    \begin{split}
        \dv{p}{s} &= kq \\
        \dv{q}{s} &= -\frac{p}{m}\\
        \dv{t}{s} &= -1.
    \end{split}
\end{equation}
which immediately yields $t = -s + c_1$ (with the immediate choice that $c_1$ be zero), and the former two equations simply resort to a time-reversed solution of the Hamiltonian problem in terms of $p$ and $q$. Hence, solving the ODE to obtain the characteristic lines is, rather unsurprisingly, equivalent to finding the phase trajectories. For the harmonic oscillator, these trajectories are
\begin{equation}
    \begin{split}
        p(s) &= c_3\cos(\omega s) + m\omega c_1 \sin(\omega s). \\
        q(s) &= c_2\cos(\omega s) - \frac{c_3}{m\omega}\sin(\omega s)\\
    \end{split}
\end{equation}
Solving for $q(0) = r_1$ and $p(0) = r_2$, yields $r_1 = c_2$ and $r_2 = c_3$. Now, because the `forcing term' $c(\cdot)$ is not present in the Liouville equation (for autonomous systems), the solution of the second ODE is trivial:
$$ \rho(s) = \rho_0(r_1, \ldots, r_2), $$
where $\rho_0$ is the initial distribution. It is an encouraging observation that the method of characteristics is easily extended towards non-autonomous systems, leaving the possibility for control action or external disturbances, which may well be of a stochastic nature themselves.

The solution to the Liouville equation is found by writing the initial distribution in terms of $p$, $q$ and $t$. Since $q$ and $p$ depend linearly on $r_1$ and $r_2$, this is a matter of taking the inverse of the associated matrix.
$$ \mqty(p\\q) = 
    \mqty( \cos(\omega s) & m\omega \sin(\omega s) \\\  -\frac{1}{m\omega}\sin(\omega s) & cos(\omega s) )\mqty(r_1\\r_2).  $$
This transformation matrix represents a symplectic transformation of the phase plane; symplectic matrices have a unit determinant\footnote{Due to the equivalence of $\spgroup{2}{\real}$ and $\slgroup{2, \real}$, having a unit determinant is a necessary and sufficient condition for a $2\times2$ matrix to be symplectic; this condition is only necessary for higher dimensional vector spaces \cite{Arnold1989}.}. Inversion and resubstitution of $t$ then yields:
$$ \mqty(r_1\\r_2) = \underbrace{\mqty( \cos(\omega t) & m\omega \sin(\omega t) \\\  -\frac{1}{m\omega}\sin(\omega t) & cos(\omega t) )}_{\Phi(t)}\mqty(p\\q).$$

\paragraph{Initial Gaussian distribution} The solution of the Liouville equation to any initial distribution is simply found by substituting the $(p,q)$ dependence with transformation stated above. For example, an initial bivariate Gaussian distribution centered at some initial point $(p_0, q_0)$ with covariance matrix $\Sigma$ subject to the linear transformation $\Phi(t)$ yields again a Gaussian: \cite{Schon2011}
$$ \mqty(p(t)\\q(t)) \quad \sim \quad \gaussian{R(t)\mqty(p_0\\q_0)}{R^\top(t)\Sigma R(t)}. $$
This result is, after all, not quite a surprise: the Gaussian distribution is transported by the convective stream of the phase space fluid; the mean drifts along its original phase space trajectory as if it where a single particle. The variance changes continuously by the similarity transform given by $R$. Interestingly, because $R$ has a unit determinant, it does not influence the determinant of the transported distribution; as such, the \emph{entropy} of the Gaussian remains constant throughout, and equal to its initial value
$$ \frac{1}{2}\log(\det(2\pic\ec\sigma)). $$

\paragraph{Averages in time and space}
The motion of the harmonic oscillator is periodic.

\section{Damped harmonic oscillator}


%% LAGRANGIAN SYSTEM
\section{Legendre involution}
In the classic, symplectic case, the Legendre transformation is used to pass from the Hamiltonian to the Lagrangian formalism and vice versa. This is because the Legendre transform facilitates a mapping between the tangent and cotangent bundle. If the Lagrangian (or Hamiltonian) is (hyper)regular (i.e. the mass matrix is invertible), this mapping is a diffeomorphism. \cite{Carinena1990}

One would be tempted to use the normal Legendre transformation on the symplectified Hamiltonian $\mathscr{H}$. This approach will meet some problems though:
\begin{itemize}
    \item A homogeneous function is not regular in the homogeneous variables --- naturally, a degree of freedom still resides in the action of the multiplicative group. Therefore, the mapping from the cotangent to the tangent bundle is not a diffeomorphism. Said otherwise, there is not a one-to-one correspondence between the homogeneous momenta and the associated velocities in the Lagrangian description.
    \item As a consequence of Euler's theorem for homogeneous functions, the Legendre transformation for a homogeneous function is necessarily equal to zero. For any homogeneous function $H$ (of degree 1), Euler's theorem states that
    $$ \sum_{i = 1}^n \rho_i \pdv{\mathscr{H}}{\rho_i} = \mathscr{H}, $$

        i.e. the function is equal to its associated `action', and therefore the expression for the Legendre transformation vanishes. \cite{Dirac1950,Dirac1933}
\end{itemize}
There is a better path to take. In essence the Legendre transform is (and was originally meant to be) a \emph{contact transformation}.


