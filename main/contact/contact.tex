\chapter{Liouville and Contact Geometry for Dissipative Systems}
\label{chap:contact_mechanics} 
\section{Introduction}
\label{sec:contact_intro}
%\subsection{Historical perspective} 
The traditional theory of Hamiltonian mechanics based on symplectic manifolds does not include energy dissipation without an explicit time dependence. A famous example is the time-dependent Hamiltonian and Lagrangian functions modeling the linearly damped harmonic oscillator, which is commonly attributed to \citet{Caldirola1941} and \citet{Kanai1948}\footnote{We will hence refer to it as the Caldirola-Kanai Hamiltonian (or Lagrangian).}. 
These systems are nonautonomous, which is to say that the phenomenon of dissipation is essentially treated to be \emph{exogeneous} from the perspective of the system.  

To the engineer, this does not seem natural at all. Every physical system, be it mechanical, electrical, economic, or otherwise, includes dissipative elements in some shape or form. These elements are inherent to the system: exogenous effects are typically reserved for externally controlled inputs or disturbances. Hence, from an engineering point of view, we seek to include the dissipative elements as phenomena that are \emph{endogeneous} to the system itself.

Some efforts have been made in the past to resolve this issue by extending the Hamiltonian formalism: the celebrated report by \citet{Dekker1981} provides an excellent overview of the advances in this field. Roughly speaking, three approaches exist:
\begin{itemize}
    \item The \emph{Bateman approach}, doubles the number of degrees of freedom in the system to create a `mirror system' that runs in the opposite time direction \cite{Bateman1931}. In the overall system the irreversible (i.e. dependent on the direction of time) effects, cancel out due to the two time directions, which is why it admits a symplectic structure.
    \item The family of \emph{complex dissipative Hamiltonians} proposed in various forms by \citet{Bopp1974}, \citet{Dekker1975}, \citet{Dedene1980}, \citet{Rajeev2007}, and more recently, \citet{Hutters2020} in the research group of the author. Originally, this method was developed to facilitate the quantization of dissipative mechanics for applications in quantum mechanics. The relation of the work in this thesis with the complex dissipative Hamiltonians is given in \cref{ssec:complex_ham}.
    \item Some purely \emph{mathematical Hamiltonians} have been devised as well, e.g. by \citet{Havas1957} solving the so-called Helmholtz conditions. These Hamiltonians produce the correct equations of motion but bear no connection to the energy in the system.
\end{itemize}
Although all of these methods work, they have some limitations for practical applications. The Bateman approach results in a system with twice the dimensions, but most (at least half) of these are redundant. This is because dissipation is in essence a first-order effect: as will be shown in this chapter, at most one additional dimension is required to model the dissipation. The complex Hamiltonians are mathematically elegant but lack physical interpretation of the canonical  coordinates and the Hamiltonian function. Finally, the mathematical Hamiltonians do not offer insights from a physical standpoint. Furthermore, they rely on singularities of the Hamiltonian function to circumvent the inevitable limitations imposed by de Rham's theorem (this point is discussed in greater detail later in \cref{sec:liouville}).

In addition, we remark that it is common practice in engineering applications to include the \emph{Rayleigh damping function} in the Lagrangian function \cite{Goldstein2011} to represent linearly damped elements. Although this damping function produces the correct equations of motion for linear damping, it does not conform to the deeper symplectic structure underlying Hamiltonian (and also Lagrangian) mechanics. Hence, we consider this an \emph{ad hoc} method, and we will not be further concerned with it in this thesis.

The purpose of this chapter is to propose a different framework that is both mathematically elegant \emph{and} physically interpretable, to make it suitable for practical applications. Instead of trying to fit the dissipation into the symplectic framework, we propose that \emph{contact manifolds} (instead of symplectic manifolds) are the natural setting for problems that include dissipation. Contact manifolds are necessarily odd-dimensional: next to the canonical pairs of positions of momenta, an extra coordinate is included that absorbs the dissipated energy. The Hamiltonian formalism can be extended from symplectic manifolds to contact manifolds as well. We will use this \emph{contact Hamiltonian formalism} to describe dissipative mechanical systems.

This chapter takes two different paths to arrive at a single contact Hamiltonian representation of the damped harmonic oscillator, which serves as an exemplary dissipative system. These two paths are based on two parallel mathematical representations of a contact manifold: one is the contact structure itself, while the other is its \emph{symplectification}. This is a procedure to cast the contact Hamiltonian system on a symplectic manifold with an additional \emph{Liouville structure}. This procedure goes at the expense of yet another dimension added to the manifold, making it even-dimensional again.

Firstly, in \cref{sec:thermodynamics}, we use the principles of thermodynamics to derive the contact Hamiltonian representation of the damped harmonic oscillator. Secondly, in \cref{sec:liouville}, we look at the symplectification of the manifold and derive an equivalent symplectified Hamiltonian system argued motivated by the form of the Caldirola-Kanai Hamiltonian.

We then proceed by transforming the contact Hamiltonian representation of the damped harmonic oscillator to the contact Lagrangian representation in \cref{sec:lagrangian}. Finally, \cref{sec:generalization} shows how the same principles can be used to model more complicated mechanical systems, including those with exogenous inputs.
%<symbol: \gamma> <expl: Damping coefficient> <tags: greek, mech>

For the reader unfamiliar with contact geometry and contact Hamiltonian systems, a concise introduction to contact geometry and contact Hamiltonian systems is given in \cref{app:contact_geometry}. A more extensive overview is given by \citet{Geiges2008} and \citet{Libermann1987}.

%% THERMODYNAMIC REASONING
\section{Contact structure from thermodynamics}
It has been known for some time that contact geometry is the proper mathematical framework for the theory of thermodynamics; dating from the original work of Gibbs, to \citet{Arnold1989b} and \citet{Hermann1973}. The contact structure arises as a consequence of the first law of the thermodynamics, that is
\begin{equation}
    \dd{U} = \eta - \beta,
\end{equation}
where $U$ is the internal energy of the system, $\eta$ the heat added to the system and $\beta$ the work done by the system on its environment. Both $\eta$ and $\beta$ are 1-forms that are not exact; which is why it makes no sense (in the context of exterior systems) to denote them by $\dd{Q}$ and $\dd{W}$. The essence of the first law really is that the difference of these forms is \emph{closed}. Locally, it can then be written as the gradient of a function, called the \emph{internal energy} $U$ of the system. As a result, the forms 
$$ \dd{\chi} = \dd{U} - \eta + \beta $$
should pull back to zero over the `allowable' states of the systems. This form defines a contact structure, which means that the allowable trajectories live on Legendre submanifolds of the overall contact manifold. \cite{Frankel2012,Bamberg1988} In literature, such submanifolds are called \emph{thermodynamic manifolds}.

For the damped harmonic oscillator, we consider the \emph{overall} system to be completely isolated; that is, there is no energy in the form of work or heat added to the system (for we consider the damper part of the system itself). As a result, the first law simply states that
$$ \dd{U} = 0. $$
Let uw know decompose the system into two subsystems: first, the mass-spring system storing the mechanical energy, and the damper, to which we may attach the conceptual picture of a heat bath (this may just as well be the damper fluid, or the surrounding air). The total internal energy then becomes:
$$ U = U_1 + U_2 = \frac{p^2}{2m} + \frac{1}{2}kq^2 + U_2. $$
If specific assumptions are made about the nature of the heat bath, an explicit expression for $U_2$ may be found as well, but we will leave this possibility open for now. The first law can also be applied to these two subsystems as well. We know that the mechanical system performs work on the damper fluid, which manifests itself as heat added to that system. We therefore have
\begin{equation}
    \begin{split}
        \dd{U}_1 = 
    \end{split}
\end{equation}










%% LIOUVILLE GEOMETRY 
\chapter{The Liouville theorem}

\section{Harmonic oscillator}
Although the Liouville theorem is usually expressed directly in terms of Poisson brackets (which, in turn, have a trivial form if expressed in Darboux coordinates), a slightly more insightful approach will be taken here. More specifically, instead of applying the Poisson brackts directly, they are formulated like so:
$$ \poisson{f}{g} = X_g(f) $$
where $X_g$ is the Hamiltonian vector field associated to $g$. The defintion of Poisson brackets in terms of Hamiltonian vector fields makes it easy to draw connection between fluid mechanics and the classical mechanics.

For the simple, undamped harmonic oscillator, the configuration manifold $M$ is simply $\real$. As such, the cotangent bundle $T^*M = \real^2$. The Hamiltonian, being a smooth function on $T^*M$, is simply a 0-form given in Darboux coordinates $(p, q)$ by:
\begin{equation}
    \ham:\quad T^*M \to \real:\quad \ham(p, q) = \frac{m}{2}p^2 + \frac{k}{2}q^2.
\end{equation}
To apply Liouville's theorem, the Hamiltonian vector field $X_\ham$ associated with $\ham$ must be found. By definition, one can do this by virtue of the natural isomorphism induced by the symplectic 2-form:
$$ \dd{\ham}(\cdot) = \omega^2(X_{\ham}, \cdot), $$
this isomorphism is sometimes called $\raiseIndex{\omega}$, or the `musical isomorphism' \cite{Abraham1978}. When applied as a simple transformation from $\real^{2n} \to \real^{2n}$, this isomorphism can be identified with the transformation matrix \cite{Arnold1989}
$$ \mqty(0_n & -I_n \\ I_n & 0_n). $$
The differential 1-form $\dd{\ham}$ is
$$ \dd{\ham} = \frac{p}{m}\dd{p} + kq\dd{q}, $$ 
such that the Hamiltonian vector field becomes (in the chart-induced basis)
$$ X_{\ham} = kq\pdv{}{p} - \frac{p}{m}\pdv{}{q}. $$
Having found the Hamiltonian vector field, Liouville's theorem can be applied to to an arbitrary distribution $\rho$ over the phase space:
\begin{equation}
    \pdv{\rho}{t} = -\poisson{H}{\rho} = \poisson{\rho}{H} = X_H(\rho) = kq\pdv{\rho}{p} - \frac{p}{m}\pdv{\rho}{q}.
    \label{eq:pde_ho}
\end{equation}
This is a simple transport equation without diffusion; hence, the initial probability distribution will simply `drift' along the streamlines of the Hamiltonian flow. As such, this problem is analogous to a flow that is purely characterized by convection. The convection equation may be readily solved using the method of characteristics.

\begin{aside}{The method of characteristics}
    \Cref{eq:pde_ho} is part of a larger class of linear first-order PDE's of the form\footnote{If the functions $a$ and $c$ depend on $\rho$, the equation is called \emph{semilinear}. This is, however, never the case for a PDE arising from the Liouville equation.} \cite[p. 207]{Farlow1989}.
    \begin{equation}
        \sum_{i=1}^n a_i(x_1, \ldots, x_n, \rho) \pdv{\rho}{x_i} = c(x_1, \ldots, x_n, \rho),
    \end{equation}
    which are traditionally solved using the \emph{method of characteristics}. This method attempts to find characteristic lines along which the solution is constant, as to convert the PDE problem into an ODE problem. More specifically, one whishes to find a parameterization of $x_i$ and $\rho$ such that:
    \begin{equation}
        \begin{split}
            \dv{x_i}{s} &= a_i\\
            \dv{\rho}{s} &= c.
        \end{split}
    \end{equation}
    Given this parameterization, the PDE can be easily rewritten as follows: \cite{Farlow1989}
    $$ \dv{\rho}{s} = \sum_{i = 1}^n \pdv{\rho}{x_i}\dv{x_i}{s}. $$
    The solution of the ODE problem then produces the trajectories for the characteristics. The reparameterization in terms of $s$ must be accompagnied by another reparameterization of the initial conditions in terms of the variable(s) $r_i$; essentially, $s$ provides the parameterization along the characteristic curves while $r_i$ is the parameterization of the initial curves. The expressions for $r_i$ are found by asserting that $x_i(0) = r_i$, and then solving for the integration constants that are still present in the found ODE solutions. Then, finally, one solves the ODE in terms of the characteristic parameterization $\qty(s, r_1, \ldots, r_n)$
    $$ \dv{\rho}{s} + c\qty(x_1(s, r), \ldots, x_n(s, r))\rho = 0, $$
    after which that solution can be written in terms of the old coordinates to obtain the solution of the PDE.

\end{aside}
As it turns out, the method of characteristics takes a particularly simple form for the harmonic oscillator (and Hamiltonian systems in general). The reparameterization in terms of $s$ is
\begin{equation}
    \begin{split}
        \dv{p}{s} &= kq \\
        \dv{q}{s} &= -\frac{p}{m}\\
        \dv{t}{s} &= -1.
    \end{split}
\end{equation}
which immediately yields $t = -s + c_1$ (with the immediate choice that $c_1$ be zero), and the former two equations simply resort to a time-reversed solution of the Hamiltonian problem in terms of $p$ and $q$. Hence, solving the ODE to obtain the characteristic lines is, rather unsurprisingly, equivalent to finding the phase trajectories. For the harmonic oscillator, these trajectories are
\begin{equation}
    \begin{split}
        p(s) &= c_3\cos(\omega s) + m\omega c_1 \sin(\omega s). \\
        q(s) &= c_2\cos(\omega s) - \frac{c_3}{m\omega}\sin(\omega s)\\
    \end{split}
\end{equation}
Solving for $q(0) = r_1$ and $p(0) = r_2$, yields $r_1 = c_2$ and $r_2 = c_3$. Now, because the `forcing term' $c(\cdot)$ is not present in the Liouville equation (for autonomous systems), the solution of the second ODE is trivial:
$$ \rho(s) = \rho_0(r_1, \ldots, r_2), $$
where $\rho_0$ is the initial distribution. It is an encouraging observation that the method of characteristics is easily extended towards non-autonomous systems, leaving the possibility for control action or external disturbances, which may well be of a stochastic nature themselves.

The solution to the Liouville equation is found by writing the initial distribution in terms of $p$, $q$ and $t$. Since $q$ and $p$ depend linearly on $r_1$ and $r_2$, this is a matter of taking the inverse of the associated matrix.
$$ \mqty(p\\q) = 
    \mqty( \cos(\omega s) & m\omega \sin(\omega s) \\\  -\frac{1}{m\omega}\sin(\omega s) & cos(\omega s) )\mqty(r_1\\r_2).  $$
This transformation matrix represents a symplectic transformation of the phase plane; symplectic matrices have a unit determinant\footnote{Due to the equivalence of $\spgroup{2}{\real}$ and $\slgroup{2, \real}$, having a unit determinant is a necessary and sufficient condition for a $2\times2$ matrix to be symplectic; this condition is only necessary for higher dimensional vector spaces \cite{Arnold1989}.}. Inversion and resubstitution of $t$ then yields:
$$ \mqty(r_1\\r_2) = \underbrace{\mqty( \cos(\omega t) & m\omega \sin(\omega t) \\\  -\frac{1}{m\omega}\sin(\omega t) & cos(\omega t) )}_{\Phi(t)}\mqty(p\\q).$$

\paragraph{Initial Gaussian distribution} The solution of the Liouville equation to any initial distribution is simply found by substituting the $(p,q)$ dependence with transformation stated above. For example, an initial bivariate Gaussian distribution centered at some initial point $(p_0, q_0)$ with covariance matrix $\Sigma$ subject to the linear transformation $\Phi(t)$ yields again a Gaussian: \cite{Schon2011}
$$ \mqty(p(t)\\q(t)) \quad \sim \quad \gaussian{R(t)\mqty(p_0\\q_0)}{R^\top(t)\Sigma R(t)}. $$
This result is, after all, not quite a surprise: the Gaussian distribution is transported by the convective stream of the phase space fluid; the mean drifts along its original phase space trajectory as if it where a single particle. The variance changes continuously by the similarity transform given by $R$. Interestingly, because $R$ has a unit determinant, it does not influence the determinant of the transported distribution; as such, the \emph{entropy} of the Gaussian remains constant throughout, and equal to its initial value
$$ \frac{1}{2}\log(\det(2\pic\ec\sigma)). $$

\paragraph{Averages in time and space}
The motion of the harmonic oscillator is periodic.

\section{Damped harmonic oscillator}


%% LAGRANGIAN SYSTEM
\section{Legendre involution}
In the classic, symplectic case, the Legendre transformation is used to pass from the Hamiltonian to the Lagrangian formalism and vice versa. This is because the Legendre transform facilitates a mapping between the tangent and cotangent bundle. If the Lagrangian (or Hamiltonian) is (hyper)regular (i.e. the mass matrix is invertible), this mapping is a diffeomorphism. \cite{Carinena1990}

One would be tempted to use the normal Legendre transformation on the symplectified Hamiltonian $\mathscr{H}$. This approach will meet some problems though:
\begin{itemize}
    \item A homogeneous function is not regular in the homogeneous variables --- naturally, a degree of freedom still resides in the action of the multiplicative group. Therefore, the mapping from the cotangent to the tangent bundle is not a diffeomorphism. Said otherwise, there is not a one-to-one correspondence between the homogeneous momenta and the associated velocities in the Lagrangian description.
    \item As a consequence of Euler's theorem for homogeneous functions, the Legendre transformation for a homogeneous function is necessarily equal to zero. For any homogeneous function $H$ (of degree 1), Euler's theorem states that
    $$ \sum_{i = 1}^n \rho_i \pdv{\mathscr{H}}{\rho_i} = \mathscr{H}, $$

        i.e. the function is equal to its associated `action', and therefore the expression for the Legendre transformation vanishes. \cite{Dirac1950,Dirac1933}
\end{itemize}
There is a better path to take. In essence the Legendre transform is (and was originally meant to be) a \emph{contact transformation}.


