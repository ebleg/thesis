\section{Legendre involution}
In the classic, symplectic case, the Legendre transformation is used to pass from the Hamiltonian to the Lagrangian formalism and vice versa. This is because the Legendre transform facilitates a mapping between the tangent and cotangent bundle. If the Lagrangian (or Hamiltonian) is (hyper)regular (i.e. the mass matrix is invertible), this mapping is a diffeomorphism. \cite{Carinena1990}

One would be tempted to use the normal Legendre transformation on the symplectified Hamiltonian $\mathscr{H}$. This approach will meet some problems though:
\begin{itemize}
    \item A homogeneous function is not regular in the homogeneous variables --- naturally, a degree of freedom still resides in the action of the multiplicative group. Therefore, the mapping from the cotangent to the tangent bundle is not a diffeomorphism. Said otherwise, there is not a one-to-one correspondence between the homogeneous momenta and the associated velocities in the Lagrangian description.
    \item As a consequence of Euler's theorem for homogeneous functions, the Legendre transformation for a homogeneous function is necessarily equal to zero. For any homogeneous function $H$ (of degree 1), Euler's theorem states that
    $$ \sum_{i = 1}^n \rho_i \pdv{\mathscr{H}}{\rho_i} = \mathscr{H}, $$

        i.e. the function is equal to its associated `action', and therefore the expression for the Legendre transformation vanishes. \cite{Dirac1950,Dirac1933}
\end{itemize}
There is a better path to take. In essence the Legendre transform is (and was originally meant to be) a \emph{contact transformation}.

