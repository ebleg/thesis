\section{Legendre involution}
In the classic, symplectic case, the Legendre transformation is used to pass from the Hamiltonian to the Lagrangian formalism and vice versa. This is because the Legendre transform facilitates a mapping between the tangent and cotangent bundle. If the Lagrangian (or Hamiltonian) is (hyper)regular (i.e. the mass matrix is invertible), this mapping is a diffeomorphism. \cite{Carinena1990}

One would be tempted to use the normal Legendre transformation on the symplectified Hamiltonian $\mathscr{H}$. This approach will meet some problems though:
\begin{itemize}
    \item A homogeneous function is not regular in the homogeneous variables --- naturally, a degree of freedom still resides in the action of the multiplicative group. Therefore, the mapping from the cotangent to the tangent bundle is not a diffeomorphism. Said otherwise, there is not a one-to-one correspondence between the homogeneous momenta and the associated velocities in the Lagrangian description.
    \item As a consequence of Euler's theorem for homogeneous functions, the Legendre transformation for a homogeneous function is necessarily equal to zero. For any homogeneous function $H$ (of degree 1), Euler's theorem states that
    $$ \sum_{i = 1}^n \rho_i \pdv{\mathscr{H}}{\rho_i} = \mathscr{H}, $$

        i.e. the function is equal to its associated `action', and therefore the expression for the Legendre transformation vanishes. \cite{Dirac1950,Dirac1933}
\end{itemize}
There is a better path to take. In essence the Legendre transform is (and was originally meant to be) a \emph{contact transformation}.

\section{Notes}

%\begin{mathbox}{Lie derivatives \& Max' question}
    The Lie derivative of the tautological form $\alpha = p\dd{q}$ with respect to the
    Hamiltonian vector field
    $$ X_H = \pdv{H}{p}\pdv{}{q} - \pdv{H}{p}\pdv{}{p}$$
    is denoted by
    $$ \lied{X_H}{\alpha}.$$
    Using Cartan's magic formula ($ \lied{V}{\theta} = \dd{(\intpr{V}{\theta})} + \intpr{V}{\dd{\theta}}$), this expression
    can be written as
    \begin{equation*} 
        \begin{split}
            \lied{X_H}{p\dd{q}} &= \dd{(\intpr{X_H}{p\dd{q}})} + \intpr{X_H}{\dd{(p\dd{q})}} \\
                                &= \dd{(\intpr{X_H}{p\dd{q}})} - \intpr{X_H}{\omega} \\
                                &= \dd{(\intpr{X_H}{p\dd{q}})} - \dd{H} \\
                                &= \dd{\qty(\pdv{H}{p}p)} - \dd{H} \\
                                &= \dd{(\dot{q}p)} - \dd{H} \\
                                &= \dd{(\dot{q}p - H)} \\
                                &= \dd{L} \\
        \end{split}
    \end{equation*}

    Explicitly in components:
    \begin{equation*}
        \lied{X_H}{\alpha} = \qty[X_H^{\nu}(\partial_{\nu}\alpha_{\mu}) + (\partial_{\mu} X_H^{\nu})\alpha_{\nu}]\dd{x}^{\mu}
    \end{equation*}

    \begin{equation*}
        \begin{split}
            \lied{X_H}{\alpha} =\, &\qty[\pdv{H}{p}\qty(\pdv{}{q}p) + \qty(\pdv{}{q}\pdv{H}{p})p - \pdv{H}{q}\qty(\pdv{}{p}p) - \qty(\pdv{}{q}\pdv{H}{q})0]\dd{q} \\
                                + &\qty[\pdv{H}{p}\qty(\pdv{}{q}0) + \qty(\pdv{}{p}\pdv{H}{p})p - \pdv{H}{q}\qty(\pdv{}{q} 0) - \qty(\pdv{}{p}\pdv{H}{q})0]\dd{p} \\
                               =\,&\qty[ \qty(\pdv{}{q}\pdv{H}{p})p - \pdv{H}{q}]\dd{q} + \qty[\qty(\pdv{}{p}\pdv{H}{p})p]\dd{p}\\
        \end{split}
    \end{equation*}
    Compare this with the expression using the Cartan equation:
    \begin{equation*}
        \begin{split}
            \dd{\qty(\pdv{H}{p}p - H)} &= \pdv{}{q}\qty(\pdv{H}{p}p)\dd{q} + \pdv{}{p}\qty(\pdv{H}{p}p)\dd{p} - \pdv{H}{q}\dd{q} - \pdv{H}{p}\dd{p}\\
                                       &= \qty[p\pdv{}{q}\qty(\pdv{H}{p}) + \pdv{H}{p}\pdv{p}{q} - \pdv{H}{q}]\dd{q} + \qty[p\pdv{}{p}\qty(\pdv{H}{p}) + \pdv{H}{p}\pdv{p}{p} - \pdv{H}{p}]\dd{p}
        \end{split}
    \end{equation*}
    which coincides with the previous expression.

\end{mathbox}


