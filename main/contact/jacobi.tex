\section{Jacobi structures for general systems}
\label{sec:jacobi}
In this section we take the ideas outlined in \cref{sec:contact,sec:symplectic} one step further to more general mechanical systems. In particular, we will focus on multi-degree of freedom (MDOF) systems, and systems with exogeneous inputs (external forces). As it turns out, a contact structure is not sufficient to describe such systems. Instead, we use a generalization of contact and symplectic structures called a \emph{Jacobi structure}.

\begin{figure}[ht!]
    \centering
    \begin{tikzpicture}[every node/.style={outer sep=0pt,thick}]
    \tikzstyle{spring}=[thick,decorate,decoration={zigzag,pre length=0.3cm,post length=0.3cm,segment length=6}]
    \tikzstyle{damper}=[thick,decoration={markings,  
      mark connection node=dmp,
      mark=at position 0.5 with 
      {
        \node (dmp) [thick,inner sep=0pt,transform shape,rotate=-90,minimum width=15pt,minimum height=3pt,draw=none] {};
        \draw [thick] ($(dmp.north east)+(2pt,0)$) -- (dmp.south east) -- (dmp.south west) -- ($(dmp.north west)+(2pt,0)$);
        \draw [thick] ($(dmp.north)+(0,-5pt)$) -- ($(dmp.north)+(0,5pt)$);
      }
    }, decorate]
    
    \tikzstyle{ground}=[fill,pattern=north east lines,draw=none,minimum width=0.75cm,minimum height=0.3cm]
    
    \node (M1) [draw,minimum width=1cm, minimum height=1.5cm] {$m_1$};
    
    \node (ground) [ground,anchor=north,yshift=-0.25cm,minimum width=2cm] at (M1.south) {};
    \draw (ground.north east) -- (ground.north west);
    \draw [thick] (M1.south west) ++ (0.2cm,-0.125cm) circle (0.125cm)  (M1.south east) ++ (-0.2cm,-0.125cm) circle (0.125cm);
    
    \node (wall) [ground, rotate=-90, minimum width=2cm,yshift=-3cm] {};
    \draw (wall.north east) -- (wall.north west);
    
    \draw [spring] (wall.160) -- ($(M1.north west)!(wall.160)!(M1.south west)$) node[pos=0.5,anchor=south, outer sep=4pt] {$k_1$};
    \draw [damper] (wall.20) -- ($(M1.north west)!(wall.20)!(M1.south west)$) node[pos=0.5,anchor=north, outer sep=10pt] {$b_1$};
    
    
    \node (M2) [xshift=3cm, draw,minimum width=1cm, minimum height=1.5cm] {$m_2$};
    \node (ground2) [ground,anchor=north,yshift=-0.25cm,minimum width=2cm] at (M2.south) {};
    \draw (ground2.north east) -- (ground2.north west);
    \draw [thick] (M2.south west) ++ (0.2cm,-0.125cm) circle (0.125cm)  (M2.south east) ++ (-0.2cm,-0.125cm) circle (0.125cm);
    
    \node (rightwall) [ground, rotate=-90, minimum width=2cm,yshift=6cm] {};
    \draw (rightwall.south east) -- (rightwall.south west);
    
    \draw [spring] (rightwall) -- ($(M2.south east)!(rightwall)!(M2.south east)$) node[pos=0.5,anchor=south, outer sep=4pt] {$k_3$};
    
    \draw [spring] (M1.40) -- ($(M2.north west)!(M1.40)!(M2.south west)$) node[pos=0.5,anchor=south, outer sep=4pt] {$k_2$};
    \draw [damper] (M1.320) -- ($(M2.north west)!(M1.320)!(M2.south west)$) node[pos=0.5,anchor=north, outer sep=10pt] {$b_2$};
    
    \draw[|->] (M1) ++(0, 1.3cm) -- ++(1cm, 0) node[pos=0.5, anchor=south] {$q_1$}; 
    \draw[|->] (M2) ++(0, 1.3cm) -- ++(1cm, 0) node[pos=0.5, anchor=south] {$q_2$}; 

    \node[bgelement, xshift=1.5cm, yshift=-4.3cm] (J2) {0};
    \node[bgelement, xshift=2cm] (J3) at (J2) {1};
    \node[bgelement, xshift=-2cm] (J1) at (J2) {1};
    \node[bgelement, yshift=-3cm] (J4) at (J2) {1};

    \node[bgelement, xshift=-1.5cm, label=west:$m_1$] (I1) at (J1) {I};
    \node[bgelement, yshift=-1.5cm, label=south:$b_2$] (R1) at (J1) {R};
    \node[bgelement, yshift=1.5cm, label=north:$k_1$]  (C1) at (J1) {C};

    \node[bgelement, xshift=-1.5cm, label=west:$b_2$] (R2) at (J4) {R};
    \node[bgelement, xshift=1.5cm, label=east:$k_2$]  (C2) at (J4) {C};

    \node[bgelement, yshift=-1.5cm, label=south:$m_2$] (I2) at (J3) {I};
    \node[bgelement, yshift=1.5cm, label=north:$k_3$]  (C3) at (J3) {C};

    %\node[bgelement, label=north:$k$] (C) at (4.5, 0.5) {C};
    %\node[bgelement, label=east:$m$]  (I) at (6, -1) {I};
    %\node[bgelement, label=west:$b$]  (R) at (3, -1) {R};

    %% test
    \draw[bonds] 
        (J1) edge[e_out] (I1)
        (J1) edge[f_out] (C1)
        (J1) edge[f_out] (R1)

        (J4) edge[f_out] (C2)
        (J4) edge[f_out] (R2)

        (J3) edge[f_out] (C3)
        (J3) edge[e_out] (I2)

        (J2) edge[e_out] (J3)
        (J1) edge[f_out] (J2)
        (J2) edge[f_out] (J4);
    
\end{tikzpicture}

    \caption{Multi-degree of freedom mechanical system with two masses, two dampers and three springs. The corresponding bond graph representation is shown below.}
    \label{fig:mdof_oscillator}
\end{figure}

To illustrate the need for a Jacobi structure, we use the mechanical MDOF system shown in \cref{fig:mdof_oscillator}. The corresponding equations of motion are
\begin{equation}
    \begin{split}
        &\dot{q}_1 = \frac{p_1}{m_1}, \\
        &\dot{q}_2 = \frac{p_2}{m_2}, \\
        &\dot{p}_1 = -\frac{b_1}{m_1}p_1 - \frac{b_2}{m_1}p_1 + \frac{b_2}{m_2}p_2 - k_1 q_1 - k_2 q_1 + k_2 q_2, \\
        &\dot{p}_2 =  - \frac{b_2}{m_2}p_2 + \frac{b_2}{m_1}p_1 - k_3 q_2 - k_2 q_2 + k_2 q_1. \\
    \end{split}
\end{equation}

To proceed with the method discussed in \cref{ssec:contact_dissipation}, we have to find the work form that specifies the work done by the system of on the dampers. The work done the first damper ($b_1$) is
$$ \beta_1 = \qty(\frac{b_1}{m_1})p_1\dd{q_1}. $$
The second damper ($b_2$) is placed between the two masses; the flow is relative. The effort is proportional to this flow; i.e.
$$ \beta_2 = b_2\qty(\frac{p_2}{m_2} - \frac{p_1}{m_1})\dd{\qty(q_2 - q_1)}. $$
Hence, the contact 1-form that specifies the dissipation is
\begin{equation}
    \begin{split}
        \alpha  &= \dd{U} - \qty(\frac{b_1}{m_1})p_1\dd{q_1} - b_2\qty(\frac{p_2}{m_2} - \frac{p_1}{m_1})\dd{\qty(q_2 - q_1)} \\
                &= \dd{U} - \qty[\qty(\frac{b_1}{m_1} + \qty(\frac{b_2}{m_1})) p_1 - \qty(\frac{b_2}{m_2})p_2]\dd{q_1}
                          - \qty[\qty(\frac{b_2}{m_2})p_2 - \qty(\frac{b_2}{m_1})p_1]\dd{q_2}.
    \end{split}
    \label{eq:mdof_form}
\end{equation}

From this expression, we can observe a crucial difference with the contact forms of the single degree of freedom systems (cf. \cref{eq:dho_contact_form,eq:serial_dho_contact_form}). This is because $\alpha$ is in this case \emph{not} of the form
$$ \dd{U} - \gamma \theta, $$
where $\theta$ is the Liouville form on the cotangent bundle of the configuration manifold $\ctbundle{Q}$. This has important ramifications, for the Liouville form (and its exterior derivative) facilitates the `pairing' between the position and momentum coordinates. In case of a single degree of freedom system, the pairing is trivial because there is only one momentum and one position coordinate. For more complicated systems this is no longer the case, as illustrated \cref{eq:mdof_form}.

The exterior derivative of $\alpha$ is
$$
    \dd{\alpha} = \qty(\frac{b_1}{m_1} + \qty(\frac{b_2}{m_1})) \wedgep{\dd{q_1}}{\dd{p_1}}
                  -\qty(\frac{b_2}{m_2})\wedgep{\dd{q_1}}{\dd{p_2}}
                  + \qty(\frac{b_2}{m_2})\wedgep{\dd{q_2}}{\dd{p_2}}
                  - \qty(\frac{b_2}{m_1})\wedgep{\dd{q_2}}{\dd{p_1}},
$$
which indicates indeed that there is also a `mixing' of $p_1$ and $q_2$ and $p_2$ and $q_1$ in the resulting 2-form. As a result, the mapping $ \fromDual{\dd{\alpha}} $ will not produce the mapping that we would expect in the purely symplectic case.

From a conceptual standpoint, this is not quite surprising: there is no inherent reason why the form that describes dissipation should somehow also include the `pairing' structure: they are distinct, and both required for the geometric description of the mechanical system. In the previous section, we were rather `lucky' to find that, \emph{that particular case}, the dissipation form $\alpha$ also included the pairing structure. 

The multi-degree of freedom systems for which $ \alpha $ is of the form $\dd{U} - \gamma \theta$ are those that do exhibit not damping on the relative velocities of the masses, and for which all dampers have the same damping coefficient. This is a very restrictive requirement, and we wish to do better. To do so, we introduce a generalization of contact and symplectic structures called a \emph{Jacobi} structure in the next section, and subsequently apply it to the system shown in \cref{fig:mdof_oscillator}.

\subsection{Jacobi structures}
A \emph{Jacobi structure} on a manifold $M$ is a bilinear mapping on the functions on $M$ \cite{marle1991}
$$ \jacobi{\,}{}: \functions{M}\times\functions{M} \to \functions{M}: \quad (f, g) \mapsto \jacobi{f}{g} $$
called the \emph{Jacobi bracket}. This mapping needs to satisfy three properties:
\begin{enumerate}[label=(\roman*), noitemsep]
    \item it must be \emph{skew-symmetric}
        $$ \jacobi{f}{g} = -\jacobi{g}{f}, $$
    \item it satisfies the \emph{Jacobi identity}
        $$ \jacobi{f}{\jacobi{g}{h}} + \jacobi{h}{\jacobi{f}{g}} + \jacobi{g}{\jacobi{h}{f}}, $$
    \item it is \emph{local}
        $$ \support{\jacobi{f}{g}} \subset \support{f} \cap \support{g}, $$
        where $\support$ denotes the support of a function.
\end{enumerate}
Manifolds equipped with a Jacobi structure are called \emph{Jacobi manifolds}.

It can be shown that any Jacobi structure can be uniquely defined in terms of a bivector field\footnote{A \emph{bivector} is the contravariant counterpart of a 2-form: it is a skew-symmetric tensor with valence (2, 0) \cite{einstein1944}.} $\Lambda$ and a vector field $R$. The corresponding Jacobi bracket is then given by: \cite{Libermann1987,marle1991}
$$ \jacobi{f}{g} = \Lambda(\dd{f}, \dd{g}) + f (\intpr{R}{\dd{g}}) - g (\intpr{R}{\dd{f}}). $$

Not just any choice of bivector field and vector field give rise to a Jacobi structure. As shown by \citet{lichnerowicz1977}, $\lambda$ and $R$ must satisfy two conditions:
$$ \schouten{\Lambda}{\Lambda} = 2 \wedgep{R}{\Lambda} \qquad \schouten{R}{\Lambda} = 0, $$
where $\schouten{\,}{}$ is the \emph{Schouten bracket}\footnote
{
    The Schouten bracket of an $r$-vector field $A$ and an $s$-vector field $B$ on a manifold is a $(r + s - 1)$-vector field $\schouten{A}{B}$, defined by its action on a closed $(r + s -1)$-form $\beta$ as follows:
    $$ \schouten{A}{B}(\beta) = (-1)^{rs + s}\intpr{A}{\dd{(\intpr{B}{\beta})}} + (-1)^r \intpr{B}{\dd{(\intpr{A}{\beta})}}. $$
    For $r = s = 1$, the Schouten bracket simply reverts to the ordinary Lie bracket \cite{dazord1991}.

}. A Jacobi manifold is therefore a triple $(M, \Lambda, R)$ \cite{Libermann1987}.

\paragraph{Symplectic manifolds are Jacobi}
For a symplectic manifold $(M, \omega)$ with dimension $2n$, the vector field $R$ is simply zero and the bivector $\Lambda$ field is defined by:
$$ \Lambda(\eta, \chi) = \omega\qty(\fromDual{\omega}(\eta), \fromDual{\omega}(\chi)) \qquad \eta, \chi \in \nforms{1}{M}, $$
with $\fromDual{\omega}$ defined as in \cref{eq:symplectic_isomorphism}. 

If $\omega$ is expressed in Darboux coordinates, i.e.
$$ \omega = \sum^n_{i = 1} \wedgep{\dd{q_i}}{\dd{p_i}},$$
then the associated bivector is \cite{chinea1998}
$$ \Lambda = \sum^n_{i = 1} \wedgep{\pdv{}{q_i}}{\pdv{}{p_i}}.$$

The associated Jacobi bracket reverts to the familiar Poisson bracket on the symplectic manifold. A Poisson structure is a particular instance of a Jacobi structure where the vector field $R$ vanishes. This makes the Poisson/Jacobi bracket into a \emph{derivation} on the algebra of smooth functions (over the real numbers): consequently, Poisson brackets satisfy the Leibniz property in addition to the conditions for Jacobi brackets given above \cite{marle1991}.

\paragraph{Contact manifolds are Jacobi}
A strictly contact manifold\footnote{For contact structures that are not globally determined by a single contact form, \citet{marle1991} introduced the concept of a \emph{Jacobi bundle}.} $(M, \alpha)$ with dimension $2n + 1$ is also a Jacobi manifold. The vector field $R = R_\alpha$ is the Reeb vector field and the bivector $\Lambda$ is equal to
$$ \Lambda(\eta, \chi) = \dd{\alpha}\qty(\fromDual{\dd{\alpha}}(\eta), \fromDual{\dd{\alpha}}(\chi)), $$
where $\fromDual{\dd{\alpha}}$ is defined as in \cref{eq:contact_isomorphism} and $\eta,\chi$ are semi-basic 1-forms on $M$.

If $\alpha$ is expressed in Darboux coordinates:
$$ \alpha = \dd{q}_0 - \sum^n_{i = 1} p_i\dd{q_i}, $$
then
$$ R = \pdv{}{q_0},$$
and
$$ \Lambda = \sum_{i = 1}^n \qty( \wedgep{\pdv{}{q_i}}{\pdv{}{p_i}}\:+\: p_i \wedgep{\pdv{}{q_0}}{\pdv{}{p_i}}). $$

\paragraph{Hamiltonian systems on Jacobi manifolds} In both cases, the Jacobi structure induces an isomorphism between the functions on the manifold and the symplectic or strictly contact vector fields on that manifold. In general, this isomorphism is stated as:
$$ \Psi: X_f = \fromDual{\Lambda}(\dd{f}) + f R, $$
where $f$ is the `Hamiltonian' function and $X_f$ the corresponding Hamiltonian vector field. The mapping $\fromDual{\Lambda}$ is defined as
$$ \fromDual{\Lambda}: \ctbundle{M} \to \tbundle{M}: \quad \qty[\fromDual{\Lambda}(\alpha)](\beta) = \Lambda(\alpha, \beta). $$

We will now apply the Jacobi structure to general mechanical systems.

\subsection{Jacobi structure of mechanical systems}
The geometric structure of a mechanical system has four components:
\begin{itemize}
    \item An odd-dimensional manifold $M = \ctbundle{Q}\times \real$, where $Q$ is the configuration manifold. It is extended by one dimension to incorporate the dissipated (internal) energy $U$ and therefore always odd-dimensional. In the following, we assume the `Darboux' coordinates $(q_1, \ldots, q_n, p_1, \ldots, p_n, U)$.
    This manifold has a bundle structure $\bundle{M}{\pi}{\ctbundle{Q}}$, where $\pi$ is the projection map that `forgets' the $U$-coordinate.
    \item A closed 2-form with constant rank $2n$, defined as the negative of the exterior derivative of the Liouville form on $\ctbundle{Q}$:
        $$ \omega = -\dd{\theta} = \sum^n_{i=1} \wedgep{\dd{q_i}}{\dd{p_i}}, $$
        i.e. $\omega$ is the canonical symplectic 2-form on $\ctbundle{Q}$.
    \item A \emph{dissipation form} $\alpha$ that encodes the work done by the system on its environment:
        $$ \alpha = \dd{U} - \beta, $$
        where $\beta = \pi^*\beta_{\ctbundle{Q}}$ is a pullback of a form on $\ctbundle{Q}$, i.e. it does not depend on $U$. When there is no dissipation, $\beta = 0$.
    \item A \emph{Hamiltonian function} $H \in \functions{M}$, equal to the sum of the mechanical energy of the system and the internal energy:
        $$ H = E + U, $$
        with $E = E(q_1, \ldots, q_n, p_1, \ldots, p_n)$ the mechanical energy of the system.
\end{itemize}

In the purely conservative case discussed in \cref{sec:symplectic}, there is no dissipation, so the extra dimension in $U$ does not play a role, and the system may be completely described on $\ctbundle{Q}$ with its symplectic structure. For the simple dissipative mechanical systems in \cref{sec:contact}, the form $\alpha$ would both encode the pairing structure \emph{and} the dissipation form, since $\dd{\alpha}$ would be of the form $\dd{U} - \gamma \theta$.  We now separate both functionalities (i.e. pairing and dissipation) to distinct, for which the symplectic and contact systems are particular cases.

To make this work from a mathematical perspective, we assume that $\omega$ and $\alpha$ satisfy the condition:
$$ \wedgep{\alpha}{(\omega)^n} \neq 0$$
everywhere on $M$; that is to say, it is a volume form on $M$ \cite{ciaglia2018}. If $M$, $\omega$ and $\alpha$ are defined as given above, this condition is clearly satisfied:
$$ \wedgep{\alpha}{(\omega)^n} = n!\:\wedgep{\dd{U}}{\qty(\bigwedge_{i=1}^{n}\wedgep{\dd{q_i}}{\dd{p_i}})}. $$

The \emph{Reeb vector field} $R$ on the manifold $M$ is uniquely defined by the conditions
$$ \intpr{R}{\alpha} = 0 \qquad \intpr{R}{\omega} = 0. $$
This vector field is unique, because the kernel of $\omega$ is always 1-dimensional, and $\alpha$ has at least a component in the zero direction of $\omega$. If this were not the case, the condition given above would not be satisfied. In the given coordinates, the Reeb vector field is equal to
$$ R = \pdv{}{U}. $$

Because $ \wedgep{\alpha}{(\omega)^n} \neq 0$, we have at every point $x \in M$:
$$ \ker{\alpha}\vert_x \cap \ker{\omega}\vert_x = \{\mathbf{0}\}, $$
and therefore \cite{dazord1991}
$$ \tspace{x}{M} = \ker{\alpha} \vert_x \oplus \ker{\omega}\vert_x. $$

Similar to the case for contact manifolds, define \emph{horizontal vector fields} as those in the kernel of $\alpha$, while \emph{vertical vector fields} are in the kernel of $\omega$. Any vector field $ X \in \vfields{M}$ can be decomposed uniquely into a horizontal component $X^\text{hor}$ and vertical component $X^\text{ver}$ like so
$$ X = \underbrace{\qty(\intpr{X}{\alpha})R}_{X^\text{ver}}\: + \: \underbrace{X - (\intpr{X}{\alpha})R}_{X^\text{hor}}. $$


