\chapter{The Liouville theorem}

\section{Harmonic oscillator}
Although the Liouville theorem is usually expressed directly in terms of Poisson brackets (which, in turn, have a trivial form if expressed in Darboux coordinates), a slightly more insightful approach will be taken here. More specifically, instead of applying the Poisson brackts directly, they are formulated like so:
$$ \poisson{f}{g} = X_g(f) $$
where $X_g$ is the Hamiltonian vector field associated to $g$. The defintion of Poisson brackets in terms of Hamiltonian vector fields makes it easy to draw connection between fluid mechanics and the classical mechanics.

For the simple, undamped harmonic oscillator, the configuration manifold $M$ is simply $\real$. As such, the cotangent bundle $T^*M = \real^2$. The Hamiltonian, being a smooth function on $T^*M$, is simply a 0-form given in Darboux coordinates $(p, q)$ by:
\begin{equation}
    \ham:\quad T^*M \to \real:\quad \ham(p, q) = \frac{m}{2}p^2 + \frac{k}{2}q^2.
\end{equation}
To apply Liouville's theorem, the Hamiltonian vector field $X_\ham$ associated with $\ham$ must be found. By definition, one can do this by virtue of the natural isomorphism induced by the symplectic 2-form:
$$ \dd{\ham}(\cdot) = \omega^2(X_{\ham}, \cdot), $$
this isomorphism is sometimes called $\raiseIndex{\omega}$, or the `musical isomorphism' \cite{Abraham1978}. When applied as a simple transformation from $\real^{2n} \to \real^{2n}$, this isomorphism can be identified with the transformation matrix \cite{Arnold1989}
$$ \mqty(0_n & -I_n \\ I_n & 0_n). $$
The differential 1-form $\dd{\ham}$ is
$$ \dd{\ham} = \frac{p}{m}\dd{p} + kq\dd{q}, $$ 
such that the Hamiltonian vector field becomes (in the chart-induced basis)
$$ X_{\ham} = kq\pdv{}{p} - \frac{p}{m}\pdv{}{q}. $$
Having found the Hamiltonian vector field, Liouville's theorem can be applied to to an arbitrary distribution $\rho$ over the phase space:
\begin{equation}
    \pdv{\rho}{t} = -\poisson{H}{\rho} = \poisson{\rho}{H} = X_H(\rho) = kq\pdv{\rho}{p} - \frac{p}{m}\pdv{\rho}{q}.
    \label{eq:pde_ho}
\end{equation}
This is a simple transport equation without diffusion; hence, the initial probability distribution will simply `drift' along the streamlines of the Hamiltonian flow. As such, this problem is analogous to a flow that is purely characterized by convection. The convection equation may be readily solved using the method of characteristics.

\begin{aside}{The method of characteristics}
    \Cref{eq:pde_ho} is part of a larger class of linear first-order PDE's of the form\footnote{If the functions $a$ and $c$ depend on $\rho$, the equation is called \emph{semilinear}. This is, however, never the case for a PDE arising from the Liouville equation.} \cite[p. 207]{Farlow1989}.
    \begin{equation}
        \sum_{i=1}^n a_i(x_1, \ldots, x_n, \rho) \pdv{\rho}{x_i} = c(x_1, \ldots, x_n, \rho),
    \end{equation}
    which are traditionally solved using the \emph{method of characteristics}. This method attempts to find characteristic lines along which the solution is constant, as to convert the PDE problem into an ODE problem. More specifically, one whishes to find a parameterization of $x_i$ and $\rho$ such that:
    \begin{equation}
        \begin{split}
            \dv{x_i}{s} &= a_i\\
            \dv{\rho}{s} &= c.
        \end{split}
    \end{equation}
    Given this parameterization, the PDE can be easily rewritten as follows: \cite{Farlow1989}
    $$ \dv{\rho}{s} = \sum_{i = 1}^n \pdv{\rho}{x_i}\dv{x_i}{s}. $$
    The solution of the ODE problem then produces the trajectories for the characteristics. The reparameterization in terms of $s$ must be accompagnied by another reparameterization of the initial conditions in terms of the variable(s) $r_i$; essentially, $s$ provides the parameterization along the characteristic curves while $r_i$ is the parameterization of the initial curves. The expressions for $r_i$ are found by asserting that $x_i(0) = r_i$, and then solving for the integration constants that are still present in the found ODE solutions. Then, finally, one solves the ODE in terms of the characteristic parameterization $\qty(s, r_1, \ldots, r_n)$
    $$ \dv{\rho}{s} + c\qty(x_1(s, r), \ldots, x_n(s, r))\rho = 0, $$
    after which that solution can be written in terms of the old coordinates to obtain the solution of the PDE.

\end{aside}
As it turns out, the method of characteristics takes a particularly simple form for the harmonic oscillator (and Hamiltonian systems in general). The reparameterization in terms of $s$ is
\begin{equation}
    \begin{split}
        \dv{p}{s} &= kq \\
        \dv{q}{s} &= -\frac{p}{m}\\
        \dv{t}{s} &= -1.
    \end{split}
\end{equation}
which immediately yields $t = -s + c_1$ (with the immediate choice that $c_1$ be zero), and the former two equations simply resort to a time-reversed solution of the Hamiltonian problem in terms of $p$ and $q$. Hence, solving the ODE to obtain the characteristic lines is, rather unsurprisingly, equivalent to finding the phase trajectories. For the harmonic oscillator, these trajectories are
\begin{equation}
    \begin{split}
        p(s) &= c_3\cos(\omega s) + m\omega c_1 \sin(\omega s). \\
        q(s) &= c_2\cos(\omega s) - \frac{c_3}{m\omega}\sin(\omega s)\\
    \end{split}
\end{equation}
Solving for $q(0) = r_1$ and $p(0) = r_2$, yields $r_1 = c_2$ and $r_2 = c_3$. Now, because the `forcing term' $c(\cdot)$ is not present in the Liouville equation (for autonomous systems), the solution of the second ODE is trivial:
$$ \rho(s) = \rho_0(r_1, \ldots, r_2), $$
where $\rho_0$ is the initial distribution. It is an encouraging observation that the method of characteristics is easily extended towards non-autonomous systems, leaving the possibility for control action or external disturbances, which may well be of a stochastic nature themselves.

The solution to the Liouville equation is found by writing the initial distribution in terms of $p$, $q$ and $t$. Since $q$ and $p$ depend linearly on $r_1$ and $r_2$, this is a matter of taking the inverse of the associated matrix.
$$ \mqty(p\\q) = 
    \mqty( \cos(\omega s) & m\omega \sin(\omega s) \\\  -\frac{1}{m\omega}\sin(\omega s) & cos(\omega s) )\mqty(r_1\\r_2).  $$
This transformation matrix represents a symplectic transformation of the phase plane; symplectic matrices have a unit determinant\footnote{Due to the equivalence of $\spgroup{2}{\real}$ and $\slgroup{2, \real}$, having a unit determinant is a necessary and sufficient condition for a $2\times2$ matrix to be symplectic; this condition is only necessary for higher dimensional vector spaces \cite{Arnold1989}.}. Inversion and resubstitution of $t$ then yields:
$$ \mqty(r_1\\r_2) = \underbrace{\mqty( \cos(\omega t) & m\omega \sin(\omega t) \\\  -\frac{1}{m\omega}\sin(\omega t) & cos(\omega t) )}_{\Phi(t)}\mqty(p\\q).$$

\paragraph{Initial Gaussian distribution} The solution of the Liouville equation to any initial distribution is simply found by substituting the $(p,q)$ dependence with transformation stated above. For example, an initial bivariate Gaussian distribution centered at some initial point $(p_0, q_0)$ with covariance matrix $\Sigma$ subject to the linear transformation $\Phi(t)$ yields again a Gaussian: \cite{Schon2011}
$$ \mqty(p(t)\\q(t)) \quad \sim \quad \gaussian{R(t)\mqty(p_0\\q_0)}{R^\top(t)\Sigma R(t)}. $$
This result is, after all, not quite a surprise: the Gaussian distribution is transported by the convective stream of the phase space fluid; the mean drifts along its original phase space trajectory as if it where a single particle. The variance changes continuously by the similarity transform given by $R$. Interestingly, because $R$ has a unit determinant, it does not influence the determinant of the transported distribution; as such, the \emph{entropy} of the Gaussian remains constant throughout, and equal to its initial value
$$ \frac{1}{2}\log(\det(2\pic\ec\sigma)). $$

\paragraph{Averages in time and space}
The motion of the harmonic oscillator is periodic.

\section{Damped harmonic oscillator}
