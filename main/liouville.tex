\chapter{The Liouville theorem}

\section{Harmonic oscillator}
Although the Liouville theorem is usually expressed directly in terms of Poisson brackets (which, in turn, have a trivial form if expressed in Darboux coordinates), a slightly more insightful approach will be taken here. More specifically, instead of applying the Poisson brackts directly, they are formulated like so:
$$ \poisson{f}{g} = X_g(f) $$
where $X_g$ is the Hamiltonian vector field associated to $g$. The defintion of Poisson brackets in terms of Hamiltonian vector fields makes it easy to draw connection between fluid mechanics and the classical mechanics.

For the simple, undamped harmonic oscillator, the configuration manifold $M$ is simply $\real$. As such, the cotangent bundle $T^*M = \real^2$. The Hamiltonian, being a smooth function on $T^*M$, is simply a 0-form given in Darboux coordinates $(q, p)$ by:
\begin{equation}
    \ham:\quad T^*M \to \real:\quad \ham(q, p) = \frac{m}{2}p^2 + \frac{k}{2}q^2.
\end{equation}
To apply Liouville's theorem, the Hamiltonian vector field $X_\ham$ associated with $\ham$ must be found. By definition, one can do this by virtue of the natural isomorphism induced by the symplectic 2-form:
$$ \dd{\ham}(\cdot) = \omega^2(X_{\ham}, \cdot), $$
this isomorphism is sometimes called $\raiseIndex{\omega}$, or the `musical isomorphism' \cite{Abraham1978}. When applied as a simple transformation from $\real^{2n} \to \real^{2n}$, this isomorphism can be identified with the transformation matrix \cite{Arnold1989}
$$ \bmqty{0_n & -I_n \\ I_n & 0_n}. $$
The differential 1-form $\dd{\ham}$ is
$$ \dd{\ham} = kq\dd{q} + \frac{p}{m}\dd{p}, $$ 
such that the Hamiltonian vector field becomes (in the chart-induced basis)
$$ X_{\ham} = kq\pdv{}{p} - \frac{p}{m}\pdv{}{q}. $$
Having found the Hamiltonian vector field, Liouville's theorem can be applied to to an arbitrary distribution $\rho$ over the phase space:
\begin{equation}
    \pdv{\rho}{t} = -\poisson{H}{\rho} = \poisson{\rho}{H} = X_H(\rho) = kq\pdv{\rho}{p} - \frac{p}{m}\pdv{\rho}{q}.
    \label{eq:pde_ho}
\end{equation}
This is a simple transport equation without diffusion; hence, the initial probability distribution will simply `drift' along the streamlines of the Hamiltonian flow. The transport equation may be solved using the method of characteristics.

\paragraph{The method of characteristics}
\Cref{eq:pde_ho} is part of a larger class of linear first-order PDE's of the form\footnote{If the functions $a$ and $c$ depend on $\rho$, the equation is called \emph{semilinear}. This is, however, never the case for a PDE arising from the Liouville equation.} \cite[p. 207]{Farlow1989}.
\begin{equation}
    \sum_{i=1}^n a_i(x_1, \ldots, x_n, \rho) \pdv{\rho}{x_i} = c(x_1, \ldots, x_n, \rho),
\end{equation}
which are traditionally solved using the \emph{method of characteristics}. This method attempts to find characteristic lines along which the solution is constant, as to convert the PDE problem into an ODE problem. More specifically, one whishes to find a parameterization of $x_i$ and $u$ such that:
\begin{equation}
    \begin{split}
        \dv{x_i}{s} &= a_i\\
        \dv{\rho}{s} &= c.
    \end{split}
\end{equation}
Given this parameterization, the PDE can be easily rewritten as follows: \cite{Farlow1989}
$$ \dv{\rho}{s} = \sum_{i = 1}^n \pdv{\rho}{x_i}\dv{x_i}{s} $$
