\chapter{Conclusions and Recommendations}
\label{chap:conclusion}

\section*{Conclusions}
The purpose of this thesis is twofold: first, to establish a fitting geometric framework for mechanical systems with dissipation that also has a clear physical interpretation, and second, to investigate the newly proposed split-quaternion representation of two-dimensional linear dynamical systems. 

We have identified that contact structures indeed provide a suitable structure for \emph{some} mechanical systems, but not all of them. We have assigned the contact structure with a thermodynamic interpretation, i.e. the 1-form that measures the work done by the dampers in the system. Using this interpretation, it can be easily constructed for any mechanical system, especially if it is presented in the form of a bond graph. Furthermore, we have supported with numerous arguments the claim that the Hamiltonian function must be numerically equal to zero, if it is to represent actual energy in the system. This important fact has not been completely appreciated by much of the literature that has appeared on the application of contact Hamiltonian systems for dissipative mechanical systems.

The physical insight into the contact structure of dissipated systems allowed us to extend the contact Hamiltonian system that has been proposed for the harmonic oscillator with parallel damping to the harmonic oscillator that has both a parallel and a serial damper. This is of great conceptual importance (also for the economic engineering interpretation), because although both serial and parallel dampers are represented by the same bond graph element, their causality is flipped. From a physical (and economic) standpoint, this means that they act very differently.

In addition, we have used the symplectified Hamiltonian system to explain the form of the Caldirola-Kanai Hamiltonian, which has been the method of choice in the past to model the damped harmonic oscillator by a Hamiltonian system. The interpretation of the strange `canonical' momentum and the presence of the exponential factor (and its inverse) in the expression arise as a direct consequence of the associated homogeneous Hamiltonian system.

The split-quaternion representation of linear dynamical systems proves to be very insightful. Since the eigenvalues of the associated matrix appear directly in the real part and vector norm of the split-quaternion, the qualitative properties of the corresponding dynamical system are easily derived. As a result, the classification of these linear systems is particularly easy within the realm of split-quaternions, especially for the degenerate cases. Evaluating the exponential function of a split-quaternions allows one to find the system solution; especially in the case of complex eigenvalues, the split-quaternions offer a computational advantage.

Finally, the normalized vector part of the split-quaternion representation unambiguously determines the shape of the solution trajectories. We have focused on the case of an underdamped mechanical system with parallel and serial damping, and showed that the normalized vector in pseudospherical coordinates directly specified the phase rotation and eccentricity of the solution trajectory. When projected to either the Poincaré disk or Cayley-Klein disk, the radial coordinate in these disks corresponds directly to known measures of eccentricity.

\section*{Recommendations}
In this thesis, we have assigned the geometric infrastructure that underlies Lagrangian mechanics with a compelling economic interpretation; being that the Lagrange 2-form is analogous to the Slutsky matrix in microeconomics. However, we have not elaborated on this result, and recommend that further research in the field of economic engineering puts this hypothesis to the test and investigates its potential consequences.

We have shown that any contact symplectic manifold can be lifted canonically to a symplectic manifold through a procedure called \emph{symplectization}. We have applied this to the contact Hamiltonian system of the damped harmonic oscillator and demonstrated its correspondence with the Caldirola-Kanai method. However, we have not symplectified the contact Hamiltonian system for the oscillator with serial and parallel damper; this can be an interesting subject for future research. According to our views, it should also be possible to derive a Caldirola-Kanai-type Hamiltonian via this method for damper with two oscillators.

Along the same line, it has been shown that any Jacobi structure can be lifted to a manifold with a homogeneous Poisson structure (symplectification is a particular case of this). Hence, it might be possible extend the practice of symplectification to the Jacobi structure for general mechanical systems. 

\begin{itemize}
    \item Further investigation of the mathematical properties of the newly defined Jacobi structure.
    \item Look into applications for port-Hamiltonian systems
    \item Interface mechanical and thermodynamic systems
    \item Hamilton-Jacobi theory
\end{itemize}

Split-quaternion representation
\begin{itemize}
    \item Applications for control
    \item Role of geodesics on the hyperboloid model in the representation of mechanical systems
    \item Hamiltonian representation using the split-quaternions
    \item Extension to higher dimensions
    \item Overdamped / critically damped systems
\end{itemize}

