\chapter{Conclusions and Recommendations}
\label{chap:conclusion}

\section*{Conclusions}
The purpose of this thesis is twofold: first, to establish a fitting geometric framework for mechanical systems with dissipation that also has a clear physical interpretation, and second, to present the properties and advantages of the newly proposed split-quaternion representation of dynamical systems. 

Contact manifolds provide a suitable setting for simple mechanical systems, in particular those with only a single degree of freedom. The contact Hamiltonian system reflects the thermodynamic/mechanical information about the system, consisting of a contact 1-form, which represents the first law of thermodynamics, and the Hamiltonian, being the sum of the mechanical and internal energy in the system. The contact form can also be interpreted as a manifestation of the work-heat equivalence as it was originally shown by James P. Joule \cite{joule1850}. Similar to the contact form in thermodynamics being the the \emph{Gibbs form}, the contact form in mechanical systems may be called the \emph{Joule form} instead. 

The newly proposed framework allows for a unification of thermodynamic and mechanical systems that would otherwise be treated separately. Indeed, the Hamiltonian contains both mechanical energy and thermodynamic energy, and the Joule form specifies the transformation of the one into the other. As a result, we expect that this approach can readily be used to handle systems that contain both thermodynamic and mechanical elements with a single Hamiltonian description.

The physical interpretation of the contact structure makes it straightforward to apply it to any mechanical system simply from inspection. We have shown this by extending the contact Hamiltonian system for the harmonic oscillator with parallel damping to the harmonic oscillator that has both a parallel and a serial damper. Apart from the compelling mathematical symmetry of the contact form, being able to handle both types of dampers has important applications in economic engineering. Whereas the parallel damper acts on the price based on the flow of goods (e.g. a transaction cost), the serial damper acts on the flow of goods based on a price (e.g. depreciation or consumption).

In addition, we have used the symplectified Hamiltonian system to explain the form of the Caldirola-Kanai Hamiltonian, has been wideley regarded as the standard method to incorporate dissipation into the Hamiltonian (in case of the damped harmonic oscillator). However, the interpretation of the form of the Caldirola-Kanai Hamiltonian has been the subject of debate in the past. We have shown that form of the Caldirola-Kanai Hamiltonian is a direct consequence of the fact that it is really the homogeneous Hamiltonian of the symplectified system with part of the solution substituted in it.
Because contact Hamiltonian systems can always be symplectified, a Caldirola-Kanai type expression can be obtained for any such system.

For more general systems, the contact structure does not suffice. This is because the contact structure derives the `Hamiltonian isomorphism' also from the contact form, which is conceptually wrong. Indeed, the nature of the dissipation in the system is by no means necessarily related to the symplectic pairing of positions and momenta. For simple mechanical systems, this is not a problem because there is only one pairing possible between the position and momentum. The generalized structure that takes care of this is not a contact structure anymore, but the physical interpretations of the associated 1-form and the Hamiltonian remain. As such, the Jacobi structure allows the unification of thermodynamic and mechanical systems just as well.

The split-quaternion representation of two-dimensional dynamical systems provides immediate insights into the geometric nature of the system. Roughly speaking, the split-quaternion are `super-eigenvalues': they have the same structure (real part plus imaginary part), where the imaginary part also contains the eigenvector information if its magnitude is disregarded.

Deriving the qualitative nature of the system follows almost immediately from inspection of the split-quaternion interpretation. Even the degenerate cases (e.g. if the eigenvalues are not simple) do not have to be handled as exceptions in this procedure. 
Furthermore, the system solution can be obtained without the need for eigenvectors. This offers, in our opinion, a computational advantage especially when the eigenvectors are complex.

We also argue that the normalized vector part of the split-quaternion presents the `shape information' of the system in a more convenient manner. This is because the normalized vector is a two-dimensional object, whereas the eigenvectors are specified as two projective two-dimensional objects. As a result, we can represent the shape of a trajectory unambiguously with a point on either the light cone, two-sheet hyperboloid or one-sheet hyperboloid depending on the regime of the system (i.e. underdamped, overdamped and critically damped).

Overall, we fully acknowledge the incontrovertible mathematical fact that because split-quaternions are isomorphic to $2\times2$-matrices, they cannot possibly do more than the matrices can. Our point is rather that the chart mapping (in the literal sense) from the split-quaternions translates to the actual behavior of the dynamical system in a more straightforward manner than the matrix representation does.

\section*{Recommendations}

\subsection*{Economic engineering}
In this thesis, we have assigned the geometric infrastructure that underlies Lagrangian mechanics with a compelling economic interpretation; being that the Lagrange 2-form is analogous to the Slutsky matrix in microeconomics. However, we have not elaborated on this result, and recommend that further research in the field of economic engineering puts this hypothesis to the test and investigates its potential consequences.

\subsection*{Geometric structures for dissipative mechanics}
We have shown that any contact symplectic manifold can be lifted canonically to a symplectic manifold through a procedure called \emph{symplectification}. We have applied this to the contact Hamiltonian system of the damped harmonic oscillator and demonstrated its correspondence with the Caldirola-Kanai method. However, we have not symplectified the contact Hamiltonian system for the oscillator with serial and parallel damper; this can be an interesting subject for future research. According to our views, it should also be possible to derive a Caldirola-Kanai-type Hamiltonian via this method for the damper with two oscillators. What is more, \emph{every} contact can be symplectified, so a Caldirola-Kanai type system can be derived for any mechanical system that can be written as a contact Hamiltonian system.

Along the same line, it has been shown that any Jacobi structure can be lifted to a manifold with a homogeneous Poisson structure, this is called \emph{Poissonization} (symplectification is a particular case of this) \cite{marle1991}. Hence, it might be possible extend the practice of symplectification to the Jacobi structure for general mechanical systems. 

On a more general note regarding the proposed Jacobi structure, we have not formally proven that the proposed structure always meets the required conditions to be a Jacobi manifold. Also, since there have been extensions for e.g. the Noether theorem and symplectic reduction to contact manifolds, there may be equivalent theorems for Jacobi manifolds as well. As such, more research is required to investigate the mathematical properties of this specific type of Jacobi structure in greater detail.

From the perspective of control theory, the `Jacobi systems' as they have been described in this thesis can be used in the framework of port-Hamiltonian systems (see \citet{VanDerSchaft2006}) and the associated control formalism to facilitate energy-based control for general mechanical systems.

Finally, we would be interested to see whether a Hamilton-Jacobi-type equation can be developed for Jacobi manifolds as well (at least, for this particular class of Jacobi structures).

\subsection*{Split-quaternion representation of dynamical systems}
Concerning the split-quaternions, we have limited ourselves to using the split-quaternions to analyze the dynamical system in a more convenient fashion. We have not, however, ventured in to the field of control using the split-quaternion representation. Although from a purely mathematical perspective, the split-quaternions cannot do more than their matrix counterparts, their properties might make some practices in control easier. For example, algorithms for pole placement are typically fairly numerically unreliable, which might be improved by using split-quaternions instead. For exactly the same reason, the split-quaternion representation may prove to be advantageous for system identification as well.

Arguably the most prominent limitation of the split-quaternions is that they are not immediately applicable to dynamical systems of dimensions greater than two. However, it may be possible to consider larger (even-dimensional) systems as interconnections of several atomic split-quaternion systems (representing a single pole pair). In any case, control applications often focus on the \emph{dominant} pole pair, which can of course easily represented by a single split-quaternion.

We have shown that the models of the hyperbolic plane can serve as a maps for the normalized vector part of the split-quaternions; and have related the shape parameters of the solution trajectories of an underdamped system. It is relatively straightforward to extend these ideas to critically damped and underdamped systems, but it is not covered in this thesis. Finally, we have also not touched upon an important feature of the hyperbolic plane and its models: the geodesics. These represent the paths of shortest distance connecting two points on the hyperboloid (or in the Poincaré or Cayley-Klein disks).  The measure of distance in the Lorentz space as a whole is specified by the Lorentzian norm, which is equivalent to the magnitude of the imaginary part of the eigenvalue. It is not altogether clear what the meaning of these paths is in terms of the associated dynamical system: they represent a specific path from the shape of one trajectory to another, thereby remaining in the same regime. We find this question a very compelling subject for future research.

