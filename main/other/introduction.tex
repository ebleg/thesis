% Introduction
\chapter{Introduction}
\label{chap:intro}

\lsymb{$\vec{v}$}{A (tangent) vector}

Idea: introduction using Arnol'ds thermodynamics quote.

%\subsection*{Notation check}
%\begin{table}[h]
%    \centering
%    \begin{tabular}{lcccc}
%    \toprule
%        \textbf{Object} & \textbf{Roman lower} & \textbf{Roman upper} & \textbf{Greek lower} & \textbf{Greek upper} \\
%    \midrule
%        Standard & $a b c d e$ & $A B C D E$ & $\alpha \beta \gamma \delta \epsilon $ & $ \Gamma \Delta \Upsilon \Omega \Theta $\\
%        Vector & $\vec{a} \vec{b} \vec{c} \vec{d} \vec{e}$ & $\vec{A} \vec{B} \vec{C} \vec{D} \vec{E}$ & $\vec{\alpha} \vec{\beta} \vec{\gamma} \vec{\delta} \vec{\epsilon} $ & $ \vec{\Gamma} \vec{\Delta} \vec{\Upsilon} \vec{\Omega} \vec{\Theta} $\\
%        Tensor & $\tens{a} \tens{b} \tens{c} \tens{d} \tens{e}$ & $\tens{A} \tens{B} \tens{C} \tens{D} \tens{E}$ & $\tens{\alpha} \tens{\beta} \tens{\gamma} \tens{\delta} \tens{\epsilon} $ & $ \tens{\Gamma} \tens{\Delta} \tens{\Upsilon} \tens{\Omega} \tens{\Theta} $\\
%    \bottomrule
%    \end{tabular}
%    \caption{Caption}
%    \label{tab:my_label}
%\end{table}
%
%%Christoffel symbol: $\christ$\\
%%Math constants: $\ii \ec \pic$\\
%%Variation: $\var S$\\
%Musical isomorphism\\
%Flat: $\toDual{X}$\\
%Sharp: $\fromDual{\omega}$\\
%Lie derivative: $\lied{X}{H}$\\
%Interior product: $\intpr{X}{\omega}$\\
%Lowercase mathcal: $\mathcal{i}$\\
%Kinematic momentum: $\mathfrak{p}p$\\
%\bundle{E}{\pi}{B}\\
%$\bsection{\tbundle{M}}$\\
%$\vfields{\tbundle{M}}$\\
%
%
%\subsection*{About mathematical notation and sign conventions}
%For symplectic geometry, the sign convention used by \citet{Abraham1978} and \citet{Cannas2001} is observed --- not the one used by Arnol'd in his \emph{Mathematical methods of classical mechanics}, nonetheless often referred to in this text.
%
%\begin{itemize}
%    \item Matrices, vectors and tensors are bold upper case.
%    \item Differential forms are typically denoted by Greek letters, with their rank as a superscript (cf. Arnol'd). 
%\end{itemize}
