% Introduction
\chapter{Introduction}
\label{chap:intro}

Real mechanical systems are never conservative. There are always elements of friction, and it is of great importance for engineers to take those into account in the practice of system modeling, control and design. Frictional forces pose no problem in Newtonian mechanics --- the traditional method of choice for mechanical engineers --- but they do in the alternative Lagrangian and Hamiltonian formalisms. Indeed, the field of analytical mechanics (as opposed to Newtonian \emph{vectorial} mechanics) does not usually consider friction. The reason for this is that the underlying geometric infrastructure of analytical mechanics, called \emph{symplectic geometry} is not suited for nonconservative systems.

For the physicists that use analytical mechanics, this is not usually a big issue, for the systems they are concerned with are so either so small or idealized that the effects of dissipation are benign or nonexistent altogheter. As mentioned, engineers do not have this luxury, given the inevitability of friction in the macroscopic world.

Some solutions have been proposed to include dissipation into the symplectic framework nonetheless. The first are the time-dependent systems, which specify explicitly how the energy in the system is changing, thereby making the system nonautonomous, e.g. the methods proposed by \citet{Caldirola1941} and \citet{Kanai1948}. However, in engineering, time-dependence is usually reserved for exogenous inputs, being either controlled inputs or uncontrolled disturbance or noise inputs into the system. A second solution is to use a complex formulation of the system states but then a modification of the underlying the complex structure is required, see for example \citet{Hutters2020}, \citet{Dedene1980} and \citet{Rajeev2007}.

In contrast to these approaches, we will not try to use a symplectic structure for dissipative systems. Instead, we draw inspiration from the mathematical theory of thermodynamics to impose a different geometric structure on the mechanical system. For simple systems, this geometric structure is a \emph{contact structure}, but for more general mechanical systems the contact structures proves to be insuficient: it has to be modified into a specific instance of a \emph{Jacobi structure}.

The applicability of contact structures and Jacobi structures has already been recognized by some authors in the past, see respectively \citet{Bravetti2017} and \citet{ciaglia2018}. However, there arguments are mainly of a mathematical nature, and do not offer direct physical insight into the system. Vladimir Arnol's once wrote\footnote{See \emph{Contact Geometry: the Geometrical Method of Gibbs' Thermodynamics} as a part of the 1989 \emph{Proceedings of the Gibbs Symposium} \cite[p. 163]{Arnold1989b}.}
\begin{quote}
``Every mathematician knows that it is impossible to understand any elementary course in thermodynamics. The reason is that thermodynamics is based [...] on a rather complicated type of geometry, called contact geometry''.
\end{quote}
We prefer to turn this issue the other way around, for thermodynamics comes entirely natural to the engineer but contact geometry certainly does not. As a result, the formulation of the geometric structures that we propose have a direct physical interpretation rooted in both thermodynamic and classical mechanics.




