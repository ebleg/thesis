% Introduction
\chapter{Introduction}
\label{chap:intro}

Real mechanical systems are never conservative. There are always elements of friction, and it is of great importance for engineers to take those into account in the practice of system modeling, control and design. Frictional forces pose no problem in Newtonian mechanics --- the traditional method of choice for mechanical engineers --- but they do in the alternative Lagrangian and Hamiltonian formalisms. Indeed, the field of analytical mechanics (as opposed to Newtonian \emph{vectorial} mechanics) does not usually consider friction. The reason for this is that the underlying geometric infrastructure of analytical mechanics, called \emph{symplectic geometry} is not suited for nonconservative systems.

For the physicists that use analytical mechanics, this is not usually a big issue, for the systems they are concerned with are so either so small or idealized that the effects of dissipation are benign or nonexistent altogheter. As mentioned, engineers do not have this luxury, given the inevitability of friction in the macroscopic world.

The reason why we want to use analytical mechanics rather than Newtonian mechanics is twofold:
\begin{itemize}
    \item First, the methods of Hamiltonian and Lagrangian mechanics offer a mathematically powerful and more economical description of mechanics. For example, one can use Noether's theorem to derive conservation principles of the system based on its Lagrangiand and Hamiltonian. For the purposes of modeling and control, several methods have been devised specifically built on the Hamiltonian formalism, e.g. port-Hamiltonian systems \cite{VanDerSchaft2006}. 
    \item The second reason originates in the discipline of \emph{economic engineering}, which is the field of study of the research group for which this thesis is essentially written. In economic engineering, analogies are used between mechanical and electrical engineering to build economic networks, so as to be able to produce \emph{causal} models for economic systems. In opposition to `classic' \emph{black box} models used by econometrists, economic engineering models are \emph{gray box}, which is to say that they are based on first principles. Perhaps contrary to classical mechanics, it is the Hamiltonian and Lagrangian formalism that is the most intuitive and fruitful for economic systems. However, just like mechanical systems, `dissipative effects' in the economy are equally inevitable. This is why we are in need for a reconciliation of analytical mechanics and energy dissipation (or the economic analogue thereof).
\end{itemize}

Some solutions have been proposed to include dissipation into the symplectic framework of Hamiltonian and Langrangian mechanics nonetheless. The first are the time-dependent systems, which specify explicitly how the energy in the system is changing, thereby making the system nonautonomous, e.g. the methods proposed by \citet{Caldirola1941} and \citet{Kanai1948}. However, in engineering, time-dependence is usually reserved for exogenous inputs, being either controlled inputs or uncontrolled disturbance or noise inputs into the system. A second solution is to use a complex formulation of the system states but then a modification of the underlying the complex structure is required, see for example \citet{Hutters2020}, \citet{Dedene1980} and \citet{Rajeev2007}.

In contrast to these approaches, we will not try to use a symplectic structure for dissipative systems. Instead, we draw inspiration from the mathematical theory of thermodynamics to impose a different geometric structure on the mechanical system. For simple systems, this geometric structure is a \emph{contact structure}, but for more general mechanical systems the contact structures proves to be insufficient: it has to be modified into a specific instance of a \emph{Jacobi structure}. These Jacobi structures include the aforementioned contact and symplectic manifolds as specific instances.

The applicability of contact structures and Jacobi structures has already been recognized by some authors in the past, see respectively \citet{Bravetti2017} and \citet{ciaglia2018}. However, there arguments are mainly of a mathematical nature, and do not offer direct physical insight into the system. As Vladimir Arnol's once wrote\footnote{See \emph{Contact Geometry: the Geometrical Method of Gibbs' Thermodynamics} as a part of the 1989 \emph{Proceedings of the Gibbs Symposium} \cite[p. 163]{Arnold1989b}.}
\begin{quote}
``Every mathematician knows that it is impossible to understand any elementary course in thermodynamics. The reason is that thermodynamics is based [...] on a rather complicated type of geometry, called contact geometry''.
\end{quote}
We prefer to turn this issue the other way around, for thermodynamics comes entirely natural to the engineer but contact geometry certainly does not. As a result, the formulation of the geometric structures that we propose have a direct physical interpretation rooted in both thermodynamic and classical mechanics.

In addition to the differential geometric structure underlying the mechanical systems, we also propose an alternative \emph{representation} of these systems in the form of split-quaternions\footnote{Also called \emph{coquaternions}}. The split-quaternions are, like regular quaternions, a four-dimensional number system. In contrast to regular quaternions, split-quaternions have both a complex and split-complex nature. It has been known for some time that the split-quaternion algebra is isomorphic to the algebra of real $2\times2$-matrices \cite{Jafari2014}. Since this matrix representation is usually the preferred field of study, there has been relatively little written about the split-quaternions. 

We make the case in this thesis that the split-quaternions, rather than matrices, provide a powerful representation of linear dynamical systems (and mechanical systems in particular). The natural properties of the split-quaternions make it comparatively easy to study the properties of the associated mechanical system. To the knowledge of the author, the relation between split-quaternions and dynamical (or mechanical) systems has never been studied in this way. As a result, the findings in this thesis pertaining to this relation are all new.

\subsection*{Outline of the thesis}
This main content of this thesis organized into three parts. 

Firstly, in \cref{chap:symplectic_economics}, we explain the role of symplectic geometry, and Hamiltonian and Lagrangian mechanics in economic engineering. Furthermore, we explain how what the analogue of mechanical dissipation is in economic engineering. In this chapter, we only focus on the economic analogies, and do not provide a comprehensive introduction to the geometry behind Hamiltonian mechanics, which is given in \cref{chap:geometric_structures}. %Hence, the reader may want to consult \cref{chap:symplectic_economics} and \cref{chap:symplectic_economics} in parallel.

Secondly, in \cref{chap:geometric_structures}, we build our way progressively from conservative mechanical systems defined on symplectic manifolds, to `simple' mechanical systems systems on contact manifolds to general mechanical systems defined on Jacobi manifolds. The part about symplectic manifolds serves as an introduction to the classical geometric theory of Hamiltonian mechanics, but does not contribute new theory. Subsequently, we use thermodynamic theory to construct a contact Hamiltonian system for both the damped harmonic oscillator in its simplest form and equipped with an additional serial damper. We then use a technique called \emph{symplectification} to cast the contact Hamiltonian system into a symplectic Hamiltonian system, and reveal its close relation with the well-known Caldirola-Kanai model.

Thirdly, in \cref{chap:quaternion}, we use the algebra of split-quaternions to represent two-dimensional linear dynamical systems. The split-quaternions can be used to conveniently classify the qualitative nature of such systems, including all the `edge cases' that are often ignored. In addition, the solution of the dynamical system is easily obtained using split-quaternion representation as well. Furthermore, we use the damped harmonic oscillator again as an exemplary system and use its split-quaternion representation to analyze the geometry of the associated solution trajectories, focusing particularly on underdamped systems.

Finally, in \cref{chap:conclusion} we conclude the thesis and provide recommendations for future research on the subject.


