% Introduction
\chapter{Introduction}
\label{chap:intro}

Symplectic manifolds are widely recognized as the appropriate geometric setting for classical mechanics. The reason is that the symplectic structure facilitates the mechanism of both Hamiltonian and Lagrangian mechanics. However, autonomous mechanical systems with a symplectic structure are necessarily conservative. By its very nature, the symplectic structure leaves no room for dissipative phenomena in the system. For the physicists that use analytical mechanics, this does not usually cause significant trouble, for the systems they are concerned with are so small or idealized that the effects of dissipation are benign or nonexistent altogether.

In contrast, energy dissipation is ubiquitous in many engineering applications, whether mechanical, electrical, or economic. This is because there are virtually always resistive or frictional elements present whose influence on the system cannot be ignored. Consequently, engineers typically revert to Newtonian (or vectorial) mechanics rather than Hamiltonian or Lagrangian (or analytical) mechanics.

Despite this shortcoming, we believe that analytical mechanics does offer substantial advantages in engineering applications over the Newtonian framework. We give two reasons to support this claim:

The first reason originates in the discipline of \emph{economic engineering}, which is the field of study of the research group for which this thesis is written. In economic engineering, analogies are used between the mechanical, electrical, and economic domains to produce \emph{causal} models for economic systems. In opposition to classic \emph{black box} models used by econometrists, economic engineering models are \emph{gray box}, which is to say that the latter are based on first principles instead of pure statistics. 

The economic engineer can barely go without analytical mechanics. This is because, perhaps contrary to classical mechanics, the Hamiltonian and Lagrangian formalisms are the most intuitive from an economic perspective, compared to Newtonian mechanics. The role of Lagrangian and Hamiltonian mechanics in economic engineering is explained in \cref{chap:symplectic_economics}. 
Because dissipative phenomena are as common in practical economic systems as in mechanical systems, we feel that there is a need to reconcile analytical mechanics with energy dissipation, or the economic analog thereof.

The second reason is that Hamiltonian and Lagrangian mechanics are based on the \emph{energy description} of the mechanical system, in contrast to using forces, as is done in Newtonian mechanics. The energy description is often much more economical and is easily constructed based on physical observation, even for complicated systems. In addition, the sophistication of these methods allows one to use mathematically powerful concepts such as symmetry to gain insights into the system, e.g., using Noether's theorem.

Some solutions have been proposed in the past to include dissipation into the symplectic framework of Hamiltonian and Langrangian mechanics nonetheless. The first is using a time-dependent formulation, which specifies explicitly how the energy in the system is changing. Notable examples of these nonautonomous systems are the methods proposed by \citet{Caldirola1941} and \citet{Kanai1948}. However, in engineering, time dependence is usually reserved for exogenous inputs, which are either controlled inputs or uncontrolled disturbances into the system. A second solution is to use a complex formulation of the system states. A drawback of these methods is that they require a modification of the underlying complex structure, see for example \citet{Hutters2020}, \citet{Dedene1980} and \citet{Rajeev2007}.

In contrast to these approaches, we will not use a symplectic structure for dissipative systems. Instead, we draw inspiration from the mathematical theory of thermodynamics to develop a different geometric structure for the mechanical system.

For simple systems, this geometric structure is a \emph{contact structure}. However, for more general multi-degree of freedom mechanical systems, a contact structure proves to be insufficient: it has to be modified into a specific instance of the overarching class of \emph{Jacobi structures}. 

Some authors have already recognized the applicability of contact and Jacobi structures in the past; see respectively \citet{Bravetti2017} and \citet{ciaglia2018}. However, the arguments made in the existing literature are mainly mathematical in nature, and the contact structure does not translate to the physical aspects of the mechanical system. As Vladimir Arnold's once wrote\footnote{See \emph{Contact Geometry: the Geometrical Method of Gibbs' Thermodynamics} as a part of the 1989 \emph{Proceedings of the Gibbs Symposium} \cite[p. 163]{Arnold1989b}.}
\begin{quote}
``Every mathematician knows that it is impossible to understand any elementary course in thermodynamics. The reason is that thermodynamics is based [...] on a rather complicated type of geometry, called contact geometry.''
\end{quote}
We prefer to turn this issue the other way around, for thermodynamics comes entirely natural to the engineer, but contact geometry certainly does not. Therefore, we propose a formulation of the geometric structure that has a direct physical interpretation rooted in both thermodynamics and classical mechanics.

In \cref{chap:geometric_structures}, we use the thermodynamic insights to progressively build our way from contact Hamiltonian systems to simple mechanical systems, ultimately leading to Jacobi-Hamiltonian systems that can be applied to any mechanical system.
%Apart from the pure development of these geometric structures, we also make two slight digressions:
%First, we use the systematic approach that we have developed to construct a contact Hamiltonian system for the harmonic oscillator with \emph{two} dampers (one in series and one in parallel): this system is particularly interesting since it exhibits a complete mathematical symmetry, and has plays an important role in economic engineering.
%
%Second, using a process called \emph{symplectification}, we lift the contact Hamiltonian system for the damped harmonic oscillator to a symplectic manifold. We show how this lifted Hamiltonian system is directly equivalent to the time-dependent Caldirola-Kanai Hamiltonian.

Whereas \Cref{chap:geometric_structures} exclusively considers mechanical systems, \cref{chap:symplectic_economics} explains the associated economic analogies. Using these analogies, the findings in \cref{chap:geometric_structures} can be readily translated to the domain of economic engineering. 
The contents of \cref{chap:symplectic_economics} can be viewed as separate and are not prerequisites for the rest of the thesis.

In addition to the differential geometric structure underlying the mechanical systems, we also propose a new representation of mechanical systems (and dynamical systems in general) in the form of split-quaternions\footnote{Also colloquially known as \emph{coquaternions}.} in \cref{chap:quaternion}. 

We make the case in this thesis that the split-quaternions provide a powerful alternative to the traditional state-space form of linear dynamical systems. This is because the natural properties of the split-quaternions coincide with the properties of the associated dynamical system. As a result, the classification of dynamical systems follows almost immediately from the corresponding split-quaternion. Furthermore, we also show how the geometry of the solution trajectories can be obtained directly from the split-quaternion representation.

To the author's knowledge, the relation between split-quaternions and dynamical (or mechanical) systems has never been studied in this way. As a result, the findings in this thesis about this relation are all new.
