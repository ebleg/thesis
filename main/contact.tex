\chapter{Symplectified Contact Mechanics for Dissipative Systems}
\label{chap:contact_mechanics}

The traditional view is that the methods of analytical mechanics, such as the Lagrangian and Hamiltonian formalisms, are
only suited for conservative systems. However, several attempts, especially in the previous century, have been made to
extend these principles to dissipative systems as well. 

\section{The damped harmonic oscillator}
This chapter (and the application in the following chapter) is primarily concerned with the prototypical dissipative mechanical system: the linearly damped harmonic oscillator depicted in \cref{fig:dho}, with the governing second-order differential equation being
\begin{equation}  
  m\ddot{q} + b\dot{q} + kq = 0.
\end{equation}
The choice for this system is rather perspicuous, since it is arguably the `easiest' dissipative system that also exhibits second-order dynamics and is linear in all terms. Furthermore, as discussed below, it serves as the test case of the overwhelming majority of research into dissipative Lagrangian and Hamiltonian mechanics
\cite{Dekker1981,Hutters2020b}. However, the method described in this section can be generalized directly to a general (possibly time-dependent) potential function $V = V(q, t)$.
\begin{figure}
    \begin{center}
        \begin{tikzpicture}[every node/.style={outer sep=0pt,thick}]
    \tikzstyle{spring}=[thick,decorate,decoration={zigzag,pre length=0.3cm,post length=0.3cm,segment length=6}]
    \tikzstyle{damper}=[thick,decoration={markings,  
      mark connection node=dmp,
      mark=at position 0.5 with 
      {
        \node (dmp) [thick,inner sep=0pt,transform shape,rotate=-90,minimum width=15pt,minimum height=3pt,draw=none] {};
        \draw [thick] ($(dmp.north east)+(2pt,0)$) -- (dmp.south east) -- (dmp.south west) -- ($(dmp.north west)+(2pt,0)$);
        \draw [thick] ($(dmp.north)+(0,-5pt)$) -- ($(dmp.north)+(0,5pt)$);
      }
    }, decorate]
    \tikzstyle{ground}=[fill,pattern=north east lines,draw=none,minimum width=0.75cm,minimum height=0.3cm]

    \node (M) [draw,minimum width=1cm, minimum height=1.5cm] {$m$};

    \node (ground) [ground,anchor=north,yshift=-0.25cm,minimum width=1.5cm] at (M.south) {};
    \draw (ground.north east) -- (ground.north west);
    \draw [thick] (M.south west) ++ (0.2cm,-0.125cm) circle (0.125cm)  (M.south east) ++ (-0.2cm,-0.125cm) circle (0.125cm);

    \node (wall) [ground, rotate=-90, minimum width=2cm,yshift=-3cm] {};
    \draw (wall.north east) -- (wall.north west);

    \draw [spring] (wall.160) -- ($(M.north west)!(wall.160)!(M.south west)$) node[pos=0.5,anchor=south, outer sep=4pt] {$k$};
    \draw [damper] (wall.20) -- ($(M.north west)!(wall.20)!(M.south west)$) node[pos=0.5,anchor=north, outer sep=10pt] {$b$};

    \path (wall) ++(0.1cm, 1.2cm) -| node (q) {} (M);
    \draw[|->] (wall) ++(0.2cm, 1.2cm) -- (q.center) node[pos=0.5, anchor=south] {$q$};
    \draw (q) ++(0, 0.1cm) -- ++(0, -0.5cm);

    \node[bgelement] (J1) at (4.5, -1) {1};
    \node[bgelement, label=north:$k$] (C) at (4.5, 0.5) {C};
    \node[bgelement, label=east:$m$]  (I) at (6, -1) {I};
    \node[bgelement, label=west:$b$]  (R) at (3, 0.5) {R};

    % test
    \draw[bonds] 
        (J1) edge[e_out] (I)
        (J1) edge[f_out] (R)
        (J1) edge[f_out] (C);


\end{tikzpicture}

    \end{center}
    \caption{Schematic of the mass-spring-damper system.}
    \label{fig:dho}
\end{figure}

\begin{table}[ht!]
    \caption{Parameter conventions of the damped harmonic oscillator. To avoid confusion with the symplectic form $\omega$, angular frequencies are denoted by $\Omega$ instead of the conventional lower case Greek letter.}
    \label{tab:dho_params}
    \begin{center}
        \begin{tabular}{llll}
            \toprule
            \textbf{Name} & \textbf{Symbol} & \textbf{Value} & \textbf{Units} \\
            \midrule
            Damping coefficient & $\gamma$ & $b/m$ & \si{\per \second }\\[0.4cm]
            Undamped frequency & $\Omega_o$ & $\sqrt{k/m}$ & \si{\per \second }\\[0.4cm]
            Damped frequency & $\Omega_d$ & $\sqrt{\Omega_0^2 - \qty(\frac{\gamma}{2})^2}$ & \si{\per \second }\\[0.4cm]  
            Damping ratio & $\zeta$ & $\frac{b}{2\sqrt{mk}}$ & -- \\[0.2cm]
            \bottomrule
        \end{tabular}
    \end{center}
\end{table}

\section{The Caldirola-Kanai method}
\label{sec:historical}
A traditional, engineering-inclined method to incorporate damping in the framework is to include a Rayleigh damping term in the Lagrangian to emulate linear damping forces, and this works `mathematically' to derive the correct equations of motion \cite{Goldstein2011}. Although frequently used for practical problems, this damping term is not really part of the \emph{actual} Lagrangian --- rather, it simply makes use of the notion of a generalized force that is not inherently part of the system. As such, this method only works on a superficial level: the pristine differential geometric foundations of mechanics do not leave room for such ad hoc tricks. There is, as a result, also no Hamiltonian counterpart for this method. 

The historical attempts to do better than the Rayleigh method were primarily motivated by the application of the (dissipative) Hamiltonian formalism in quantum mechanics through discretization. For this application, a sound mathematical structure is of the essence, which calls for a more rigorous approach. A celebrated paper by
\citet{Dekker1981} provides an excellent summary of many attempts up to 1981. Indeed, the well-studied approach developed by \citet{Caldirola1941} and \citet{Kanai1948} was intended exactly for this purpose. This method features an explicit time-dependence both in the Lagrangian function
\begin{equation}
    \Lck(q, \dot{q}, t) = \ec^{\gamma t}\qty(\frac{1}{2}m\dot{q}^2 - \frac{1}{2}kq^2),
    \label{eq:lag_CK}
\end{equation}
and the corresponding Hamiltonian function:
\begin{equation}
    \Hck(q, \Pcan, t) = \frac{\Pcan^2}{2m}\ec^{-\gamma t} + \frac{1}{2}kq^2\ec^{\gamma t}.
    \label{eq:ham_CK}
\end{equation}
In latter equation, $\Pcan$ refers to a special `canonical momentum', that is
\begin{equation}
    \Pcan \equiv \pdv{\Lck}{\dot{q}},
    \label{eq:can_momentum}
\end{equation}
which is related to the `true` kinematic momentum by the relation $\Pcan = p\ec^{\gamma
t} = m\dot{q}\ec^{\gamma t}$. As such, it is also clear that the Caldirola-Kanai Lagrangian and Hamiltonian functions are related by the Legendre transform \emph{with respect to the canonical momentum}:\footnote{The `Legendre transform'
refers, in the context of fiber bundles, to the so-called fiber derivative. On a manifold $M$, let $L \in
\functions{M}$. Then the fiber derivative is defined als 
    $$ \fiberder{L}: \tbundle{L}\to\ctbundle{L}: \fiberder{L}(\vec{v})\cdot\vec{w} = \left. \dv{}{s}\right\vert_{s = 0} L(\vec{v} +
    s\vec{w}). $$
 Hence, the Legendre transform is in the first place the mapping that associates the generalized velocities with the
 associated (canonical) generalized momenta. Importantly, this mapping is a diffeomorphism (that is, invertible and onto) if the Hessian of
 $L$ is nondegenerate - roughly equivalent to the statement that every generalized velocity has an associated `mass` to
 it. \cite{Marsden1998}}
$$ \Hck = \Pcan \dot{q} - \Lck. $$
From either \cref{eq:lag_CK} or \cref{eq:ham_CK}, the equations of motion are readily
derived (for the Hamiltonian case with respect to $\Pcan$ after which the transformation to $p$ can be effected).
Indeed, after taking the appropriate derivatives, one obtains:
\begin{equation*} 
    \begin{split}
        \dv{}{t}\qty(\pdv{\Lck}{\dot{q}}) - \pdv{\Lck}{q} &= 0 \\
        \Rightarrow \ec^{\gamma t}\qty(m\ddot{q} + m\gamma\dot{q} + kq) &= 0
    \end{split}
\end{equation*}
for the Lagrangian case. Likewise, Hamilton's equations yield: \cite{Tokieda2021}
\begin{equation*}
    \begin{split}
        \dot{q} &= \pdv{\Hck}{\Pcan} = \frac{\Pcan}{m}\ec^{-\gamma t} =  \frac{p}{m}, \\
        \dot{\Pcan} &= -\pdv{\Hck}{q} = -kq\ec^{\gamma t}.\\
    \end{split}
\end{equation*}
The relation between the time derivatives of the momenta $\dot{p}$ and $\dot{\Pcan}$ is slightly more involved since one must invoke the product rule as a result of their time-dependent relation:
    $$ \dot{\Pcan} = \ec^{\gamma t}\qty(\dot{p} + \gamma p). $$
Substition yields the correct equation for $p$, though the equation is again multiplied by $\ec^{\gamma t}$. Because the latter is sufficiently well-behaved (that is, it has no zeros), it can be removed without any problems.

\paragraph{Geometric perspective}
To put the above derivation in a geometric setting, define the Liouville 1-form as
$$ \alpha = \Pcan\dd{q} \quad \Rightarrow \quad \omega = -\dd{\alpha} = \wedgep{\dd{q}}{\dd{\Pcan}},$$
where the symplectic 2-form will be used to obtain Hamilton's equations. The Hamiltonian \cref{eq:ham_CK} is explicitly time-dependent. This will give rise to a time-dependent vector field governing the solution curves.\footnote
{A \emph{time-dependent vector field} on a manifold $M$ is a mapping $X: M\times\real \to \tbundle{M}$ such that for each $t \in \real$, the restriction $X_t$ of $X$ to $M \times \{t\}$ is a vector field on $M$. \cite{Libermann1987} An additional construction of importance, called the \emph{suspension} of the vector field, is a mapping $$ \tilde{X}: \real \times M \to \tbundle{(\real \times M)} \quad (t, m) \mapsto ((t, 1), (m, X(t, m))),$$ that is to say, it lifts the vector field to the extended space that also includes $t$ and assigns the time coordinate with a trivial velocity of 1. \cite{Abraham1978}}
The construction of the vector field associated with a time-dependent Hamiltonian follows the same construction rules as a normal Hamiltonian (using the isomorphism given by $\omega$), but `frozen' at each instant of $t$. Even more bluntly speaking, one simply ignores the $t$-coordinate during the derivation, only to acknowledge the dependence at the very end. This leads to the following vector field, `suspended' on the $\real\times Q$ space:
$$ \tilde{X}_{\Hck} = -\ec^{\gamma t}kq\pdv{}{\Pcan} + \ec^{-\gamma t}\frac{\Pcan}{m}\pdv{}{q} + \pdv{}{t}.$$
The suspension is important to make the final coordinate transform from $\rho$ to $p$ work properly. Indeed, effecting the transformation $(q, \rho, t) \mapsto (q, \ec^{-\gamma t}, t)$, one obtains
$$ \tilde{X}_{\Hck} = \qty(-kq - \gamma p)\pdv{}{p} + \frac{p}{m}\pdv{}{q} + \pdv{}{t}.$$
It is worthwile to ponder on some apparent peculiarities in the Caldirola-Kanai method, for they will be explained elegantly by the contact-Hamiltonian formalism. Firstly, the role of the two-different momenta is not very clear from the get-go, apart from being a consequence of the way the Caldirola-Kanai Lagrangian is formulated. This has also been the reason for considerable confusion in the academic community (see \citet{Schuch1997}). Furthermore, there is the 

\section{Symplectification and Liouville structures}
We start with an $n$-dimensional \emph{base manifold} $Q$. In the context mechanical systems, this manifold is the configuration manifold of the system, \emph{extended} with an additional, `special` position coordinate coordinate that will be interpreted later. Let us assume that $Q$ has coordinates $\vec{q} = \qty(q_0, q_1, \ldots, q_n)$. For the damped harmonic oscillator, this manifold is two-dimensional, for it contains just the special coordinate and the position of the mass. Without loss of generalization, we will denote the special position coordinate by $q_0$.

Now, introduce the \emph{manifold of contact elements} $\cbundle{Q}$, to the base manifold $Q$. This is the manifold of all points in $Q$, with the space of all possible tangent hyperplanes at every point. This manifold has dimension $2n-1$.

The manifold of contact elements to $Q$ can be identified with the projectivization of the cotangent bundle $\ctbundle{Q}$, denoted by $\pctbundle{Q}$. This manifold is of dimension $2n-1$. 

\section{Dissipative contact Hamiltonian mechanics}
The contact-geometric counterpart of Hamiltonian and Lagrangian mechanics has been the subject of increasing academic interest in recent years, see for example \citet{VanderSchaft2021a,VanderSchaft2018,Maschke2018,Bravetti2017,DeLeon2020}, etc. The conception of the idea arguably traces back to the work of Herglotz \cite{Guenther1996}, who derived it using the variational principle, and the developments in differential geometry, by e.g. \citet{Arnold1989} and \citet{Libermann1987}.

In this section, the direct connection is made between the Caldirola-Kanai Hamiltonian given by \cref{eq:ham_CK} and the contact Hamiltonian described by \citet{Bravetti2017}, using Liouville geometry\footnote{It is interesting to note that Bravetti gives the Caldirola-Kanai method as an example of dissipative Hamiltonians in his paper, but fails to make the connection with his own method.}. It then proceeds to \emph{contact Lagrangian mechanics}, strongly related to the Herglotz' work. Finally, the whole theory can be explained from a thermodynamic perspective. While it was already known for some time (dating back to Arnol'd) that contact geometry is the preferred geometry for thermodynamics, its equivalence to contact geometry in (dissipative) classical mechanics has not been desribed in past literature. This somehow underpins a famous statement by Vladimir Arnol'd that `Contact geometry is all geometry', in the sense that conservative mechanical systems can be considered as part of a larger class of systems for which energy dissipation \emph{is} allowed. \cite{Geiges2008}

\subsection{From Caldirola-Kanai to contact mechanics}
Recall the Caldirola-Kanai Hamiltonian from \cref{sec:historical},
\begin{equation}
    \Hck(q, \Pcan, t) = \frac{\Pcan^2}{2m}\ec^{-\gamma t} + \frac{1}{2}kq^2\ec^{\gamma t},
    \label{eq:ham_CK2}
\end{equation}
and rewrite it as
\begin{equation}
    \Hck(q, \Pcan, t) = \ec^{\gamma t}\qty(\frac{1}{2m}\qty(\frac{\Pcan}{\ec^{\gamma t}})^2 + \frac{1}{2}kq^2).
\end{equation}
The idea is to view this Hamiltonian as a homogeneous function by introducing appropriate coordinates. Homogeneous momenta are denoted by $\rho_i$. By inspection, choose (with slight abuse of notation)
$$ \rho_1 = \rho,\quad \rho_0 = \ec^{\gamma t}\quad \text{and } q_1 = q$$
such that the Caldirola-Kanai Hamiltonian becomes
\begin{equation}
    H(q, \rho_0, \rho_1) = \rho_0\qty(\frac{1}{2m}\qty(\frac{\rho_1}{\rho_0})^2 + \frac{1}{2}kq_1^2).
\end{equation}
which is homogeneous in fiber dimensions $\rho_0$ and $\rho_1$. That is to say,
$$ H(q, \lambda \rho_0, \lambda \rho_1) = \lambda H(q, \rho_0, \rho_1) \quad \lambda \in \real_0$$
From the definition in \cref{eq:can_momentum}, one can see that the `real', kinematic momentum is equal to
$$ p \equiv \frac{\rho_1}{\rho_0}. $$
As such, we can make the distinction between two Hamiltonian functions: firstly, there is the \emph{contact Hamiltonian}
$$ \hat{H}(p, q_1) = \qty(\frac{1}{2m}p^2 + \frac{1}{2}kq^2), $$
and secondly, the \emph{homogeneous Hamiltonian},
$$ H(\rho_0, \rho_1, q_1) = \rho_0 \hat{H}\qty(\frac{\rho_1}{\rho_0}, q_1).$$
which is the `symplectified' version of the contact Hamiltonian. This allows us to derive the equations of motion using the symplectic structure
$$ \omega = \wedgep{\dd{q_0}}{\dd{\rho_0}} + \wedgep{\dd{q_1}}{\dd{\rho_1}}$$
and proceed as usual. The contact equations of motion are then obtained through the relation between $p, \rho_0$ and $\rho_1$.

\section{Dissipative contact Lagrangian mechanics}
TODO: fix notation, always include `1` subscripts?
$\ctzbundle{Q}$



%\begin{itemize}
%    \item The \textbf{mirror system} or \textbf{Bateman} approach, doubles the number of system dimensions by including a mirror system that runs opposite in time. Arguably the most flexible of all methods, it is 
%    \item Expressing the Hamiltonian in \textbf{complex coordinates} has also produced promising results: notable are the attempts of \citet{Bopp1974}, \citet{Dedene1980} and the very recent contribution by \citet{Hutters2020b} in the research group to which the author belongs as well.
%    \item A different approach, related to the contact method, are the 
%    \item Contact mechanics
%    \item Mathematical Hamiltonians
%\end{itemize}
%
%[Mention Max as contributor]
