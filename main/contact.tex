\chapter{Symplectified Contact Mechanics for Dissipative Systems}

The traditional view is that the methods of analytical mechanics, such as the Lagrangian and Hamiltonian formalisms, are
only suited for conservative systems. However, several attempts, especially in the previous century, have been made to
extend these principles to dissipative systems as well. 

\section{The damped harmonic oscillator}
This chapter (and the application in the following chapter) is
primarily concerned with the prototypical dissipative mechanical system: the linearly damped harmonic oscillator
depicted in \cref{fig:dho}, with the governing second-order differential equation being
\begin{equation}  
  m\ddot{q} + b\dot{q} + kq = 0.
\end{equation}
The choice for this system is rather perspicuous, since it is arguably the `easiest` dissipative system that also
exhibits second-order dynamics and is linear in all terms. Furthermore, as discussed below, it serves as the test case
of the overwhelming majority of research into dissipative Lagrangian and Hamiltonian mechanics
\cite{Dekker1981,Hutters2020b}. However, the method described in this section can be generalized directly to a general
(possibly time-dependent) potential function $V = V(q, t)$.
\begin{figure}
    \begin{center}
        \begin{tikzpicture}[every node/.style={outer sep=0pt,thick}]
    \tikzstyle{spring}=[thick,decorate,decoration={zigzag,pre length=0.3cm,post length=0.3cm,segment length=6}]
    \tikzstyle{damper}=[thick,decoration={markings,  
      mark connection node=dmp,
      mark=at position 0.5 with 
      {
        \node (dmp) [thick,inner sep=0pt,transform shape,rotate=-90,minimum width=15pt,minimum height=3pt,draw=none] {};
        \draw [thick] ($(dmp.north east)+(2pt,0)$) -- (dmp.south east) -- (dmp.south west) -- ($(dmp.north west)+(2pt,0)$);
        \draw [thick] ($(dmp.north)+(0,-5pt)$) -- ($(dmp.north)+(0,5pt)$);
      }
    }, decorate]
    \tikzstyle{ground}=[fill,pattern=north east lines,draw=none,minimum width=0.75cm,minimum height=0.3cm]

    \node (M) [draw,minimum width=1cm, minimum height=1.5cm] {$m$};

    \node (ground) [ground,anchor=north,yshift=-0.25cm,minimum width=1.5cm] at (M.south) {};
    \draw (ground.north east) -- (ground.north west);
    \draw [thick] (M.south west) ++ (0.2cm,-0.125cm) circle (0.125cm)  (M.south east) ++ (-0.2cm,-0.125cm) circle (0.125cm);

    \node (wall) [ground, rotate=-90, minimum width=2cm,yshift=-3cm] {};
    \draw (wall.north east) -- (wall.north west);

    \draw [spring] (wall.160) -- ($(M.north west)!(wall.160)!(M.south west)$) node[pos=0.5,anchor=south, outer sep=4pt] {$k$};
    \draw [damper] (wall.20) -- ($(M.north west)!(wall.20)!(M.south west)$) node[pos=0.5,anchor=north, outer sep=10pt] {$b$};

    \path (wall) ++(0.1cm, 1.2cm) -| node (q) {} (M);
    \draw[|->] (wall) ++(0.2cm, 1.2cm) -- (q.center) node[pos=0.5, anchor=south] {$q$};
    \draw (q) ++(0, 0.1cm) -- ++(0, -0.5cm);

    \node[bgelement] (J1) at (4.5, -1) {1};
    \node[bgelement, label=north:$k$] (C) at (4.5, 0.5) {C};
    \node[bgelement, label=east:$m$]  (I) at (6, -1) {I};
    \node[bgelement, label=west:$b$]  (R) at (3, 0.5) {R};

    % test
    \draw[bonds] 
        (J1) edge[e_out] (I)
        (J1) edge[f_out] (R)
        (J1) edge[f_out] (C);


\end{tikzpicture}

    \end{center}
    \caption{Schematic of the mass-spring-damper system.}
    \label{fig:dho}
\end{figure}
\towrite{Discuss parameter notation}

\begin{table}[h!]
    \caption{Parameter conventions of the damped harmonic oscillator. To avoid confusion with the symplectic form $\omega$, angular frequencies are denoted by $\Omega$ instead of the conventional lower case Greek letter.}
    \label{tab:dho_params}
    \begin{center}
        \begin{tabular}{llll}
            \toprule
            \textbf{Name} & \textbf{Symbol} & \textbf{Value} & \textbf{Units} \\
            \midrule
            Damping coefficient & $\gamma$ & $b/m$ & \si{\per \second }\\[0.4cm]
            Undamped frequency & $\Omega_o$ & $\sqrt{k/m}$ & \si{\per \second }\\[0.4cm]
            Damped frequency & $\Omega_d$ & $\sqrt{\Omega_0^2 - \qty(\frac{\gamma}{2})^2}$ & \si{\per \second }\\[0.4cm]  
            Damping ratio & $\zeta$ & $\frac{b}{2\sqrt{mk}}$ & -- \\[0.2cm]
            \bottomrule
        \end{tabular}
    \end{center}
\end{table}

%The first attempt is arguably made in the work of \citet{Bateman1931} in 1931, who used a `mirror` system in
%combination with the real system that runs with an opposite time direction --- as such, the irreversible dynamics are
%compensated by the mirror system, which ensures that the overal system can be cast in the necessarily even-dimensional
%(the Bateman approach automatically doubles the number of dimensions in the system) symplectic setting that underpins
%the Hamiltonian formalism\footnote{Strictly speaking, Bateman only covered the Lagrangian part of the story in his
%paper}.

\section{Historical perspectives}
A traditional, engineering-inclined method to incorporate damping in the framework is to include a Rayleigh damping term in the Lagrangian to emulate linear damping forces, and this works `mathematically` to derive the correct equations of motion \cite{Goldstein2011}. Although frequently used for practical problems, this damping term is not really part of the \emph{actual} Lagrangian --- rather, it simply makes use of the notion of a generalized force that is not inherently part of the system. As such, this method only `works' on a superficial level: the pristine differential geometric foundations of mechanics do not leave room for such ad hoc tricks. There is, as a result, also no Hamiltonian counterpart for this method. 

The historical attempts to do better than the Rayleigh method were primarily motivated by the application of the (dissipative) Hamiltonian formalism in quantum mechanics through discretization. For this application, a sound mathematical structure is of the essence, which calls for a more rigorous approach. A celebrated paper by
\citet{Dekker1981} provides an excellent summary of many attempts up to 1981. Indeed, the well-studied approach developed by \citet{Caldirola1941} and \citet{Kanai1948} was intended exactly for this purpose. This method features an explicit time-dependence both in the Lagrangian function
\begin{equation}
    \Lck(q, \dot{q}, t) = \ec^{\gamma t}\qty(\frac{1}{2}m\dot{q}^2 - \frac{1}{2}kq^2),
    \label{eq:lag_CK}
\end{equation}
and the corresponding Hamiltonian function:
\begin{equation}
    \Hck(q, \Pcan, t) = \frac{\Pcan^2}{2m}\ec^{-\gamma t} + \frac{1}{2}kq^2\ec^{\gamma t}.
    \label{eq:ham_CK}
\end{equation}
In latter equation, $\Pcan$ refers to a special `canonical momentum', that is
$$ \Pcan \equiv \pdv{\Lck}{\dot{q}}, $$
which is related to the `true` kinematic momentum by the relation $\Pcan = p\ec^{\gamma
t} = m\dot{q}\ec^{\gamma t}$. As such, it is also clear that the Caldirola-Kanai Lagrangian and Hamiltonian functions are related by the Legendre transform \emph{with respect to the canonical momentum}:\footnote{The `Legendre transform'
refers, in the context of fiber bundles, to the so-called fiber derivative. On a manifold $M$, let $L \in
\functions{M}$. Then the fiber derivative is defined als 
    $$ \fiberder{L}: \tbundle{L}\to\ctbundle{L}: \fiberder{L}(\vec{v})\cdot\vec{w} = \left. \dv{}{s}\right\vert_{s = 0} L(\vec{v} +
    s\vec{w}). $$
 Hence, the Legendre transform is in the first place the mapping that associates the generalized velocities with the
 associated (canonical) generalized momenta. Importantly, this mapping is a diffeomorphism (that is, invertible and onto) if the Hessian of
 $L$ is nondegenerate - roughly equivalent to the statement that every generalized velocity has an associated `mass` to
 it. \cite{Marsden1998}.}
$$ \Hck = \Pcan \dot{q} - \Lck. $$
From either \cref{eq:lag_CK} or \cref{eq:ham_CK}, the equations of motion are readily
derived (for the Hamiltonian case with respect to $\Pcan$ after which the transformation to $p$ can be effected).
Indeed, after taking the appropriate derivatives, one obtains:
\begin{equation*} 
    \begin{split}
        \dv{}{t}\qty(\pdv{\Lck}{\dot{q}}) - \pdv{\Lck}{q} &= 0 \\
        \Rightarrow \ec^{\gamma t}\qty(m\ddot{q} + m\gamma\dot{q} + kq) &= 0
    \end{split}
\end{equation*}
for the Lagrangian case. Hamilton's equations amount to: \cite{Tokieda2021}
\begin{equation*}
    \begin{split}
        \dot{q} &= \pdv{\Hck}{\Pcan} = \frac{\Pcan}{m}\ec^{-\gamma t} =  \frac{p}{m}, \\
        \dot{\Pcan} &= -\pdv{\Hck}{q} = -kq\ec^{\gamma t}.\\
    \end{split}
\end{equation*}
The relation between the time derivatives of the momenta $\dot{p}$ and $\dot{\Pcan}$ is slightly more involved since one must invoke the product rule as a result of their time-dependencent relation:
    $$ \dot{\Pcan} = \ec^{\gamma t}\qty(\dot{p} + \gamma p). $$
Substition yields the correct equation for $p$, though the equation is again multiplied by $\ec^{\gamma t}$. Because the latter is sufficiently well-behaved (that is, it has no zeros), it can be removed without any problems.

\paragraph{Geometric perspective}
To put the above derivation in a geometric setting, define the Liouville 1-form as
$$ \alpha = \Pcan\dd{q} \quad \Rightarrow \quad \omega = -\dd{\alpha} = \wedgep{\dd{q}}{\dd{\Pcan}},$$
where the symplectic 2-form will be used to obtain Hamilton's equations. The Hamiltonian \cref{eq:ham_CK} is explicitly time-dependent. This will give rise to a time-dependent vector field governing the solution curves.\footnote
{A \emph{time-dependent vector field} on a manifold $M$ is a mapping $X: M\times\real \to \tbundle{M}$ such that for each $t \in \real$, the restriction $X_t$ of $X$ to $M \times \{t\}$ is a vector field on $M$. \cite{Libermann1987} An additional construction of importance, called the \emph{suspension} of the vector field, is a mapping $$ \tilde{X}: \real \times M \to \tbundle{(\real \times M)} \quad (t, m) \mapsto ((t, 1), (m, X(t, m))),$$ that is to say, it lifts the vector field to the extended space that also includes $t$ and assigns the time coordinate with a trivial velocity of 1. \cite{Abraham1978}}
The construction of the vector field associated with a time-dependent Hamiltonian follows the same construction rules as a normal Hamiltonian (using the isomorphism given by $\omega$), but `frozen' at each instant of $t$. Even more bluntly speaking, one simply ignores the $t$-coordinate during the derivation, only to acknowledge the dependence at the very end. This leads to the following vector field, `suspended' on the $\real\times Q$ space:
$$ \tilde{X}_{\Hck} = -\ec^{\gamma t}kq\pdv{}{\Pcan} + \ec^{-\gamma t}\frac{\Pcan}{m}\pdv{}{q} + \pdv{}{t}.$$
The suspension is important to make the final coordinate transform from $\rho$ to $p$ work properly. Indeed, effecting the transformation $(q, \rho, t) \mapsto (q, \ec^{-\gamma t}, t)$, one obtains
$$ \tilde{X}_{\Hck} = \qty(-kq - \gamma p)\pdv{}{p} + \frac{p}{m}\pdv{}{q} + \pdv{}{t}.$$
It is worthwile to ponder on some apparent peculiarities in the Caldirola-Kanai method, for they will be explained elegantly by the contact-Hamiltonian formalism. Firstly, the role of the two-different momenta is not very clear from the get-go, apart from being a consequence of the way the Caldirola-Kanai Lagrangian is formulated. This has also been the reason for considerable confusion in the academic community (see \citet{Schuch1997}). Furthermore, there is the 

\section{Dissipative contact Hamiltonian mechanics}
The contact-geometric counterpart of Hamiltonian and Lagrangian mechanics has been the subject of increasing academic interest in recent years, see for example \citet{VanderSchaft2021a,VanderSchaft2018,Maschke2018,Bravetti2017,DeLeon2020}, etc. The conception of the idea arguably traces back to the work of Herglotz \cite{Guenther1996}, who derived it using the variational principle, and the developments in differential geometry, by e.g. \citet{Arnold1989} and \citet{Libermann1987}.

\begin{mathbox}{Contact manifolds}
    A \textbf{contact element} on a manifold $M$ is a point $m \in M$ combined with a tangent hyperplane $\xi_m \subset \tspace{m}{M}$ (i.e. a subspace of the tangent space  with codimension 1). The term `contact' refers to the intuitive notion that if two submanifolds `touch', they share a contact element: they are \emph{in contact} (which is a slightly weaker condition than tangency). \cite{Cannas2001}

    A \textbf{contact manifold} is a manifold $M$ (of dimension $2n+1$) with a \textbf{contact structure}, that is a smooth field (or distribution) of contact elements on that manifold. Locally, any contact element determines a 1-form $\alpha$ (up to multiplication by a nonzero scalar) whose kernel constitutes the tangent hyperplane distribution, i.e. 
    $$ \xi_m = \ker \alpha_m $$
    This $\alpha$ is called the (local) \emph{contact form}. The field of hyperplanes must satisfy a nondegeneracy condition: it must be \emph{nonintegrable}, i.e. it \emph{never} satisfies the Frobenius integrability condition: \cite{Cannas2001,Abraham1978,Arnold1989}
    $$ \wedgep{\alpha}{(\dd{\alpha})^n} \neq 0, $$
    where integrable distributions would have this expression vanish everywhere. Roughly equivalent statements are that (i) one cannot find foliations of $M$ such that the $\xi$ is everywhere tangent to it, or (ii) that $\dd{\alpha}\vert_\xi$ is a \emph{symplectic form}. In this treatment, all contact forms are assumed to be global, which is the case if the quotient $TM/\xi$ is a trivial line bundle, i.e. the orientation is preserved across the entire manifold \cite{Geiges2008}.

    \tclower
    The \textbf{Darboux theorem} for contact manifolds states that it is always possible to find coordinates $z, x_i, y_i$ such that locally, the contact form is equal to 
    $$ \dd{z} - \Sum y_i\dd{x_i}, $$
    which is also called the standard or natural contact structure. The standard contact structure on $\real^3$ is illustrated below.
        \begin{center}
            % This file was created by matlab2tikz.
%
%The latest updates can be retrieved from
%  http://www.mathworks.com/matlabcentral/fileexchange/22022-matlab2tikz-matlab2tikz
%where you can also make suggestions and rate matlab2tikz.
%
\begin{tikzpicture}

\begin{axis}[%
    width=4.in,
    height=2.8in,
    at={(0.772in,0.457in)},
    scale only axis,
    plot box ratio=3 3 1,
    xmin=-1.2,
    grid,
    3d box = complete,
    xmax=1.2,
    tick align=outside,
    ymin=-1.2,
    ymax=1.2,
    zmin=-0.4,
    zmax=0.4,
    view={-39.1473836532351}{27.2327575315137},
    xlabel = $x$,
    ylabel = $y$,
    zlabel = $z$,
    axis background/.style={fill=white},
    %axis x line*=origin,
    %axis y line*=origin,
    %axis z line*=origin
    %axis lines = middle,
]

\addplot3[area legend, draw=black, fill=accent1, forget plot]
table[row sep=crcr] {%
x	y	z\\
0.936360389693211	0.91	-0.0636396103067893\\
0.936360389693211	1.09	-0.0636396103067893\\
1.06363961030679	1.09	0.0636396103067893\\
1.06363961030679	0.91	0.0636396103067893\\
}--cycle;

\addplot3[area legend, draw=black, fill=accent1, forget plot]
table[row sep=crcr] {%
x	y	z\\
0.929721807150127	0.71	-0.0562225542798982\\
0.929721807150127	0.89	-0.0562225542798982\\
1.07027819284987	0.89	0.0562225542798982\\
1.07027819284987	0.71	0.0562225542798982\\
}--cycle;

\addplot3[area legend, draw=black, fill=accent1, forget plot]
table[row sep=crcr] {%
x	y	z\\
0.922825636685871	0.51	-0.0463046179884774\\
0.922825636685871	0.69	-0.0463046179884774\\
1.07717436331413	0.69	0.0463046179884774\\
1.07717436331413	0.51	0.0463046179884774\\
}--cycle;

\addplot3[area legend, draw=black, fill=accent1, forget plot]
table[row sep=crcr] {%
x	y	z\\
0.916437097820327	0.31	-0.0334251608718693\\
0.916437097820327	0.49	-0.0334251608718693\\
1.08356290217967	0.49	0.0334251608718693\\
1.08356290217967	0.31	0.0334251608718693\\
}--cycle;

\addplot3[area legend, draw=black, fill=accent1, forget plot]
table[row sep=crcr] {%
x	y	z\\
0.911747739187817	0.11	-0.0176504521624366\\
0.911747739187817	0.29	-0.0176504521624366\\
1.08825226081218	0.29	0.0176504521624366\\
1.08825226081218	0.11	0.0176504521624366\\
}--cycle;

\addplot3[area legend, draw=black, fill=accent1, forget plot]
table[row sep=crcr] {%
x	y	z\\
0.91	-0.09	-0\\
0.91	0.09	-0\\
1.09	0.09	0\\
1.09	-0.09	0\\
}--cycle;

\addplot3[area legend, draw=black, fill=accent1, forget plot]
table[row sep=crcr] {%
x	y	z\\
0.911747739187817	-0.29	0.0176504521624366\\
0.911747739187817	-0.11	0.0176504521624366\\
1.08825226081218	-0.11	-0.0176504521624366\\
1.08825226081218	-0.29	-0.0176504521624366\\
}--cycle;

\addplot3[area legend, draw=black, fill=accent1, forget plot]
table[row sep=crcr] {%
x	y	z\\
0.916437097820327	-0.49	0.0334251608718693\\
0.916437097820327	-0.31	0.0334251608718693\\
1.08356290217967	-0.31	-0.0334251608718693\\
1.08356290217967	-0.49	-0.0334251608718693\\
}--cycle;

\addplot3[area legend, draw=black, fill=accent1, forget plot]
table[row sep=crcr] {%
x	y	z\\
0.922825636685871	-0.69	0.0463046179884774\\
0.922825636685871	-0.51	0.0463046179884774\\
1.07717436331413	-0.51	-0.0463046179884774\\
1.07717436331413	-0.69	-0.0463046179884774\\
}--cycle;

\addplot3[area legend, draw=black, fill=accent1, forget plot]
table[row sep=crcr] {%
x	y	z\\
0.929721807150127	-0.89	0.0562225542798982\\
0.929721807150127	-0.71	0.0562225542798982\\
1.07027819284987	-0.71	-0.0562225542798982\\
1.07027819284987	-0.89	-0.0562225542798982\\
}--cycle;

\addplot3[area legend, draw=black, fill=accent1, forget plot]
table[row sep=crcr] {%
x	y	z\\
0.936360389693211	-1.09	0.0636396103067893\\
0.936360389693211	-0.91	0.0636396103067893\\
1.06363961030679	-0.91	-0.0636396103067893\\
1.06363961030679	-1.09	-0.0636396103067893\\
}--cycle;

\addplot3[area legend, draw=black, fill=accent1, forget plot]
table[row sep=crcr] {%
x	y	z\\
0.736360389693211	0.91	-0.0636396103067893\\
0.736360389693211	1.09	-0.0636396103067893\\
0.863639610306789	1.09	0.0636396103067893\\
0.863639610306789	0.91	0.0636396103067893\\
}--cycle;

\addplot3[area legend, draw=black, fill=accent1, forget plot]
table[row sep=crcr] {%
x	y	z\\
0.729721807150127	0.71	-0.0562225542798982\\
0.729721807150127	0.89	-0.0562225542798982\\
0.870278192849873	0.89	0.0562225542798982\\
0.870278192849873	0.71	0.0562225542798982\\
}--cycle;

\addplot3[area legend, draw=black, fill=accent1, forget plot]
table[row sep=crcr] {%
x	y	z\\
0.722825636685871	0.51	-0.0463046179884774\\
0.722825636685871	0.69	-0.0463046179884774\\
0.877174363314129	0.69	0.0463046179884774\\
0.877174363314129	0.51	0.0463046179884774\\
}--cycle;

\addplot3[area legend, draw=black, fill=accent1, forget plot]
table[row sep=crcr] {%
x	y	z\\
0.716437097820327	0.31	-0.0334251608718693\\
0.716437097820327	0.49	-0.0334251608718693\\
0.883562902179673	0.49	0.0334251608718693\\
0.883562902179673	0.31	0.0334251608718693\\
}--cycle;

\addplot3[area legend, draw=black, fill=accent1, forget plot]
table[row sep=crcr] {%
x	y	z\\
0.711747739187817	0.11	-0.0176504521624366\\
0.711747739187817	0.29	-0.0176504521624366\\
0.888252260812183	0.29	0.0176504521624366\\
0.888252260812183	0.11	0.0176504521624366\\
}--cycle;

\addplot3[area legend, draw=black, fill=accent1, forget plot]
table[row sep=crcr] {%
x	y	z\\
0.71	-0.09	-0\\
0.71	0.09	-0\\
0.89	0.09	0\\
0.89	-0.09	0\\
}--cycle;

\addplot3[area legend, draw=black, fill=accent1, forget plot]
table[row sep=crcr] {%
x	y	z\\
0.711747739187817	-0.29	0.0176504521624366\\
0.711747739187817	-0.11	0.0176504521624366\\
0.888252260812183	-0.11	-0.0176504521624366\\
0.888252260812183	-0.29	-0.0176504521624366\\
}--cycle;

\addplot3[area legend, draw=black, fill=accent1, forget plot]
table[row sep=crcr] {%
x	y	z\\
0.716437097820327	-0.49	0.0334251608718693\\
0.716437097820327	-0.31	0.0334251608718693\\
0.883562902179673	-0.31	-0.0334251608718693\\
0.883562902179673	-0.49	-0.0334251608718693\\
}--cycle;

\addplot3[area legend, draw=black, fill=accent1, forget plot]
table[row sep=crcr] {%
x	y	z\\
0.722825636685871	-0.69	0.0463046179884774\\
0.722825636685871	-0.51	0.0463046179884774\\
0.877174363314129	-0.51	-0.0463046179884774\\
0.877174363314129	-0.69	-0.0463046179884774\\
}--cycle;

\addplot3[area legend, draw=black, fill=accent1, forget plot]
table[row sep=crcr] {%
x	y	z\\
0.729721807150127	-0.89	0.0562225542798982\\
0.729721807150127	-0.71	0.0562225542798982\\
0.870278192849873	-0.71	-0.0562225542798982\\
0.870278192849873	-0.89	-0.0562225542798982\\
}--cycle;

\addplot3[area legend, draw=black, fill=accent1, forget plot]
table[row sep=crcr] {%
x	y	z\\
0.736360389693211	-1.09	0.0636396103067893\\
0.736360389693211	-0.91	0.0636396103067893\\
0.863639610306789	-0.91	-0.0636396103067893\\
0.863639610306789	-1.09	-0.0636396103067893\\
}--cycle;

\addplot3[area legend, draw=black, fill=accent1, forget plot]
table[row sep=crcr] {%
x	y	z\\
0.536360389693211	0.91	-0.0636396103067893\\
0.536360389693211	1.09	-0.0636396103067893\\
0.663639610306789	1.09	0.0636396103067893\\
0.663639610306789	0.91	0.0636396103067893\\
}--cycle;

\addplot3[area legend, draw=black, fill=accent1, forget plot]
table[row sep=crcr] {%
x	y	z\\
0.529721807150127	0.71	-0.0562225542798982\\
0.529721807150127	0.89	-0.0562225542798982\\
0.670278192849873	0.89	0.0562225542798982\\
0.670278192849873	0.71	0.0562225542798982\\
}--cycle;

\addplot3[area legend, draw=black, fill=accent1, forget plot]
table[row sep=crcr] {%
x	y	z\\
0.522825636685871	0.51	-0.0463046179884774\\
0.522825636685871	0.69	-0.0463046179884774\\
0.677174363314129	0.69	0.0463046179884774\\
0.677174363314129	0.51	0.0463046179884774\\
}--cycle;

\addplot3[area legend, draw=black, fill=accent1, forget plot]
table[row sep=crcr] {%
x	y	z\\
0.516437097820327	0.31	-0.0334251608718693\\
0.516437097820327	0.49	-0.0334251608718693\\
0.683562902179673	0.49	0.0334251608718693\\
0.683562902179673	0.31	0.0334251608718693\\
}--cycle;

\addplot3[area legend, draw=black, fill=accent1, forget plot]
table[row sep=crcr] {%
x	y	z\\
0.511747739187817	0.11	-0.0176504521624366\\
0.511747739187817	0.29	-0.0176504521624366\\
0.688252260812183	0.29	0.0176504521624366\\
0.688252260812183	0.11	0.0176504521624366\\
}--cycle;

\addplot3[area legend, draw=black, fill=accent1, forget plot]
table[row sep=crcr] {%
x	y	z\\
0.51	-0.09	-0\\
0.51	0.09	-0\\
0.69	0.09	0\\
0.69	-0.09	0\\
}--cycle;

\addplot3[area legend, draw=black, fill=accent1, forget plot]
table[row sep=crcr] {%
x	y	z\\
0.511747739187817	-0.29	0.0176504521624366\\
0.511747739187817	-0.11	0.0176504521624366\\
0.688252260812183	-0.11	-0.0176504521624366\\
0.688252260812183	-0.29	-0.0176504521624366\\
}--cycle;

\addplot3[area legend, draw=black, fill=accent1, forget plot]
table[row sep=crcr] {%
x	y	z\\
0.516437097820327	-0.49	0.0334251608718693\\
0.516437097820327	-0.31	0.0334251608718693\\
0.683562902179673	-0.31	-0.0334251608718693\\
0.683562902179673	-0.49	-0.0334251608718693\\
}--cycle;

\addplot3[area legend, draw=black, fill=accent1, forget plot]
table[row sep=crcr] {%
x	y	z\\
0.522825636685871	-0.69	0.0463046179884774\\
0.522825636685871	-0.51	0.0463046179884774\\
0.677174363314129	-0.51	-0.0463046179884774\\
0.677174363314129	-0.69	-0.0463046179884774\\
}--cycle;

\addplot3[area legend, draw=black, fill=accent1, forget plot]
table[row sep=crcr] {%
x	y	z\\
0.529721807150127	-0.89	0.0562225542798982\\
0.529721807150127	-0.71	0.0562225542798982\\
0.670278192849873	-0.71	-0.0562225542798982\\
0.670278192849873	-0.89	-0.0562225542798982\\
}--cycle;

\addplot3[area legend, draw=black, fill=accent1, forget plot]
table[row sep=crcr] {%
x	y	z\\
0.536360389693211	-1.09	0.0636396103067893\\
0.536360389693211	-0.91	0.0636396103067893\\
0.663639610306789	-0.91	-0.0636396103067893\\
0.663639610306789	-1.09	-0.0636396103067893\\
}--cycle;

\addplot3[area legend, draw=black, fill=accent1, forget plot]
table[row sep=crcr] {%
x	y	z\\
0.336360389693211	0.91	-0.0636396103067893\\
0.336360389693211	1.09	-0.0636396103067893\\
0.463639610306789	1.09	0.0636396103067893\\
0.463639610306789	0.91	0.0636396103067893\\
}--cycle;

\addplot3[area legend, draw=black, fill=accent1, forget plot]
table[row sep=crcr] {%
x	y	z\\
0.329721807150127	0.71	-0.0562225542798982\\
0.329721807150127	0.89	-0.0562225542798982\\
0.470278192849873	0.89	0.0562225542798982\\
0.470278192849873	0.71	0.0562225542798982\\
}--cycle;

\addplot3[area legend, draw=black, fill=accent1, forget plot]
table[row sep=crcr] {%
x	y	z\\
0.322825636685871	0.51	-0.0463046179884774\\
0.322825636685871	0.69	-0.0463046179884774\\
0.477174363314129	0.69	0.0463046179884774\\
0.477174363314129	0.51	0.0463046179884774\\
}--cycle;

\addplot3[area legend, draw=black, fill=accent1, forget plot]
table[row sep=crcr] {%
x	y	z\\
0.316437097820327	0.31	-0.0334251608718693\\
0.316437097820327	0.49	-0.0334251608718693\\
0.483562902179673	0.49	0.0334251608718693\\
0.483562902179673	0.31	0.0334251608718693\\
}--cycle;

\addplot3[area legend, draw=black, fill=accent1, forget plot]
table[row sep=crcr] {%
x	y	z\\
0.311747739187817	0.11	-0.0176504521624366\\
0.311747739187817	0.29	-0.0176504521624366\\
0.488252260812183	0.29	0.0176504521624366\\
0.488252260812183	0.11	0.0176504521624366\\
}--cycle;

\addplot3[area legend, draw=black, fill=accent1, forget plot]
table[row sep=crcr] {%
x	y	z\\
0.31	-0.09	-0\\
0.31	0.09	-0\\
0.49	0.09	0\\
0.49	-0.09	0\\
}--cycle;

\addplot3[area legend, draw=black, fill=accent1, forget plot]
table[row sep=crcr] {%
x	y	z\\
0.311747739187817	-0.29	0.0176504521624366\\
0.311747739187817	-0.11	0.0176504521624366\\
0.488252260812183	-0.11	-0.0176504521624366\\
0.488252260812183	-0.29	-0.0176504521624366\\
}--cycle;

\addplot3[area legend, draw=black, fill=accent1, forget plot]
table[row sep=crcr] {%
x	y	z\\
0.316437097820327	-0.49	0.0334251608718693\\
0.316437097820327	-0.31	0.0334251608718693\\
0.483562902179673	-0.31	-0.0334251608718693\\
0.483562902179673	-0.49	-0.0334251608718693\\
}--cycle;

\addplot3[area legend, draw=black, fill=accent1, forget plot]
table[row sep=crcr] {%
x	y	z\\
0.322825636685871	-0.69	0.0463046179884774\\
0.322825636685871	-0.51	0.0463046179884774\\
0.477174363314129	-0.51	-0.0463046179884774\\
0.477174363314129	-0.69	-0.0463046179884774\\
}--cycle;

\addplot3[area legend, draw=black, fill=accent1, forget plot]
table[row sep=crcr] {%
x	y	z\\
0.329721807150127	-0.89	0.0562225542798982\\
0.329721807150127	-0.71	0.0562225542798982\\
0.470278192849873	-0.71	-0.0562225542798982\\
0.470278192849873	-0.89	-0.0562225542798982\\
}--cycle;

\addplot3[area legend, draw=black, fill=accent1, forget plot]
table[row sep=crcr] {%
x	y	z\\
0.336360389693211	-1.09	0.0636396103067893\\
0.336360389693211	-0.91	0.0636396103067893\\
0.463639610306789	-0.91	-0.0636396103067893\\
0.463639610306789	-1.09	-0.0636396103067893\\
}--cycle;

\addplot3[area legend, draw=black, fill=accent1, forget plot]
table[row sep=crcr] {%
x	y	z\\
0.136360389693211	0.91	-0.0636396103067893\\
0.136360389693211	1.09	-0.0636396103067893\\
0.263639610306789	1.09	0.0636396103067893\\
0.263639610306789	0.91	0.0636396103067893\\
}--cycle;

\addplot3[area legend, draw=black, fill=accent1, forget plot]
table[row sep=crcr] {%
x	y	z\\
0.129721807150127	0.71	-0.0562225542798982\\
0.129721807150127	0.89	-0.0562225542798982\\
0.270278192849873	0.89	0.0562225542798982\\
0.270278192849873	0.71	0.0562225542798982\\
}--cycle;

\addplot3[area legend, draw=black, fill=accent1, forget plot]
table[row sep=crcr] {%
x	y	z\\
0.122825636685871	0.51	-0.0463046179884774\\
0.122825636685871	0.69	-0.0463046179884774\\
0.277174363314129	0.69	0.0463046179884774\\
0.277174363314129	0.51	0.0463046179884774\\
}--cycle;

\addplot3[area legend, draw=black, fill=accent1, forget plot]
table[row sep=crcr] {%
x	y	z\\
0.116437097820327	0.31	-0.0334251608718693\\
0.116437097820327	0.49	-0.0334251608718693\\
0.283562902179673	0.49	0.0334251608718693\\
0.283562902179673	0.31	0.0334251608718693\\
}--cycle;

\addplot3[area legend, draw=black, fill=accent1, forget plot]
table[row sep=crcr] {%
x	y	z\\
0.111747739187817	0.11	-0.0176504521624366\\
0.111747739187817	0.29	-0.0176504521624366\\
0.288252260812183	0.29	0.0176504521624366\\
0.288252260812183	0.11	0.0176504521624366\\
}--cycle;

\addplot3[area legend, draw=black, fill=accent1, forget plot]
table[row sep=crcr] {%
x	y	z\\
0.11	-0.09	-0\\
0.11	0.09	-0\\
0.29	0.09	0\\
0.29	-0.09	0\\
}--cycle;

\addplot3[area legend, draw=black, fill=accent1, forget plot]
table[row sep=crcr] {%
x	y	z\\
0.111747739187817	-0.29	0.0176504521624366\\
0.111747739187817	-0.11	0.0176504521624366\\
0.288252260812183	-0.11	-0.0176504521624366\\
0.288252260812183	-0.29	-0.0176504521624366\\
}--cycle;

\addplot3[area legend, draw=black, fill=accent1, forget plot]
table[row sep=crcr] {%
x	y	z\\
0.116437097820327	-0.49	0.0334251608718693\\
0.116437097820327	-0.31	0.0334251608718693\\
0.283562902179673	-0.31	-0.0334251608718693\\
0.283562902179673	-0.49	-0.0334251608718693\\
}--cycle;

\addplot3[area legend, draw=black, fill=accent1, forget plot]
table[row sep=crcr] {%
x	y	z\\
0.122825636685871	-0.69	0.0463046179884774\\
0.122825636685871	-0.51	0.0463046179884774\\
0.277174363314129	-0.51	-0.0463046179884774\\
0.277174363314129	-0.69	-0.0463046179884774\\
}--cycle;

\addplot3[area legend, draw=black, fill=accent1, forget plot]
table[row sep=crcr] {%
x	y	z\\
0.129721807150127	-0.89	0.0562225542798982\\
0.129721807150127	-0.71	0.0562225542798982\\
0.270278192849873	-0.71	-0.0562225542798982\\
0.270278192849873	-0.89	-0.0562225542798982\\
}--cycle;

\addplot3[area legend, draw=black, fill=accent1, forget plot]
table[row sep=crcr] {%
x	y	z\\
0.136360389693211	-1.09	0.0636396103067893\\
0.136360389693211	-0.91	0.0636396103067893\\
0.263639610306789	-0.91	-0.0636396103067893\\
0.263639610306789	-1.09	-0.0636396103067893\\
}--cycle;

\addplot3[area legend, draw=black, fill=accent1, forget plot]
table[row sep=crcr] {%
x	y	z\\
-0.0636396103067893	0.91	-0.0636396103067893\\
-0.0636396103067893	1.09	-0.0636396103067893\\
0.0636396103067893	1.09	0.0636396103067893\\
0.0636396103067893	0.91	0.0636396103067893\\
}--cycle;

\addplot3[area legend, draw=black, fill=accent1, forget plot]
table[row sep=crcr] {%
x	y	z\\
-0.0702781928498727	0.71	-0.0562225542798982\\
-0.0702781928498727	0.89	-0.0562225542798982\\
0.0702781928498727	0.89	0.0562225542798982\\
0.0702781928498727	0.71	0.0562225542798982\\
}--cycle;

\addplot3[area legend, draw=black, fill=accent1, forget plot]
table[row sep=crcr] {%
x	y	z\\
-0.077174363314129	0.51	-0.0463046179884774\\
-0.077174363314129	0.69	-0.0463046179884774\\
0.077174363314129	0.69	0.0463046179884774\\
0.077174363314129	0.51	0.0463046179884774\\
}--cycle;

\addplot3[area legend, draw=black, fill=accent1, forget plot]
table[row sep=crcr] {%
x	y	z\\
-0.0835629021796733	0.31	-0.0334251608718693\\
-0.0835629021796733	0.49	-0.0334251608718693\\
0.0835629021796733	0.49	0.0334251608718693\\
0.0835629021796733	0.31	0.0334251608718693\\
}--cycle;

\addplot3[area legend, draw=black, fill=accent1, forget plot]
table[row sep=crcr] {%
x	y	z\\
-0.0882522608121828	0.11	-0.0176504521624366\\
-0.0882522608121828	0.29	-0.0176504521624366\\
0.0882522608121828	0.29	0.0176504521624366\\
0.0882522608121828	0.11	0.0176504521624366\\
}--cycle;

\addplot3[area legend, draw=black, fill=accent1, forget plot]
table[row sep=crcr] {%
x	y	z\\
-0.09	-0.09	-0\\
-0.09	0.09	-0\\
0.09	0.09	0\\
0.09	-0.09	0\\
}--cycle;

\addplot3[area legend, draw=black, fill=accent1, forget plot]
table[row sep=crcr] {%
x	y	z\\
-0.0882522608121828	-0.29	0.0176504521624366\\
-0.0882522608121828	-0.11	0.0176504521624366\\
0.0882522608121828	-0.11	-0.0176504521624366\\
0.0882522608121828	-0.29	-0.0176504521624366\\
}--cycle;

\addplot3[area legend, draw=black, fill=accent1, forget plot]
table[row sep=crcr] {%
x	y	z\\
-0.0835629021796733	-0.49	0.0334251608718693\\
-0.0835629021796733	-0.31	0.0334251608718693\\
0.0835629021796733	-0.31	-0.0334251608718693\\
0.0835629021796733	-0.49	-0.0334251608718693\\
}--cycle;

\addplot3[area legend, draw=black, fill=accent1, forget plot]
table[row sep=crcr] {%
x	y	z\\
-0.077174363314129	-0.69	0.0463046179884774\\
-0.077174363314129	-0.51	0.0463046179884774\\
0.077174363314129	-0.51	-0.0463046179884774\\
0.077174363314129	-0.69	-0.0463046179884774\\
}--cycle;

\addplot3[area legend, draw=black, fill=accent1, forget plot]
table[row sep=crcr] {%
x	y	z\\
-0.0702781928498727	-0.89	0.0562225542798982\\
-0.0702781928498727	-0.71	0.0562225542798982\\
0.0702781928498727	-0.71	-0.0562225542798982\\
0.0702781928498727	-0.89	-0.0562225542798982\\
}--cycle;

\addplot3[area legend, draw=black, fill=accent1, forget plot]
table[row sep=crcr] {%
x	y	z\\
-0.0636396103067893	-1.09	0.0636396103067893\\
-0.0636396103067893	-0.91	0.0636396103067893\\
0.0636396103067893	-0.91	-0.0636396103067893\\
0.0636396103067893	-1.09	-0.0636396103067893\\
}--cycle;

\addplot3[area legend, draw=black, fill=accent1, forget plot]
table[row sep=crcr] {%
x	y	z\\
-0.263639610306789	0.91	-0.0636396103067893\\
-0.263639610306789	1.09	-0.0636396103067893\\
-0.136360389693211	1.09	0.0636396103067893\\
-0.136360389693211	0.91	0.0636396103067893\\
}--cycle;

\addplot3[area legend, draw=black, fill=accent1, forget plot]
table[row sep=crcr] {%
x	y	z\\
-0.270278192849873	0.71	-0.0562225542798982\\
-0.270278192849873	0.89	-0.0562225542798982\\
-0.129721807150127	0.89	0.0562225542798982\\
-0.129721807150127	0.71	0.0562225542798982\\
}--cycle;

\addplot3[area legend, draw=black, fill=accent1, forget plot]
table[row sep=crcr] {%
x	y	z\\
-0.277174363314129	0.51	-0.0463046179884774\\
-0.277174363314129	0.69	-0.0463046179884774\\
-0.122825636685871	0.69	0.0463046179884774\\
-0.122825636685871	0.51	0.0463046179884774\\
}--cycle;

\addplot3[area legend, draw=black, fill=accent1, forget plot]
table[row sep=crcr] {%
x	y	z\\
-0.283562902179673	0.31	-0.0334251608718693\\
-0.283562902179673	0.49	-0.0334251608718693\\
-0.116437097820327	0.49	0.0334251608718693\\
-0.116437097820327	0.31	0.0334251608718693\\
}--cycle;

\addplot3[area legend, draw=black, fill=accent1, forget plot]
table[row sep=crcr] {%
x	y	z\\
-0.288252260812183	0.11	-0.0176504521624366\\
-0.288252260812183	0.29	-0.0176504521624366\\
-0.111747739187817	0.29	0.0176504521624366\\
-0.111747739187817	0.11	0.0176504521624366\\
}--cycle;

\addplot3[area legend, draw=black, fill=accent1, forget plot]
table[row sep=crcr] {%
x	y	z\\
-0.29	-0.09	-0\\
-0.29	0.09	-0\\
-0.11	0.09	0\\
-0.11	-0.09	0\\
}--cycle;

\addplot3[area legend, draw=black, fill=accent1, forget plot]
table[row sep=crcr] {%
x	y	z\\
-0.288252260812183	-0.29	0.0176504521624366\\
-0.288252260812183	-0.11	0.0176504521624366\\
-0.111747739187817	-0.11	-0.0176504521624366\\
-0.111747739187817	-0.29	-0.0176504521624366\\
}--cycle;

\addplot3[area legend, draw=black, fill=accent1, forget plot]
table[row sep=crcr] {%
x	y	z\\
-0.283562902179673	-0.49	0.0334251608718693\\
-0.283562902179673	-0.31	0.0334251608718693\\
-0.116437097820327	-0.31	-0.0334251608718693\\
-0.116437097820327	-0.49	-0.0334251608718693\\
}--cycle;

\addplot3[area legend, draw=black, fill=accent1, forget plot]
table[row sep=crcr] {%
x	y	z\\
-0.277174363314129	-0.69	0.0463046179884774\\
-0.277174363314129	-0.51	0.0463046179884774\\
-0.122825636685871	-0.51	-0.0463046179884774\\
-0.122825636685871	-0.69	-0.0463046179884774\\
}--cycle;

\addplot3[area legend, draw=black, fill=accent1, forget plot]
table[row sep=crcr] {%
x	y	z\\
-0.270278192849873	-0.89	0.0562225542798982\\
-0.270278192849873	-0.71	0.0562225542798982\\
-0.129721807150127	-0.71	-0.0562225542798982\\
-0.129721807150127	-0.89	-0.0562225542798982\\
}--cycle;

\addplot3[area legend, draw=black, fill=accent1, forget plot]
table[row sep=crcr] {%
x	y	z\\
-0.263639610306789	-1.09	0.0636396103067893\\
-0.263639610306789	-0.91	0.0636396103067893\\
-0.136360389693211	-0.91	-0.0636396103067893\\
-0.136360389693211	-1.09	-0.0636396103067893\\
}--cycle;

\addplot3[area legend, draw=black, fill=accent1, forget plot]
table[row sep=crcr] {%
x	y	z\\
-0.463639610306789	0.91	-0.0636396103067893\\
-0.463639610306789	1.09	-0.0636396103067893\\
-0.336360389693211	1.09	0.0636396103067893\\
-0.336360389693211	0.91	0.0636396103067893\\
}--cycle;

\addplot3[area legend, draw=black, fill=accent1, forget plot]
table[row sep=crcr] {%
x	y	z\\
-0.470278192849873	0.71	-0.0562225542798982\\
-0.470278192849873	0.89	-0.0562225542798982\\
-0.329721807150127	0.89	0.0562225542798982\\
-0.329721807150127	0.71	0.0562225542798982\\
}--cycle;

\addplot3[area legend, draw=black, fill=accent1, forget plot]
table[row sep=crcr] {%
x	y	z\\
-0.477174363314129	0.51	-0.0463046179884774\\
-0.477174363314129	0.69	-0.0463046179884774\\
-0.322825636685871	0.69	0.0463046179884774\\
-0.322825636685871	0.51	0.0463046179884774\\
}--cycle;

\addplot3[area legend, draw=black, fill=accent1, forget plot]
table[row sep=crcr] {%
x	y	z\\
-0.483562902179673	0.31	-0.0334251608718693\\
-0.483562902179673	0.49	-0.0334251608718693\\
-0.316437097820327	0.49	0.0334251608718693\\
-0.316437097820327	0.31	0.0334251608718693\\
}--cycle;

\addplot3[area legend, draw=black, fill=accent1, forget plot]
table[row sep=crcr] {%
x	y	z\\
-0.488252260812183	0.11	-0.0176504521624366\\
-0.488252260812183	0.29	-0.0176504521624366\\
-0.311747739187817	0.29	0.0176504521624366\\
-0.311747739187817	0.11	0.0176504521624366\\
}--cycle;

\addplot3[area legend, draw=black, fill=accent1, forget plot]
table[row sep=crcr] {%
x	y	z\\
-0.49	-0.09	-0\\
-0.49	0.09	-0\\
-0.31	0.09	0\\
-0.31	-0.09	0\\
}--cycle;

\addplot3[area legend, draw=black, fill=accent1, forget plot]
table[row sep=crcr] {%
x	y	z\\
-0.488252260812183	-0.29	0.0176504521624366\\
-0.488252260812183	-0.11	0.0176504521624366\\
-0.311747739187817	-0.11	-0.0176504521624366\\
-0.311747739187817	-0.29	-0.0176504521624366\\
}--cycle;

\addplot3[area legend, draw=black, fill=accent1, forget plot]
table[row sep=crcr] {%
x	y	z\\
-0.483562902179673	-0.49	0.0334251608718693\\
-0.483562902179673	-0.31	0.0334251608718693\\
-0.316437097820327	-0.31	-0.0334251608718693\\
-0.316437097820327	-0.49	-0.0334251608718693\\
}--cycle;

\addplot3[area legend, draw=black, fill=accent1, forget plot]
table[row sep=crcr] {%
x	y	z\\
-0.477174363314129	-0.69	0.0463046179884774\\
-0.477174363314129	-0.51	0.0463046179884774\\
-0.322825636685871	-0.51	-0.0463046179884774\\
-0.322825636685871	-0.69	-0.0463046179884774\\
}--cycle;

\addplot3[area legend, draw=black, fill=accent1, forget plot]
table[row sep=crcr] {%
x	y	z\\
-0.470278192849873	-0.89	0.0562225542798982\\
-0.470278192849873	-0.71	0.0562225542798982\\
-0.329721807150127	-0.71	-0.0562225542798982\\
-0.329721807150127	-0.89	-0.0562225542798982\\
}--cycle;

\addplot3[area legend, draw=black, fill=accent1, forget plot]
table[row sep=crcr] {%
x	y	z\\
-0.463639610306789	-1.09	0.0636396103067893\\
-0.463639610306789	-0.91	0.0636396103067893\\
-0.336360389693211	-0.91	-0.0636396103067893\\
-0.336360389693211	-1.09	-0.0636396103067893\\
}--cycle;

\addplot3[area legend, draw=black, fill=accent1, forget plot]
table[row sep=crcr] {%
x	y	z\\
-0.663639610306789	0.91	-0.0636396103067893\\
-0.663639610306789	1.09	-0.0636396103067893\\
-0.536360389693211	1.09	0.0636396103067893\\
-0.536360389693211	0.91	0.0636396103067893\\
}--cycle;

\addplot3[area legend, draw=black, fill=accent1, forget plot]
table[row sep=crcr] {%
x	y	z\\
-0.670278192849873	0.71	-0.0562225542798982\\
-0.670278192849873	0.89	-0.0562225542798982\\
-0.529721807150127	0.89	0.0562225542798982\\
-0.529721807150127	0.71	0.0562225542798982\\
}--cycle;

\addplot3[area legend, draw=black, fill=accent1, forget plot]
table[row sep=crcr] {%
x	y	z\\
-0.677174363314129	0.51	-0.0463046179884774\\
-0.677174363314129	0.69	-0.0463046179884774\\
-0.522825636685871	0.69	0.0463046179884774\\
-0.522825636685871	0.51	0.0463046179884774\\
}--cycle;

\addplot3[area legend, draw=black, fill=accent1, forget plot]
table[row sep=crcr] {%
x	y	z\\
-0.683562902179673	0.31	-0.0334251608718693\\
-0.683562902179673	0.49	-0.0334251608718693\\
-0.516437097820327	0.49	0.0334251608718693\\
-0.516437097820327	0.31	0.0334251608718693\\
}--cycle;

\addplot3[area legend, draw=black, fill=accent1, forget plot]
table[row sep=crcr] {%
x	y	z\\
-0.688252260812183	0.11	-0.0176504521624366\\
-0.688252260812183	0.29	-0.0176504521624366\\
-0.511747739187817	0.29	0.0176504521624366\\
-0.511747739187817	0.11	0.0176504521624366\\
}--cycle;

\addplot3[area legend, draw=black, fill=accent1, forget plot]
table[row sep=crcr] {%
x	y	z\\
-0.69	-0.09	-0\\
-0.69	0.09	-0\\
-0.51	0.09	0\\
-0.51	-0.09	0\\
}--cycle;

\addplot3[area legend, draw=black, fill=accent1, forget plot]
table[row sep=crcr] {%
x	y	z\\
-0.688252260812183	-0.29	0.0176504521624366\\
-0.688252260812183	-0.11	0.0176504521624366\\
-0.511747739187817	-0.11	-0.0176504521624366\\
-0.511747739187817	-0.29	-0.0176504521624366\\
}--cycle;

\addplot3[area legend, draw=black, fill=accent1, forget plot]
table[row sep=crcr] {%
x	y	z\\
-0.683562902179673	-0.49	0.0334251608718693\\
-0.683562902179673	-0.31	0.0334251608718693\\
-0.516437097820327	-0.31	-0.0334251608718693\\
-0.516437097820327	-0.49	-0.0334251608718693\\
}--cycle;

\addplot3[area legend, draw=black, fill=accent1, forget plot]
table[row sep=crcr] {%
x	y	z\\
-0.677174363314129	-0.69	0.0463046179884774\\
-0.677174363314129	-0.51	0.0463046179884774\\
-0.522825636685871	-0.51	-0.0463046179884774\\
-0.522825636685871	-0.69	-0.0463046179884774\\
}--cycle;

\addplot3[area legend, draw=black, fill=accent1, forget plot]
table[row sep=crcr] {%
x	y	z\\
-0.670278192849873	-0.89	0.0562225542798982\\
-0.670278192849873	-0.71	0.0562225542798982\\
-0.529721807150127	-0.71	-0.0562225542798982\\
-0.529721807150127	-0.89	-0.0562225542798982\\
}--cycle;

\addplot3[area legend, draw=black, fill=accent1, forget plot]
table[row sep=crcr] {%
x	y	z\\
-0.663639610306789	-1.09	0.0636396103067893\\
-0.663639610306789	-0.91	0.0636396103067893\\
-0.536360389693211	-0.91	-0.0636396103067893\\
-0.536360389693211	-1.09	-0.0636396103067893\\
}--cycle;

\addplot3[area legend, draw=black, fill=accent1, forget plot]
table[row sep=crcr] {%
x	y	z\\
-0.863639610306789	0.91	-0.0636396103067893\\
-0.863639610306789	1.09	-0.0636396103067893\\
-0.736360389693211	1.09	0.0636396103067893\\
-0.736360389693211	0.91	0.0636396103067893\\
}--cycle;

\addplot3[area legend, draw=black, fill=accent1, forget plot]
table[row sep=crcr] {%
x	y	z\\
-0.870278192849873	0.71	-0.0562225542798982\\
-0.870278192849873	0.89	-0.0562225542798982\\
-0.729721807150127	0.89	0.0562225542798982\\
-0.729721807150127	0.71	0.0562225542798982\\
}--cycle;

\addplot3[area legend, draw=black, fill=accent1, forget plot]
table[row sep=crcr] {%
x	y	z\\
-0.877174363314129	0.51	-0.0463046179884774\\
-0.877174363314129	0.69	-0.0463046179884774\\
-0.722825636685871	0.69	0.0463046179884774\\
-0.722825636685871	0.51	0.0463046179884774\\
}--cycle;

\addplot3[area legend, draw=black, fill=accent1, forget plot]
table[row sep=crcr] {%
x	y	z\\
-0.883562902179673	0.31	-0.0334251608718693\\
-0.883562902179673	0.49	-0.0334251608718693\\
-0.716437097820327	0.49	0.0334251608718693\\
-0.716437097820327	0.31	0.0334251608718693\\
}--cycle;

\addplot3[area legend, draw=black, fill=accent1, forget plot]
table[row sep=crcr] {%
x	y	z\\
-0.888252260812183	0.11	-0.0176504521624366\\
-0.888252260812183	0.29	-0.0176504521624366\\
-0.711747739187817	0.29	0.0176504521624366\\
-0.711747739187817	0.11	0.0176504521624366\\
}--cycle;

\addplot3[area legend, draw=black, fill=accent1, forget plot]
table[row sep=crcr] {%
x	y	z\\
-0.89	-0.09	-0\\
-0.89	0.09	-0\\
-0.71	0.09	0\\
-0.71	-0.09	0\\
}--cycle;

\addplot3[area legend, draw=black, fill=accent1, forget plot]
table[row sep=crcr] {%
x	y	z\\
-0.888252260812183	-0.29	0.0176504521624366\\
-0.888252260812183	-0.11	0.0176504521624366\\
-0.711747739187817	-0.11	-0.0176504521624366\\
-0.711747739187817	-0.29	-0.0176504521624366\\
}--cycle;

\addplot3[area legend, draw=black, fill=accent1, forget plot]
table[row sep=crcr] {%
x	y	z\\
-0.883562902179673	-0.49	0.0334251608718693\\
-0.883562902179673	-0.31	0.0334251608718693\\
-0.716437097820327	-0.31	-0.0334251608718693\\
-0.716437097820327	-0.49	-0.0334251608718693\\
}--cycle;

\addplot3[area legend, draw=black, fill=accent1, forget plot]
table[row sep=crcr] {%
x	y	z\\
-0.877174363314129	-0.69	0.0463046179884774\\
-0.877174363314129	-0.51	0.0463046179884774\\
-0.722825636685871	-0.51	-0.0463046179884774\\
-0.722825636685871	-0.69	-0.0463046179884774\\
}--cycle;

\addplot3[area legend, draw=black, fill=accent1, forget plot]
table[row sep=crcr] {%
x	y	z\\
-0.870278192849873	-0.89	0.0562225542798982\\
-0.870278192849873	-0.71	0.0562225542798982\\
-0.729721807150127	-0.71	-0.0562225542798982\\
-0.729721807150127	-0.89	-0.0562225542798982\\
}--cycle;

\addplot3[area legend, draw=black, fill=accent1, forget plot]
table[row sep=crcr] {%
x	y	z\\
-0.863639610306789	-1.09	0.0636396103067893\\
-0.863639610306789	-0.91	0.0636396103067893\\
-0.736360389693211	-0.91	-0.0636396103067893\\
-0.736360389693211	-1.09	-0.0636396103067893\\
}--cycle;

\addplot3[area legend, draw=black, fill=accent1, forget plot]
table[row sep=crcr] {%
x	y	z\\
-1.06363961030679	0.91	-0.0636396103067893\\
-1.06363961030679	1.09	-0.0636396103067893\\
-0.936360389693211	1.09	0.0636396103067893\\
-0.936360389693211	0.91	0.0636396103067893\\
}--cycle;

\addplot3[area legend, draw=black, fill=accent1, forget plot]
table[row sep=crcr] {%
x	y	z\\
-1.07027819284987	0.71	-0.0562225542798982\\
-1.07027819284987	0.89	-0.0562225542798982\\
-0.929721807150127	0.89	0.0562225542798982\\
-0.929721807150127	0.71	0.0562225542798982\\
}--cycle;

\addplot3[area legend, draw=black, fill=accent1, forget plot]
table[row sep=crcr] {%
x	y	z\\
-1.07717436331413	0.51	-0.0463046179884774\\
-1.07717436331413	0.69	-0.0463046179884774\\
-0.922825636685871	0.69	0.0463046179884774\\
-0.922825636685871	0.51	0.0463046179884774\\
}--cycle;

\addplot3[area legend, draw=black, fill=accent1, forget plot]
table[row sep=crcr] {%
x	y	z\\
-1.08356290217967	0.31	-0.0334251608718693\\
-1.08356290217967	0.49	-0.0334251608718693\\
-0.916437097820327	0.49	0.0334251608718693\\
-0.916437097820327	0.31	0.0334251608718693\\
}--cycle;

\addplot3[area legend, draw=black, fill=accent1, forget plot]
table[row sep=crcr] {%
x	y	z\\
-1.08825226081218	0.11	-0.0176504521624366\\
-1.08825226081218	0.29	-0.0176504521624366\\
-0.911747739187817	0.29	0.0176504521624366\\
-0.911747739187817	0.11	0.0176504521624366\\
}--cycle;

\addplot3[area legend, draw=black, fill=accent1, forget plot]
table[row sep=crcr] {%
x	y	z\\
-1.09	-0.09	-0\\
-1.09	0.09	-0\\
-0.91	0.09	0\\
-0.91	-0.09	0\\
}--cycle;

\addplot3[area legend, draw=black, fill=accent1, forget plot]
table[row sep=crcr] {%
x	y	z\\
-1.08825226081218	-0.29	0.0176504521624366\\
-1.08825226081218	-0.11	0.0176504521624366\\
-0.911747739187817	-0.11	-0.0176504521624366\\
-0.911747739187817	-0.29	-0.0176504521624366\\
}--cycle;

\addplot3[area legend, draw=black, fill=accent1, forget plot]
table[row sep=crcr] {%
x	y	z\\
-1.08356290217967	-0.49	0.0334251608718693\\
-1.08356290217967	-0.31	0.0334251608718693\\
-0.916437097820327	-0.31	-0.0334251608718693\\
-0.916437097820327	-0.49	-0.0334251608718693\\
}--cycle;

\addplot3[area legend, draw=black, fill=accent1, forget plot]
table[row sep=crcr] {%
x	y	z\\
-1.07717436331413	-0.69	0.0463046179884774\\
-1.07717436331413	-0.51	0.0463046179884774\\
-0.922825636685871	-0.51	-0.0463046179884774\\
-0.922825636685871	-0.69	-0.0463046179884774\\
}--cycle;

\addplot3[area legend, draw=black, fill=accent1, forget plot]
table[row sep=crcr] {%
x	y	z\\
-1.07027819284987	-0.89	0.0562225542798982\\
-1.07027819284987	-0.71	0.0562225542798982\\
-0.929721807150127	-0.71	-0.0562225542798982\\
-0.929721807150127	-0.89	-0.0562225542798982\\
}--cycle;

\addplot3[area legend, draw=black, fill=accent1, forget plot]
table[row sep=crcr] {%
x	y	z\\
-1.06363961030679	-1.09	0.0636396103067893\\
-1.06363961030679	-0.91	0.0636396103067893\\
-0.936360389693211	-0.91	-0.0636396103067893\\
-0.936360389693211	-1.09	-0.0636396103067893\\
}--cycle;
\end{axis}

\begin{axis}[%
width=5.938in,
height=3.854in,
at={(0in,0in)},
scale only axis,
xmin=0,
xmax=1,
ymin=0,
ymax=1,
axis line style={draw=none},
ticks=none,
axis x line*=bottom,
axis y line*=left
]
\end{axis}
\end{tikzpicture}%

            %\vspace{0.3cm}
        \end{center}
        \begin{center}
            \begin{minipage}{12cm}
                {\small The standard contact structure on $\real^3$, given by the contact form $\dd{z} - y\dd{x}$; the hyperplanes tilt more in the increasing $y$-direction.}
            \end{minipage}
        \end{center}
    Finally, it is clear that the contact form singles out a `special direction' in the tangent space at every point of the manifold. This direction is given by the unique \textbf{Reeb vector field},
    $$ R_\alpha \in \vfields{M}:\quad \intpr{R_\alpha}{\dd{\alpha}}= 0 \quad \text{and} \quad \intpr{R_\alpha}{\alpha} = 1,$$
    that is, it locally points in the `direction' of the contact form.
\end{mathbox}

\begin{mathbox}{Manifold of contact elements}
    It was prevously noted that a contact manifold is simply a manifold with a contact structure. However, there is, associated to any manifold $N$ a \emph{canonical} contact manifold, just like one can always find a canonical symplectic structure on $\ctbundle{N}$.
\end{mathbox}

\begin{mathbox}{Liouville geometry}
    blabla
\end{mathbox}


\begin{itemize}
    \item Cosymplectic
    \item Presymplectic
    \item Precontact
\end{itemize}


\begin{enumerate}
    \item (Briefly) introduce contact Hamiltions á la van der Schaft 
    \item Briefly explain Bravetti's Hamiltonian
    \item From CK to Bravetti
    \item Integral invariants + Lagrangian?
\end{enumerate}

%\begin{itemize}
%    \item The \textbf{mirror system} or \textbf{Bateman} approach, doubles the number of system dimensions by including a mirror system that runs opposite in time. Arguably the most flexible of all methods, it is 
%    \item Expressing the Hamiltonian in \textbf{complex coordinates} has also produced promising results: notable are the attempts of \citet{Bopp1974}, \citet{Dedene1980} and the very recent contribution by \citet{Hutters2020b} in the research group to which the author belongs as well.
%    \item A different approach, related to the contact method, are the 
%    \item Contact mechanics
%    \item Mathematical Hamiltonians
%\end{itemize}
%
%[Mention Max as contributor]
