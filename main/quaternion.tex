\chapter{Split-Quaternions as Dynamical Systems}
\label{chap:quaternion}
%\lsymb{$\vec{\alpha}$}{A differential form}

In this chapter, the geometric connection is made between the algebra of split-quaternions and the qualitative behavior of two-dimensional linear dynamical systems. 

\section{Split-quaternion algebra}
\subsection{Basic properties}
Like the conventional quaternions, the split-quaternions are a number system that consists of linear combinations of four basis elements, which will be denoted by \quati, \quatj and \quatk. The algebra of split-quaternions is associative but not commutative --- formally speaking, the algebraic structure is a \emph{noncommutative ring}. The multiplication table for the split-quaternion algebra is shown in \cref{tab:quat_table}. The set of split-quaternions is denoted by \spquaternions (since \quaternions is reserved for conventional quaternions).
\begin{table}[ht!]
    \centering
    \caption{Multiplication table for the split-quaternion algebra.}
    \label{tab:quat_table}
    \begin{tabular}{c|cccc}
        \toprule
        &         1      & \quati  & \quatj  & \quatk \\ 
        \midrule
        1       & 1      & \quati  & \quatj  & \quatk \\ 
        \quati  & \quati & -1      & \quatk  & -\quatj \\ 
        \quatj  & \quatj & -\quatk & 1       & -\quati \\ 
        \quatk  & \quatk & \quatj  & \quati  & 1 \\ 
        \bottomrule
    \end{tabular}
\end{table}

The important distinction from conventional quaternions resides in the diagonal elements of \cref{tab:quat_table}. Whereas for quaternions all the nonreal basis elements square to $-1$, this is not the case for the split-quaternions (only \quati does). This is precisely the reason why split-quaternions are `split', for this difference in sign gives rise to an indefinite quadratic form when computing the norm of the split-quadratic form. Because the quadratic form is indefinite, it classifies the set of split-quaternions into several subsets, which is to be discussed later.

\paragraph{Dihedral group} he basis elements of the split-quaternions form a group under multiplication, namely the \emph{dihedral group} \digroup{4}, which represents all the symmetries of a square: the identity, a 90-degree rotation and two reflections (cf. \cref{fig:square_symmetry}).
\begin{figure}[h!]
    \begin{center}
        \begin{tikzpicture}
    \node[draw, inner sep=3mm,thick, fill=accent1!40] at (0, 0) {};
    \node[draw, inner sep=3mm,thick, fill=accent1!40] at (2, 0) {};
    \node[draw, inner sep=3mm,thick, fill=accent1!40] at (4, 0) {};
    \node[draw, inner sep=3mm,thick, fill=accent1!40] at (6, 0) {};
    \draw[->] (2, 0.7) arc (90:0:0.7);
    \draw[->] (2, -0.7) arc (270:180:0.7);
    \draw[dashdotted] (3.5, 0.5) -- (4.5, -0.5);
    \draw[<->] (3.75, -0.25) -- (4.25, 0.25);
    \draw[dashdotted] (5.5, -0.5) -- (6.5, 0.5);
    \draw[<->] (5.75, 0.25) -- (6.25, -0.25);
    \node[] at (0, -1.1) {1};
    \node[] at (2, -1.1) {\quati};
    \node[] at (4, -1.1) {\quatj};
    \node[] at (6, -1.1) {\quatk};

\end{tikzpicture}

    \end{center}
    \caption{The dihedral group \digroup{4} is the symmetry group of a square. This group is isomorphic to the group formed by $1, \quati, \quatj$ and \quatk under multiplication.}
    \label{fig:square_symmetry}
\end{figure}

The structure of the dihedral group can be visualized by its \emph{cycle graph} in \cref{fig:cycle_graph}. Many important properties of the split-quaternion algebra and the applications in this chapter can be traced back to the topology of this cycle graph. One example is the split nature of the quaternions: the \quati-element generates an order four cycle, while \quatj and \quatk generate order two cycles (in contrast, the cycle graph for conventional quaternions is entirely symmetric for all these elements).
\begin{figure}[h!]
    \begin{center}
        \begin{tikzpicture}
    \node[draw, thick, circle, fill=accent1!40] (e) at (0, 0) {};
    
    \node[thick, draw, circle, below right=0.8cm and 0.5cm of e] (r1) {};
    \node[thick, draw, circle, below left=0.8cm and 0.5cm of e] (r2) {};
    \node[thick, draw, circle, left of= r2] (r3) {};
    \node[thick, draw, circle, right of= r1] (r4) {};
    \node[thick, draw, circle, above=1.5cm of e] (ro2) {};
    \node[thick, draw, circle, above right=0.7cm and 1cm of e] (ro1) {};
    \node[thick, draw, circle, above left=0.7cm and 1cm of e] (ro3) {};
    
    \draw[thick] (e) -- (r1);
    \draw[thick] (e) -- (r2);
    \draw[thick] (e) -- (r3);
    \draw[thick] (e) -- (r4);
    
    \draw[thick] (e) -- (ro1) -- (ro2) -- (ro3) -- (e);

\end{tikzpicture}

    \end{center}
    \caption{Cycle graph of the dihedral group. There are five cycles: one of order four which represents the rotations (or the element \quati), and four order 2 cycles, which are all the possible reflections. The colored element represents the identity.}
    \label{fig:cycle_graph}
\end{figure}

\paragraph{Split-quaternion norm} Complex numbers have a real and imaginary part. Likewise, (split)-quaternions have a similar notion: a \emph{scalar} (or real) and \emph{vector} (or imaginary) components. For an arbitrary split-quaternion $q \in \spquaternions$, \cite{Jafari2014}
$$ a = \quat{a_0}{a_1}{a_2}{a_3} $$
the real part is $\scapart{h} = a_0$ and the vector part is $ \vecpart{a} = \quatvec{a_1}{a_2}{a_3}$. For convenience, the vector part will be referred to in traditional bold vector notation:
$$ \vec{a} = \vecpart{a} = \quatvec{a_1}{a_2}{a_3}. $$

Furthermore, for every split-quaternion there is a unique \emph{conjugate}
$$ \conj{a} = \scapart{a} - \vecpart{a} = a_0 - a_1\quati - a_2\quatj - a_3\quatk, $$
through which the \emph{split-quaternion norm} is defined:
\begin{equation}
    \mathscr{N}(a) = a\conj{h} = a_0^2 + a_1^2 - a_2^2 - a_3^2. 
    \label{eq:quat_norm}
\end{equation}
As mentioned, this norm is not positive definite, in stark contrast to quaternions or complex numbers. Split-quaternions can be categorized into three classes based on their norm being negative, zero or positive. In the tradition of special relativity, these classes are named \emph{spacelike}, \emph{lightlike} and \emph{timelike} respectively: \cite{Misner1970,Landau1971}
\begin{itemize}
    \item \textbf{Timelike}: $ \mathscr{N}(a) > 0 $, with real length $ \norm{a} = \sqrt{a \conj{a}}$. 
    \item \textbf{Lightlike}: $ \mathscr{N}(a) = 0 $, with zero length $\norm{a} = 0$. 
    \item \textbf{Spacelike}: $ \mathscr{N}(a) < 0 $, with imaginary length $\norm{a} = \ii \sqrt{\abs{a\conj{a}}}$.
\end{itemize}
Even though they behave similarly, the imaginary unit $\ii$ is not to be confused with the split-quaternion basis element \quati, because they belong to different number systems.

\paragraph{Vector norm}
Apart from the split-quaternion norm, we can also define a norm that only considers the vector part of the split-quaternion. This norm is defined in accordance with the overall quaternion norm given by \cref{eq:quat_norm}:
$$ \mathscr{N}_v(\vec{a}) = a_1^2 - a^2_2 - a^2_3. $$
The above expression is equivalent the Lorentz norm applied to a vector in $\real^3$; we will denote $\real^3$ equipped with the Lorentz norm by $\real^{2, 1}$. \cite{Jafari2014} Observe that this quadratic form is not positive-definite either; in the the same veign as before, we can therefore classify quaternions by the `sign' of their vector part again. We refer to these classes as \emph{timelike (vectors)}, \emph{spacelike (vectors)} and \emph{lightlike (vectors)} in the same way. 

Observe that $ \mathscr{N}(a) < 0 \Rightarrow \mathscr{N}_v(\vec{a}) < 0$; that is to say, a spacelike split-quaternion always has a spacelike vector part. The converse is not necessarily true. Along the same line, a lightlike split-quaternion can only have a lightlike or spacelike vector part. All possible combinaions are listed in \cref{tab:class_combinations}. This classification is important because, as discussed in \cref{sec:system_classification}, this classification is precisely equivalent to the qualitative classification of dynamic systems.

\begin{table}[ht]
    \centering
    \caption{All the possible combinations of the class of a split-quaternion and its vector part. Spacelike split-quaternions can only have a spacelike vector, while lightlike split-quaternions can only have lightlike or spacelieke vector parts.}
    \label{tab:class_combinations}
    \begin{tabular}{c|cccc}
        \toprule
        &  & \multicolumn{3}{c}{$ \mathscr{N}_v(\vb{a}) $} \\
        \hline
        &  & \emph{spacelike} & \emph{lightlike} & \emph{timelike} \\
        & \emph{spacelike} & \circled{1} & --- & --- \\
        & \emph{lightlike} & \circled{2} & \circled{3} & --- \\
        \multirow{-3}{*}{$ \mathscr{N}(a) $} & \emph{timelike} & \circled{4} & \circled{5} & \circled{6} \\
        \bottomrule
    \end{tabular}
\end{table}

%As such, we can identify a grand total of \emph{nine} categories for split-quaternions, based on each possible combination of quaternion and vector norm `sign' (these can be chosen completely independent from each other). 
 
\subsection{Relation with two-dimensional matrix algebra}
The algebra of split-quaternions is isomorphic to the algebra of real two-dimensional matrices. This fact underlies this entire chapter, for it allows us to find an alternative for the traditional matrix description of linear dynamical systems. 

Formally, an algebra is a vector space combined with a vector space $V$ over a field \field, combined with an addition operation, scalar multiplication, and an \field-bilinear product operation $V\times V \to V$. \cite{Schuller2014}
\begin{itemize}
    \item The split-quaternion algebra is an algebra over the field real numbers ($\field = \real$), where the multiplication is governed by the split-quaternion multiplication rules (see \cref{tab:quat_table}).
    \item The set of $2\times2$-matrices also constitutes an \real-vector space; matrix multiplication makes it into an algebra.
\end{itemize}
An algebra isomorphism is an isomorphism between vector spaces that also commutes with the respective product operations in both vector spaces. If $(V, \bullet)$ and $(W, \diamond)$ are vector spaces equipped with their product operations, then $\phi: V \to W$ is an algebra isomorphism if (i) $\phi$ is a vector space isomorphism between $V$ and $W$, and (ii)
$$ \phi(v_1 \bullet v_2) = \phi(v_1)\diamond\phi(v_2) \qquad v_1, v_2 \in V. $$
In the case of the split-quaternions and the matrices, it is sufficient to map the basis elements of the split-quaternions to four linearly independent `basis' matrices, and show that the resulting matrices observe the same multiplication rules as defined in \cref{tab:quat_table}. Indeed, define the mapping $\phi$ by 
\begin{equation}
    \begin{split}
        \phi: \spquaternions \to \real^{2\times2}: \quad &  
         1 \mapsto  \mqty(1 & 0 \\ 0 & 1) \qquad
        \quati \mapsto  \mqty(0 & 1 \\  -1 & 0) \\
        & \quatj \mapsto  \mqty(0 & 1 \\  1 & 0)\qquad 
        \quatk \mapsto  \mqty(1 & 0 \\  0 & -1) \\
    \end{split}
\end{equation}
It is easily verified that (i) these matrices span $\real^{2\times2}$ and (ii) that the multiplication rules for split-quaternions are in accordance when translated to the respective matrices under matrix multiplication. Due to the bilinearity of the product, any linear combination of the basis elements will therefore satisfy the rules as well. Hence, we have established an algebra isomorphism between the split-quaternions and the $2\times 2$-matrices. Based on this mapping for the basis vectors, a general quaternion maps to 
$$ \phi(\quat{a_0}{a_1}{a_2}{a_3}) \quad = \quad \mqty(a_0 + a_3 & a_1 + a_2 \\ a_2 - a_1 & a_0 - a_3). $$
Likewise, the inverse mapping on an arbitrary matrix yields
$$ \phi^{-1}\mqty(b_0 & b_1 \\ b_2 & b_3) \quad = \quad \quat{\frac{b_0 + b_3}{2}}{\qty(\frac{b_1 - b_2}{2})}{\qty(\frac{b_1 + b_2}{2})}{\qty(\frac{b_0 - b_3}{2})}.$$

One of the powerful features of the mapping $\phi$ is that it maps natural properties of the split-quaternion to natural properties of the associated matrix. Hence, given that $A = \phi(a)$ with $a \in \spquaternions$ and $A \in \real^{2\times2}$, we have the following properties: 
\begin{itemize}
    \item The \emph{conjugate} of the split-quaternion maps to the \emph{adjugate} of the matrix:\footnote
        {The adjugate of a matrix is the transpose of its cofactor matrix.}
        $$ \phi(\conj{a}) = \adjugate(A). $$
    \item The \emph{trace} of the matrix coincides with the \emph{real or scalar part} of the split-quaternion:
        $$ \scapart{a} = \frac{\trace(A)}{2}. $$
    \item The \emph{determinant} of the matrix is equal to \emph{norm} of the split-quaternion:
        $$ \mathscr{N}(a) = \det(A). $$
    \item The equivalence of the determinant and the split-quaternion norm hints at the fact that the multiplicative inverse of a split-quaternion does not always exist: only when its norm is nonzero. In that case, it is clear that
        $$ \phi\qty(a^{-1}) = A^{-1} \qquad \mathscr{N}(A) \neq 0. $$
    The determinant properties also learns us something about the behavior of the classification under the split-quaternion multiplication
        \begin{table}[h!]
        \centering
        \caption{Propagation of the class of split-quaternions when multiplied with a split-quaternion of another class. The timelike split-quaternions form a subgroup under multiplication, the timelike and spacelike split-quaternions do not: timelike split-quaternions do not have an inverse and the spacelike split-quaternions are not closed.}
        \label{tab:multiplication_class}
        \begin{tabular}{c|ccc}
            \toprule
            $\times$ & \emph{space} & \emph{light} & \emph{time} \\
            \hline
            \emph{space} & time  & light & space \\
            \emph{light} & light & light & light \\
            \emph{time} &  space & light & time \\
            \bottomrule
        \end{tabular}
        \end{table}
    \item The eigenvalues of a $2\times2$-matrix can be expressed in terms of its trace and its determinant:
        $$ \lambda_A = \frac{\trace(A) \pm \sqrt{\trace^2(A) - 4\det(A)}}{2}.$$
        The argument of the square root is equal to the \emph{negative of the vector norm} of $a$. We therefore have:
        $$ \lambda_A = \frac{2a_0 \pm \sqrt{4 a_0^2 - 4\mathscr{N}(a)}}{2} = a_0 \pm \ii\norm{\vb{a}}.$$
        Hence, the `real' (scalar) and the magnitude of the `imaginary' (vector) parts of the quaternion coincide with the real and imaginary part of the eigenvalues of the matrix.
\end{itemize}

The algebra of $2\times2$-matrices (or equivalently, of the split-quaternions) also consitute the Lie algebra $\glalg{2}{\real}$ of the two-dimensional general linear group $\glgroup{2}{\real}$. Furthermore, the traceless matrices, or equivalently, the split-quaternions with zero real part form the subalgebra $\slalg{2}{\real}$ of the special linear group $\slgroup{2}{\real}$. These are the volume-preserving automorphisms on $\real^2$. Because in $\real^2$, volume and area coincide, the special linear group and the symplectic group $\spgroup{1}$ are equivalent. For higher dimensions, this is not the case: area preservation is generally a stronger condition than volume preservation. The Lie algebra elements of the symplectic group are called Hamiltonian matrices; therefore, split-quaternions without real part are referred to as \emph{Hamiltonian}.

\section{Classification of dynamical systems}
\label{sec:system_classification}
The classification of two-dimensional linear dynamical systems is important, or they also locally represent the fixed points of general nonlinear systems. Traditionally, this decomposition is done according to the eigenvalues of the state transition matrix matrix $A$, or equivalently, through a Poincaré diagram as shown in \cref{fig:poincare_diagram}.
\begin{figure}[h!]
    \begin{center}
        \begin{tikzpicture}[line cap=round,line join=round]
  % BACKGROUND FILLS
  \fill [main, domain=-4.3:4.3, accent1!40] plot (\x, {0.25*\x*\x}) -- cycle; % main graph
  \fill [main, domain=-4.3:4.3, accent2!40] (-4.3, 0) -- plot (\x, {0.25*\x*\x}) -- (4.3, 0) -- cycle; % main graph
  \fill [main, todoGray!30] (-4.3, 0) -- (-4.3, -2.7) -- (4.3, -2.7) -- (4.3, 0) -- cycle; % main graph

  % MAIN DIAGRAM
  \draw [main,->, thick] (0,-0.3) -- (0,4.8)  % vertical axis
    node [label={[above,yshift=-0.2cm] $\mathscr{N}(a)$}] {};

  \draw [main,->, thick] (-4.5,0) -- (4.5,0)  % horizontal axis
    node [label={[right,yshift=-1ex] $a_0$}] {}; 

    \draw [main, domain=-4.3:4.5, smooth, double] plot (\x, {0.25*\x*\x}) node [anchor=south] {$\mathscr{N}(\vec{a}) = 0$}; % main graph
  %\node at (-4,4) [pin={[above]$\scriptstyle\Delta=0$}] {};

  %\node at (-4,4) [pin={[above]$\scriptstyle\Delta=0$}] {};

  %\node at (4,4.2) [pin={[align=left] {$ \mathscr{N}(\vec{a}) = 0$}}] {};

  % TEMPLATES describing areas
  \node at ( 0  ,-1.4) {\template\saddle};
  \node at (-3.5  , 1  ) {\template\sink};
  \node at ( 3.5  , 1  ) {\template\source}; 
  \node at (-1.8, 3.7) {\template\spiralsink};
  \node at ( 1.8, 3.7) {\template\spiralsource};

  % TEMPLATES labeling lines and points
  \node at ( 0  , 1.1) [inner sep= 1mm, outer sep = 0mm, pin={[draw,fill=white,right,xshift=0.2cm]%
    \template\centre}] {};
  \node at (-3  , 0  ) [inner sep= 1mm, outer sep = 0mm, pin={[draw,fill=white,below,yshift=-1cm]%
    \template\stablefp}] {};
  \node at ( 3  , 0  ) [inner sep= 1mm, outer sep = 0mm, pin={[draw,fill=white,below,yshift=-1cm]%
    \template\unstablefp}] {};
  \node at (-3.5,{0.25*3.5*3.5}) [inner sep= 1mm, outer sep = 0mm, pin={[draw,left,fill=white,xshift=-1.15cm,yshift=-0.3cm]%
    \template\degensink}] {};
  \node at (3.5,{0.25*3.5*3.5}) [inner sep= 1mm, outer sep = 0mm, pin={[draw,fill=white,right,xshift=1.15cm,yshift=-0.3cm]%
    \template\degensource}] {};
  \node at ( 0  , 0  ) [inner sep= 1mm, outer sep = 0mm, pin={[draw, above left,fill=white,align=center,xshift=-0.3cm]%
    ~\\[-0.9ex]\templatecaption{\normalsize{\circled{3}}}\\[-0.6ex]\templatecaption{uniform}\\[-1ex]\templatecaption{motion}}] {};
%% sdafasd

    \node[fill=accent1!40, draw=none, label=right:{\footnotesize $\mathscr{N}(a) > 0,\:\mathscr{N}(\vec{a}) > 0$}] at (4.7,-1) {};
    \node[fill=accent2!40, draw=none, label=right:{\footnotesize $\mathscr{N}(a) > 0,\:\mathscr{N}(\vec{a}) < 0$}] at (4.7,-1.6) {};
    \node[fill=todoGray!30, draw=none, label=right:{\footnotesize $\mathscr{N}(a) < 0$}] at (4.7,-2.2) {};

 % Place nodes for numbering classes
    \node at (-3.7, 2.3) {\circled{4}};
    \node at (3.7, 2.3) {\circled{4}};
    \node at (-1.8, 2.3) {\circled{6}};
    \node at (3, 4) {\circled{6}};
    \node at (1, -2) {\circled{1}};

\end{tikzpicture}

    \end{center}
    \caption{The classic Poincaré diagram, based on the conventional classification of fixed points based on the trace and determinant of the state transition matrix $A$.}
    \label{fig:poincare_diagram}
\end{figure}

Because the split-quaternion norms are directly related to the real and imaginary part of the eigenvalues of the associated matrix, this classification is more naturally done in the realm of split-quaternions.

\subsubsection*{Spacelike split-quaternion norm}
    \begin{itemize}
        \item[\circled{1}] For spacelike split-quaternions, there is only one possibility: a negative split-quaternion norm corresponds to a negative determinant, which means that the fixed point is a \emph{saddle}. We can distinguish one particular case: if the scalar part of the split-quaternion is zero ($a_0 = 0$), the saddle is `balanced', and generates a proper \emph{squeeze mapping}, which is a symplectomorphism of the phase space. The split-quaternion is therefore Hamiltonian. An example of the latter is the linearization of the unstable fixed point of a rotational pendulum.
    \end{itemize}

\subsubsection*{Lightlike split-quaternion norm}
    \begin{itemize}
        \item[\circled{2}] \emph{Spacelike vector norm}: in this case, there is not just a fixed point but a fixed line in the phase space. This fixed line is stable or unstable depending on the sign of the scalar part of the quaternion. 
        \item[\circled{3}] \emph{Lightlike vector norm}: this case is degenerate of the second degree; it coincides with the origin in the Poincaré diagram. The associated vector field is purely translational. An example is an object in uniform motion.
    \end{itemize}
\subsubsection*{Timelike split-quaternion norm}
    \begin{itemize}
        \item[\circled{4}] \emph{Spacelike vector norm}: this case gives rise to eigenvalues that are purely real; the fixed point is called a \emph{node}. Depending on the sign of the scalar part, the fixed point can be an unstable node or \emph{source} ($a_0 > 0$) or a stable node or \emph{sink} ($a_0 < 0$). An example of such a system is the overdamped harmonic oscillator.
        \item[\circled{5}] \emph{Lightlike vector norm}: the eigenvalues of the associated matrix are real and equal; this type of fixed point is named a \emph{degenerate node}. More specifically, in the unstable case ($a_0 > 0$) it is called a \emph{degenerate source}, while in the stable case it is referred to as a \emph{degenerate sink}.
        \item[\circled{6}] Blabla
    \end{itemize}

\todo{Connection with root locus}
\todo{Connection with Jordan decomposition}

\todo{Basis vector fields}


\section{Notes}
! orthogonal refers to `regular' orthogonal, Lorentz-orthogonal makes the distinction.

Motivation: $\vec{u}$ seems to be `aligned' with major direction of the elliptic trajectory in the Lorentz-orthogonal subspace, generated by the action of its cross-product. Show this formally by making use of the eigenvectors.

The basis vectors $ \qty{\vec{e}_2, \vec{e}_3}$, where $\vec{e}_2$ is the orthogonal projection of the vector $\vec{e}_1 = \uvec{u}$ on its Lorentz-orthogonal subspace, and $\vec{e}_3 \triangleq \lorcrossp{\vec{e}_1}{\vec{e}_2}$, form the real and imaginary parts of two of the eigenvectors of the matrix $\mat{U}_{\lorcrossp{}{}}$. 

Because the basis vectors $\vec{e}_2$ and $\vec{e}_3$ are also orthogonal in the Euclidean sense, the 

\begin{proof}
    Let $\uvec{u} = u_1\uvec{i} + u_2\uvec{j} + u_3\uvec{k}$. A normal vector to the Lorentz-orthogonal subspace is $
    \uvec{n} = u_1\uvec{i} - u_2\uvec{j} - u_3\uvec{k}$. Then, the basis vectors are
    \begin{equation}
        \begin{split}
            \vec{e}_2 &= 
            \uvec{u} - \frac{ \inner{\uvec{u}}{\uvec{n}} }{ \inner{\uvec{n}}{\uvec{n}} } \uvec{n} \\
            \vec{e}_3 &= \lorcrossp{\uvec{u}}{\vec{e}_2} = -\frac{ \inner{\uvec{u}}{\uvec{n}} }{ \inner{\uvec{n}}{\uvec{n}} } \qty(\lorcrossp{\uvec{u}}{\uvec{n}}),
        \end{split}
    \end{equation}
    because the Lorentz-cross product distributes over addition and $\lorcrossp{\uvec{u}}{\uvec{u}} = \vec{0}$. The proposition above claims that $\vec{e}_2 + \ii\vec{e}_3$ is an eigenvector of the matrix $\mat{U}_{\lorcrossp{}{}}$. Hence, it must be the case that $\mat{U}_{\lorcrossp{}{}}(\vec{e}_2 + \ii\vec{e}_3) = \lambda(\vec{e}_2 + \ii\vec{e}_3)$, where $\lambda$ is then an eigenvalue of the matrix. This can be verified by replacing the action of $\mat{U}_{\lorcrossp{}{}}$ with the cross product. Plugging in the definition and exploiting the linearity of the Lorentz cross-product, we obtain:
    \begin{equation*}
        \begin{split}
            \lorcrossp{\uvec{u}}{\qty(\vec{e}_2 + \ii\vec{e}_3)} 
            &= \lorcrossp{\uvec{u}}{\vec{e}_2} +
        \ii\qty(\lorcrossp{\uvec{u}}{\vec{e}_3}) \\
            &= \vec{e}_3 + \qty(\lorcrossp{\uvec{u}}{\vec{e}_3})\ii \\ 
            &=\vec{e}_3 +  \qty(\lorcrossp{\uvec{u}}{\qty(\lorcrossp{\uvec{u}}{\vec{e}_2})})\ii \\
            &=\vec{e}_3 -  \frac{ \inner{\uvec{u}}{\uvec{n}} }{ \inner{\uvec{n}}{\uvec{n}} }\qty(\lorcrossp{\uvec{u}}{\qty(\lorcrossp{\uvec{u}}{\uvec{n}})})\ii.  \\
        \end{split}
    \end{equation*}
The triple cross-product expansion, or `Lagrange formula', relates the regular cross product to the corresponding dot product:
    $$ \vec{a}\times\qty(\vec{b}\times\vec{c}) = \vec{b}\:\inner{\vec{c}}{\vec{a}} - \vec{c}\:\inner{\vec{a}}{\vec{b}}. $$
This well-known identity generalizes (easily verified) to the Lorentzian counterpart of the cross- and inner products:
    $$ 
        \lorcrossp{\vec{a}}{\qty(\lorcrossp{\vec{b}}{\vec{c}})} 
       = \vec{b}\:\lorinner{\vec{c}}{\vec{a}} - \vec{c}\:\lorinner{\vec{a}}{\vec{b}}. 
    $$
Using the Lagrange formula, the above expression becomes
    \begin{equation*}
        \begin{split}
            & \vec{e}_3 - \frac{ \inner{\uvec{u}}{\uvec{n}} }{ \inner{\uvec{n}}{\uvec{n}} }\qty(\uvec{u}\,\lorinner{\uvec{u}}{\uvec{n}} - \uvec{n}\lorinner{\uvec{u}}{\uvec{u}})\ii \\
            & =\, \vec{e}_3 - \qty(\uvec{u}\,\frac{ \lorinner{\uvec{u}}{\uvec{n}} \, \inner{\uvec{u}}{\uvec{n}} }{ \inner{\uvec{n}}{\uvec{n}} } - \uvec{n}\frac{ \inner{\uvec{u}}{\uvec{n}} }{ \inner{\uvec{n}}{\uvec{n}} })\ii \\
            & =\, \vec{e}_3 - \qty(\uvec{u} - \uvec{n}\frac{ \inner{\uvec{u}}{\uvec{n}} }{ \inner{\uvec{n}}{\uvec{n}} })\ii \\
            & =\, \vec{e}_3 - \vec{e}_2\ii. 
        \end{split}
    \end{equation*}
    The latter is the scalar multiple of the vector $\vec{e}_2 + \vec{e}_3$ by $-\ii$ - hence, this is indeed an eigenvector of the corresponding matrix.
\end{proof}
Because $\vec{e}_2$ and $\vec{e}_3$ are also orthogonal in the normal sense, they are aligned with the major axes of the elliptic trajectories generated by the cross product. Hence, they can be used to find a basis of the invariant subspace which makes the trajectories identical to those in the phase plane.


