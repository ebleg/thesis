\chapter{Split-Quaternions as Dynamical Systems}
\label{chap:quaternion}
%\lsymb{$\vec{\alpha}$}{A differential form}

In this chapter, the geometric connection is made between the algebra of split-quaternions and the qualitative behavior of two-dimensional linear dynamical systems. 

\section{Split-quaternion algebra}
\subsection{Basic properties}
Like the conventional quaternions, the split-quaternions are a number system that consists of linear combinations of four basis elements, which will be denoted by \quati, \quatj and \quatk. The algebra of split-quaternions is associative but not commutative --- formally speaking, the algebraic structure is a \emph{noncommutative ring}. The multiplication table for the split-quaternion algebra is shown in \cref{tab:quat_table}. The set of split-quaternions is denoted by \spquaternions (since \quaternions is reserved for conventional quaternions).
\begin{table}[ht!]
    \centering
    \caption{Multiplication table for the split-quaternion algebra.}
    \label{tab:quat_table}
    \begin{tabular}{c|cccc}
        \toprule
        &         1      & \quati  & \quatj  & \quatk \\ 
        \midrule
        1       & 1      & \quati  & \quatj  & \quatk \\ 
        \quati  & \quati & -1      & \quatk  & -\quatj \\ 
        \quatj  & \quatj & -\quatk & 1       & -\quati \\ 
        \quatk  & \quatk & \quatj  & \quati  & 1 \\ 
        \bottomrule
    \end{tabular}
\end{table}

The important distinction from conventional quaternions resides in the diagonal elements of \cref{tab:quat_table}. Whereas for quaternions all the nonreal basis elements square to $-1$, this is not the case for the split-quaternions (only \quati does). This is precisely the reason why split-quaternions are `split', for this difference in sign gives rise to an indefinite quadratic form when computing the norm of the split-quadratic form. Because the quadratic form is indefinite, it classifies the set of split-quaternions into several subsets, which is to be discussed later.

\paragraph{Dihedral group} he basis elements of the split-quaternions form a group under multiplication, namely the \emph{dihedral group} \digroup{4}, which represents all the symmetries of a square: the identity, a 90-degree rotation and two reflections (cf. \cref{fig:square_symmetry}).
\begin{figure}[h!]
    \begin{center}
        \begin{tikzpicture}
    \node[draw, inner sep=3mm,thick, fill=accent1!40] at (0, 0) {};
    \node[draw, inner sep=3mm,thick, fill=accent1!40] at (2, 0) {};
    \node[draw, inner sep=3mm,thick, fill=accent1!40] at (4, 0) {};
    \node[draw, inner sep=3mm,thick, fill=accent1!40] at (6, 0) {};
    \draw[->] (2, 0.7) arc (90:0:0.7);
    \draw[->] (2, -0.7) arc (270:180:0.7);
    \draw[dashdotted] (3.5, 0.5) -- (4.5, -0.5);
    \draw[<->] (3.75, -0.25) -- (4.25, 0.25);
    \draw[dashdotted] (5.5, -0.5) -- (6.5, 0.5);
    \draw[<->] (5.75, 0.25) -- (6.25, -0.25);
    \node[] at (0, -1.1) {1};
    \node[] at (2, -1.1) {\quati};
    \node[] at (4, -1.1) {\quatj};
    \node[] at (6, -1.1) {\quatk};

\end{tikzpicture}

    \end{center}
    \caption{The dihedral group \digroup{4} is the symmetry group of a square. This group is isomorphic to the group formed by $1, \quati, \quatj$ and \quatk under multiplication.}
    \label{fig:square_symmetry}
\end{figure}

The structure of the dihedral group can be visualized by its \emph{cycle graph} in \cref{fig:cycle_graph}. Many important properties of the split-quaternion algebra and the applications in this chapter can be traced back to the topology of this cycle graph. One example is the split nature of the quaternions: the \quati-element generates an order four cycle, while \quatj and \quatk generate order two cycles (in contrast, the cycle graph for conventional quaternions is entirely symmetric for all these elements).
\begin{figure}[h!]
    \begin{center}
        \begin{tikzpicture}
    \node[draw, thick, circle, fill=accent1!40] (e) at (0, 0) {};
    
    \node[thick, draw, circle, below right=0.8cm and 0.5cm of e] (r1) {};
    \node[thick, draw, circle, below left=0.8cm and 0.5cm of e] (r2) {};
    \node[thick, draw, circle, left of= r2] (r3) {};
    \node[thick, draw, circle, right of= r1] (r4) {};
    \node[thick, draw, circle, above=1.5cm of e] (ro2) {};
    \node[thick, draw, circle, above right=0.7cm and 1cm of e] (ro1) {};
    \node[thick, draw, circle, above left=0.7cm and 1cm of e] (ro3) {};
    
    \draw[thick] (e) -- (r1);
    \draw[thick] (e) -- (r2);
    \draw[thick] (e) -- (r3);
    \draw[thick] (e) -- (r4);
    
    \draw[thick] (e) -- (ro1) -- (ro2) -- (ro3) -- (e);

\end{tikzpicture}

    \end{center}
    \caption{Cycle graph of the dihedral group. There are five cycles: one of order four which represents the rotations (or the element \quati), and four order 2 cycles, which are all the possible reflections. The colored element represents the identity.}
    \label{fig:cycle_graph}
\end{figure}

\paragraph{Split-quaternion norm} Complex numbers have a real and imaginary part. Likewise, (split)-quaternions have a similar notion: a \emph{scalar} (or real) and \emph{vector} (or imaginary) components. For an arbitrary split-quaternion $q \in \spquaternions$, \cite{Jafari2014}
$$ a = \quat{a_0}{a_1}{a_2}{a_3} $$
the real part is $\scapart{h} = a_0$ and the vector part is $ \vecpart{a} = \quatvec{a_1}{a_2}{a_3}$. For convenience, the vector part will be referred to in traditional bold vector notation:
$$ \vec{a} = \vecpart{a} = \quatvec{a_1}{a_2}{a_3}. $$

Furthermore, for every split-quaternion there is a unique \emph{conjugate}
$$ \conj{a} = \scapart{a} - \vecpart{a} = a_0 - a_1\quati - a_2\quatj - a_3\quatk, $$
through which the \emph{split-quaternion norm} is defined:
\begin{equation}
    \mathscr{N}(a) = a\conj{h} = a_0^2 + a_1^2 - a_2^2 - a_3^2. 
    \label{eq:quat_norm}
\end{equation}
As mentioned, this norm is not positive definite, in stark contrast to quaternions or complex numbers. Split-quaternions can be categorized into three classes based on their norm being negative, zero or positive. In the tradition of special relativity, these classes are named \emph{spacelike}, \emph{lightlike} and \emph{timelike} respectively: \cite{Misner1970,Landau1971}
\begin{itemize}
    \item \textbf{Timelike}: $ \mathscr{N}(a) > 0 $, with real length $ \norm{a} = \sqrt{a \conj{a}}$. 
    \item \textbf{Lightlike}: $ \mathscr{N}(a) = 0 $, with zero length $\norm{a} = 0$. 
    \item \textbf{Spacelike}: $ \mathscr{N}(a) < 0 $, with imaginary length $\norm{a} = \ii \sqrt{\abs{a\conj{a}}}$.
\end{itemize}
Even though they behave similarly, the imaginary unit $\ii$ is not to be confused with the split-quaternion basis element \quati, because they belong to different number systems.

\paragraph{Vector norm}
Apart from the split-quaternion norm, we can also define a norm that only considers the vector part of the split-quaternion. This norm is defined in accordance with the overall quaternion norm given by \cref{eq:quat_norm}:
$$ \mathscr{N}_v(\vec{a}) = a_1^2 - a^2_2 - a^2_3. $$
The above expression is equivalent the Lorentz norm applied to a vector in $\real^3$; we will denote $\real^3$ equipped with the Lorentz norm by $\real^{2, 1}$. \cite{Jafari2014} Observe that this quadratic form is not positive-definite either; in the the same veign as before, we can therefore classify quaternions by the `sign' of their vector part again. We refer to these classes as \emph{timelike (vectors)}, \emph{spacelike (vectors)} and \emph{lightlike (vectors)} in the same way. 

Observe that $ \mathscr{N}(a) < 0 \Rightarrow \mathscr{N}_v(\vec{a}) < 0$; that is to say, a spacelike split-quaternion always has a spacelike vector part. The converse is not necessarily true. Along the same line, a lightlike split-quaternion can only have a lightlike or spacelike vector part. All possible combinaions are listed in \cref{tab:class_combinations}. This classification is important because, as discussed in \cref{sec:system_classification}, this classification is precisely equivalent to the qualitative classification of dynamic systems.

\begin{table}[ht]
    \centering
    \caption{All the possible combinations of the class of a split-quaternion and its vector part. Spacelike split-quaternions can only have a spacelike vector, while lightlike split-quaternions can only have lightlike or spacelieke vector parts.}
    \label{tab:class_combinations}
    \begin{tabular}{c|cccc}
        \toprule
        &  & \multicolumn{3}{c}{$ \mathscr{N}_v(\vb{a}) $} \\
        \hline
        &  & \emph{spacelike} & \emph{lightlike} & \emph{timelike} \\
        & \emph{spacelike} & \circled{1} & --- & --- \\
        & \emph{lightlike} & \circled{2} & \circled{3} & --- \\
        \multirow{-3}{*}{$ \mathscr{N}(a) $} & \emph{timelike} & \circled{4} & \circled{5} & \circled{6} \\
        \bottomrule
    \end{tabular}
\end{table}

%As such, we can identify a grand total of \emph{nine} categories for split-quaternions, based on each possible combination of quaternion and vector norm `sign' (these can be chosen completely independent from each other). 
 
\subsection{Relation with two-dimensional matrix algebra}
The algebra of split-quaternions is isomorphic to the algebra of real two-dimensional matrices. This fact underlies this entire chapter, for it allows us to find an alternative for the traditional matrix description of linear dynamical systems. 

Formally, an algebra is a vector space combined with a vector space $V$ over a field \field, combined with an addition operation, scalar multiplication, and an \field-bilinear product operation $V\times V \to V$. \cite{Schuller2014}
\begin{itemize}
    \item The split-quaternion algebra is an algebra over the field real numbers ($\field = \real$), where the multiplication is governed by the split-quaternion multiplication rules (see \cref{tab:quat_table}).
    \item The set of $2\times2$-matrices also constitutes an \real-vector space; matrix multiplication makes it into an algebra.
\end{itemize}
An algebra isomorphism is an isomorphism between vector spaces that also commutes with the respective product operations in both vector spaces. If $(V, \bullet)$ and $(W, \diamond)$ are vector spaces equipped with their product operations, then $\phi: V \to W$ is an algebra isomorphism if (i) $\phi$ is a vector space isomorphism between $V$ and $W$, and (ii)
$$ \phi(v_1 \bullet v_2) = \phi(v_1)\diamond\phi(v_2) \qquad v_1, v_2 \in V. $$
In the case of the split-quaternions and the matrices, it is sufficient to map the basis elements of the split-quaternions to four linearly independent `basis' matrices, and show that the resulting matrices observe the same multiplication rules as defined in \cref{tab:quat_table}. Indeed, define the mapping $\phi$ by 
\begin{equation}
    \begin{split}
        \phi: \spquaternions \to \real^{2\times2}: \quad &  
         1 \mapsto  \mqty(1 & 0 \\ 0 & 1) \qquad
        \quati \mapsto  \mqty(0 & 1 \\  -1 & 0) \\
        & \quatj \mapsto  \mqty(0 & 1 \\  1 & 0)\qquad 
        \quatk \mapsto  \mqty(1 & 0 \\  0 & -1) \\
    \end{split}
\end{equation}
It is easily verified that (i) these matrices span $\real^{2\times2}$ and (ii) that the multiplication rules for split-quaternions are in accordance when translated to the respective matrices under matrix multiplication. Due to the bilinearity of the product, any linear combination of the basis elements will therefore satisfy the rules as well. Hence, we have established an algebra isomorphism between the split-quaternions and the $2\times 2$-matrices. Based on this mapping for the basis vectors, a general quaternion maps to 
$$ \phi(\quat{a_0}{a_1}{a_2}{a_3}) \quad = \quad \mqty(a_0 + a_3 & a_1 + a_2 \\ a_2 - a_1 & a_0 - a_3). $$
Likewise, the inverse mapping on an arbitrary matrix yields
$$ \phi^{-1}\mqty(b_0 & b_1 \\ b_2 & b_3) \quad = \quad \quat{\frac{b_0 + b_3}{2}}{\qty(\frac{b_1 - b_2}{2})}{\qty(\frac{b_1 + b_2}{2})}{\qty(\frac{b_0 - b_3}{2})}.$$

One of the powerful features of the mapping $\phi$ is that it maps natural properties of the split-quaternion to natural properties of the associated matrix. Hence, given that $A = \phi(a)$ with $a \in \spquaternions$ and $A \in \real^{2\times2}$, we have the following properties: 
\begin{itemize}
    \item The \emph{conjugate} of the split-quaternion maps to the \emph{adjugate} of the matrix:\footnote
        {The adjugate of a matrix is the transpose of its cofactor matrix.}
        $$ \phi(\conj{a}) = \adjugate(A). $$
    \item The \emph{trace} of the matrix coincides with the \emph{real or scalar part} of the split-quaternion:
        $$ \scapart{a} = \frac{\trace(A)}{2}. $$
    \item The \emph{determinant} of the matrix is equal to \emph{norm} of the split-quaternion:
        $$ \mathscr{N}(a) = \det(A). $$
    \item The equivalence of the determinant and the split-quaternion norm hints at the fact that the multiplicative inverse of a split-quaternion does not always exist: only when its norm is nonzero. In that case, it is clear that
        $$ \phi\qty(a^{-1}) = A^{-1} \qquad \mathscr{N}(A) \neq 0. $$
    The determinant properties also learns us something about the behavior of the classification under the split-quaternion multiplication
        \begin{table}[h!]
        \centering
        \caption{Propagation of the class of split-quaternions when multiplied with a split-quaternion of another class. The timelike split-quaternions form a subgroup under multiplication, the timelike and spacelike split-quaternions do not: timelike split-quaternions do not have an inverse and the spacelike split-quaternions are not closed.}
        \label{tab:multiplication_class}
        \begin{tabular}{c|ccc}
            \toprule
            $\times$ & \emph{space} & \emph{light} & \emph{time} \\
            \hline
            \emph{space} & time  & light & space \\
            \emph{light} & light & light & light \\
            \emph{time} &  space & light & time \\
            \bottomrule
        \end{tabular}
        \end{table}
    \item The eigenvalues of a $2\times2$-matrix can be expressed in terms of its trace and its determinant:
        $$ \lambda_A = \frac{\trace(A) \pm \sqrt{\trace^2(A) - 4\det(A)}}{2}.$$
        The argument of the square root is equal to the \emph{negative of the vector norm} of $a$. We therefore have:
        \begin{equation}
            \lambda_A = \frac{2a_0 \pm \sqrt{4 a_0^2 - 4\mathscr{N}(a)}}{2} = a_0 \pm \ii\norm{\vb{a}}. 
            \label{eq:quat_eigvals}
        \end{equation}
        Hence, the `real' (scalar) and the magnitude of the `imaginary' (vector) parts of the quaternion coincide with the real and imaginary part of the eigenvalues of the matrix.
\end{itemize}

The algebra of $2\times2$-matrices (or equivalently, of the split-quaternions) also consitute the Lie algebra $\glalg{2}{\real}$ of the two-dimensional general linear group $\glgroup{2}{\real}$. Furthermore, the traceless matrices, or equivalently, the split-quaternions with zero real part form the subalgebra $\slalg{2}{\real}$ of the special linear group $\slgroup{2}{\real}$. These are the volume-preserving automorphisms on $\real^2$. Because in $\real^2$, volume and area coincide, the special linear group and the symplectic group $\spgroup{1}$ are equivalent. For higher dimensions, this is not the case: area preservation is generally a stronger condition than volume preservation. The Lie algebra elements of the symplectic group are called Hamiltonian matrices; therefore, split-quaternions without real part are referred to as \emph{Hamiltonian}.

\section{Classification of dynamical systems}
\label{sec:system_classification}
The classification of two-dimensional linear dynamical systems is important, or they also locally represent the fixed points of general nonlinear systems. Traditionally, this decomposition is done according to the eigenvalues of the state transition matrix matrix $A$, or equivalently, through a Poincaré diagram as shown in \cref{fig:poincare_diagram}.
\begin{figure}[h!]
    \begin{center}
        \begin{tikzpicture}[line cap=round,line join=round]
  % BACKGROUND FILLS
  \fill [main, domain=-4.3:4.3, accent1!40] plot (\x, {0.25*\x*\x}) -- cycle; % main graph
  \fill [main, domain=-4.3:4.3, accent2!40] (-4.3, 0) -- plot (\x, {0.25*\x*\x}) -- (4.3, 0) -- cycle; % main graph
  \fill [main, todoGray!30] (-4.3, 0) -- (-4.3, -2.7) -- (4.3, -2.7) -- (4.3, 0) -- cycle; % main graph

  % MAIN DIAGRAM
  \draw [main,->, thick] (0,-0.3) -- (0,4.8)  % vertical axis
    node [label={[above,yshift=-0.2cm] $\mathscr{N}(a)$}] {};

  \draw [main,->, thick] (-4.5,0) -- (4.5,0)  % horizontal axis
    node [label={[right,yshift=-1ex] $a_0$}] {}; 

    \draw [main, domain=-4.3:4.5, smooth, double] plot (\x, {0.25*\x*\x}) node [anchor=south] {$\mathscr{N}(\vec{a}) = 0$}; % main graph
  %\node at (-4,4) [pin={[above]$\scriptstyle\Delta=0$}] {};

  %\node at (-4,4) [pin={[above]$\scriptstyle\Delta=0$}] {};

  %\node at (4,4.2) [pin={[align=left] {$ \mathscr{N}(\vec{a}) = 0$}}] {};

  % TEMPLATES describing areas
  \node at ( 0  ,-1.4) {\template\saddle};
  \node at (-3.5  , 1  ) {\template\sink};
  \node at ( 3.5  , 1  ) {\template\source}; 
  \node at (-1.8, 3.7) {\template\spiralsink};
  \node at ( 1.8, 3.7) {\template\spiralsource};

  % TEMPLATES labeling lines and points
  \node at ( 0  , 1.1) [inner sep= 1mm, outer sep = 0mm, pin={[draw,fill=white,right,xshift=0.2cm]%
    \template\centre}] {};
  \node at (-3  , 0  ) [inner sep= 1mm, outer sep = 0mm, pin={[draw,fill=white,below,yshift=-1cm]%
    \template\stablefp}] {};
  \node at ( 3  , 0  ) [inner sep= 1mm, outer sep = 0mm, pin={[draw,fill=white,below,yshift=-1cm]%
    \template\unstablefp}] {};
  \node at (-3.5,{0.25*3.5*3.5}) [inner sep= 1mm, outer sep = 0mm, pin={[draw,left,fill=white,xshift=-1.15cm,yshift=-0.3cm]%
    \template\degensink}] {};
  \node at (3.5,{0.25*3.5*3.5}) [inner sep= 1mm, outer sep = 0mm, pin={[draw,fill=white,right,xshift=1.15cm,yshift=-0.3cm]%
    \template\degensource}] {};
  \node at ( 0  , 0  ) [inner sep= 1mm, outer sep = 0mm, pin={[draw, above left,fill=white,align=center,xshift=-0.3cm]%
    ~\\[-0.9ex]\templatecaption{\normalsize{\circled{3}}}\\[-0.6ex]\templatecaption{uniform}\\[-1ex]\templatecaption{motion}}] {};
%% sdafasd

    \node[fill=accent1!40, draw=none, label=right:{\footnotesize $\mathscr{N}(a) > 0,\:\mathscr{N}(\vec{a}) > 0$}] at (4.7,-1) {};
    \node[fill=accent2!40, draw=none, label=right:{\footnotesize $\mathscr{N}(a) > 0,\:\mathscr{N}(\vec{a}) < 0$}] at (4.7,-1.6) {};
    \node[fill=todoGray!30, draw=none, label=right:{\footnotesize $\mathscr{N}(a) < 0$}] at (4.7,-2.2) {};

 % Place nodes for numbering classes
    \node at (-3.7, 2.3) {\circled{4}};
    \node at (3.7, 2.3) {\circled{4}};
    \node at (-1.8, 2.3) {\circled{6}};
    \node at (3, 4) {\circled{6}};
    \node at (1, -2) {\circled{1}};

\end{tikzpicture}

    \end{center}
    \caption{The classic Poincaré diagram, based on the conventional classification of fixed points based on the trace and determinant of the state transition matrix $A$.}
    \label{fig:poincare_diagram}
\end{figure}

Because the split-quaternion norms are directly related to the real and imaginary part of the eigenvalues of the associated matrix, this classification is more naturally done in the realm of split-quaternions.

\subsubsection*{Spacelike split-quaternion norm}
    \begin{itemize}
        \item[\circled{1}] For spacelike split-quaternions, there is only one possibility: a negative split-quaternion norm corresponds to a negative determinant, which means that the fixed point is a \emph{saddle}. We can distinguish one particular case: if the scalar part of the split-quaternion is zero ($a_0 = 0$), the saddle is `balanced', and generates a proper \emph{squeeze mapping}, which is a symplectomorphism of the phase space. The split-quaternion is therefore Hamiltonian. An example of the latter is the linearization of the unstable fixed point of a rotational pendulum.
    \end{itemize}

\subsubsection*{Lightlike split-quaternion norm}
    \begin{itemize}
        \item[\circled{2}] \emph{Spacelike vector norm}: in this case, there is not just a fixed point but a fixed line in the phase space. This fixed line is stable or unstable depending on the sign of the scalar part of the quaternion. 
        \item[\circled{3}] \emph{Lightlike vector norm}: this case is degenerate of the second degree; it coincides with the origin in the Poincaré diagram. The associated vector field is purely translational. An example is an object in uniform motion.
    \end{itemize}
\subsubsection*{Timelike split-quaternion norm}
    \begin{itemize}
        \item[\circled{4}] \emph{Spacelike vector norm}: this case gives rise to eigenvalues that are purely real; the fixed point is called a \emph{node}. Depending on the sign of the scalar part, the fixed point can be an unstable node or \emph{source} ($a_0 > 0$) or a stable node or \emph{sink} ($a_0 < 0$). An example of such a system is the overdamped harmonic oscillator.
        \item[\circled{5}] \emph{Lightlike vector norm}: the eigenvalues of the associated matrix are real and equal; this type of fixed point is named a \emph{degenerate node}. More specifically, in the unstable case ($a_0 > 0$) it is called a \emph{degenerate source}, while in the stable case it is referred to as a \emph{degenerate sink}. An example is a critically damped harmonic oscillator.
        \item[\circled{6}] \emph{Timelike vector norm}: this really is the only general case where the eigenvalues of $A$ are complex. If $a_0 = 0$, the eigenvalues are imaginary and the fixed point is a \emph{center}. Likewise, for $a_0 > 0$ it is an \emph{unstable spiral node} and for $a_0 < 0$ a \emph{stable spiral node}. An example is an underdamped (or even undamped) harmonic oscillator.
    \end{itemize}

\todo{Connection with root locus}
\todo{Connection with Jordan decomposition}

\todo{Basis vector fields}

\begin{figure}
    \begin{center}
        \begin{tikzpicture}

\begin{axis}[%
width=2.367in,
height=2.367in,
at={(0.388in,3.158in)},
scale only axis,
xmin=-1,
xmax=1,
ymin=-1,
ymax=1,
axis background/.style={fill=white},
%title style={font=\bfseries},
title={$1\quad \qty(x\pdv{}{x} + y\pdv{}{y})$},
axis lines = box,
]
\addplot [color=black!40, line width=0.4pt, forget plot]
  table[row sep=crcr]{%
0.005	0\\
0.0055	0\\
0.0061	0\\
0.0067	0\\
0.0075	0\\
0.0082	0\\
0.0091	0\\
0.0101	0\\
0.0111	0\\
0.0123	0\\
0.0136	0\\
0.015	0\\
0.0166	0\\
0.0183	0\\
0.0203	0\\
0.0224	0\\
0.0248	0\\
0.0274	0\\
0.0302	0\\
0.0334	0\\
0.0369	0\\
0.0408	0\\
0.0451	0\\
0.0499	0\\
0.0551	0\\
0.0609	0\\
0.0673	0\\
0.0744	0\\
0.0822	0\\
0.0909	0\\
0.1004	0\\
0.111	0\\
0.1227	0\\
0.1356	0\\
0.1498	0\\
0.1656	0\\
0.183	0\\
0.2022	0\\
0.2235	0\\
0.247	0\\
0.273	0\\
0.3017	0\\
0.3334	0\\
0.3685	0\\
0.4073	0\\
0.4501	0\\
0.4974	0\\
0.5497	0\\
0.6076	0\\
0.6714	0\\
0.7421	0\\
0.8201	0\\
0.9064	0\\
1.0017	0\\
1.107	0\\
};
\addplot [color=black!40, line width=0.4pt, forget plot]
  table[row sep=crcr]{%
0.0047	0.0016\\
0.0052	0.0018\\
0.0058	0.002\\
0.0064	0.0022\\
0.0071	0.0024\\
0.0078	0.0027\\
0.0086	0.003\\
0.0095	0.0033\\
0.0105	0.0036\\
0.0116	0.004\\
0.0129	0.0044\\
0.0142	0.0049\\
0.0157	0.0054\\
0.0174	0.006\\
0.0192	0.0066\\
0.0212	0.0073\\
0.0234	0.008\\
0.0259	0.0089\\
0.0286	0.0098\\
0.0316	0.0109\\
0.0349	0.012\\
0.0386	0.0133\\
0.0427	0.0147\\
0.0472	0.0162\\
0.0521	0.0179\\
0.0576	0.0198\\
0.0637	0.0219\\
0.0704	0.0242\\
0.0778	0.0267\\
0.0859	0.0295\\
0.095	0.0326\\
0.105	0.036\\
0.116	0.0398\\
0.1282	0.044\\
0.1417	0.0486\\
0.1566	0.0538\\
0.1731	0.0594\\
0.1913	0.0657\\
0.2114	0.0726\\
0.2336	0.0802\\
0.2582	0.0886\\
0.2854	0.098\\
0.3154	0.1083\\
0.3485	0.1197\\
0.3852	0.1322\\
0.4257	0.1461\\
0.4705	0.1615\\
0.5199	0.1785\\
0.5746	0.1973\\
0.6351	0.218\\
0.7019	0.2409\\
0.7757	0.2663\\
0.8573	0.2943\\
0.9474	0.3252\\
1.0471	0.3595\\
1.1572	0.3973\\
};
\addplot [color=black!40, line width=0.4pt, forget plot]
  table[row sep=crcr]{%
0.0039	0.0031\\
0.0044	0.0034\\
0.0048	0.0038\\
0.0053	0.0041\\
0.0059	0.0046\\
0.0065	0.0051\\
0.0072	0.0056\\
0.0079	0.0062\\
0.0088	0.0068\\
0.0097	0.0076\\
0.0107	0.0083\\
0.0119	0.0092\\
0.0131	0.0102\\
0.0145	0.0113\\
0.016	0.0125\\
0.0177	0.0138\\
0.0195	0.0152\\
0.0216	0.0168\\
0.0239	0.0186\\
0.0264	0.0205\\
0.0292	0.0227\\
0.0322	0.0251\\
0.0356	0.0277\\
0.0394	0.0306\\
0.0435	0.0339\\
0.0481	0.0374\\
0.0531	0.0413\\
0.0587	0.0457\\
0.0649	0.0505\\
0.0717	0.0558\\
0.0793	0.0617\\
0.0876	0.0682\\
0.0968	0.0753\\
0.107	0.0833\\
0.1182	0.092\\
0.1307	0.1017\\
0.1444	0.1124\\
0.1596	0.1242\\
0.1764	0.1373\\
0.1949	0.1517\\
0.2154	0.1677\\
0.2381	0.1853\\
0.2631	0.2048\\
0.2908	0.2263\\
0.3214	0.2501\\
0.3552	0.2764\\
0.3925	0.3055\\
0.4338	0.3377\\
0.4794	0.3732\\
0.5299	0.4124\\
0.5856	0.4558\\
0.6472	0.5037\\
0.7152	0.5567\\
0.7905	0.6152\\
0.8736	0.68\\
0.9655	0.7515\\
1.067	0.8305\\
1.1792	0.9178\\
};
\addplot [color=black!40, line width=0.4pt, forget plot]
  table[row sep=crcr]{%
0.0027	0.0042\\
0.003	0.0046\\
0.0033	0.0051\\
0.0037	0.0057\\
0.0041	0.0062\\
0.0045	0.0069\\
0.005	0.0076\\
0.0055	0.0084\\
0.0061	0.0093\\
0.0067	0.0103\\
0.0074	0.0114\\
0.0082	0.0126\\
0.0091	0.0139\\
0.01	0.0154\\
0.0111	0.017\\
0.0123	0.0188\\
0.0135	0.0207\\
0.015	0.0229\\
0.0165	0.0253\\
0.0183	0.028\\
0.0202	0.0309\\
0.0223	0.0342\\
0.0247	0.0378\\
0.0273	0.0418\\
0.0301	0.0461\\
0.0333	0.051\\
0.0368	0.0564\\
0.0407	0.0623\\
0.045	0.0688\\
0.0497	0.0761\\
0.0549	0.0841\\
0.0607	0.0929\\
0.0671	0.1027\\
0.0741	0.1135\\
0.0819	0.1254\\
0.0906	0.1386\\
0.1001	0.1532\\
0.1106	0.1693\\
0.1222	0.1871\\
0.1351	0.2068\\
0.1493	0.2285\\
0.165	0.2526\\
0.1824	0.2791\\
0.2015	0.3085\\
0.2227	0.3409\\
0.2462	0.3768\\
0.2721	0.4164\\
0.3007	0.4602\\
0.3323	0.5086\\
0.3672	0.5621\\
0.4059	0.6212\\
0.4486	0.6866\\
0.4957	0.7588\\
0.5479	0.8386\\
0.6055	0.9268\\
0.6692	1.0242\\
0.7395	1.132\\
};
\addplot [color=black!40, line width=0.4pt, forget plot]
  table[row sep=crcr]{%
0.0012	0.0048\\
0.0014	0.0054\\
0.0015	0.0059\\
0.0017	0.0065\\
0.0018	0.0072\\
0.002	0.008\\
0.0022	0.0088\\
0.0025	0.0098\\
0.0027	0.0108\\
0.003	0.0119\\
0.0033	0.0132\\
0.0037	0.0146\\
0.0041	0.0161\\
0.0045	0.0178\\
0.005	0.0197\\
0.0055	0.0217\\
0.0061	0.024\\
0.0067	0.0265\\
0.0074	0.0293\\
0.0082	0.0324\\
0.0091	0.0358\\
0.01	0.0396\\
0.0111	0.0437\\
0.0122	0.0483\\
0.0135	0.0534\\
0.015	0.059\\
0.0165	0.0653\\
0.0183	0.0721\\
0.0202	0.0797\\
0.0223	0.0881\\
0.0247	0.0974\\
0.0272	0.1076\\
0.0301	0.1189\\
0.0333	0.1314\\
0.0368	0.1452\\
0.0406	0.1605\\
0.0449	0.1774\\
0.0496	0.196\\
0.0549	0.2167\\
0.0606	0.2395\\
0.067	0.2646\\
0.0741	0.2925\\
0.0819	0.3232\\
0.0905	0.3572\\
0.1	0.3948\\
0.1105	0.4363\\
0.1221	0.4822\\
0.135	0.5329\\
0.1491	0.589\\
0.1648	0.6509\\
0.1822	0.7194\\
0.2013	0.795\\
0.2225	0.8786\\
0.2459	0.971\\
0.2718	1.0732\\
0.3003	1.186\\
};
\addplot [color=black!40, line width=0.4pt, forget plot]
  table[row sep=crcr]{%
-0.0004	0.005\\
-0.0005	0.0055\\
-0.0005	0.0061\\
-0.0006	0.0067\\
-0.0006	0.0074\\
-0.0007	0.0082\\
-0.0008	0.0091\\
-0.0008	0.01\\
-0.0009	0.0111\\
-0.001	0.0123\\
-0.0011	0.0135\\
-0.0012	0.015\\
-0.0014	0.0165\\
-0.0015	0.0183\\
-0.0017	0.0202\\
-0.0019	0.0223\\
-0.002	0.0247\\
-0.0023	0.0273\\
-0.0025	0.0301\\
-0.0028	0.0333\\
-0.0031	0.0368\\
-0.0034	0.0407\\
-0.0037	0.045\\
-0.0041	0.0497\\
-0.0046	0.0549\\
-0.005	0.0607\\
-0.0056	0.0671\\
-0.0061	0.0741\\
-0.0068	0.0819\\
-0.0075	0.0906\\
-0.0083	0.1001\\
-0.0092	0.1106\\
-0.0101	0.1222\\
-0.0112	0.1351\\
-0.0124	0.1493\\
-0.0137	0.165\\
-0.0151	0.1824\\
-0.0167	0.2015\\
-0.0185	0.2227\\
-0.0204	0.2462\\
-0.0225	0.2721\\
-0.0249	0.3007\\
-0.0275	0.3323\\
-0.0304	0.3672\\
-0.0336	0.4059\\
-0.0372	0.4485\\
-0.0411	0.4957\\
-0.0454	0.5479\\
-0.0502	0.6055\\
-0.0554	0.6692\\
-0.0613	0.7395\\
-0.0677	0.8173\\
-0.0748	0.9033\\
-0.0827	0.9983\\
-0.0914	1.1033\\
};
\addplot [color=black!40, line width=0.4pt, forget plot]
  table[row sep=crcr]{%
-0.002	0.0046\\
-0.0022	0.0051\\
-0.0025	0.0056\\
-0.0027	0.0062\\
-0.003	0.0068\\
-0.0033	0.0075\\
-0.0037	0.0083\\
-0.004	0.0092\\
-0.0045	0.0102\\
-0.0049	0.0113\\
-0.0055	0.0124\\
-0.006	0.0138\\
-0.0067	0.0152\\
-0.0074	0.0168\\
-0.0081	0.0186\\
-0.009	0.0205\\
-0.0099	0.0227\\
-0.011	0.0251\\
-0.0122	0.0277\\
-0.0134	0.0306\\
-0.0148	0.0338\\
-0.0164	0.0374\\
-0.0181	0.0413\\
-0.02	0.0457\\
-0.0221	0.0505\\
-0.0245	0.0558\\
-0.027	0.0616\\
-0.0299	0.0681\\
-0.033	0.0753\\
-0.0365	0.0832\\
-0.0403	0.092\\
-0.0446	0.1016\\
-0.0493	0.1123\\
-0.0545	0.1241\\
-0.0602	0.1372\\
-0.0665	0.1516\\
-0.0735	0.1676\\
-0.0812	0.1852\\
-0.0898	0.2047\\
-0.0992	0.2262\\
-0.1097	0.25\\
-0.1212	0.2763\\
-0.1339	0.3053\\
-0.148	0.3375\\
-0.1636	0.373\\
-0.1808	0.4122\\
-0.1998	0.4555\\
-0.2208	0.5034\\
-0.2441	0.5564\\
-0.2697	0.6149\\
-0.2981	0.6796\\
-0.3294	0.751\\
-0.3641	0.83\\
-0.4024	0.9173\\
-0.4447	1.0138\\
-0.4915	1.1204\\
};
\addplot [color=black!40, line width=0.4pt, forget plot]
  table[row sep=crcr]{%
-0.0034	0.0037\\
-0.0037	0.0041\\
-0.0041	0.0045\\
-0.0046	0.005\\
-0.0051	0.0055\\
-0.0056	0.0061\\
-0.0062	0.0067\\
-0.0068	0.0074\\
-0.0075	0.0082\\
-0.0083	0.009\\
-0.0092	0.01\\
-0.0102	0.0111\\
-0.0112	0.0122\\
-0.0124	0.0135\\
-0.0137	0.0149\\
-0.0152	0.0165\\
-0.0168	0.0182\\
-0.0185	0.0201\\
-0.0205	0.0223\\
-0.0226	0.0246\\
-0.025	0.0272\\
-0.0277	0.03\\
-0.0306	0.0332\\
-0.0338	0.0367\\
-0.0373	0.0406\\
-0.0413	0.0448\\
-0.0456	0.0495\\
-0.0504	0.0547\\
-0.0557	0.0605\\
-0.0615	0.0669\\
-0.068	0.0739\\
-0.0752	0.0817\\
-0.0831	0.0902\\
-0.0918	0.0997\\
-0.1015	0.1102\\
-0.1121	0.1218\\
-0.1239	0.1346\\
-0.137	0.1488\\
-0.1514	0.1644\\
-0.1673	0.1817\\
-0.1849	0.2008\\
-0.2043	0.222\\
-0.2258	0.2453\\
-0.2496	0.2711\\
-0.2758	0.2996\\
-0.3048	0.3311\\
-0.3369	0.366\\
-0.3723	0.4045\\
-0.4115	0.447\\
-0.4548	0.494\\
-0.5026	0.546\\
-0.5554	0.6034\\
-0.6139	0.6668\\
-0.6784	0.737\\
-0.7498	0.8145\\
-0.8286	0.9001\\
-0.9158	0.9948\\
-1.0121	1.0994\\
};
\addplot [color=black!40, line width=0.4pt, forget plot]
  table[row sep=crcr]{%
-0.0044	0.0024\\
-0.0049	0.0026\\
-0.0054	0.0029\\
-0.0059	0.0032\\
-0.0066	0.0036\\
-0.0073	0.0039\\
-0.008	0.0043\\
-0.0089	0.0048\\
-0.0098	0.0053\\
-0.0108	0.0059\\
-0.012	0.0065\\
-0.0132	0.0071\\
-0.0146	0.0079\\
-0.0161	0.0087\\
-0.0178	0.0097\\
-0.0197	0.0107\\
-0.0218	0.0118\\
-0.0241	0.013\\
-0.0266	0.0144\\
-0.0294	0.0159\\
-0.0325	0.0176\\
-0.0359	0.0194\\
-0.0397	0.0215\\
-0.0439	0.0237\\
-0.0485	0.0262\\
-0.0536	0.029\\
-0.0592	0.032\\
-0.0654	0.0354\\
-0.0723	0.0391\\
-0.0799	0.0432\\
-0.0883	0.0478\\
-0.0976	0.0528\\
-0.1079	0.0584\\
-0.1192	0.0645\\
-0.1318	0.0713\\
-0.1456	0.0788\\
-0.1609	0.0871\\
-0.1779	0.0963\\
-0.1966	0.1064\\
-0.2172	0.1176\\
-0.2401	0.1299\\
-0.2653	0.1436\\
-0.2932	0.1587\\
-0.3241	0.1754\\
-0.3582	0.1938\\
-0.3958	0.2142\\
-0.4375	0.2367\\
-0.4835	0.2616\\
-0.5343	0.2892\\
-0.5905	0.3196\\
-0.6526	0.3532\\
-0.7213	0.3903\\
-0.7971	0.4314\\
-0.881	0.4767\\
-0.9736	0.5269\\
-1.076	0.5823\\
-1.1892	0.6435\\
};
\addplot [color=black!40, line width=0.4pt, forget plot]
  table[row sep=crcr]{%
-0.0049	0.0008\\
-0.0055	0.0009\\
-0.006	0.001\\
-0.0067	0.0011\\
-0.0074	0.0012\\
-0.0081	0.0014\\
-0.009	0.0015\\
-0.0099	0.0017\\
-0.011	0.0018\\
-0.0121	0.002\\
-0.0134	0.0022\\
-0.0148	0.0025\\
-0.0164	0.0027\\
-0.0181	0.003\\
-0.02	0.0033\\
-0.0221	0.0037\\
-0.0244	0.0041\\
-0.027	0.0045\\
-0.0298	0.005\\
-0.033	0.0055\\
-0.0364	0.0061\\
-0.0403	0.0067\\
-0.0445	0.0074\\
-0.0492	0.0082\\
-0.0544	0.0091\\
-0.0601	0.01\\
-0.0664	0.0111\\
-0.0734	0.0122\\
-0.0811	0.0135\\
-0.0896	0.015\\
-0.0991	0.0165\\
-0.1095	0.0183\\
-0.121	0.0202\\
-0.1337	0.0223\\
-0.1478	0.0247\\
-0.1633	0.0273\\
-0.1805	0.0301\\
-0.1995	0.0333\\
-0.2205	0.0368\\
-0.2436	0.0407\\
-0.2693	0.0449\\
-0.2976	0.0497\\
-0.3289	0.0549\\
-0.3635	0.0607\\
-0.4017	0.067\\
-0.4439	0.0741\\
-0.4906	0.0819\\
-0.5422	0.0905\\
-0.5993	0.1\\
-0.6623	0.1105\\
-0.7319	0.1221\\
-0.8089	0.135\\
-0.894	0.1492\\
-0.988	0.1649\\
-1.0919	0.1822\\
};
\addplot [color=black!40, line width=0.4pt, forget plot]
  table[row sep=crcr]{%
-0.0049	-0.0008\\
-0.0055	-0.0009\\
-0.006	-0.001\\
-0.0067	-0.0011\\
-0.0074	-0.0012\\
-0.0081	-0.0014\\
-0.009	-0.0015\\
-0.0099	-0.0017\\
-0.011	-0.0018\\
-0.0121	-0.002\\
-0.0134	-0.0022\\
-0.0148	-0.0025\\
-0.0164	-0.0027\\
-0.0181	-0.003\\
-0.02	-0.0033\\
-0.0221	-0.0037\\
-0.0244	-0.0041\\
-0.027	-0.0045\\
-0.0298	-0.005\\
-0.033	-0.0055\\
-0.0364	-0.0061\\
-0.0403	-0.0067\\
-0.0445	-0.0074\\
-0.0492	-0.0082\\
-0.0544	-0.0091\\
-0.0601	-0.01\\
-0.0664	-0.0111\\
-0.0734	-0.0122\\
-0.0811	-0.0135\\
-0.0896	-0.015\\
-0.0991	-0.0165\\
-0.1095	-0.0183\\
-0.121	-0.0202\\
-0.1337	-0.0223\\
-0.1478	-0.0247\\
-0.1633	-0.0273\\
-0.1805	-0.0301\\
-0.1995	-0.0333\\
-0.2205	-0.0368\\
-0.2436	-0.0407\\
-0.2693	-0.0449\\
-0.2976	-0.0497\\
-0.3289	-0.0549\\
-0.3635	-0.0607\\
-0.4017	-0.067\\
-0.4439	-0.0741\\
-0.4906	-0.0819\\
-0.5422	-0.0905\\
-0.5993	-0.1\\
-0.6623	-0.1105\\
-0.7319	-0.1221\\
-0.8089	-0.135\\
-0.894	-0.1492\\
-0.988	-0.1649\\
-1.0919	-0.1822\\
};
\addplot [color=black!40, line width=0.4pt, forget plot]
  table[row sep=crcr]{%
-0.0044	-0.0024\\
-0.0049	-0.0026\\
-0.0054	-0.0029\\
-0.0059	-0.0032\\
-0.0066	-0.0036\\
-0.0073	-0.0039\\
-0.008	-0.0043\\
-0.0089	-0.0048\\
-0.0098	-0.0053\\
-0.0108	-0.0059\\
-0.012	-0.0065\\
-0.0132	-0.0071\\
-0.0146	-0.0079\\
-0.0161	-0.0087\\
-0.0178	-0.0097\\
-0.0197	-0.0107\\
-0.0218	-0.0118\\
-0.0241	-0.013\\
-0.0266	-0.0144\\
-0.0294	-0.0159\\
-0.0325	-0.0176\\
-0.0359	-0.0194\\
-0.0397	-0.0215\\
-0.0439	-0.0237\\
-0.0485	-0.0262\\
-0.0536	-0.029\\
-0.0592	-0.032\\
-0.0654	-0.0354\\
-0.0723	-0.0391\\
-0.0799	-0.0432\\
-0.0883	-0.0478\\
-0.0976	-0.0528\\
-0.1079	-0.0584\\
-0.1192	-0.0645\\
-0.1318	-0.0713\\
-0.1456	-0.0788\\
-0.1609	-0.0871\\
-0.1779	-0.0963\\
-0.1966	-0.1064\\
-0.2172	-0.1176\\
-0.2401	-0.1299\\
-0.2653	-0.1436\\
-0.2932	-0.1587\\
-0.3241	-0.1754\\
-0.3582	-0.1938\\
-0.3958	-0.2142\\
-0.4375	-0.2367\\
-0.4835	-0.2616\\
-0.5343	-0.2892\\
-0.5905	-0.3196\\
-0.6526	-0.3532\\
-0.7213	-0.3903\\
-0.7971	-0.4314\\
-0.881	-0.4767\\
-0.9736	-0.5269\\
-1.076	-0.5823\\
-1.1892	-0.6435\\
};
\addplot [color=black!40, line width=0.4pt, forget plot]
  table[row sep=crcr]{%
-0.0034	-0.0037\\
-0.0037	-0.0041\\
-0.0041	-0.0045\\
-0.0046	-0.005\\
-0.0051	-0.0055\\
-0.0056	-0.0061\\
-0.0062	-0.0067\\
-0.0068	-0.0074\\
-0.0075	-0.0082\\
-0.0083	-0.009\\
-0.0092	-0.01\\
-0.0102	-0.0111\\
-0.0112	-0.0122\\
-0.0124	-0.0135\\
-0.0137	-0.0149\\
-0.0152	-0.0165\\
-0.0168	-0.0182\\
-0.0185	-0.0201\\
-0.0205	-0.0223\\
-0.0226	-0.0246\\
-0.025	-0.0272\\
-0.0277	-0.03\\
-0.0306	-0.0332\\
-0.0338	-0.0367\\
-0.0373	-0.0406\\
-0.0413	-0.0448\\
-0.0456	-0.0495\\
-0.0504	-0.0547\\
-0.0557	-0.0605\\
-0.0615	-0.0669\\
-0.068	-0.0739\\
-0.0752	-0.0817\\
-0.0831	-0.0902\\
-0.0918	-0.0997\\
-0.1015	-0.1102\\
-0.1121	-0.1218\\
-0.1239	-0.1346\\
-0.137	-0.1488\\
-0.1514	-0.1644\\
-0.1673	-0.1817\\
-0.1849	-0.2008\\
-0.2043	-0.222\\
-0.2258	-0.2453\\
-0.2496	-0.2711\\
-0.2758	-0.2996\\
-0.3048	-0.3311\\
-0.3369	-0.366\\
-0.3723	-0.4045\\
-0.4115	-0.447\\
-0.4548	-0.494\\
-0.5026	-0.546\\
-0.5554	-0.6034\\
-0.6139	-0.6668\\
-0.6784	-0.737\\
-0.7498	-0.8145\\
-0.8286	-0.9001\\
-0.9158	-0.9948\\
-1.0121	-1.0994\\
};
\addplot [color=black!40, line width=0.4pt, forget plot]
  table[row sep=crcr]{%
-0.002	-0.0046\\
-0.0022	-0.0051\\
-0.0025	-0.0056\\
-0.0027	-0.0062\\
-0.003	-0.0068\\
-0.0033	-0.0075\\
-0.0037	-0.0083\\
-0.004	-0.0092\\
-0.0045	-0.0102\\
-0.0049	-0.0113\\
-0.0055	-0.0124\\
-0.006	-0.0138\\
-0.0067	-0.0152\\
-0.0074	-0.0168\\
-0.0081	-0.0186\\
-0.009	-0.0205\\
-0.0099	-0.0227\\
-0.011	-0.0251\\
-0.0122	-0.0277\\
-0.0134	-0.0306\\
-0.0148	-0.0338\\
-0.0164	-0.0374\\
-0.0181	-0.0413\\
-0.02	-0.0457\\
-0.0221	-0.0505\\
-0.0245	-0.0558\\
-0.027	-0.0616\\
-0.0299	-0.0681\\
-0.033	-0.0753\\
-0.0365	-0.0832\\
-0.0403	-0.092\\
-0.0446	-0.1016\\
-0.0493	-0.1123\\
-0.0545	-0.1241\\
-0.0602	-0.1372\\
-0.0665	-0.1516\\
-0.0735	-0.1676\\
-0.0812	-0.1852\\
-0.0898	-0.2047\\
-0.0992	-0.2262\\
-0.1097	-0.25\\
-0.1212	-0.2763\\
-0.1339	-0.3053\\
-0.148	-0.3375\\
-0.1636	-0.373\\
-0.1808	-0.4122\\
-0.1998	-0.4555\\
-0.2208	-0.5034\\
-0.2441	-0.5564\\
-0.2697	-0.6149\\
-0.2981	-0.6796\\
-0.3294	-0.751\\
-0.3641	-0.83\\
-0.4024	-0.9173\\
-0.4447	-1.0138\\
-0.4915	-1.1204\\
};
\addplot [color=black!40, line width=0.4pt, forget plot]
  table[row sep=crcr]{%
-0.0004	-0.005\\
-0.0005	-0.0055\\
-0.0005	-0.0061\\
-0.0006	-0.0067\\
-0.0006	-0.0074\\
-0.0007	-0.0082\\
-0.0008	-0.0091\\
-0.0008	-0.01\\
-0.0009	-0.0111\\
-0.001	-0.0123\\
-0.0011	-0.0135\\
-0.0012	-0.015\\
-0.0014	-0.0165\\
-0.0015	-0.0183\\
-0.0017	-0.0202\\
-0.0019	-0.0223\\
-0.002	-0.0247\\
-0.0023	-0.0273\\
-0.0025	-0.0301\\
-0.0028	-0.0333\\
-0.0031	-0.0368\\
-0.0034	-0.0407\\
-0.0037	-0.045\\
-0.0041	-0.0497\\
-0.0046	-0.0549\\
-0.005	-0.0607\\
-0.0056	-0.0671\\
-0.0061	-0.0741\\
-0.0068	-0.0819\\
-0.0075	-0.0906\\
-0.0083	-0.1001\\
-0.0092	-0.1106\\
-0.0101	-0.1222\\
-0.0112	-0.1351\\
-0.0124	-0.1493\\
-0.0137	-0.165\\
-0.0151	-0.1824\\
-0.0167	-0.2015\\
-0.0185	-0.2227\\
-0.0204	-0.2462\\
-0.0225	-0.2721\\
-0.0249	-0.3007\\
-0.0275	-0.3323\\
-0.0304	-0.3672\\
-0.0336	-0.4059\\
-0.0372	-0.4485\\
-0.0411	-0.4957\\
-0.0454	-0.5479\\
-0.0502	-0.6055\\
-0.0554	-0.6692\\
-0.0613	-0.7395\\
-0.0677	-0.8173\\
-0.0748	-0.9033\\
-0.0827	-0.9983\\
-0.0914	-1.1033\\
};
\addplot [color=black!40, line width=0.4pt, forget plot]
  table[row sep=crcr]{%
0.0012	-0.0048\\
0.0014	-0.0054\\
0.0015	-0.0059\\
0.0017	-0.0065\\
0.0018	-0.0072\\
0.002	-0.008\\
0.0022	-0.0088\\
0.0025	-0.0098\\
0.0027	-0.0108\\
0.003	-0.0119\\
0.0033	-0.0132\\
0.0037	-0.0146\\
0.0041	-0.0161\\
0.0045	-0.0178\\
0.005	-0.0197\\
0.0055	-0.0217\\
0.0061	-0.024\\
0.0067	-0.0265\\
0.0074	-0.0293\\
0.0082	-0.0324\\
0.0091	-0.0358\\
0.01	-0.0396\\
0.0111	-0.0437\\
0.0122	-0.0483\\
0.0135	-0.0534\\
0.015	-0.059\\
0.0165	-0.0653\\
0.0183	-0.0721\\
0.0202	-0.0797\\
0.0223	-0.0881\\
0.0247	-0.0974\\
0.0272	-0.1076\\
0.0301	-0.1189\\
0.0333	-0.1314\\
0.0368	-0.1452\\
0.0406	-0.1605\\
0.0449	-0.1774\\
0.0496	-0.196\\
0.0549	-0.2167\\
0.0606	-0.2395\\
0.067	-0.2646\\
0.0741	-0.2925\\
0.0819	-0.3232\\
0.0905	-0.3572\\
0.1	-0.3948\\
0.1105	-0.4363\\
0.1221	-0.4822\\
0.135	-0.5329\\
0.1491	-0.589\\
0.1648	-0.6509\\
0.1822	-0.7194\\
0.2013	-0.795\\
0.2225	-0.8786\\
0.2459	-0.971\\
0.2718	-1.0732\\
0.3003	-1.186\\
};
\addplot [color=black!40, line width=0.4pt, forget plot]
  table[row sep=crcr]{%
0.0027	-0.0042\\
0.003	-0.0046\\
0.0033	-0.0051\\
0.0037	-0.0057\\
0.0041	-0.0062\\
0.0045	-0.0069\\
0.005	-0.0076\\
0.0055	-0.0084\\
0.0061	-0.0093\\
0.0067	-0.0103\\
0.0074	-0.0114\\
0.0082	-0.0126\\
0.0091	-0.0139\\
0.01	-0.0154\\
0.0111	-0.017\\
0.0123	-0.0188\\
0.0135	-0.0207\\
0.015	-0.0229\\
0.0165	-0.0253\\
0.0183	-0.028\\
0.0202	-0.0309\\
0.0223	-0.0342\\
0.0247	-0.0378\\
0.0273	-0.0418\\
0.0301	-0.0461\\
0.0333	-0.051\\
0.0368	-0.0564\\
0.0407	-0.0623\\
0.045	-0.0688\\
0.0497	-0.0761\\
0.0549	-0.0841\\
0.0607	-0.0929\\
0.0671	-0.1027\\
0.0741	-0.1135\\
0.0819	-0.1254\\
0.0906	-0.1386\\
0.1001	-0.1532\\
0.1106	-0.1693\\
0.1222	-0.1871\\
0.1351	-0.2068\\
0.1493	-0.2285\\
0.165	-0.2526\\
0.1824	-0.2791\\
0.2015	-0.3085\\
0.2227	-0.3409\\
0.2462	-0.3768\\
0.2721	-0.4164\\
0.3007	-0.4602\\
0.3323	-0.5086\\
0.3672	-0.5621\\
0.4059	-0.6212\\
0.4486	-0.6866\\
0.4957	-0.7588\\
0.5479	-0.8386\\
0.6055	-0.9268\\
0.6692	-1.0242\\
0.7395	-1.132\\
};
\addplot [color=black!40, line width=0.4pt, forget plot]
  table[row sep=crcr]{%
0.0039	-0.0031\\
0.0044	-0.0034\\
0.0048	-0.0038\\
0.0053	-0.0041\\
0.0059	-0.0046\\
0.0065	-0.0051\\
0.0072	-0.0056\\
0.0079	-0.0062\\
0.0088	-0.0068\\
0.0097	-0.0076\\
0.0107	-0.0083\\
0.0119	-0.0092\\
0.0131	-0.0102\\
0.0145	-0.0113\\
0.016	-0.0125\\
0.0177	-0.0138\\
0.0195	-0.0152\\
0.0216	-0.0168\\
0.0239	-0.0186\\
0.0264	-0.0205\\
0.0292	-0.0227\\
0.0322	-0.0251\\
0.0356	-0.0277\\
0.0394	-0.0306\\
0.0435	-0.0339\\
0.0481	-0.0374\\
0.0531	-0.0413\\
0.0587	-0.0457\\
0.0649	-0.0505\\
0.0717	-0.0558\\
0.0793	-0.0617\\
0.0876	-0.0682\\
0.0968	-0.0753\\
0.107	-0.0833\\
0.1182	-0.092\\
0.1307	-0.1017\\
0.1444	-0.1124\\
0.1596	-0.1242\\
0.1764	-0.1373\\
0.1949	-0.1517\\
0.2154	-0.1677\\
0.2381	-0.1853\\
0.2631	-0.2048\\
0.2908	-0.2263\\
0.3214	-0.2501\\
0.3552	-0.2764\\
0.3925	-0.3055\\
0.4338	-0.3377\\
0.4794	-0.3732\\
0.5299	-0.4124\\
0.5856	-0.4558\\
0.6472	-0.5037\\
0.7152	-0.5567\\
0.7905	-0.6152\\
0.8736	-0.68\\
0.9655	-0.7515\\
1.067	-0.8305\\
1.1792	-0.9178\\
};
\addplot [color=black!40, line width=0.4pt, forget plot]
  table[row sep=crcr]{%
0.0047	-0.0016\\
0.0052	-0.0018\\
0.0058	-0.002\\
0.0064	-0.0022\\
0.0071	-0.0024\\
0.0078	-0.0027\\
0.0086	-0.003\\
0.0095	-0.0033\\
0.0105	-0.0036\\
0.0116	-0.004\\
0.0129	-0.0044\\
0.0142	-0.0049\\
0.0157	-0.0054\\
0.0174	-0.006\\
0.0192	-0.0066\\
0.0212	-0.0073\\
0.0234	-0.008\\
0.0259	-0.0089\\
0.0286	-0.0098\\
0.0316	-0.0109\\
0.0349	-0.012\\
0.0386	-0.0133\\
0.0427	-0.0147\\
0.0472	-0.0162\\
0.0521	-0.0179\\
0.0576	-0.0198\\
0.0637	-0.0219\\
0.0704	-0.0242\\
0.0778	-0.0267\\
0.0859	-0.0295\\
0.095	-0.0326\\
0.105	-0.036\\
0.116	-0.0398\\
0.1282	-0.044\\
0.1417	-0.0486\\
0.1566	-0.0538\\
0.1731	-0.0594\\
0.1913	-0.0657\\
0.2114	-0.0726\\
0.2336	-0.0802\\
0.2582	-0.0886\\
0.2854	-0.098\\
0.3154	-0.1083\\
0.3485	-0.1197\\
0.3852	-0.1322\\
0.4257	-0.1461\\
0.4705	-0.1615\\
0.5199	-0.1785\\
0.5746	-0.1973\\
0.6351	-0.218\\
0.7019	-0.2409\\
0.7757	-0.2663\\
0.8573	-0.2943\\
0.9474	-0.3252\\
1.0471	-0.3595\\
1.1572	-0.3973\\
};
\addplot [color=black!40, line width=0.4pt, forget plot]
  table[row sep=crcr]{%
0.005	-0\\
0.0055	-0\\
0.0061	-0\\
0.0067	-0\\
0.0075	-0\\
0.0082	-0\\
0.0091	-0\\
0.0101	-0\\
0.0111	-0\\
0.0123	-0\\
0.0136	-0\\
0.015	-0\\
0.0166	-0\\
0.0183	-0\\
0.0203	-0\\
0.0224	-0\\
0.0248	-0\\
0.0274	-0\\
0.0302	-0\\
0.0334	-0\\
0.0369	-0\\
0.0408	-0\\
0.0451	-0\\
0.0499	-0\\
0.0551	-0\\
0.0609	-0\\
0.0673	-0\\
0.0744	-0\\
0.0822	-0\\
0.0909	-0\\
0.1004	-0\\
0.111	-0\\
0.1227	-0\\
0.1356	-0\\
0.1498	-0\\
0.1656	-0\\
0.183	-0\\
0.2022	-0\\
0.2235	-0\\
0.247	-0\\
0.273	-0\\
0.3017	-0\\
0.3334	-0\\
0.3685	-0\\
0.4073	-0\\
0.4501	-0\\
0.4974	-0\\
0.5497	-0\\
0.6076	-0\\
0.6714	-0\\
0.7421	-0\\
0.8201	-0\\
0.9064	-0\\
1.0017	-0\\
1.107	-0\\
};
\addplot[-stealth, color=accent1, point meta={sqrt((\thisrow{u})^2+(\thisrow{v})^2)}, point meta min=0, quiver={u=\thisrow{u}, v=\thisrow{v}, scale arrows = 1.45, every arrow/.append style={line width=1pt*\pgfplotspointmetatransformed/1000}}]
 table[row sep=crcr] {%
x	y	u	v\\
-1	-1	-0.09	-0.09\\
-1	-0.894736842105263	-0.09	-0.0805263157894737\\
-1	-0.789473684210526	-0.09	-0.0710526315789474\\
-1	-0.684210526315789	-0.09	-0.0615789473684211\\
-1	-0.578947368421053	-0.09	-0.0521052631578948\\
-1	-0.473684210526316	-0.09	-0.0426315789473684\\
-1	-0.368421052631579	-0.09	-0.0331578947368421\\
-1	-0.263157894736842	-0.09	-0.0236842105263158\\
-1	-0.157894736842105	-0.09	-0.0142105263157895\\
-1	-0.0526315789473684	-0.09	-0.00473684210526316\\
-1	0.0526315789473684	-0.09	0.00473684210526316\\
-1	0.157894736842105	-0.09	0.0142105263157895\\
-1	0.263157894736842	-0.09	0.0236842105263158\\
-1	0.368421052631579	-0.09	0.0331578947368421\\
-1	0.473684210526316	-0.09	0.0426315789473684\\
-1	0.578947368421053	-0.09	0.0521052631578948\\
-1	0.684210526315789	-0.09	0.0615789473684211\\
-1	0.789473684210526	-0.09	0.0710526315789474\\
-1	0.894736842105263	-0.09	0.0805263157894737\\
-1	1	-0.09	0.09\\
-0.894736842105263	-1	-0.0805263157894737	-0.09\\
-0.894736842105263	-0.894736842105263	-0.0805263157894737	-0.0805263157894737\\
-0.894736842105263	-0.789473684210526	-0.0805263157894737	-0.0710526315789474\\
-0.894736842105263	-0.684210526315789	-0.0805263157894737	-0.0615789473684211\\
-0.894736842105263	-0.578947368421053	-0.0805263157894737	-0.0521052631578948\\
-0.894736842105263	-0.473684210526316	-0.0805263157894737	-0.0426315789473684\\
-0.894736842105263	-0.368421052631579	-0.0805263157894737	-0.0331578947368421\\
-0.894736842105263	-0.263157894736842	-0.0805263157894737	-0.0236842105263158\\
-0.894736842105263	-0.157894736842105	-0.0805263157894737	-0.0142105263157895\\
-0.894736842105263	-0.0526315789473684	-0.0805263157894737	-0.00473684210526316\\
-0.894736842105263	0.0526315789473684	-0.0805263157894737	0.00473684210526316\\
-0.894736842105263	0.157894736842105	-0.0805263157894737	0.0142105263157895\\
-0.894736842105263	0.263157894736842	-0.0805263157894737	0.0236842105263158\\
-0.894736842105263	0.368421052631579	-0.0805263157894737	0.0331578947368421\\
-0.894736842105263	0.473684210526316	-0.0805263157894737	0.0426315789473684\\
-0.894736842105263	0.578947368421053	-0.0805263157894737	0.0521052631578948\\
-0.894736842105263	0.684210526315789	-0.0805263157894737	0.0615789473684211\\
-0.894736842105263	0.789473684210526	-0.0805263157894737	0.0710526315789474\\
-0.894736842105263	0.894736842105263	-0.0805263157894737	0.0805263157894737\\
-0.894736842105263	1	-0.0805263157894737	0.09\\
-0.789473684210526	-1	-0.0710526315789474	-0.09\\
-0.789473684210526	-0.894736842105263	-0.0710526315789474	-0.0805263157894737\\
-0.789473684210526	-0.789473684210526	-0.0710526315789474	-0.0710526315789474\\
-0.789473684210526	-0.684210526315789	-0.0710526315789474	-0.0615789473684211\\
-0.789473684210526	-0.578947368421053	-0.0710526315789474	-0.0521052631578948\\
-0.789473684210526	-0.473684210526316	-0.0710526315789474	-0.0426315789473684\\
-0.789473684210526	-0.368421052631579	-0.0710526315789474	-0.0331578947368421\\
-0.789473684210526	-0.263157894736842	-0.0710526315789474	-0.0236842105263158\\
-0.789473684210526	-0.157894736842105	-0.0710526315789474	-0.0142105263157895\\
-0.789473684210526	-0.0526315789473684	-0.0710526315789474	-0.00473684210526316\\
-0.789473684210526	0.0526315789473684	-0.0710526315789474	0.00473684210526316\\
-0.789473684210526	0.157894736842105	-0.0710526315789474	0.0142105263157895\\
-0.789473684210526	0.263157894736842	-0.0710526315789474	0.0236842105263158\\
-0.789473684210526	0.368421052631579	-0.0710526315789474	0.0331578947368421\\
-0.789473684210526	0.473684210526316	-0.0710526315789474	0.0426315789473684\\
-0.789473684210526	0.578947368421053	-0.0710526315789474	0.0521052631578948\\
-0.789473684210526	0.684210526315789	-0.0710526315789474	0.0615789473684211\\
-0.789473684210526	0.789473684210526	-0.0710526315789474	0.0710526315789474\\
-0.789473684210526	0.894736842105263	-0.0710526315789474	0.0805263157894737\\
-0.789473684210526	1	-0.0710526315789474	0.09\\
-0.684210526315789	-1	-0.0615789473684211	-0.09\\
-0.684210526315789	-0.894736842105263	-0.0615789473684211	-0.0805263157894737\\
-0.684210526315789	-0.789473684210526	-0.0615789473684211	-0.0710526315789474\\
-0.684210526315789	-0.684210526315789	-0.0615789473684211	-0.0615789473684211\\
-0.684210526315789	-0.578947368421053	-0.0615789473684211	-0.0521052631578948\\
-0.684210526315789	-0.473684210526316	-0.0615789473684211	-0.0426315789473684\\
-0.684210526315789	-0.368421052631579	-0.0615789473684211	-0.0331578947368421\\
-0.684210526315789	-0.263157894736842	-0.0615789473684211	-0.0236842105263158\\
-0.684210526315789	-0.157894736842105	-0.0615789473684211	-0.0142105263157895\\
-0.684210526315789	-0.0526315789473684	-0.0615789473684211	-0.00473684210526316\\
-0.684210526315789	0.0526315789473684	-0.0615789473684211	0.00473684210526316\\
-0.684210526315789	0.157894736842105	-0.0615789473684211	0.0142105263157895\\
-0.684210526315789	0.263157894736842	-0.0615789473684211	0.0236842105263158\\
-0.684210526315789	0.368421052631579	-0.0615789473684211	0.0331578947368421\\
-0.684210526315789	0.473684210526316	-0.0615789473684211	0.0426315789473684\\
-0.684210526315789	0.578947368421053	-0.0615789473684211	0.0521052631578948\\
-0.684210526315789	0.684210526315789	-0.0615789473684211	0.0615789473684211\\
-0.684210526315789	0.789473684210526	-0.0615789473684211	0.0710526315789474\\
-0.684210526315789	0.894736842105263	-0.0615789473684211	0.0805263157894737\\
-0.684210526315789	1	-0.0615789473684211	0.09\\
-0.578947368421053	-1	-0.0521052631578948	-0.09\\
-0.578947368421053	-0.894736842105263	-0.0521052631578948	-0.0805263157894737\\
-0.578947368421053	-0.789473684210526	-0.0521052631578948	-0.0710526315789474\\
-0.578947368421053	-0.684210526315789	-0.0521052631578948	-0.0615789473684211\\
-0.578947368421053	-0.578947368421053	-0.0521052631578948	-0.0521052631578948\\
-0.578947368421053	-0.473684210526316	-0.0521052631578948	-0.0426315789473684\\
-0.578947368421053	-0.368421052631579	-0.0521052631578948	-0.0331578947368421\\
-0.578947368421053	-0.263157894736842	-0.0521052631578948	-0.0236842105263158\\
-0.578947368421053	-0.157894736842105	-0.0521052631578948	-0.0142105263157895\\
-0.578947368421053	-0.0526315789473684	-0.0521052631578948	-0.00473684210526316\\
-0.578947368421053	0.0526315789473684	-0.0521052631578948	0.00473684210526316\\
-0.578947368421053	0.157894736842105	-0.0521052631578948	0.0142105263157895\\
-0.578947368421053	0.263157894736842	-0.0521052631578948	0.0236842105263158\\
-0.578947368421053	0.368421052631579	-0.0521052631578948	0.0331578947368421\\
-0.578947368421053	0.473684210526316	-0.0521052631578948	0.0426315789473684\\
-0.578947368421053	0.578947368421053	-0.0521052631578948	0.0521052631578948\\
-0.578947368421053	0.684210526315789	-0.0521052631578948	0.0615789473684211\\
-0.578947368421053	0.789473684210526	-0.0521052631578948	0.0710526315789474\\
-0.578947368421053	0.894736842105263	-0.0521052631578948	0.0805263157894737\\
-0.578947368421053	1	-0.0521052631578948	0.09\\
-0.473684210526316	-1	-0.0426315789473684	-0.09\\
-0.473684210526316	-0.894736842105263	-0.0426315789473684	-0.0805263157894737\\
-0.473684210526316	-0.789473684210526	-0.0426315789473684	-0.0710526315789474\\
-0.473684210526316	-0.684210526315789	-0.0426315789473684	-0.0615789473684211\\
-0.473684210526316	-0.578947368421053	-0.0426315789473684	-0.0521052631578948\\
-0.473684210526316	-0.473684210526316	-0.0426315789473684	-0.0426315789473684\\
-0.473684210526316	-0.368421052631579	-0.0426315789473684	-0.0331578947368421\\
-0.473684210526316	-0.263157894736842	-0.0426315789473684	-0.0236842105263158\\
-0.473684210526316	-0.157894736842105	-0.0426315789473684	-0.0142105263157895\\
-0.473684210526316	-0.0526315789473684	-0.0426315789473684	-0.00473684210526316\\
-0.473684210526316	0.0526315789473684	-0.0426315789473684	0.00473684210526316\\
-0.473684210526316	0.157894736842105	-0.0426315789473684	0.0142105263157895\\
-0.473684210526316	0.263157894736842	-0.0426315789473684	0.0236842105263158\\
-0.473684210526316	0.368421052631579	-0.0426315789473684	0.0331578947368421\\
-0.473684210526316	0.473684210526316	-0.0426315789473684	0.0426315789473684\\
-0.473684210526316	0.578947368421053	-0.0426315789473684	0.0521052631578948\\
-0.473684210526316	0.684210526315789	-0.0426315789473684	0.0615789473684211\\
-0.473684210526316	0.789473684210526	-0.0426315789473684	0.0710526315789474\\
-0.473684210526316	0.894736842105263	-0.0426315789473684	0.0805263157894737\\
-0.473684210526316	1	-0.0426315789473684	0.09\\
-0.368421052631579	-1	-0.0331578947368421	-0.09\\
-0.368421052631579	-0.894736842105263	-0.0331578947368421	-0.0805263157894737\\
-0.368421052631579	-0.789473684210526	-0.0331578947368421	-0.0710526315789474\\
-0.368421052631579	-0.684210526315789	-0.0331578947368421	-0.0615789473684211\\
-0.368421052631579	-0.578947368421053	-0.0331578947368421	-0.0521052631578948\\
-0.368421052631579	-0.473684210526316	-0.0331578947368421	-0.0426315789473684\\
-0.368421052631579	-0.368421052631579	-0.0331578947368421	-0.0331578947368421\\
-0.368421052631579	-0.263157894736842	-0.0331578947368421	-0.0236842105263158\\
-0.368421052631579	-0.157894736842105	-0.0331578947368421	-0.0142105263157895\\
-0.368421052631579	-0.0526315789473684	-0.0331578947368421	-0.00473684210526316\\
-0.368421052631579	0.0526315789473684	-0.0331578947368421	0.00473684210526316\\
-0.368421052631579	0.157894736842105	-0.0331578947368421	0.0142105263157895\\
-0.368421052631579	0.263157894736842	-0.0331578947368421	0.0236842105263158\\
-0.368421052631579	0.368421052631579	-0.0331578947368421	0.0331578947368421\\
-0.368421052631579	0.473684210526316	-0.0331578947368421	0.0426315789473684\\
-0.368421052631579	0.578947368421053	-0.0331578947368421	0.0521052631578948\\
-0.368421052631579	0.684210526315789	-0.0331578947368421	0.0615789473684211\\
-0.368421052631579	0.789473684210526	-0.0331578947368421	0.0710526315789474\\
-0.368421052631579	0.894736842105263	-0.0331578947368421	0.0805263157894737\\
-0.368421052631579	1	-0.0331578947368421	0.09\\
-0.263157894736842	-1	-0.0236842105263158	-0.09\\
-0.263157894736842	-0.894736842105263	-0.0236842105263158	-0.0805263157894737\\
-0.263157894736842	-0.789473684210526	-0.0236842105263158	-0.0710526315789474\\
-0.263157894736842	-0.684210526315789	-0.0236842105263158	-0.0615789473684211\\
-0.263157894736842	-0.578947368421053	-0.0236842105263158	-0.0521052631578948\\
-0.263157894736842	-0.473684210526316	-0.0236842105263158	-0.0426315789473684\\
-0.263157894736842	-0.368421052631579	-0.0236842105263158	-0.0331578947368421\\
-0.263157894736842	-0.263157894736842	-0.0236842105263158	-0.0236842105263158\\
-0.263157894736842	-0.157894736842105	-0.0236842105263158	-0.0142105263157895\\
-0.263157894736842	-0.0526315789473684	-0.0236842105263158	-0.00473684210526316\\
-0.263157894736842	0.0526315789473684	-0.0236842105263158	0.00473684210526316\\
-0.263157894736842	0.157894736842105	-0.0236842105263158	0.0142105263157895\\
-0.263157894736842	0.263157894736842	-0.0236842105263158	0.0236842105263158\\
-0.263157894736842	0.368421052631579	-0.0236842105263158	0.0331578947368421\\
-0.263157894736842	0.473684210526316	-0.0236842105263158	0.0426315789473684\\
-0.263157894736842	0.578947368421053	-0.0236842105263158	0.0521052631578948\\
-0.263157894736842	0.684210526315789	-0.0236842105263158	0.0615789473684211\\
-0.263157894736842	0.789473684210526	-0.0236842105263158	0.0710526315789474\\
-0.263157894736842	0.894736842105263	-0.0236842105263158	0.0805263157894737\\
-0.263157894736842	1	-0.0236842105263158	0.09\\
-0.157894736842105	-1	-0.0142105263157895	-0.09\\
-0.157894736842105	-0.894736842105263	-0.0142105263157895	-0.0805263157894737\\
-0.157894736842105	-0.789473684210526	-0.0142105263157895	-0.0710526315789474\\
-0.157894736842105	-0.684210526315789	-0.0142105263157895	-0.0615789473684211\\
-0.157894736842105	-0.578947368421053	-0.0142105263157895	-0.0521052631578948\\
-0.157894736842105	-0.473684210526316	-0.0142105263157895	-0.0426315789473684\\
-0.157894736842105	-0.368421052631579	-0.0142105263157895	-0.0331578947368421\\
-0.157894736842105	-0.263157894736842	-0.0142105263157895	-0.0236842105263158\\
-0.157894736842105	-0.157894736842105	-0.0142105263157895	-0.0142105263157895\\
-0.157894736842105	-0.0526315789473684	-0.0142105263157895	-0.00473684210526316\\
-0.157894736842105	0.0526315789473684	-0.0142105263157895	0.00473684210526316\\
-0.157894736842105	0.157894736842105	-0.0142105263157895	0.0142105263157895\\
-0.157894736842105	0.263157894736842	-0.0142105263157895	0.0236842105263158\\
-0.157894736842105	0.368421052631579	-0.0142105263157895	0.0331578947368421\\
-0.157894736842105	0.473684210526316	-0.0142105263157895	0.0426315789473684\\
-0.157894736842105	0.578947368421053	-0.0142105263157895	0.0521052631578948\\
-0.157894736842105	0.684210526315789	-0.0142105263157895	0.0615789473684211\\
-0.157894736842105	0.789473684210526	-0.0142105263157895	0.0710526315789474\\
-0.157894736842105	0.894736842105263	-0.0142105263157895	0.0805263157894737\\
-0.157894736842105	1	-0.0142105263157895	0.09\\
-0.0526315789473684	-1	-0.00473684210526316	-0.09\\
-0.0526315789473684	-0.894736842105263	-0.00473684210526316	-0.0805263157894737\\
-0.0526315789473684	-0.789473684210526	-0.00473684210526316	-0.0710526315789474\\
-0.0526315789473684	-0.684210526315789	-0.00473684210526316	-0.0615789473684211\\
-0.0526315789473684	-0.578947368421053	-0.00473684210526316	-0.0521052631578948\\
-0.0526315789473684	-0.473684210526316	-0.00473684210526316	-0.0426315789473684\\
-0.0526315789473684	-0.368421052631579	-0.00473684210526316	-0.0331578947368421\\
-0.0526315789473684	-0.263157894736842	-0.00473684210526316	-0.0236842105263158\\
-0.0526315789473684	-0.157894736842105	-0.00473684210526316	-0.0142105263157895\\
-0.0526315789473684	-0.0526315789473684	-0.00473684210526316	-0.00473684210526316\\
-0.0526315789473684	0.0526315789473684	-0.00473684210526316	0.00473684210526316\\
-0.0526315789473684	0.157894736842105	-0.00473684210526316	0.0142105263157895\\
-0.0526315789473684	0.263157894736842	-0.00473684210526316	0.0236842105263158\\
-0.0526315789473684	0.368421052631579	-0.00473684210526316	0.0331578947368421\\
-0.0526315789473684	0.473684210526316	-0.00473684210526316	0.0426315789473684\\
-0.0526315789473684	0.578947368421053	-0.00473684210526316	0.0521052631578948\\
-0.0526315789473684	0.684210526315789	-0.00473684210526316	0.0615789473684211\\
-0.0526315789473684	0.789473684210526	-0.00473684210526316	0.0710526315789474\\
-0.0526315789473684	0.894736842105263	-0.00473684210526316	0.0805263157894737\\
-0.0526315789473684	1	-0.00473684210526316	0.09\\
0.0526315789473684	-1	0.00473684210526316	-0.09\\
0.0526315789473684	-0.894736842105263	0.00473684210526316	-0.0805263157894737\\
0.0526315789473684	-0.789473684210526	0.00473684210526316	-0.0710526315789474\\
0.0526315789473684	-0.684210526315789	0.00473684210526316	-0.0615789473684211\\
0.0526315789473684	-0.578947368421053	0.00473684210526316	-0.0521052631578948\\
0.0526315789473684	-0.473684210526316	0.00473684210526316	-0.0426315789473684\\
0.0526315789473684	-0.368421052631579	0.00473684210526316	-0.0331578947368421\\
0.0526315789473684	-0.263157894736842	0.00473684210526316	-0.0236842105263158\\
0.0526315789473684	-0.157894736842105	0.00473684210526316	-0.0142105263157895\\
0.0526315789473684	-0.0526315789473684	0.00473684210526316	-0.00473684210526316\\
0.0526315789473684	0.0526315789473684	0.00473684210526316	0.00473684210526316\\
0.0526315789473684	0.157894736842105	0.00473684210526316	0.0142105263157895\\
0.0526315789473684	0.263157894736842	0.00473684210526316	0.0236842105263158\\
0.0526315789473684	0.368421052631579	0.00473684210526316	0.0331578947368421\\
0.0526315789473684	0.473684210526316	0.00473684210526316	0.0426315789473684\\
0.0526315789473684	0.578947368421053	0.00473684210526316	0.0521052631578948\\
0.0526315789473684	0.684210526315789	0.00473684210526316	0.0615789473684211\\
0.0526315789473684	0.789473684210526	0.00473684210526316	0.0710526315789474\\
0.0526315789473684	0.894736842105263	0.00473684210526316	0.0805263157894737\\
0.0526315789473684	1	0.00473684210526316	0.09\\
0.157894736842105	-1	0.0142105263157895	-0.09\\
0.157894736842105	-0.894736842105263	0.0142105263157895	-0.0805263157894737\\
0.157894736842105	-0.789473684210526	0.0142105263157895	-0.0710526315789474\\
0.157894736842105	-0.684210526315789	0.0142105263157895	-0.0615789473684211\\
0.157894736842105	-0.578947368421053	0.0142105263157895	-0.0521052631578948\\
0.157894736842105	-0.473684210526316	0.0142105263157895	-0.0426315789473684\\
0.157894736842105	-0.368421052631579	0.0142105263157895	-0.0331578947368421\\
0.157894736842105	-0.263157894736842	0.0142105263157895	-0.0236842105263158\\
0.157894736842105	-0.157894736842105	0.0142105263157895	-0.0142105263157895\\
0.157894736842105	-0.0526315789473684	0.0142105263157895	-0.00473684210526316\\
0.157894736842105	0.0526315789473684	0.0142105263157895	0.00473684210526316\\
0.157894736842105	0.157894736842105	0.0142105263157895	0.0142105263157895\\
0.157894736842105	0.263157894736842	0.0142105263157895	0.0236842105263158\\
0.157894736842105	0.368421052631579	0.0142105263157895	0.0331578947368421\\
0.157894736842105	0.473684210526316	0.0142105263157895	0.0426315789473684\\
0.157894736842105	0.578947368421053	0.0142105263157895	0.0521052631578948\\
0.157894736842105	0.684210526315789	0.0142105263157895	0.0615789473684211\\
0.157894736842105	0.789473684210526	0.0142105263157895	0.0710526315789474\\
0.157894736842105	0.894736842105263	0.0142105263157895	0.0805263157894737\\
0.157894736842105	1	0.0142105263157895	0.09\\
0.263157894736842	-1	0.0236842105263158	-0.09\\
0.263157894736842	-0.894736842105263	0.0236842105263158	-0.0805263157894737\\
0.263157894736842	-0.789473684210526	0.0236842105263158	-0.0710526315789474\\
0.263157894736842	-0.684210526315789	0.0236842105263158	-0.0615789473684211\\
0.263157894736842	-0.578947368421053	0.0236842105263158	-0.0521052631578948\\
0.263157894736842	-0.473684210526316	0.0236842105263158	-0.0426315789473684\\
0.263157894736842	-0.368421052631579	0.0236842105263158	-0.0331578947368421\\
0.263157894736842	-0.263157894736842	0.0236842105263158	-0.0236842105263158\\
0.263157894736842	-0.157894736842105	0.0236842105263158	-0.0142105263157895\\
0.263157894736842	-0.0526315789473684	0.0236842105263158	-0.00473684210526316\\
0.263157894736842	0.0526315789473684	0.0236842105263158	0.00473684210526316\\
0.263157894736842	0.157894736842105	0.0236842105263158	0.0142105263157895\\
0.263157894736842	0.263157894736842	0.0236842105263158	0.0236842105263158\\
0.263157894736842	0.368421052631579	0.0236842105263158	0.0331578947368421\\
0.263157894736842	0.473684210526316	0.0236842105263158	0.0426315789473684\\
0.263157894736842	0.578947368421053	0.0236842105263158	0.0521052631578948\\
0.263157894736842	0.684210526315789	0.0236842105263158	0.0615789473684211\\
0.263157894736842	0.789473684210526	0.0236842105263158	0.0710526315789474\\
0.263157894736842	0.894736842105263	0.0236842105263158	0.0805263157894737\\
0.263157894736842	1	0.0236842105263158	0.09\\
0.368421052631579	-1	0.0331578947368421	-0.09\\
0.368421052631579	-0.894736842105263	0.0331578947368421	-0.0805263157894737\\
0.368421052631579	-0.789473684210526	0.0331578947368421	-0.0710526315789474\\
0.368421052631579	-0.684210526315789	0.0331578947368421	-0.0615789473684211\\
0.368421052631579	-0.578947368421053	0.0331578947368421	-0.0521052631578948\\
0.368421052631579	-0.473684210526316	0.0331578947368421	-0.0426315789473684\\
0.368421052631579	-0.368421052631579	0.0331578947368421	-0.0331578947368421\\
0.368421052631579	-0.263157894736842	0.0331578947368421	-0.0236842105263158\\
0.368421052631579	-0.157894736842105	0.0331578947368421	-0.0142105263157895\\
0.368421052631579	-0.0526315789473684	0.0331578947368421	-0.00473684210526316\\
0.368421052631579	0.0526315789473684	0.0331578947368421	0.00473684210526316\\
0.368421052631579	0.157894736842105	0.0331578947368421	0.0142105263157895\\
0.368421052631579	0.263157894736842	0.0331578947368421	0.0236842105263158\\
0.368421052631579	0.368421052631579	0.0331578947368421	0.0331578947368421\\
0.368421052631579	0.473684210526316	0.0331578947368421	0.0426315789473684\\
0.368421052631579	0.578947368421053	0.0331578947368421	0.0521052631578948\\
0.368421052631579	0.684210526315789	0.0331578947368421	0.0615789473684211\\
0.368421052631579	0.789473684210526	0.0331578947368421	0.0710526315789474\\
0.368421052631579	0.894736842105263	0.0331578947368421	0.0805263157894737\\
0.368421052631579	1	0.0331578947368421	0.09\\
0.473684210526316	-1	0.0426315789473684	-0.09\\
0.473684210526316	-0.894736842105263	0.0426315789473684	-0.0805263157894737\\
0.473684210526316	-0.789473684210526	0.0426315789473684	-0.0710526315789474\\
0.473684210526316	-0.684210526315789	0.0426315789473684	-0.0615789473684211\\
0.473684210526316	-0.578947368421053	0.0426315789473684	-0.0521052631578948\\
0.473684210526316	-0.473684210526316	0.0426315789473684	-0.0426315789473684\\
0.473684210526316	-0.368421052631579	0.0426315789473684	-0.0331578947368421\\
0.473684210526316	-0.263157894736842	0.0426315789473684	-0.0236842105263158\\
0.473684210526316	-0.157894736842105	0.0426315789473684	-0.0142105263157895\\
0.473684210526316	-0.0526315789473684	0.0426315789473684	-0.00473684210526316\\
0.473684210526316	0.0526315789473684	0.0426315789473684	0.00473684210526316\\
0.473684210526316	0.157894736842105	0.0426315789473684	0.0142105263157895\\
0.473684210526316	0.263157894736842	0.0426315789473684	0.0236842105263158\\
0.473684210526316	0.368421052631579	0.0426315789473684	0.0331578947368421\\
0.473684210526316	0.473684210526316	0.0426315789473684	0.0426315789473684\\
0.473684210526316	0.578947368421053	0.0426315789473684	0.0521052631578948\\
0.473684210526316	0.684210526315789	0.0426315789473684	0.0615789473684211\\
0.473684210526316	0.789473684210526	0.0426315789473684	0.0710526315789474\\
0.473684210526316	0.894736842105263	0.0426315789473684	0.0805263157894737\\
0.473684210526316	1	0.0426315789473684	0.09\\
0.578947368421053	-1	0.0521052631578948	-0.09\\
0.578947368421053	-0.894736842105263	0.0521052631578948	-0.0805263157894737\\
0.578947368421053	-0.789473684210526	0.0521052631578948	-0.0710526315789474\\
0.578947368421053	-0.684210526315789	0.0521052631578948	-0.0615789473684211\\
0.578947368421053	-0.578947368421053	0.0521052631578948	-0.0521052631578948\\
0.578947368421053	-0.473684210526316	0.0521052631578948	-0.0426315789473684\\
0.578947368421053	-0.368421052631579	0.0521052631578948	-0.0331578947368421\\
0.578947368421053	-0.263157894736842	0.0521052631578948	-0.0236842105263158\\
0.578947368421053	-0.157894736842105	0.0521052631578948	-0.0142105263157895\\
0.578947368421053	-0.0526315789473684	0.0521052631578948	-0.00473684210526316\\
0.578947368421053	0.0526315789473684	0.0521052631578948	0.00473684210526316\\
0.578947368421053	0.157894736842105	0.0521052631578948	0.0142105263157895\\
0.578947368421053	0.263157894736842	0.0521052631578948	0.0236842105263158\\
0.578947368421053	0.368421052631579	0.0521052631578948	0.0331578947368421\\
0.578947368421053	0.473684210526316	0.0521052631578948	0.0426315789473684\\
0.578947368421053	0.578947368421053	0.0521052631578948	0.0521052631578948\\
0.578947368421053	0.684210526315789	0.0521052631578948	0.0615789473684211\\
0.578947368421053	0.789473684210526	0.0521052631578948	0.0710526315789474\\
0.578947368421053	0.894736842105263	0.0521052631578948	0.0805263157894737\\
0.578947368421053	1	0.0521052631578948	0.09\\
0.684210526315789	-1	0.0615789473684211	-0.09\\
0.684210526315789	-0.894736842105263	0.0615789473684211	-0.0805263157894737\\
0.684210526315789	-0.789473684210526	0.0615789473684211	-0.0710526315789474\\
0.684210526315789	-0.684210526315789	0.0615789473684211	-0.0615789473684211\\
0.684210526315789	-0.578947368421053	0.0615789473684211	-0.0521052631578948\\
0.684210526315789	-0.473684210526316	0.0615789473684211	-0.0426315789473684\\
0.684210526315789	-0.368421052631579	0.0615789473684211	-0.0331578947368421\\
0.684210526315789	-0.263157894736842	0.0615789473684211	-0.0236842105263158\\
0.684210526315789	-0.157894736842105	0.0615789473684211	-0.0142105263157895\\
0.684210526315789	-0.0526315789473684	0.0615789473684211	-0.00473684210526316\\
0.684210526315789	0.0526315789473684	0.0615789473684211	0.00473684210526316\\
0.684210526315789	0.157894736842105	0.0615789473684211	0.0142105263157895\\
0.684210526315789	0.263157894736842	0.0615789473684211	0.0236842105263158\\
0.684210526315789	0.368421052631579	0.0615789473684211	0.0331578947368421\\
0.684210526315789	0.473684210526316	0.0615789473684211	0.0426315789473684\\
0.684210526315789	0.578947368421053	0.0615789473684211	0.0521052631578948\\
0.684210526315789	0.684210526315789	0.0615789473684211	0.0615789473684211\\
0.684210526315789	0.789473684210526	0.0615789473684211	0.0710526315789474\\
0.684210526315789	0.894736842105263	0.0615789473684211	0.0805263157894737\\
0.684210526315789	1	0.0615789473684211	0.09\\
0.789473684210526	-1	0.0710526315789474	-0.09\\
0.789473684210526	-0.894736842105263	0.0710526315789474	-0.0805263157894737\\
0.789473684210526	-0.789473684210526	0.0710526315789474	-0.0710526315789474\\
0.789473684210526	-0.684210526315789	0.0710526315789474	-0.0615789473684211\\
0.789473684210526	-0.578947368421053	0.0710526315789474	-0.0521052631578948\\
0.789473684210526	-0.473684210526316	0.0710526315789474	-0.0426315789473684\\
0.789473684210526	-0.368421052631579	0.0710526315789474	-0.0331578947368421\\
0.789473684210526	-0.263157894736842	0.0710526315789474	-0.0236842105263158\\
0.789473684210526	-0.157894736842105	0.0710526315789474	-0.0142105263157895\\
0.789473684210526	-0.0526315789473684	0.0710526315789474	-0.00473684210526316\\
0.789473684210526	0.0526315789473684	0.0710526315789474	0.00473684210526316\\
0.789473684210526	0.157894736842105	0.0710526315789474	0.0142105263157895\\
0.789473684210526	0.263157894736842	0.0710526315789474	0.0236842105263158\\
0.789473684210526	0.368421052631579	0.0710526315789474	0.0331578947368421\\
0.789473684210526	0.473684210526316	0.0710526315789474	0.0426315789473684\\
0.789473684210526	0.578947368421053	0.0710526315789474	0.0521052631578948\\
0.789473684210526	0.684210526315789	0.0710526315789474	0.0615789473684211\\
0.789473684210526	0.789473684210526	0.0710526315789474	0.0710526315789474\\
0.789473684210526	0.894736842105263	0.0710526315789474	0.0805263157894737\\
0.789473684210526	1	0.0710526315789474	0.09\\
0.894736842105263	-1	0.0805263157894737	-0.09\\
0.894736842105263	-0.894736842105263	0.0805263157894737	-0.0805263157894737\\
0.894736842105263	-0.789473684210526	0.0805263157894737	-0.0710526315789474\\
0.894736842105263	-0.684210526315789	0.0805263157894737	-0.0615789473684211\\
0.894736842105263	-0.578947368421053	0.0805263157894737	-0.0521052631578948\\
0.894736842105263	-0.473684210526316	0.0805263157894737	-0.0426315789473684\\
0.894736842105263	-0.368421052631579	0.0805263157894737	-0.0331578947368421\\
0.894736842105263	-0.263157894736842	0.0805263157894737	-0.0236842105263158\\
0.894736842105263	-0.157894736842105	0.0805263157894737	-0.0142105263157895\\
0.894736842105263	-0.0526315789473684	0.0805263157894737	-0.00473684210526316\\
0.894736842105263	0.0526315789473684	0.0805263157894737	0.00473684210526316\\
0.894736842105263	0.157894736842105	0.0805263157894737	0.0142105263157895\\
0.894736842105263	0.263157894736842	0.0805263157894737	0.0236842105263158\\
0.894736842105263	0.368421052631579	0.0805263157894737	0.0331578947368421\\
0.894736842105263	0.473684210526316	0.0805263157894737	0.0426315789473684\\
0.894736842105263	0.578947368421053	0.0805263157894737	0.0521052631578948\\
0.894736842105263	0.684210526315789	0.0805263157894737	0.0615789473684211\\
0.894736842105263	0.789473684210526	0.0805263157894737	0.0710526315789474\\
0.894736842105263	0.894736842105263	0.0805263157894737	0.0805263157894737\\
0.894736842105263	1	0.0805263157894737	0.09\\
1	-1	0.09	-0.09\\
1	-0.894736842105263	0.09	-0.0805263157894737\\
1	-0.789473684210526	0.09	-0.0710526315789474\\
1	-0.684210526315789	0.09	-0.0615789473684211\\
1	-0.578947368421053	0.09	-0.0521052631578948\\
1	-0.473684210526316	0.09	-0.0426315789473684\\
1	-0.368421052631579	0.09	-0.0331578947368421\\
1	-0.263157894736842	0.09	-0.0236842105263158\\
1	-0.157894736842105	0.09	-0.0142105263157895\\
1	-0.0526315789473684	0.09	-0.00473684210526316\\
1	0.0526315789473684	0.09	0.00473684210526316\\
1	0.157894736842105	0.09	0.0142105263157895\\
1	0.263157894736842	0.09	0.0236842105263158\\
1	0.368421052631579	0.09	0.0331578947368421\\
1	0.473684210526316	0.09	0.0426315789473684\\
1	0.578947368421053	0.09	0.0521052631578948\\
1	0.684210526315789	0.09	0.0615789473684211\\
1	0.789473684210526	0.09	0.0710526315789474\\
1	0.894736842105263	0.09	0.0805263157894737\\
1	1	0.09	0.09\\
};
\end{axis}

\begin{axis}[%
width=2.367in,
height=2.367in,
at={(3.181in,3.158in)},
scale only axis,
xmin=-1,
xmax=1,
ymin=-1,
ymax=1,
axis background/.style={fill=white},
%title style={font=\bfseries},
title={$\quati\quad \qty(y\pdv{}{x} - x\pdv{}{y})$},
axis lines=box,
]
\addplot [color=black!40, line width=0.4pt, forget plot]
  table[row sep=crcr]{%
0	0.1\\
0.01	0.0995\\
0.0199	0.098\\
0.0296	0.0955\\
0.0389	0.0921\\
0.0479	0.0878\\
0.0565	0.0825\\
0.0644	0.0765\\
0.0717	0.0697\\
0.0783	0.0622\\
0.0841	0.054\\
0.0891	0.0454\\
0.0932	0.0362\\
0.0964	0.0267\\
0.0985	0.017\\
0.0997	0.0071\\
0.1	-0.0029\\
0.0992	-0.0129\\
0.0974	-0.0227\\
0.0946	-0.0323\\
0.0909	-0.0416\\
0.0863	-0.0505\\
0.0808	-0.0589\\
0.0746	-0.0666\\
0.0675	-0.0737\\
0.0598	-0.0801\\
0.0516	-0.0857\\
0.0427	-0.0904\\
0.0335	-0.0942\\
0.0239	-0.0971\\
0.0141	-0.099\\
0.0042	-0.0999\\
-0.0058	-0.0998\\
-0.0158	-0.0987\\
-0.0256	-0.0967\\
-0.0351	-0.0936\\
-0.0443	-0.0897\\
-0.053	-0.0848\\
-0.0612	-0.0791\\
-0.0688	-0.0726\\
-0.0757	-0.0654\\
-0.0818	-0.0575\\
-0.0872	-0.049\\
-0.0916	-0.0401\\
-0.0952	-0.0307\\
-0.0978	-0.0211\\
-0.0994	-0.0112\\
-0.1	-0.0012\\
-0.0996	0.0087\\
-0.0982	0.0187\\
-0.0959	0.0284\\
-0.0926	0.0378\\
-0.0883	0.0469\\
-0.0832	0.0554\\
-0.0773	0.0635\\
-0.0706	0.0709\\
-0.0631	0.0776\\
-0.0551	0.0835\\
-0.0465	0.0886\\
-0.0374	0.0927\\
-0.0279	0.096\\
-0.0182	0.0983\\
-0.0083	0.0997\\
0.0017	0.1\\
0.0117	0.0993\\
0.0215	0.0977\\
0.0312	0.095\\
0.0405	0.0914\\
0.0494	0.0869\\
0.0578	0.0816\\
0.0657	0.0754\\
0.0729	0.0685\\
0.0794	0.0608\\
0.085	0.0526\\
0.0899	0.0439\\
0.0938	0.0347\\
0.0968	0.0251\\
0.0988	0.0153\\
0.0999	0.0054\\
0.0999	-0.0046\\
0.0989	-0.0146\\
};
\addplot [color=black!40, line width=0.4pt, forget plot]
  table[row sep=crcr]{%
0	0.246\\
0.0246	0.2448\\
0.0489	0.2411\\
0.0727	0.235\\
0.0958	0.2266\\
0.118	0.2159\\
0.1389	0.2031\\
0.1585	0.1882\\
0.1765	0.1714\\
0.1927	0.1529\\
0.207	0.1329\\
0.2193	0.1116\\
0.2293	0.0891\\
0.2371	0.0658\\
0.2424	0.0418\\
0.2454	0.0174\\
0.2459	-0.0072\\
0.244	-0.0317\\
0.2396	-0.0559\\
0.2328	-0.0795\\
0.2237	-0.1024\\
0.2124	-0.1242\\
0.1989	-0.1448\\
0.1835	-0.1639\\
0.1662	-0.1814\\
0.1472	-0.1971\\
0.1268	-0.2108\\
0.1051	-0.2224\\
0.0824	-0.2318\\
0.0589	-0.2389\\
0.0347	-0.2436\\
0.0102	-0.2458\\
-0.0144	-0.2456\\
-0.0388	-0.2429\\
-0.0629	-0.2379\\
-0.0863	-0.2304\\
-0.1089	-0.2206\\
-0.1304	-0.2087\\
-0.1505	-0.1946\\
-0.1692	-0.1786\\
-0.1862	-0.1608\\
-0.2013	-0.1414\\
-0.2144	-0.1206\\
-0.2254	-0.0986\\
-0.2341	-0.0756\\
-0.2405	-0.0519\\
-0.2445	-0.0276\\
-0.246	-0.003\\
-0.2451	0.0215\\
-0.2417	0.0459\\
-0.2359	0.0698\\
-0.2278	0.093\\
-0.2174	0.1153\\
-0.2048	0.1364\\
-0.1901	0.1561\\
-0.1736	0.1743\\
-0.1553	0.1908\\
-0.1355	0.2054\\
-0.1143	0.2179\\
-0.092	0.2282\\
-0.0687	0.2362\\
-0.0448	0.2419\\
-0.0204	0.2452\\
0.0041	0.246\\
0.0287	0.2443\\
0.0529	0.2403\\
0.0766	0.2338\\
0.0996	0.225\\
0.1216	0.2139\\
0.1423	0.2007\\
0.1616	0.1855\\
0.1793	0.1684\\
0.1953	0.1497\\
0.2092	0.1294\\
0.2211	0.1079\\
0.2308	0.0853\\
0.2381	0.0618\\
0.2431	0.0377\\
0.2457	0.0133\\
0.2458	-0.0113\\
0.2434	-0.0358\\
};
\addplot [color=black!40, line width=0.4pt, forget plot]
  table[row sep=crcr]{%
0	0.392\\
0.0391	0.3901\\
0.0779	0.3842\\
0.1159	0.3745\\
0.1527	0.3611\\
0.188	0.3441\\
0.2214	0.3236\\
0.2526	0.2999\\
0.2812	0.2731\\
0.3071	0.2437\\
0.3299	0.2118\\
0.3494	0.1778\\
0.3654	0.1421\\
0.3778	0.1049\\
0.3863	0.0666\\
0.3911	0.0277\\
0.3919	-0.0114\\
0.3888	-0.0505\\
0.3818	-0.0891\\
0.371	-0.1267\\
0.3565	-0.1631\\
0.3384	-0.1979\\
0.317	-0.2307\\
0.2924	-0.2612\\
0.2648	-0.2891\\
0.2346	-0.3141\\
0.2021	-0.3359\\
0.1676	-0.3544\\
0.1313	-0.3694\\
0.0938	-0.3807\\
0.0553	-0.3881\\
0.0163	-0.3917\\
-0.0229	-0.3914\\
-0.0618	-0.3871\\
-0.1002	-0.379\\
-0.1375	-0.3671\\
-0.1735	-0.3516\\
-0.2077	-0.3325\\
-0.2399	-0.3101\\
-0.2696	-0.2846\\
-0.2967	-0.2563\\
-0.3208	-0.2254\\
-0.3417	-0.1922\\
-0.3592	-0.1571\\
-0.3731	-0.1205\\
-0.3832	-0.0826\\
-0.3896	-0.044\\
-0.392	-0.0049\\
-0.3905	0.0343\\
-0.3852	0.0731\\
-0.3759	0.1112\\
-0.363	0.1482\\
-0.3464	0.1837\\
-0.3263	0.2173\\
-0.303	0.2488\\
-0.2766	0.2778\\
-0.2475	0.3041\\
-0.2159	0.3272\\
-0.1821	0.3472\\
-0.1466	0.3636\\
-0.1095	0.3764\\
-0.0714	0.3855\\
-0.0326	0.3907\\
0.0066	0.392\\
0.0457	0.3894\\
0.0843	0.3829\\
0.1221	0.3725\\
0.1587	0.3585\\
0.1937	0.3408\\
0.2268	0.3198\\
0.2576	0.2956\\
0.2858	0.2684\\
0.3112	0.2385\\
0.3334	0.2062\\
0.3523	0.1719\\
0.3677	0.1359\\
0.3795	0.0985\\
0.3874	0.0601\\
0.3915	0.0212\\
0.3916	-0.018\\
0.3879	-0.057\\
};
\addplot [color=black!40, line width=0.4pt, forget plot]
  table[row sep=crcr]{%
0	0.5381\\
0.0537	0.5354\\
0.1069	0.5273\\
0.159	0.514\\
0.2095	0.4956\\
0.258	0.4722\\
0.3038	0.4441\\
0.3466	0.4115\\
0.386	0.3749\\
0.4215	0.3345\\
0.4528	0.2907\\
0.4795	0.2441\\
0.5015	0.195\\
0.5185	0.1439\\
0.5302	0.0915\\
0.5367	0.0381\\
0.5378	-0.0157\\
0.5336	-0.0693\\
0.524	-0.1223\\
0.5092	-0.174\\
0.4893	-0.2239\\
0.4645	-0.2716\\
0.435	-0.3167\\
0.4012	-0.3585\\
0.3634	-0.3968\\
0.322	-0.4311\\
0.2774	-0.4611\\
0.23	-0.4865\\
0.1802	-0.507\\
0.1287	-0.5224\\
0.0759	-0.5327\\
0.0224	-0.5376\\
-0.0314	-0.5372\\
-0.0849	-0.5313\\
-0.1375	-0.5202\\
-0.1887	-0.5039\\
-0.2381	-0.4825\\
-0.2851	-0.4563\\
-0.3292	-0.4256\\
-0.3701	-0.3906\\
-0.4072	-0.3517\\
-0.4403	-0.3093\\
-0.469	-0.2638\\
-0.493	-0.2157\\
-0.512	-0.1654\\
-0.526	-0.1134\\
-0.5347	-0.0603\\
-0.538	-0.0067\\
-0.536	0.0471\\
-0.5286	0.1004\\
-0.516	0.1526\\
-0.4982	0.2034\\
-0.4754	0.2521\\
-0.4478	0.2983\\
-0.4158	0.3415\\
-0.3796	0.3813\\
-0.3397	0.4173\\
-0.2963	0.4491\\
-0.25	0.4765\\
-0.2012	0.499\\
-0.1503	0.5166\\
-0.098	0.5291\\
-0.0447	0.5362\\
0.009	0.538\\
0.0627	0.5344\\
0.1157	0.5255\\
0.1676	0.5113\\
0.2178	0.492\\
0.2659	0.4678\\
0.3112	0.4389\\
0.3535	0.4057\\
0.3922	0.3683\\
0.427	0.3273\\
0.4576	0.2831\\
0.4836	0.236\\
0.5047	0.1865\\
0.5208	0.1352\\
0.5317	0.0825\\
0.5373	0.029\\
0.5375	-0.0248\\
0.5323	-0.0783\\
};
\addplot [color=black!40, line width=0.4pt, forget plot]
  table[row sep=crcr]{%
0	0.6841\\
0.0683	0.6807\\
0.1359	0.6705\\
0.2022	0.6535\\
0.2664	0.6301\\
0.328	0.6003\\
0.3863	0.5646\\
0.4407	0.5232\\
0.4907	0.4766\\
0.5359	0.4252\\
0.5756	0.3696\\
0.6097	0.3103\\
0.6376	0.2479\\
0.6592	0.183\\
0.6741	0.1163\\
0.6824	0.0484\\
0.6838	-0.02\\
0.6784	-0.0881\\
0.6662	-0.1554\\
0.6474	-0.2212\\
0.622	-0.2847\\
0.5905	-0.3454\\
0.5531	-0.4026\\
0.5101	-0.4558\\
0.4621	-0.5044\\
0.4094	-0.5481\\
0.3527	-0.5862\\
0.2924	-0.6185\\
0.2292	-0.6446\\
0.1637	-0.6642\\
0.0965	-0.6772\\
0.0284	-0.6835\\
-0.0399	-0.6829\\
-0.1079	-0.6755\\
-0.1748	-0.6614\\
-0.24	-0.6406\\
-0.3027	-0.6135\\
-0.3625	-0.5802\\
-0.4186	-0.5411\\
-0.4705	-0.4966\\
-0.5177	-0.4472\\
-0.5598	-0.3932\\
-0.5962	-0.3354\\
-0.6267	-0.2742\\
-0.651	-0.2102\\
-0.6687	-0.1442\\
-0.6798	-0.0767\\
-0.684	-0.0085\\
-0.6815	0.0599\\
-0.6721	0.1276\\
-0.656	0.1941\\
-0.6333	0.2586\\
-0.6044	0.3205\\
-0.5693	0.3792\\
-0.5286	0.4342\\
-0.4827	0.4848\\
-0.4318	0.5306\\
-0.3767	0.571\\
-0.3178	0.6058\\
-0.2558	0.6345\\
-0.1911	0.6568\\
-0.1246	0.6726\\
-0.0568	0.6817\\
0.0115	0.684\\
0.0797	0.6794\\
0.1472	0.6681\\
0.2131	0.65\\
0.277	0.6255\\
0.338	0.5948\\
0.3957	0.558\\
0.4494	0.5157\\
0.4987	0.4683\\
0.5429	0.4162\\
0.5818	0.3599\\
0.6148	0.3\\
0.6417	0.2371\\
0.6621	0.1719\\
0.676	0.1049\\
0.6831	0.0369\\
0.6834	-0.0315\\
0.6768	-0.0995\\
};
\addplot [color=black!40, line width=0.4pt, forget plot]
  table[row sep=crcr]{%
0	0.8301\\
0.0829	0.826\\
0.1649	0.8136\\
0.2453	0.793\\
0.3233	0.7646\\
0.398	0.7285\\
0.4687	0.6851\\
0.5348	0.6349\\
0.5955	0.5783\\
0.6503	0.516\\
0.6985	0.4485\\
0.7398	0.3765\\
0.7737	0.3008\\
0.7999	0.2221\\
0.818	0.1411\\
0.828	0.0587\\
0.8298	-0.0242\\
0.8232	-0.107\\
0.8084	-0.1886\\
0.7855	-0.2684\\
0.7548	-0.3455\\
0.7166	-0.4191\\
0.6711	-0.4885\\
0.619	-0.5531\\
0.5607	-0.6121\\
0.4968	-0.665\\
0.4279	-0.7113\\
0.3548	-0.7505\\
0.2781	-0.7822\\
0.1986	-0.806\\
0.1171	-0.8218\\
0.0345	-0.8294\\
-0.0485	-0.8287\\
-0.1309	-0.8197\\
-0.2121	-0.8026\\
-0.2912	-0.7774\\
-0.3673	-0.7444\\
-0.4398	-0.704\\
-0.5079	-0.6566\\
-0.5709	-0.6026\\
-0.6282	-0.5426\\
-0.6793	-0.4772\\
-0.7235	-0.407\\
-0.7605	-0.3327\\
-0.7899	-0.2551\\
-0.8115	-0.175\\
-0.8249	-0.0931\\
-0.8301	-0.0103\\
-0.8269	0.0726\\
-0.8156	0.1548\\
-0.796	0.2355\\
-0.7685	0.3138\\
-0.7334	0.3889\\
-0.6909	0.4602\\
-0.6415	0.5269\\
-0.5857	0.5883\\
-0.524	0.6438\\
-0.4571	0.6929\\
-0.3857	0.7351\\
-0.3104	0.7699\\
-0.2319	0.7971\\
-0.1512	0.8162\\
-0.069	0.8272\\
0.014	0.83\\
0.0967	0.8245\\
0.1786	0.8107\\
0.2586	0.7888\\
0.3361	0.759\\
0.4102	0.7217\\
0.4802	0.6771\\
0.5454	0.6258\\
0.6051	0.5683\\
0.6588	0.505\\
0.706	0.4367\\
0.746	0.364\\
0.7787	0.2877\\
0.8035	0.2086\\
0.8203	0.1273\\
0.8289	0.0448\\
0.8292	-0.0382\\
0.8213	-0.1208\\
};
\addplot [color=black!40, line width=0.4pt, forget plot]
  table[row sep=crcr]{%
0	0.9761\\
0.0975	0.9713\\
0.1939	0.9567\\
0.2885	0.9325\\
0.3801	0.8991\\
0.468	0.8566\\
0.5512	0.8056\\
0.6288	0.7466\\
0.7002	0.6801\\
0.7646	0.6068\\
0.8214	0.5274\\
0.8699	0.4428\\
0.9098	0.3537\\
0.9406	0.2611\\
0.9619	0.1659\\
0.9737	0.069\\
0.9757	-0.0285\\
0.968	-0.1258\\
0.9506	-0.2218\\
0.9237	-0.3156\\
0.8876	-0.4062\\
0.8426	-0.4928\\
0.7892	-0.5745\\
0.7279	-0.6504\\
0.6593	-0.7198\\
0.5842	-0.782\\
0.5032	-0.8364\\
0.4172	-0.8825\\
0.327	-0.9197\\
0.2335	-0.9478\\
0.1378	-0.9664\\
0.0406	-0.9753\\
-0.057	-0.9745\\
-0.154	-0.9639\\
-0.2494	-0.9437\\
-0.3424	-0.9141\\
-0.432	-0.8754\\
-0.5172	-0.8279\\
-0.5973	-0.7721\\
-0.6714	-0.7086\\
-0.7387	-0.638\\
-0.7988	-0.5611\\
-0.8508	-0.4786\\
-0.8943	-0.3912\\
-0.9289	-0.3\\
-0.9542	-0.2058\\
-0.97	-0.1095\\
-0.9761	-0.0121\\
-0.9724	0.0854\\
-0.959	0.1821\\
-0.936	0.2769\\
-0.9037	0.369\\
-0.8624	0.4573\\
-0.8124	0.5411\\
-0.7543	0.6196\\
-0.6887	0.6918\\
-0.6162	0.7571\\
-0.5375	0.8148\\
-0.4535	0.8644\\
-0.365	0.9054\\
-0.2727	0.9373\\
-0.1778	0.9598\\
-0.0811	0.9728\\
0.0164	0.976\\
0.1138	0.9695\\
0.21	0.9533\\
0.3041	0.9276\\
0.3952	0.8926\\
0.4823	0.8487\\
0.5646	0.7963\\
0.6413	0.7359\\
0.7116	0.6682\\
0.7747	0.5938\\
0.8301	0.5135\\
0.8773	0.4281\\
0.9156	0.3384\\
0.9448	0.2453\\
0.9646	0.1497\\
0.9747	0.0527\\
0.9751	-0.0449\\
0.9658	-0.142\\
};
\addplot [color=black!40, line width=0.4pt, forget plot]
  table[row sep=crcr]{%
0	1.1222\\
0.112	1.1166\\
0.2229	1.0998\\
0.3316	1.072\\
0.437	1.0336\\
0.538	0.9848\\
0.6336	0.9262\\
0.7229	0.8583\\
0.805	0.7818\\
0.879	0.6975\\
0.9443	0.6063\\
1.0001	0.509\\
1.0459	0.4066\\
1.0813	0.3002\\
1.1058	0.1907\\
1.1194	0.0794\\
1.1217	-0.0328\\
1.1128	-0.1446\\
1.0928	-0.255\\
1.0619	-0.3628\\
1.0204	-0.467\\
0.9687	-0.5665\\
0.9073	-0.6604\\
0.8368	-0.7477\\
0.758	-0.8275\\
0.6716	-0.899\\
0.5785	-0.9616\\
0.4796	-1.0145\\
0.3759	-1.0573\\
0.2685	-1.0896\\
0.1584	-1.1109\\
0.0467	-1.1212\\
-0.0655	-1.1203\\
-0.177	-1.1081\\
-0.2868	-1.0849\\
-0.3936	-1.0509\\
-0.4966	-1.0063\\
-0.5946	-0.9517\\
-0.6866	-0.8876\\
-0.7718	-0.8146\\
-0.8493	-0.7335\\
-0.9182	-0.645\\
-0.9781	-0.5502\\
-1.0281	-0.4498\\
-1.0679	-0.3449\\
-1.097	-0.2365\\
-1.1151	-0.1259\\
-1.1221	-0.0139\\
-1.1179	0.0982\\
-1.1025	0.2093\\
-1.0761	0.3183\\
-1.0389	0.4242\\
-0.9914	0.5258\\
-0.9339	0.6221\\
-0.8672	0.7122\\
-0.7917	0.7952\\
-0.7084	0.8703\\
-0.618	0.9367\\
-0.5214	0.9937\\
-0.4196	1.0408\\
-0.3136	1.0775\\
-0.2044	1.1034\\
-0.0932	1.1183\\
0.0189	1.122\\
0.1308	1.1145\\
0.2414	1.0959\\
0.3496	1.0663\\
0.4543	1.0261\\
0.5545	0.9756\\
0.6491	0.9154\\
0.7372	0.846\\
0.818	0.7682\\
0.8906	0.6827\\
0.9543	0.5903\\
1.0085	0.4921\\
1.0526	0.389\\
1.0862	0.282\\
1.1089	0.1721\\
1.1205	0.0605\\
1.121	-0.0516\\
1.1102	-0.1633\\
};
\addplot [color=black!40, line width=0.4pt, forget plot]
  table[row sep=crcr]{%
0.4939	1.1681\\
0.608	1.1129\\
0.7161	1.0467\\
0.817	0.97\\
0.9097	0.8836\\
0.9934	0.7883\\
1.0671	0.6852\\
1.1302	0.5752\\
1.182	0.4595\\
1.1532	-0.5278\\
1.0947	-0.6402\\
1.0253	-0.7463\\
0.9457	-0.845\\
0.8566	-0.9352\\
0.759	-1.016\\
0.6538	-1.0867\\
0.542	-1.1465\\
0.4248	-1.1949\\
-0.4449	-1.1876\\
-0.5612	-1.1373\\
-0.6719	-1.0756\\
-0.776	-1.0031\\
-0.8722	-0.9206\\
-0.9598	-0.8289\\
-1.0377	-0.729\\
-1.1053	-0.6217\\
-1.1619	-0.5083\\
-1.1741	0.4793\\
-1.1204	0.5942\\
-1.0555	0.7031\\
-0.98	0.8049\\
-0.8948	0.8987\\
-0.8006	0.9836\\
-0.6984	1.0586\\
-0.5892	1.123\\
-0.4741	1.1762\\
0.5134	1.1596\\
0.6266	1.1026\\
0.7336	1.0345\\
0.8332	0.9561\\
0.9245	0.8681\\
1.0065	0.7715\\
1.0785	0.6672\\
1.1397	0.5562\\
1.1896	0.4396\\
};
\addplot [color=black!40, line width=0.4pt, forget plot]
  table[row sep=crcr]{%
0.7985	1.1672\\
0.9111	1.0817\\
1.0145	0.9853\\
1.1078	0.8791\\
1.19	0.7641\\
1.1434	-0.8323\\
1.0546	-0.9423\\
0.9552	-1.0428\\
0.8464	-1.133\\
-0.7493	-1.1994\\
-0.8653	-1.1186\\
-0.9726	-1.0266\\
-1.0703	-0.9244\\
-1.1572	-0.8129\\
-1.177	0.784\\
-1.0929	0.8976\\
-0.9978	1.0022\\
-0.8927	1.0968\\
-0.7788	1.1805\\
0.818	1.1536\\
0.9291	1.0662\\
1.0309	0.9681\\
1.1224	0.8603\\
};
\addplot[-stealth, color=accent1, point meta={sqrt((\thisrow{u})^2+(\thisrow{v})^2)}, point meta min=0, quiver={u=\thisrow{u}, v=\thisrow{v}, scale arrows = 1.45, every arrow/.append style={line width=1pt*\pgfplotspointmetatransformed/1000}}]
 table[row sep=crcr] {%
x	y	u	v\\
-1	-1	-0.09	0.09\\
-1	-0.894736842105263	-0.0805263157894737	0.09\\
-1	-0.789473684210526	-0.0710526315789474	0.09\\
-1	-0.684210526315789	-0.0615789473684211	0.09\\
-1	-0.578947368421053	-0.0521052631578948	0.09\\
-1	-0.473684210526316	-0.0426315789473684	0.09\\
-1	-0.368421052631579	-0.0331578947368421	0.09\\
-1	-0.263157894736842	-0.0236842105263158	0.09\\
-1	-0.157894736842105	-0.0142105263157895	0.09\\
-1	-0.0526315789473684	-0.00473684210526316	0.09\\
-1	0.0526315789473684	0.00473684210526316	0.09\\
-1	0.157894736842105	0.0142105263157895	0.09\\
-1	0.263157894736842	0.0236842105263158	0.09\\
-1	0.368421052631579	0.0331578947368421	0.09\\
-1	0.473684210526316	0.0426315789473684	0.09\\
-1	0.578947368421053	0.0521052631578948	0.09\\
-1	0.684210526315789	0.0615789473684211	0.09\\
-1	0.789473684210526	0.0710526315789474	0.09\\
-1	0.894736842105263	0.0805263157894737	0.09\\
-1	1	0.09	0.09\\
-0.894736842105263	-1	-0.09	0.0805263157894737\\
-0.894736842105263	-0.894736842105263	-0.0805263157894737	0.0805263157894737\\
-0.894736842105263	-0.789473684210526	-0.0710526315789474	0.0805263157894737\\
-0.894736842105263	-0.684210526315789	-0.0615789473684211	0.0805263157894737\\
-0.894736842105263	-0.578947368421053	-0.0521052631578948	0.0805263157894737\\
-0.894736842105263	-0.473684210526316	-0.0426315789473684	0.0805263157894737\\
-0.894736842105263	-0.368421052631579	-0.0331578947368421	0.0805263157894737\\
-0.894736842105263	-0.263157894736842	-0.0236842105263158	0.0805263157894737\\
-0.894736842105263	-0.157894736842105	-0.0142105263157895	0.0805263157894737\\
-0.894736842105263	-0.0526315789473684	-0.00473684210526316	0.0805263157894737\\
-0.894736842105263	0.0526315789473684	0.00473684210526316	0.0805263157894737\\
-0.894736842105263	0.157894736842105	0.0142105263157895	0.0805263157894737\\
-0.894736842105263	0.263157894736842	0.0236842105263158	0.0805263157894737\\
-0.894736842105263	0.368421052631579	0.0331578947368421	0.0805263157894737\\
-0.894736842105263	0.473684210526316	0.0426315789473684	0.0805263157894737\\
-0.894736842105263	0.578947368421053	0.0521052631578948	0.0805263157894737\\
-0.894736842105263	0.684210526315789	0.0615789473684211	0.0805263157894737\\
-0.894736842105263	0.789473684210526	0.0710526315789474	0.0805263157894737\\
-0.894736842105263	0.894736842105263	0.0805263157894737	0.0805263157894737\\
-0.894736842105263	1	0.09	0.0805263157894737\\
-0.789473684210526	-1	-0.09	0.0710526315789474\\
-0.789473684210526	-0.894736842105263	-0.0805263157894737	0.0710526315789474\\
-0.789473684210526	-0.789473684210526	-0.0710526315789474	0.0710526315789474\\
-0.789473684210526	-0.684210526315789	-0.0615789473684211	0.0710526315789474\\
-0.789473684210526	-0.578947368421053	-0.0521052631578948	0.0710526315789474\\
-0.789473684210526	-0.473684210526316	-0.0426315789473684	0.0710526315789474\\
-0.789473684210526	-0.368421052631579	-0.0331578947368421	0.0710526315789474\\
-0.789473684210526	-0.263157894736842	-0.0236842105263158	0.0710526315789474\\
-0.789473684210526	-0.157894736842105	-0.0142105263157895	0.0710526315789474\\
-0.789473684210526	-0.0526315789473684	-0.00473684210526316	0.0710526315789474\\
-0.789473684210526	0.0526315789473684	0.00473684210526316	0.0710526315789474\\
-0.789473684210526	0.157894736842105	0.0142105263157895	0.0710526315789474\\
-0.789473684210526	0.263157894736842	0.0236842105263158	0.0710526315789474\\
-0.789473684210526	0.368421052631579	0.0331578947368421	0.0710526315789474\\
-0.789473684210526	0.473684210526316	0.0426315789473684	0.0710526315789474\\
-0.789473684210526	0.578947368421053	0.0521052631578948	0.0710526315789474\\
-0.789473684210526	0.684210526315789	0.0615789473684211	0.0710526315789474\\
-0.789473684210526	0.789473684210526	0.0710526315789474	0.0710526315789474\\
-0.789473684210526	0.894736842105263	0.0805263157894737	0.0710526315789474\\
-0.789473684210526	1	0.09	0.0710526315789474\\
-0.684210526315789	-1	-0.09	0.0615789473684211\\
-0.684210526315789	-0.894736842105263	-0.0805263157894737	0.0615789473684211\\
-0.684210526315789	-0.789473684210526	-0.0710526315789474	0.0615789473684211\\
-0.684210526315789	-0.684210526315789	-0.0615789473684211	0.0615789473684211\\
-0.684210526315789	-0.578947368421053	-0.0521052631578948	0.0615789473684211\\
-0.684210526315789	-0.473684210526316	-0.0426315789473684	0.0615789473684211\\
-0.684210526315789	-0.368421052631579	-0.0331578947368421	0.0615789473684211\\
-0.684210526315789	-0.263157894736842	-0.0236842105263158	0.0615789473684211\\
-0.684210526315789	-0.157894736842105	-0.0142105263157895	0.0615789473684211\\
-0.684210526315789	-0.0526315789473684	-0.00473684210526316	0.0615789473684211\\
-0.684210526315789	0.0526315789473684	0.00473684210526316	0.0615789473684211\\
-0.684210526315789	0.157894736842105	0.0142105263157895	0.0615789473684211\\
-0.684210526315789	0.263157894736842	0.0236842105263158	0.0615789473684211\\
-0.684210526315789	0.368421052631579	0.0331578947368421	0.0615789473684211\\
-0.684210526315789	0.473684210526316	0.0426315789473684	0.0615789473684211\\
-0.684210526315789	0.578947368421053	0.0521052631578948	0.0615789473684211\\
-0.684210526315789	0.684210526315789	0.0615789473684211	0.0615789473684211\\
-0.684210526315789	0.789473684210526	0.0710526315789474	0.0615789473684211\\
-0.684210526315789	0.894736842105263	0.0805263157894737	0.0615789473684211\\
-0.684210526315789	1	0.09	0.0615789473684211\\
-0.578947368421053	-1	-0.09	0.0521052631578948\\
-0.578947368421053	-0.894736842105263	-0.0805263157894737	0.0521052631578948\\
-0.578947368421053	-0.789473684210526	-0.0710526315789474	0.0521052631578948\\
-0.578947368421053	-0.684210526315789	-0.0615789473684211	0.0521052631578948\\
-0.578947368421053	-0.578947368421053	-0.0521052631578948	0.0521052631578948\\
-0.578947368421053	-0.473684210526316	-0.0426315789473684	0.0521052631578948\\
-0.578947368421053	-0.368421052631579	-0.0331578947368421	0.0521052631578948\\
-0.578947368421053	-0.263157894736842	-0.0236842105263158	0.0521052631578948\\
-0.578947368421053	-0.157894736842105	-0.0142105263157895	0.0521052631578948\\
-0.578947368421053	-0.0526315789473684	-0.00473684210526316	0.0521052631578948\\
-0.578947368421053	0.0526315789473684	0.00473684210526316	0.0521052631578948\\
-0.578947368421053	0.157894736842105	0.0142105263157895	0.0521052631578948\\
-0.578947368421053	0.263157894736842	0.0236842105263158	0.0521052631578948\\
-0.578947368421053	0.368421052631579	0.0331578947368421	0.0521052631578948\\
-0.578947368421053	0.473684210526316	0.0426315789473684	0.0521052631578948\\
-0.578947368421053	0.578947368421053	0.0521052631578948	0.0521052631578948\\
-0.578947368421053	0.684210526315789	0.0615789473684211	0.0521052631578948\\
-0.578947368421053	0.789473684210526	0.0710526315789474	0.0521052631578948\\
-0.578947368421053	0.894736842105263	0.0805263157894737	0.0521052631578948\\
-0.578947368421053	1	0.09	0.0521052631578948\\
-0.473684210526316	-1	-0.09	0.0426315789473684\\
-0.473684210526316	-0.894736842105263	-0.0805263157894737	0.0426315789473684\\
-0.473684210526316	-0.789473684210526	-0.0710526315789474	0.0426315789473684\\
-0.473684210526316	-0.684210526315789	-0.0615789473684211	0.0426315789473684\\
-0.473684210526316	-0.578947368421053	-0.0521052631578948	0.0426315789473684\\
-0.473684210526316	-0.473684210526316	-0.0426315789473684	0.0426315789473684\\
-0.473684210526316	-0.368421052631579	-0.0331578947368421	0.0426315789473684\\
-0.473684210526316	-0.263157894736842	-0.0236842105263158	0.0426315789473684\\
-0.473684210526316	-0.157894736842105	-0.0142105263157895	0.0426315789473684\\
-0.473684210526316	-0.0526315789473684	-0.00473684210526316	0.0426315789473684\\
-0.473684210526316	0.0526315789473684	0.00473684210526316	0.0426315789473684\\
-0.473684210526316	0.157894736842105	0.0142105263157895	0.0426315789473684\\
-0.473684210526316	0.263157894736842	0.0236842105263158	0.0426315789473684\\
-0.473684210526316	0.368421052631579	0.0331578947368421	0.0426315789473684\\
-0.473684210526316	0.473684210526316	0.0426315789473684	0.0426315789473684\\
-0.473684210526316	0.578947368421053	0.0521052631578948	0.0426315789473684\\
-0.473684210526316	0.684210526315789	0.0615789473684211	0.0426315789473684\\
-0.473684210526316	0.789473684210526	0.0710526315789474	0.0426315789473684\\
-0.473684210526316	0.894736842105263	0.0805263157894737	0.0426315789473684\\
-0.473684210526316	1	0.09	0.0426315789473684\\
-0.368421052631579	-1	-0.09	0.0331578947368421\\
-0.368421052631579	-0.894736842105263	-0.0805263157894737	0.0331578947368421\\
-0.368421052631579	-0.789473684210526	-0.0710526315789474	0.0331578947368421\\
-0.368421052631579	-0.684210526315789	-0.0615789473684211	0.0331578947368421\\
-0.368421052631579	-0.578947368421053	-0.0521052631578948	0.0331578947368421\\
-0.368421052631579	-0.473684210526316	-0.0426315789473684	0.0331578947368421\\
-0.368421052631579	-0.368421052631579	-0.0331578947368421	0.0331578947368421\\
-0.368421052631579	-0.263157894736842	-0.0236842105263158	0.0331578947368421\\
-0.368421052631579	-0.157894736842105	-0.0142105263157895	0.0331578947368421\\
-0.368421052631579	-0.0526315789473684	-0.00473684210526316	0.0331578947368421\\
-0.368421052631579	0.0526315789473684	0.00473684210526316	0.0331578947368421\\
-0.368421052631579	0.157894736842105	0.0142105263157895	0.0331578947368421\\
-0.368421052631579	0.263157894736842	0.0236842105263158	0.0331578947368421\\
-0.368421052631579	0.368421052631579	0.0331578947368421	0.0331578947368421\\
-0.368421052631579	0.473684210526316	0.0426315789473684	0.0331578947368421\\
-0.368421052631579	0.578947368421053	0.0521052631578948	0.0331578947368421\\
-0.368421052631579	0.684210526315789	0.0615789473684211	0.0331578947368421\\
-0.368421052631579	0.789473684210526	0.0710526315789474	0.0331578947368421\\
-0.368421052631579	0.894736842105263	0.0805263157894737	0.0331578947368421\\
-0.368421052631579	1	0.09	0.0331578947368421\\
-0.263157894736842	-1	-0.09	0.0236842105263158\\
-0.263157894736842	-0.894736842105263	-0.0805263157894737	0.0236842105263158\\
-0.263157894736842	-0.789473684210526	-0.0710526315789474	0.0236842105263158\\
-0.263157894736842	-0.684210526315789	-0.0615789473684211	0.0236842105263158\\
-0.263157894736842	-0.578947368421053	-0.0521052631578948	0.0236842105263158\\
-0.263157894736842	-0.473684210526316	-0.0426315789473684	0.0236842105263158\\
-0.263157894736842	-0.368421052631579	-0.0331578947368421	0.0236842105263158\\
-0.263157894736842	-0.263157894736842	-0.0236842105263158	0.0236842105263158\\
-0.263157894736842	-0.157894736842105	-0.0142105263157895	0.0236842105263158\\
-0.263157894736842	-0.0526315789473684	-0.00473684210526316	0.0236842105263158\\
-0.263157894736842	0.0526315789473684	0.00473684210526316	0.0236842105263158\\
-0.263157894736842	0.157894736842105	0.0142105263157895	0.0236842105263158\\
-0.263157894736842	0.263157894736842	0.0236842105263158	0.0236842105263158\\
-0.263157894736842	0.368421052631579	0.0331578947368421	0.0236842105263158\\
-0.263157894736842	0.473684210526316	0.0426315789473684	0.0236842105263158\\
-0.263157894736842	0.578947368421053	0.0521052631578948	0.0236842105263158\\
-0.263157894736842	0.684210526315789	0.0615789473684211	0.0236842105263158\\
-0.263157894736842	0.789473684210526	0.0710526315789474	0.0236842105263158\\
-0.263157894736842	0.894736842105263	0.0805263157894737	0.0236842105263158\\
-0.263157894736842	1	0.09	0.0236842105263158\\
-0.157894736842105	-1	-0.09	0.0142105263157895\\
-0.157894736842105	-0.894736842105263	-0.0805263157894737	0.0142105263157895\\
-0.157894736842105	-0.789473684210526	-0.0710526315789474	0.0142105263157895\\
-0.157894736842105	-0.684210526315789	-0.0615789473684211	0.0142105263157895\\
-0.157894736842105	-0.578947368421053	-0.0521052631578948	0.0142105263157895\\
-0.157894736842105	-0.473684210526316	-0.0426315789473684	0.0142105263157895\\
-0.157894736842105	-0.368421052631579	-0.0331578947368421	0.0142105263157895\\
-0.157894736842105	-0.263157894736842	-0.0236842105263158	0.0142105263157895\\
-0.157894736842105	-0.157894736842105	-0.0142105263157895	0.0142105263157895\\
-0.157894736842105	-0.0526315789473684	-0.00473684210526316	0.0142105263157895\\
-0.157894736842105	0.0526315789473684	0.00473684210526316	0.0142105263157895\\
-0.157894736842105	0.157894736842105	0.0142105263157895	0.0142105263157895\\
-0.157894736842105	0.263157894736842	0.0236842105263158	0.0142105263157895\\
-0.157894736842105	0.368421052631579	0.0331578947368421	0.0142105263157895\\
-0.157894736842105	0.473684210526316	0.0426315789473684	0.0142105263157895\\
-0.157894736842105	0.578947368421053	0.0521052631578948	0.0142105263157895\\
-0.157894736842105	0.684210526315789	0.0615789473684211	0.0142105263157895\\
-0.157894736842105	0.789473684210526	0.0710526315789474	0.0142105263157895\\
-0.157894736842105	0.894736842105263	0.0805263157894737	0.0142105263157895\\
-0.157894736842105	1	0.09	0.0142105263157895\\
-0.0526315789473684	-1	-0.09	0.00473684210526316\\
-0.0526315789473684	-0.894736842105263	-0.0805263157894737	0.00473684210526316\\
-0.0526315789473684	-0.789473684210526	-0.0710526315789474	0.00473684210526316\\
-0.0526315789473684	-0.684210526315789	-0.0615789473684211	0.00473684210526316\\
-0.0526315789473684	-0.578947368421053	-0.0521052631578948	0.00473684210526316\\
-0.0526315789473684	-0.473684210526316	-0.0426315789473684	0.00473684210526316\\
-0.0526315789473684	-0.368421052631579	-0.0331578947368421	0.00473684210526316\\
-0.0526315789473684	-0.263157894736842	-0.0236842105263158	0.00473684210526316\\
-0.0526315789473684	-0.157894736842105	-0.0142105263157895	0.00473684210526316\\
-0.0526315789473684	-0.0526315789473684	-0.00473684210526316	0.00473684210526316\\
-0.0526315789473684	0.0526315789473684	0.00473684210526316	0.00473684210526316\\
-0.0526315789473684	0.157894736842105	0.0142105263157895	0.00473684210526316\\
-0.0526315789473684	0.263157894736842	0.0236842105263158	0.00473684210526316\\
-0.0526315789473684	0.368421052631579	0.0331578947368421	0.00473684210526316\\
-0.0526315789473684	0.473684210526316	0.0426315789473684	0.00473684210526316\\
-0.0526315789473684	0.578947368421053	0.0521052631578948	0.00473684210526316\\
-0.0526315789473684	0.684210526315789	0.0615789473684211	0.00473684210526316\\
-0.0526315789473684	0.789473684210526	0.0710526315789474	0.00473684210526316\\
-0.0526315789473684	0.894736842105263	0.0805263157894737	0.00473684210526316\\
-0.0526315789473684	1	0.09	0.00473684210526316\\
0.0526315789473684	-1	-0.09	-0.00473684210526316\\
0.0526315789473684	-0.894736842105263	-0.0805263157894737	-0.00473684210526316\\
0.0526315789473684	-0.789473684210526	-0.0710526315789474	-0.00473684210526316\\
0.0526315789473684	-0.684210526315789	-0.0615789473684211	-0.00473684210526316\\
0.0526315789473684	-0.578947368421053	-0.0521052631578948	-0.00473684210526316\\
0.0526315789473684	-0.473684210526316	-0.0426315789473684	-0.00473684210526316\\
0.0526315789473684	-0.368421052631579	-0.0331578947368421	-0.00473684210526316\\
0.0526315789473684	-0.263157894736842	-0.0236842105263158	-0.00473684210526316\\
0.0526315789473684	-0.157894736842105	-0.0142105263157895	-0.00473684210526316\\
0.0526315789473684	-0.0526315789473684	-0.00473684210526316	-0.00473684210526316\\
0.0526315789473684	0.0526315789473684	0.00473684210526316	-0.00473684210526316\\
0.0526315789473684	0.157894736842105	0.0142105263157895	-0.00473684210526316\\
0.0526315789473684	0.263157894736842	0.0236842105263158	-0.00473684210526316\\
0.0526315789473684	0.368421052631579	0.0331578947368421	-0.00473684210526316\\
0.0526315789473684	0.473684210526316	0.0426315789473684	-0.00473684210526316\\
0.0526315789473684	0.578947368421053	0.0521052631578948	-0.00473684210526316\\
0.0526315789473684	0.684210526315789	0.0615789473684211	-0.00473684210526316\\
0.0526315789473684	0.789473684210526	0.0710526315789474	-0.00473684210526316\\
0.0526315789473684	0.894736842105263	0.0805263157894737	-0.00473684210526316\\
0.0526315789473684	1	0.09	-0.00473684210526316\\
0.157894736842105	-1	-0.09	-0.0142105263157895\\
0.157894736842105	-0.894736842105263	-0.0805263157894737	-0.0142105263157895\\
0.157894736842105	-0.789473684210526	-0.0710526315789474	-0.0142105263157895\\
0.157894736842105	-0.684210526315789	-0.0615789473684211	-0.0142105263157895\\
0.157894736842105	-0.578947368421053	-0.0521052631578948	-0.0142105263157895\\
0.157894736842105	-0.473684210526316	-0.0426315789473684	-0.0142105263157895\\
0.157894736842105	-0.368421052631579	-0.0331578947368421	-0.0142105263157895\\
0.157894736842105	-0.263157894736842	-0.0236842105263158	-0.0142105263157895\\
0.157894736842105	-0.157894736842105	-0.0142105263157895	-0.0142105263157895\\
0.157894736842105	-0.0526315789473684	-0.00473684210526316	-0.0142105263157895\\
0.157894736842105	0.0526315789473684	0.00473684210526316	-0.0142105263157895\\
0.157894736842105	0.157894736842105	0.0142105263157895	-0.0142105263157895\\
0.157894736842105	0.263157894736842	0.0236842105263158	-0.0142105263157895\\
0.157894736842105	0.368421052631579	0.0331578947368421	-0.0142105263157895\\
0.157894736842105	0.473684210526316	0.0426315789473684	-0.0142105263157895\\
0.157894736842105	0.578947368421053	0.0521052631578948	-0.0142105263157895\\
0.157894736842105	0.684210526315789	0.0615789473684211	-0.0142105263157895\\
0.157894736842105	0.789473684210526	0.0710526315789474	-0.0142105263157895\\
0.157894736842105	0.894736842105263	0.0805263157894737	-0.0142105263157895\\
0.157894736842105	1	0.09	-0.0142105263157895\\
0.263157894736842	-1	-0.09	-0.0236842105263158\\
0.263157894736842	-0.894736842105263	-0.0805263157894737	-0.0236842105263158\\
0.263157894736842	-0.789473684210526	-0.0710526315789474	-0.0236842105263158\\
0.263157894736842	-0.684210526315789	-0.0615789473684211	-0.0236842105263158\\
0.263157894736842	-0.578947368421053	-0.0521052631578948	-0.0236842105263158\\
0.263157894736842	-0.473684210526316	-0.0426315789473684	-0.0236842105263158\\
0.263157894736842	-0.368421052631579	-0.0331578947368421	-0.0236842105263158\\
0.263157894736842	-0.263157894736842	-0.0236842105263158	-0.0236842105263158\\
0.263157894736842	-0.157894736842105	-0.0142105263157895	-0.0236842105263158\\
0.263157894736842	-0.0526315789473684	-0.00473684210526316	-0.0236842105263158\\
0.263157894736842	0.0526315789473684	0.00473684210526316	-0.0236842105263158\\
0.263157894736842	0.157894736842105	0.0142105263157895	-0.0236842105263158\\
0.263157894736842	0.263157894736842	0.0236842105263158	-0.0236842105263158\\
0.263157894736842	0.368421052631579	0.0331578947368421	-0.0236842105263158\\
0.263157894736842	0.473684210526316	0.0426315789473684	-0.0236842105263158\\
0.263157894736842	0.578947368421053	0.0521052631578948	-0.0236842105263158\\
0.263157894736842	0.684210526315789	0.0615789473684211	-0.0236842105263158\\
0.263157894736842	0.789473684210526	0.0710526315789474	-0.0236842105263158\\
0.263157894736842	0.894736842105263	0.0805263157894737	-0.0236842105263158\\
0.263157894736842	1	0.09	-0.0236842105263158\\
0.368421052631579	-1	-0.09	-0.0331578947368421\\
0.368421052631579	-0.894736842105263	-0.0805263157894737	-0.0331578947368421\\
0.368421052631579	-0.789473684210526	-0.0710526315789474	-0.0331578947368421\\
0.368421052631579	-0.684210526315789	-0.0615789473684211	-0.0331578947368421\\
0.368421052631579	-0.578947368421053	-0.0521052631578948	-0.0331578947368421\\
0.368421052631579	-0.473684210526316	-0.0426315789473684	-0.0331578947368421\\
0.368421052631579	-0.368421052631579	-0.0331578947368421	-0.0331578947368421\\
0.368421052631579	-0.263157894736842	-0.0236842105263158	-0.0331578947368421\\
0.368421052631579	-0.157894736842105	-0.0142105263157895	-0.0331578947368421\\
0.368421052631579	-0.0526315789473684	-0.00473684210526316	-0.0331578947368421\\
0.368421052631579	0.0526315789473684	0.00473684210526316	-0.0331578947368421\\
0.368421052631579	0.157894736842105	0.0142105263157895	-0.0331578947368421\\
0.368421052631579	0.263157894736842	0.0236842105263158	-0.0331578947368421\\
0.368421052631579	0.368421052631579	0.0331578947368421	-0.0331578947368421\\
0.368421052631579	0.473684210526316	0.0426315789473684	-0.0331578947368421\\
0.368421052631579	0.578947368421053	0.0521052631578948	-0.0331578947368421\\
0.368421052631579	0.684210526315789	0.0615789473684211	-0.0331578947368421\\
0.368421052631579	0.789473684210526	0.0710526315789474	-0.0331578947368421\\
0.368421052631579	0.894736842105263	0.0805263157894737	-0.0331578947368421\\
0.368421052631579	1	0.09	-0.0331578947368421\\
0.473684210526316	-1	-0.09	-0.0426315789473684\\
0.473684210526316	-0.894736842105263	-0.0805263157894737	-0.0426315789473684\\
0.473684210526316	-0.789473684210526	-0.0710526315789474	-0.0426315789473684\\
0.473684210526316	-0.684210526315789	-0.0615789473684211	-0.0426315789473684\\
0.473684210526316	-0.578947368421053	-0.0521052631578948	-0.0426315789473684\\
0.473684210526316	-0.473684210526316	-0.0426315789473684	-0.0426315789473684\\
0.473684210526316	-0.368421052631579	-0.0331578947368421	-0.0426315789473684\\
0.473684210526316	-0.263157894736842	-0.0236842105263158	-0.0426315789473684\\
0.473684210526316	-0.157894736842105	-0.0142105263157895	-0.0426315789473684\\
0.473684210526316	-0.0526315789473684	-0.00473684210526316	-0.0426315789473684\\
0.473684210526316	0.0526315789473684	0.00473684210526316	-0.0426315789473684\\
0.473684210526316	0.157894736842105	0.0142105263157895	-0.0426315789473684\\
0.473684210526316	0.263157894736842	0.0236842105263158	-0.0426315789473684\\
0.473684210526316	0.368421052631579	0.0331578947368421	-0.0426315789473684\\
0.473684210526316	0.473684210526316	0.0426315789473684	-0.0426315789473684\\
0.473684210526316	0.578947368421053	0.0521052631578948	-0.0426315789473684\\
0.473684210526316	0.684210526315789	0.0615789473684211	-0.0426315789473684\\
0.473684210526316	0.789473684210526	0.0710526315789474	-0.0426315789473684\\
0.473684210526316	0.894736842105263	0.0805263157894737	-0.0426315789473684\\
0.473684210526316	1	0.09	-0.0426315789473684\\
0.578947368421053	-1	-0.09	-0.0521052631578948\\
0.578947368421053	-0.894736842105263	-0.0805263157894737	-0.0521052631578948\\
0.578947368421053	-0.789473684210526	-0.0710526315789474	-0.0521052631578948\\
0.578947368421053	-0.684210526315789	-0.0615789473684211	-0.0521052631578948\\
0.578947368421053	-0.578947368421053	-0.0521052631578948	-0.0521052631578948\\
0.578947368421053	-0.473684210526316	-0.0426315789473684	-0.0521052631578948\\
0.578947368421053	-0.368421052631579	-0.0331578947368421	-0.0521052631578948\\
0.578947368421053	-0.263157894736842	-0.0236842105263158	-0.0521052631578948\\
0.578947368421053	-0.157894736842105	-0.0142105263157895	-0.0521052631578948\\
0.578947368421053	-0.0526315789473684	-0.00473684210526316	-0.0521052631578948\\
0.578947368421053	0.0526315789473684	0.00473684210526316	-0.0521052631578948\\
0.578947368421053	0.157894736842105	0.0142105263157895	-0.0521052631578948\\
0.578947368421053	0.263157894736842	0.0236842105263158	-0.0521052631578948\\
0.578947368421053	0.368421052631579	0.0331578947368421	-0.0521052631578948\\
0.578947368421053	0.473684210526316	0.0426315789473684	-0.0521052631578948\\
0.578947368421053	0.578947368421053	0.0521052631578948	-0.0521052631578948\\
0.578947368421053	0.684210526315789	0.0615789473684211	-0.0521052631578948\\
0.578947368421053	0.789473684210526	0.0710526315789474	-0.0521052631578948\\
0.578947368421053	0.894736842105263	0.0805263157894737	-0.0521052631578948\\
0.578947368421053	1	0.09	-0.0521052631578948\\
0.684210526315789	-1	-0.09	-0.0615789473684211\\
0.684210526315789	-0.894736842105263	-0.0805263157894737	-0.0615789473684211\\
0.684210526315789	-0.789473684210526	-0.0710526315789474	-0.0615789473684211\\
0.684210526315789	-0.684210526315789	-0.0615789473684211	-0.0615789473684211\\
0.684210526315789	-0.578947368421053	-0.0521052631578948	-0.0615789473684211\\
0.684210526315789	-0.473684210526316	-0.0426315789473684	-0.0615789473684211\\
0.684210526315789	-0.368421052631579	-0.0331578947368421	-0.0615789473684211\\
0.684210526315789	-0.263157894736842	-0.0236842105263158	-0.0615789473684211\\
0.684210526315789	-0.157894736842105	-0.0142105263157895	-0.0615789473684211\\
0.684210526315789	-0.0526315789473684	-0.00473684210526316	-0.0615789473684211\\
0.684210526315789	0.0526315789473684	0.00473684210526316	-0.0615789473684211\\
0.684210526315789	0.157894736842105	0.0142105263157895	-0.0615789473684211\\
0.684210526315789	0.263157894736842	0.0236842105263158	-0.0615789473684211\\
0.684210526315789	0.368421052631579	0.0331578947368421	-0.0615789473684211\\
0.684210526315789	0.473684210526316	0.0426315789473684	-0.0615789473684211\\
0.684210526315789	0.578947368421053	0.0521052631578948	-0.0615789473684211\\
0.684210526315789	0.684210526315789	0.0615789473684211	-0.0615789473684211\\
0.684210526315789	0.789473684210526	0.0710526315789474	-0.0615789473684211\\
0.684210526315789	0.894736842105263	0.0805263157894737	-0.0615789473684211\\
0.684210526315789	1	0.09	-0.0615789473684211\\
0.789473684210526	-1	-0.09	-0.0710526315789474\\
0.789473684210526	-0.894736842105263	-0.0805263157894737	-0.0710526315789474\\
0.789473684210526	-0.789473684210526	-0.0710526315789474	-0.0710526315789474\\
0.789473684210526	-0.684210526315789	-0.0615789473684211	-0.0710526315789474\\
0.789473684210526	-0.578947368421053	-0.0521052631578948	-0.0710526315789474\\
0.789473684210526	-0.473684210526316	-0.0426315789473684	-0.0710526315789474\\
0.789473684210526	-0.368421052631579	-0.0331578947368421	-0.0710526315789474\\
0.789473684210526	-0.263157894736842	-0.0236842105263158	-0.0710526315789474\\
0.789473684210526	-0.157894736842105	-0.0142105263157895	-0.0710526315789474\\
0.789473684210526	-0.0526315789473684	-0.00473684210526316	-0.0710526315789474\\
0.789473684210526	0.0526315789473684	0.00473684210526316	-0.0710526315789474\\
0.789473684210526	0.157894736842105	0.0142105263157895	-0.0710526315789474\\
0.789473684210526	0.263157894736842	0.0236842105263158	-0.0710526315789474\\
0.789473684210526	0.368421052631579	0.0331578947368421	-0.0710526315789474\\
0.789473684210526	0.473684210526316	0.0426315789473684	-0.0710526315789474\\
0.789473684210526	0.578947368421053	0.0521052631578948	-0.0710526315789474\\
0.789473684210526	0.684210526315789	0.0615789473684211	-0.0710526315789474\\
0.789473684210526	0.789473684210526	0.0710526315789474	-0.0710526315789474\\
0.789473684210526	0.894736842105263	0.0805263157894737	-0.0710526315789474\\
0.789473684210526	1	0.09	-0.0710526315789474\\
0.894736842105263	-1	-0.09	-0.0805263157894737\\
0.894736842105263	-0.894736842105263	-0.0805263157894737	-0.0805263157894737\\
0.894736842105263	-0.789473684210526	-0.0710526315789474	-0.0805263157894737\\
0.894736842105263	-0.684210526315789	-0.0615789473684211	-0.0805263157894737\\
0.894736842105263	-0.578947368421053	-0.0521052631578948	-0.0805263157894737\\
0.894736842105263	-0.473684210526316	-0.0426315789473684	-0.0805263157894737\\
0.894736842105263	-0.368421052631579	-0.0331578947368421	-0.0805263157894737\\
0.894736842105263	-0.263157894736842	-0.0236842105263158	-0.0805263157894737\\
0.894736842105263	-0.157894736842105	-0.0142105263157895	-0.0805263157894737\\
0.894736842105263	-0.0526315789473684	-0.00473684210526316	-0.0805263157894737\\
0.894736842105263	0.0526315789473684	0.00473684210526316	-0.0805263157894737\\
0.894736842105263	0.157894736842105	0.0142105263157895	-0.0805263157894737\\
0.894736842105263	0.263157894736842	0.0236842105263158	-0.0805263157894737\\
0.894736842105263	0.368421052631579	0.0331578947368421	-0.0805263157894737\\
0.894736842105263	0.473684210526316	0.0426315789473684	-0.0805263157894737\\
0.894736842105263	0.578947368421053	0.0521052631578948	-0.0805263157894737\\
0.894736842105263	0.684210526315789	0.0615789473684211	-0.0805263157894737\\
0.894736842105263	0.789473684210526	0.0710526315789474	-0.0805263157894737\\
0.894736842105263	0.894736842105263	0.0805263157894737	-0.0805263157894737\\
0.894736842105263	1	0.09	-0.0805263157894737\\
1	-1	-0.09	-0.09\\
1	-0.894736842105263	-0.0805263157894737	-0.09\\
1	-0.789473684210526	-0.0710526315789474	-0.09\\
1	-0.684210526315789	-0.0615789473684211	-0.09\\
1	-0.578947368421053	-0.0521052631578948	-0.09\\
1	-0.473684210526316	-0.0426315789473684	-0.09\\
1	-0.368421052631579	-0.0331578947368421	-0.09\\
1	-0.263157894736842	-0.0236842105263158	-0.09\\
1	-0.157894736842105	-0.0142105263157895	-0.09\\
1	-0.0526315789473684	-0.00473684210526316	-0.09\\
1	0.0526315789473684	0.00473684210526316	-0.09\\
1	0.157894736842105	0.0142105263157895	-0.09\\
1	0.263157894736842	0.0236842105263158	-0.09\\
1	0.368421052631579	0.0331578947368421	-0.09\\
1	0.473684210526316	0.0426315789473684	-0.09\\
1	0.578947368421053	0.0521052631578948	-0.09\\
1	0.684210526315789	0.0615789473684211	-0.09\\
1	0.789473684210526	0.0710526315789474	-0.09\\
1	0.894736842105263	0.0805263157894737	-0.09\\
1	1	0.09	-0.09\\
};
\end{axis}

\begin{axis}[%
width=2.367in,
height=2.367in,
at={(0.388in,0.319in)},
scale only axis,
xmin=-1,
xmax=1,
ymin=-1,
ymax=1,
axis background/.style={fill=white},
%title style={font=\bfseries},
title={$\quatj\quad\qty(y\pdv{}{x} + x\pdv{}{y})$},
axis lines = box,
]
\addplot [color=black!40, line width=0.4pt, forget plot]
  table[row sep=crcr]{%
-1	0.4\\
-0.9649	0.3018\\
-0.9395	0.2067\\
-0.9235	0.1136\\
-0.9168	0.0217\\
-0.9192	-0.07\\
-0.9308	-0.1625\\
-0.9517	-0.2565\\
-0.9822	-0.3531\\
-1.0225	-0.4533\\
-1.073	-0.558\\
-1.1343	-0.6682\\
};
\addplot [color=black!40, line width=0.4pt, forget plot]
  table[row sep=crcr]{%
-1	0.4857\\
-0.9564	0.388\\
-0.9223	0.2941\\
-0.8974	0.2032\\
-0.8816	0.1143\\
-0.8745	0.0266\\
-0.8762	-0.0609\\
-0.8867	-0.1489\\
-0.9061	-0.2385\\
-0.9345	-0.3304\\
-0.9723	-0.4257\\
-1.0198	-0.5252\\
-1.0775	-0.63\\
-1.146	-0.7411\\
};
\addplot [color=black!40, line width=0.4pt, forget plot]
  table[row sep=crcr]{%
-1	0.5714\\
-0.9478	0.4741\\
-0.905	0.3816\\
-0.8713	0.2928\\
-0.8464	0.207\\
-0.8299	0.1233\\
-0.8217	0.0408\\
-0.8217	-0.0413\\
-0.8299	-0.1239\\
-0.8465	-0.2076\\
-0.8715	-0.2934\\
-0.9053	-0.3822\\
-0.9481	-0.4748\\
-1.0004	-0.5721\\
-1.0627	-0.6752\\
-1.1357	-0.785\\
};
\addplot [color=black!40, line width=0.4pt, forget plot]
  table[row sep=crcr]{%
-1	0.6571\\
-0.9392	0.5603\\
-0.8878	0.469\\
-0.8452	0.3824\\
-0.8111	0.2997\\
-0.7852	0.2199\\
-0.7671	0.1424\\
-0.7567	0.0662\\
-0.7538	-0.0092\\
-0.7585	-0.0848\\
-0.7708	-0.1612\\
-0.7908	-0.2392\\
-0.8187	-0.3196\\
-0.8548	-0.4032\\
-0.8995	-0.4909\\
-0.9532	-0.5834\\
-1.0164	-0.6818\\
-1.0898	-0.787\\
-1.174	-0.9001\\
};
\addplot [color=black!40, line width=0.4pt, forget plot]
  table[row sep=crcr]{%
-1	0.7429\\
-0.9306	0.6464\\
-0.8705	0.5564\\
-0.8191	0.472\\
-0.7759	0.3923\\
-0.7405	0.3166\\
-0.7125	0.244\\
-0.6916	0.1738\\
-0.6777	0.1054\\
-0.6705	0.0381\\
-0.6701	-0.0289\\
-0.6763	-0.0962\\
-0.6893	-0.1644\\
-0.7093	-0.2343\\
-0.7363	-0.3065\\
-0.7707	-0.3818\\
-0.8128	-0.4609\\
-0.863	-0.5446\\
-0.9219	-0.6338\\
-0.99	-0.7293\\
-1.068	-0.8321\\
-1.1566	-0.9432\\
};
\addplot [color=black!40, line width=0.4pt, forget plot]
  table[row sep=crcr]{%
-1	0.8286\\
-0.922	0.7326\\
-0.8532	0.6439\\
-0.793	0.5616\\
-0.7407	0.485\\
-0.6959	0.4132\\
-0.658	0.3456\\
-0.6266	0.2814\\
-0.6016	0.2201\\
-0.5825	0.1609\\
-0.5693	0.1034\\
-0.5618	0.0468\\
-0.56	-0.0092\\
-0.5637	-0.0653\\
-0.573	-0.1221\\
-0.5881	-0.1801\\
-0.6091	-0.24\\
-0.6362	-0.3022\\
-0.6697	-0.3674\\
-0.7098	-0.4363\\
-0.7571	-0.5096\\
-0.8119	-0.588\\
-0.8749	-0.6723\\
-0.9466	-0.7633\\
-1.0278	-0.8619\\
-1.1193	-0.9692\\
};
\addplot [color=black!40, line width=0.4pt, forget plot]
  table[row sep=crcr]{%
-1	0.9143\\
-0.9134	0.8187\\
-0.836	0.7313\\
-0.7669	0.6512\\
-0.7055	0.5777\\
-0.6512	0.5099\\
-0.6034	0.4472\\
-0.5616	0.389\\
-0.5255	0.3347\\
-0.4946	0.2837\\
-0.4686	0.2356\\
-0.4474	0.1899\\
-0.4306	0.146\\
-0.4181	0.1036\\
-0.4098	0.0622\\
-0.4056	0.0215\\
-0.4055	-0.019\\
-0.4095	-0.0597\\
-0.4175	-0.1011\\
-0.4297	-0.1434\\
-0.4462	-0.1871\\
-0.4672	-0.2328\\
-0.4928	-0.2807\\
-0.5234	-0.3315\\
-0.5593	-0.3856\\
-0.6007	-0.4435\\
-0.6481	-0.5059\\
-0.702	-0.5734\\
-0.763	-0.6466\\
-0.8316	-0.7262\\
-0.9085	-0.8132\\
-0.9945	-0.9082\\
-1.0904	-1.0124\\
-1.1973	-1.1267\\
};
\addplot [color=black!40, line width=0.4pt, forget plot]
  table[row sep=crcr]{%
-1	1\\
-0.9048	0.9048\\
-0.8187	0.8187\\
-0.7408	0.7408\\
-0.6703	0.6703\\
-0.6065	0.6065\\
-0.5488	0.5488\\
-0.4966	0.4966\\
-0.4493	0.4493\\
-0.4066	0.4066\\
-0.3679	0.3679\\
-0.3329	0.3329\\
-0.3012	0.3012\\
-0.2725	0.2725\\
-0.2466	0.2466\\
-0.2231	0.2231\\
-0.2019	0.2019\\
-0.1827	0.1827\\
-0.1653	0.1653\\
-0.1496	0.1496\\
-0.1353	0.1353\\
-0.1225	0.1225\\
-0.1108	0.1108\\
-0.1003	0.1003\\
-0.0907	0.0907\\
-0.0821	0.0821\\
-0.0743	0.0743\\
-0.0672	0.0672\\
-0.0608	0.0608\\
-0.055	0.055\\
-0.0498	0.0498\\
-0.045	0.045\\
-0.0408	0.0408\\
-0.0369	0.0369\\
-0.0334	0.0334\\
-0.0302	0.0302\\
-0.0273	0.0273\\
-0.0247	0.0247\\
-0.0224	0.0224\\
-0.0202	0.0202\\
-0.0183	0.0183\\
-0.0166	0.0166\\
-0.015	0.015\\
-0.0136	0.0136\\
-0.0123	0.0123\\
-0.0111	0.0111\\
-0.0101	0.0101\\
-0.0091	0.0091\\
-0.0082	0.0082\\
-0.0074	0.0074\\
-0.0067	0.0067\\
-0.0061	0.0061\\
-0.0055	0.0055\\
-0.005	0.005\\
-0.0045	0.0045\\
-0.0041	0.0041\\
-0.0037	0.0037\\
-0.0033	0.0033\\
-0.003	0.003\\
-0.0027	0.0027\\
-0.0025	0.0025\\
-0.0022	0.0022\\
-0.002	0.002\\
-0.0018	0.0018\\
-0.0017	0.0017\\
-0.0015	0.0015\\
-0.0014	0.0014\\
-0.0012	0.0012\\
-0.0011	0.0011\\
-0.001	0.001\\
-0.0009	0.0009\\
-0.0008	0.0008\\
-0.0007	0.0007\\
-0.0007	0.0007\\
-0.0006	0.0006\\
-0.0006	0.0006\\
-0.0005	0.0005\\
-0.0005	0.0005\\
-0.0004	0.0004\\
-0.0004	0.0004\\
-0.0003	0.0003\\
};
\addplot [color=black!40, line width=0.4pt, forget plot]
  table[row sep=crcr]{%
0.4	-1\\
0.3018	-0.9649\\
0.2067	-0.9395\\
0.1136	-0.9235\\
0.0217	-0.9168\\
-0.07	-0.9192\\
-0.1625	-0.9308\\
-0.2565	-0.9517\\
-0.3531	-0.9822\\
-0.4533	-1.0225\\
-0.558	-1.073\\
-0.6682	-1.1343\\
};
\addplot [color=black!40, line width=0.4pt, forget plot]
  table[row sep=crcr]{%
0.4857	-1\\
0.388	-0.9564\\
0.2941	-0.9223\\
0.2032	-0.8974\\
0.1143	-0.8816\\
0.0266	-0.8745\\
-0.0609	-0.8762\\
-0.1489	-0.8867\\
-0.2385	-0.9061\\
-0.3304	-0.9345\\
-0.4257	-0.9723\\
-0.5252	-1.0198\\
-0.63	-1.0775\\
-0.7411	-1.146\\
};
\addplot [color=black!40, line width=0.4pt, forget plot]
  table[row sep=crcr]{%
0.5714	-1\\
0.4741	-0.9478\\
0.3816	-0.905\\
0.2928	-0.8713\\
0.207	-0.8464\\
0.1233	-0.8299\\
0.0408	-0.8217\\
-0.0413	-0.8217\\
-0.1239	-0.8299\\
-0.2076	-0.8465\\
-0.2934	-0.8715\\
-0.3822	-0.9053\\
-0.4748	-0.9481\\
-0.5721	-1.0004\\
-0.6752	-1.0627\\
-0.785	-1.1357\\
};
\addplot [color=black!40, line width=0.4pt, forget plot]
  table[row sep=crcr]{%
0.6571	-1\\
0.5603	-0.9392\\
0.469	-0.8878\\
0.3824	-0.8452\\
0.2997	-0.8111\\
0.2199	-0.7852\\
0.1424	-0.7671\\
0.0662	-0.7567\\
-0.0092	-0.7538\\
-0.0848	-0.7585\\
-0.1612	-0.7708\\
-0.2392	-0.7908\\
-0.3196	-0.8187\\
-0.4032	-0.8548\\
-0.4909	-0.8995\\
-0.5834	-0.9532\\
-0.6818	-1.0164\\
-0.787	-1.0898\\
-0.9001	-1.174\\
};
\addplot [color=black!40, line width=0.4pt, forget plot]
  table[row sep=crcr]{%
0.7429	-1\\
0.6464	-0.9306\\
0.5564	-0.8705\\
0.472	-0.8191\\
0.3923	-0.7759\\
0.3166	-0.7405\\
0.244	-0.7125\\
0.1738	-0.6916\\
0.1054	-0.6777\\
0.0381	-0.6705\\
-0.0289	-0.6701\\
-0.0962	-0.6763\\
-0.1644	-0.6893\\
-0.2343	-0.7093\\
-0.3065	-0.7363\\
-0.3818	-0.7707\\
-0.4609	-0.8128\\
-0.5446	-0.863\\
-0.6338	-0.9219\\
-0.7293	-0.99\\
-0.8321	-1.068\\
-0.9432	-1.1566\\
};
\addplot [color=black!40, line width=0.4pt, forget plot]
  table[row sep=crcr]{%
0.8286	-1\\
0.7326	-0.922\\
0.6439	-0.8532\\
0.5616	-0.793\\
0.485	-0.7407\\
0.4132	-0.6959\\
0.3456	-0.658\\
0.2814	-0.6266\\
0.2201	-0.6016\\
0.1609	-0.5825\\
0.1034	-0.5693\\
0.0468	-0.5618\\
-0.0092	-0.56\\
-0.0653	-0.5637\\
-0.1221	-0.573\\
-0.1801	-0.5881\\
-0.24	-0.6091\\
-0.3022	-0.6362\\
-0.3674	-0.6697\\
-0.4363	-0.7098\\
-0.5096	-0.7571\\
-0.588	-0.8119\\
-0.6723	-0.8749\\
-0.7633	-0.9466\\
-0.8619	-1.0278\\
-0.9692	-1.1193\\
};
\addplot [color=black!40, line width=0.4pt, forget plot]
  table[row sep=crcr]{%
0.9143	-1\\
0.8187	-0.9134\\
0.7313	-0.836\\
0.6512	-0.7669\\
0.5777	-0.7055\\
0.5099	-0.6512\\
0.4472	-0.6034\\
0.389	-0.5616\\
0.3347	-0.5255\\
0.2837	-0.4946\\
0.2356	-0.4686\\
0.1899	-0.4474\\
0.146	-0.4306\\
0.1036	-0.4181\\
0.0622	-0.4098\\
0.0215	-0.4056\\
-0.019	-0.4055\\
-0.0597	-0.4095\\
-0.1011	-0.4175\\
-0.1434	-0.4297\\
-0.1871	-0.4462\\
-0.2328	-0.4672\\
-0.2807	-0.4928\\
-0.3315	-0.5234\\
-0.3856	-0.5593\\
-0.4435	-0.6007\\
-0.5059	-0.6481\\
-0.5734	-0.702\\
-0.6466	-0.763\\
-0.7262	-0.8316\\
-0.8132	-0.9085\\
-0.9082	-0.9945\\
-1.0124	-1.0904\\
-1.1267	-1.1973\\
};
\addplot [color=black!40, line width=0.4pt, forget plot]
  table[row sep=crcr]{%
1	-1\\
0.9048	-0.9048\\
0.8187	-0.8187\\
0.7408	-0.7408\\
0.6703	-0.6703\\
0.6065	-0.6065\\
0.5488	-0.5488\\
0.4966	-0.4966\\
0.4493	-0.4493\\
0.4066	-0.4066\\
0.3679	-0.3679\\
0.3329	-0.3329\\
0.3012	-0.3012\\
0.2725	-0.2725\\
0.2466	-0.2466\\
0.2231	-0.2231\\
0.2019	-0.2019\\
0.1827	-0.1827\\
0.1653	-0.1653\\
0.1496	-0.1496\\
0.1353	-0.1353\\
0.1225	-0.1225\\
0.1108	-0.1108\\
0.1003	-0.1003\\
0.0907	-0.0907\\
0.0821	-0.0821\\
0.0743	-0.0743\\
0.0672	-0.0672\\
0.0608	-0.0608\\
0.055	-0.055\\
0.0498	-0.0498\\
0.045	-0.045\\
0.0408	-0.0408\\
0.0369	-0.0369\\
0.0334	-0.0334\\
0.0302	-0.0302\\
0.0273	-0.0273\\
0.0247	-0.0247\\
0.0224	-0.0224\\
0.0202	-0.0202\\
0.0183	-0.0183\\
0.0166	-0.0166\\
0.015	-0.015\\
0.0136	-0.0136\\
0.0123	-0.0123\\
0.0111	-0.0111\\
0.0101	-0.0101\\
0.0091	-0.0091\\
0.0082	-0.0082\\
0.0074	-0.0074\\
0.0067	-0.0067\\
0.0061	-0.0061\\
0.0055	-0.0055\\
0.005	-0.005\\
0.0045	-0.0045\\
0.0041	-0.0041\\
0.0037	-0.0037\\
0.0033	-0.0033\\
0.003	-0.003\\
0.0027	-0.0027\\
0.0025	-0.0025\\
0.0022	-0.0022\\
0.002	-0.002\\
0.0018	-0.0018\\
0.0017	-0.0017\\
0.0015	-0.0015\\
0.0014	-0.0014\\
0.0012	-0.0012\\
0.0011	-0.0011\\
0.001	-0.001\\
0.0009	-0.0009\\
0.0008	-0.0008\\
0.0007	-0.0007\\
0.0007	-0.0007\\
0.0006	-0.0006\\
0.0006	-0.0006\\
0.0005	-0.0005\\
0.0005	-0.0005\\
0.0004	-0.0004\\
0.0004	-0.0004\\
0.0003	-0.0003\\
};
\addplot [color=black!40, line width=0.4pt, forget plot]
  table[row sep=crcr]{%
-0.4	1\\
-0.3018	0.9649\\
-0.2067	0.9395\\
-0.1136	0.9235\\
-0.0217	0.9168\\
0.07	0.9192\\
0.1625	0.9308\\
0.2565	0.9517\\
0.3531	0.9822\\
0.4533	1.0225\\
0.558	1.073\\
0.6682	1.1343\\
};
\addplot [color=black!40, line width=0.4pt, forget plot]
  table[row sep=crcr]{%
-0.4857	1\\
-0.388	0.9564\\
-0.2941	0.9223\\
-0.2032	0.8974\\
-0.1143	0.8816\\
-0.0266	0.8745\\
0.0609	0.8762\\
0.1489	0.8867\\
0.2385	0.9061\\
0.3304	0.9345\\
0.4257	0.9723\\
0.5252	1.0198\\
0.63	1.0775\\
0.7411	1.146\\
};
\addplot [color=black!40, line width=0.4pt, forget plot]
  table[row sep=crcr]{%
-0.5714	1\\
-0.4741	0.9478\\
-0.3816	0.905\\
-0.2928	0.8713\\
-0.207	0.8464\\
-0.1233	0.8299\\
-0.0408	0.8217\\
0.0413	0.8217\\
0.1239	0.8299\\
0.2076	0.8465\\
0.2934	0.8715\\
0.3822	0.9053\\
0.4748	0.9481\\
0.5721	1.0004\\
0.6752	1.0627\\
0.785	1.1357\\
};
\addplot [color=black!40, line width=0.4pt, forget plot]
  table[row sep=crcr]{%
-0.6571	1\\
-0.5603	0.9392\\
-0.469	0.8878\\
-0.3824	0.8452\\
-0.2997	0.8111\\
-0.2199	0.7852\\
-0.1424	0.7671\\
-0.0662	0.7567\\
0.0092	0.7538\\
0.0848	0.7585\\
0.1612	0.7708\\
0.2392	0.7908\\
0.3196	0.8187\\
0.4032	0.8548\\
0.4909	0.8995\\
0.5834	0.9532\\
0.6818	1.0164\\
0.787	1.0898\\
0.9001	1.174\\
};
\addplot [color=black!40, line width=0.4pt, forget plot]
  table[row sep=crcr]{%
-0.7429	1\\
-0.6464	0.9306\\
-0.5564	0.8705\\
-0.472	0.8191\\
-0.3923	0.7759\\
-0.3166	0.7405\\
-0.244	0.7125\\
-0.1738	0.6916\\
-0.1054	0.6777\\
-0.0381	0.6705\\
0.0289	0.6701\\
0.0962	0.6763\\
0.1644	0.6893\\
0.2343	0.7093\\
0.3065	0.7363\\
0.3818	0.7707\\
0.4609	0.8128\\
0.5446	0.863\\
0.6338	0.9219\\
0.7293	0.99\\
0.8321	1.068\\
0.9432	1.1566\\
};
\addplot [color=black!40, line width=0.4pt, forget plot]
  table[row sep=crcr]{%
-0.8286	1\\
-0.7326	0.922\\
-0.6439	0.8532\\
-0.5616	0.793\\
-0.485	0.7407\\
-0.4132	0.6959\\
-0.3456	0.658\\
-0.2814	0.6266\\
-0.2201	0.6016\\
-0.1609	0.5825\\
-0.1034	0.5693\\
-0.0468	0.5618\\
0.0092	0.56\\
0.0653	0.5637\\
0.1221	0.573\\
0.1801	0.5881\\
0.24	0.6091\\
0.3022	0.6362\\
0.3674	0.6697\\
0.4363	0.7098\\
0.5096	0.7571\\
0.588	0.8119\\
0.6723	0.8749\\
0.7633	0.9466\\
0.8619	1.0278\\
0.9692	1.1193\\
};
\addplot [color=black!40, line width=0.4pt, forget plot]
  table[row sep=crcr]{%
-0.9143	1\\
-0.8187	0.9134\\
-0.7313	0.836\\
-0.6512	0.7669\\
-0.5777	0.7055\\
-0.5099	0.6512\\
-0.4472	0.6034\\
-0.389	0.5616\\
-0.3347	0.5255\\
-0.2837	0.4946\\
-0.2356	0.4686\\
-0.1899	0.4474\\
-0.146	0.4306\\
-0.1036	0.4181\\
-0.0622	0.4098\\
-0.0215	0.4056\\
0.019	0.4055\\
0.0597	0.4095\\
0.1011	0.4175\\
0.1434	0.4297\\
0.1871	0.4462\\
0.2328	0.4672\\
0.2807	0.4928\\
0.3315	0.5234\\
0.3856	0.5593\\
0.4435	0.6007\\
0.5059	0.6481\\
0.5734	0.702\\
0.6466	0.763\\
0.7262	0.8316\\
0.8132	0.9085\\
0.9082	0.9945\\
1.0124	1.0904\\
1.1267	1.1973\\
};
\addplot [color=black!40, line width=0.4pt, forget plot]
  table[row sep=crcr]{%
-1	1\\
-0.9048	0.9048\\
-0.8187	0.8187\\
-0.7408	0.7408\\
-0.6703	0.6703\\
-0.6065	0.6065\\
-0.5488	0.5488\\
-0.4966	0.4966\\
-0.4493	0.4493\\
-0.4066	0.4066\\
-0.3679	0.3679\\
-0.3329	0.3329\\
-0.3012	0.3012\\
-0.2725	0.2725\\
-0.2466	0.2466\\
-0.2231	0.2231\\
-0.2019	0.2019\\
-0.1827	0.1827\\
-0.1653	0.1653\\
-0.1496	0.1496\\
-0.1353	0.1353\\
-0.1225	0.1225\\
-0.1108	0.1108\\
-0.1003	0.1003\\
-0.0907	0.0907\\
-0.0821	0.0821\\
-0.0743	0.0743\\
-0.0672	0.0672\\
-0.0608	0.0608\\
-0.055	0.055\\
-0.0498	0.0498\\
-0.045	0.045\\
-0.0408	0.0408\\
-0.0369	0.0369\\
-0.0334	0.0334\\
-0.0302	0.0302\\
-0.0273	0.0273\\
-0.0247	0.0247\\
-0.0224	0.0224\\
-0.0202	0.0202\\
-0.0183	0.0183\\
-0.0166	0.0166\\
-0.015	0.015\\
-0.0136	0.0136\\
-0.0123	0.0123\\
-0.0111	0.0111\\
-0.0101	0.0101\\
-0.0091	0.0091\\
-0.0082	0.0082\\
-0.0074	0.0074\\
-0.0067	0.0067\\
-0.0061	0.0061\\
-0.0055	0.0055\\
-0.005	0.005\\
-0.0045	0.0045\\
-0.0041	0.0041\\
-0.0037	0.0037\\
-0.0033	0.0033\\
-0.003	0.003\\
-0.0027	0.0027\\
-0.0025	0.0025\\
-0.0022	0.0022\\
-0.002	0.002\\
-0.0018	0.0018\\
-0.0017	0.0017\\
-0.0015	0.0015\\
-0.0014	0.0014\\
-0.0012	0.0012\\
-0.0011	0.0011\\
-0.001	0.001\\
-0.0009	0.0009\\
-0.0008	0.0008\\
-0.0007	0.0007\\
-0.0007	0.0007\\
-0.0006	0.0006\\
-0.0006	0.0006\\
-0.0005	0.0005\\
-0.0005	0.0005\\
-0.0004	0.0004\\
-0.0004	0.0004\\
-0.0003	0.0003\\
};
\addplot [color=black!40, line width=0.4pt, forget plot]
  table[row sep=crcr]{%
1	-0.4\\
0.9649	-0.3018\\
0.9395	-0.2067\\
0.9235	-0.1136\\
0.9168	-0.0217\\
0.9192	0.07\\
0.9308	0.1625\\
0.9517	0.2565\\
0.9822	0.3531\\
1.0225	0.4533\\
1.073	0.558\\
1.1343	0.6682\\
};
\addplot [color=black!40, line width=0.4pt, forget plot]
  table[row sep=crcr]{%
1	-0.4857\\
0.9564	-0.388\\
0.9223	-0.2941\\
0.8974	-0.2032\\
0.8816	-0.1143\\
0.8745	-0.0266\\
0.8762	0.0609\\
0.8867	0.1489\\
0.9061	0.2385\\
0.9345	0.3304\\
0.9723	0.4257\\
1.0198	0.5252\\
1.0775	0.63\\
1.146	0.7411\\
};
\addplot [color=black!40, line width=0.4pt, forget plot]
  table[row sep=crcr]{%
1	-0.5714\\
0.9478	-0.4741\\
0.905	-0.3816\\
0.8713	-0.2928\\
0.8464	-0.207\\
0.8299	-0.1233\\
0.8217	-0.0408\\
0.8217	0.0413\\
0.8299	0.1239\\
0.8465	0.2076\\
0.8715	0.2934\\
0.9053	0.3822\\
0.9481	0.4748\\
1.0004	0.5721\\
1.0627	0.6752\\
1.1357	0.785\\
};
\addplot [color=black!40, line width=0.4pt, forget plot]
  table[row sep=crcr]{%
1	-0.6571\\
0.9392	-0.5603\\
0.8878	-0.469\\
0.8452	-0.3824\\
0.8111	-0.2997\\
0.7852	-0.2199\\
0.7671	-0.1424\\
0.7567	-0.0662\\
0.7538	0.0092\\
0.7585	0.0848\\
0.7708	0.1612\\
0.7908	0.2392\\
0.8187	0.3196\\
0.8548	0.4032\\
0.8995	0.4909\\
0.9532	0.5834\\
1.0164	0.6818\\
1.0898	0.787\\
1.174	0.9001\\
};
\addplot [color=black!40, line width=0.4pt, forget plot]
  table[row sep=crcr]{%
1	-0.7429\\
0.9306	-0.6464\\
0.8705	-0.5564\\
0.8191	-0.472\\
0.7759	-0.3923\\
0.7405	-0.3166\\
0.7125	-0.244\\
0.6916	-0.1738\\
0.6777	-0.1054\\
0.6705	-0.0381\\
0.6701	0.0289\\
0.6763	0.0962\\
0.6893	0.1644\\
0.7093	0.2343\\
0.7363	0.3065\\
0.7707	0.3818\\
0.8128	0.4609\\
0.863	0.5446\\
0.9219	0.6338\\
0.99	0.7293\\
1.068	0.8321\\
1.1566	0.9432\\
};
\addplot [color=black!40, line width=0.4pt, forget plot]
  table[row sep=crcr]{%
1	-0.8286\\
0.922	-0.7326\\
0.8532	-0.6439\\
0.793	-0.5616\\
0.7407	-0.485\\
0.6959	-0.4132\\
0.658	-0.3456\\
0.6266	-0.2814\\
0.6016	-0.2201\\
0.5825	-0.1609\\
0.5693	-0.1034\\
0.5618	-0.0468\\
0.56	0.0092\\
0.5637	0.0653\\
0.573	0.1221\\
0.5881	0.1801\\
0.6091	0.24\\
0.6362	0.3022\\
0.6697	0.3674\\
0.7098	0.4363\\
0.7571	0.5096\\
0.8119	0.588\\
0.8749	0.6723\\
0.9466	0.7633\\
1.0278	0.8619\\
1.1193	0.9692\\
};
\addplot [color=black!40, line width=0.4pt, forget plot]
  table[row sep=crcr]{%
1	-0.9143\\
0.9134	-0.8187\\
0.836	-0.7313\\
0.7669	-0.6512\\
0.7055	-0.5777\\
0.6512	-0.5099\\
0.6034	-0.4472\\
0.5616	-0.389\\
0.5255	-0.3347\\
0.4946	-0.2837\\
0.4686	-0.2356\\
0.4474	-0.1899\\
0.4306	-0.146\\
0.4181	-0.1036\\
0.4098	-0.0622\\
0.4056	-0.0215\\
0.4055	0.019\\
0.4095	0.0597\\
0.4175	0.1011\\
0.4297	0.1434\\
0.4462	0.1871\\
0.4672	0.2328\\
0.4928	0.2807\\
0.5234	0.3315\\
0.5593	0.3856\\
0.6007	0.4435\\
0.6481	0.5059\\
0.702	0.5734\\
0.763	0.6466\\
0.8316	0.7262\\
0.9085	0.8132\\
0.9945	0.9082\\
1.0904	1.0124\\
1.1973	1.1267\\
};
\addplot [color=black!40, line width=0.4pt, forget plot]
  table[row sep=crcr]{%
1	-1\\
0.9048	-0.9048\\
0.8187	-0.8187\\
0.7408	-0.7408\\
0.6703	-0.6703\\
0.6065	-0.6065\\
0.5488	-0.5488\\
0.4966	-0.4966\\
0.4493	-0.4493\\
0.4066	-0.4066\\
0.3679	-0.3679\\
0.3329	-0.3329\\
0.3012	-0.3012\\
0.2725	-0.2725\\
0.2466	-0.2466\\
0.2231	-0.2231\\
0.2019	-0.2019\\
0.1827	-0.1827\\
0.1653	-0.1653\\
0.1496	-0.1496\\
0.1353	-0.1353\\
0.1225	-0.1225\\
0.1108	-0.1108\\
0.1003	-0.1003\\
0.0907	-0.0907\\
0.0821	-0.0821\\
0.0743	-0.0743\\
0.0672	-0.0672\\
0.0608	-0.0608\\
0.055	-0.055\\
0.0498	-0.0498\\
0.045	-0.045\\
0.0408	-0.0408\\
0.0369	-0.0369\\
0.0334	-0.0334\\
0.0302	-0.0302\\
0.0273	-0.0273\\
0.0247	-0.0247\\
0.0224	-0.0224\\
0.0202	-0.0202\\
0.0183	-0.0183\\
0.0166	-0.0166\\
0.015	-0.015\\
0.0136	-0.0136\\
0.0123	-0.0123\\
0.0111	-0.0111\\
0.0101	-0.0101\\
0.0091	-0.0091\\
0.0082	-0.0082\\
0.0074	-0.0074\\
0.0067	-0.0067\\
0.0061	-0.0061\\
0.0055	-0.0055\\
0.005	-0.005\\
0.0045	-0.0045\\
0.0041	-0.0041\\
0.0037	-0.0037\\
0.0033	-0.0033\\
0.003	-0.003\\
0.0027	-0.0027\\
0.0025	-0.0025\\
0.0022	-0.0022\\
0.002	-0.002\\
0.0018	-0.0018\\
0.0017	-0.0017\\
0.0015	-0.0015\\
0.0014	-0.0014\\
0.0012	-0.0012\\
0.0011	-0.0011\\
0.001	-0.001\\
0.0009	-0.0009\\
0.0008	-0.0008\\
0.0007	-0.0007\\
0.0007	-0.0007\\
0.0006	-0.0006\\
0.0006	-0.0006\\
0.0005	-0.0005\\
0.0005	-0.0005\\
0.0004	-0.0004\\
0.0004	-0.0004\\
0.0003	-0.0003\\
};
\addplot[-stealth, color=accent1, point meta={sqrt((\thisrow{u})^2+(\thisrow{v})^2)}, point meta min=0, quiver={u=\thisrow{u}, v=\thisrow{v}, scale arrows = 1.45, every arrow/.append style={
line width=1pt*\pgfplotspointmetatransformed/1000}}]
 table[row sep=crcr] {%
x	y	u	v\\
-1	-1	-0.09	-0.09\\
-1	-0.894736842105263	-0.0805263157894737	-0.09\\
-1	-0.789473684210526	-0.0710526315789474	-0.09\\
-1	-0.684210526315789	-0.0615789473684211	-0.09\\
-1	-0.578947368421053	-0.0521052631578948	-0.09\\
-1	-0.473684210526316	-0.0426315789473684	-0.09\\
-1	-0.368421052631579	-0.0331578947368421	-0.09\\
-1	-0.263157894736842	-0.0236842105263158	-0.09\\
-1	-0.157894736842105	-0.0142105263157895	-0.09\\
-1	-0.0526315789473684	-0.00473684210526316	-0.09\\
-1	0.0526315789473684	0.00473684210526316	-0.09\\
-1	0.157894736842105	0.0142105263157895	-0.09\\
-1	0.263157894736842	0.0236842105263158	-0.09\\
-1	0.368421052631579	0.0331578947368421	-0.09\\
-1	0.473684210526316	0.0426315789473684	-0.09\\
-1	0.578947368421053	0.0521052631578948	-0.09\\
-1	0.684210526315789	0.0615789473684211	-0.09\\
-1	0.789473684210526	0.0710526315789474	-0.09\\
-1	0.894736842105263	0.0805263157894737	-0.09\\
-1	1	0.09	-0.09\\
-0.894736842105263	-1	-0.09	-0.0805263157894737\\
-0.894736842105263	-0.894736842105263	-0.0805263157894737	-0.0805263157894737\\
-0.894736842105263	-0.789473684210526	-0.0710526315789474	-0.0805263157894737\\
-0.894736842105263	-0.684210526315789	-0.0615789473684211	-0.0805263157894737\\
-0.894736842105263	-0.578947368421053	-0.0521052631578948	-0.0805263157894737\\
-0.894736842105263	-0.473684210526316	-0.0426315789473684	-0.0805263157894737\\
-0.894736842105263	-0.368421052631579	-0.0331578947368421	-0.0805263157894737\\
-0.894736842105263	-0.263157894736842	-0.0236842105263158	-0.0805263157894737\\
-0.894736842105263	-0.157894736842105	-0.0142105263157895	-0.0805263157894737\\
-0.894736842105263	-0.0526315789473684	-0.00473684210526316	-0.0805263157894737\\
-0.894736842105263	0.0526315789473684	0.00473684210526316	-0.0805263157894737\\
-0.894736842105263	0.157894736842105	0.0142105263157895	-0.0805263157894737\\
-0.894736842105263	0.263157894736842	0.0236842105263158	-0.0805263157894737\\
-0.894736842105263	0.368421052631579	0.0331578947368421	-0.0805263157894737\\
-0.894736842105263	0.473684210526316	0.0426315789473684	-0.0805263157894737\\
-0.894736842105263	0.578947368421053	0.0521052631578948	-0.0805263157894737\\
-0.894736842105263	0.684210526315789	0.0615789473684211	-0.0805263157894737\\
-0.894736842105263	0.789473684210526	0.0710526315789474	-0.0805263157894737\\
-0.894736842105263	0.894736842105263	0.0805263157894737	-0.0805263157894737\\
-0.894736842105263	1	0.09	-0.0805263157894737\\
-0.789473684210526	-1	-0.09	-0.0710526315789474\\
-0.789473684210526	-0.894736842105263	-0.0805263157894737	-0.0710526315789474\\
-0.789473684210526	-0.789473684210526	-0.0710526315789474	-0.0710526315789474\\
-0.789473684210526	-0.684210526315789	-0.0615789473684211	-0.0710526315789474\\
-0.789473684210526	-0.578947368421053	-0.0521052631578948	-0.0710526315789474\\
-0.789473684210526	-0.473684210526316	-0.0426315789473684	-0.0710526315789474\\
-0.789473684210526	-0.368421052631579	-0.0331578947368421	-0.0710526315789474\\
-0.789473684210526	-0.263157894736842	-0.0236842105263158	-0.0710526315789474\\
-0.789473684210526	-0.157894736842105	-0.0142105263157895	-0.0710526315789474\\
-0.789473684210526	-0.0526315789473684	-0.00473684210526316	-0.0710526315789474\\
-0.789473684210526	0.0526315789473684	0.00473684210526316	-0.0710526315789474\\
-0.789473684210526	0.157894736842105	0.0142105263157895	-0.0710526315789474\\
-0.789473684210526	0.263157894736842	0.0236842105263158	-0.0710526315789474\\
-0.789473684210526	0.368421052631579	0.0331578947368421	-0.0710526315789474\\
-0.789473684210526	0.473684210526316	0.0426315789473684	-0.0710526315789474\\
-0.789473684210526	0.578947368421053	0.0521052631578948	-0.0710526315789474\\
-0.789473684210526	0.684210526315789	0.0615789473684211	-0.0710526315789474\\
-0.789473684210526	0.789473684210526	0.0710526315789474	-0.0710526315789474\\
-0.789473684210526	0.894736842105263	0.0805263157894737	-0.0710526315789474\\
-0.789473684210526	1	0.09	-0.0710526315789474\\
-0.684210526315789	-1	-0.09	-0.0615789473684211\\
-0.684210526315789	-0.894736842105263	-0.0805263157894737	-0.0615789473684211\\
-0.684210526315789	-0.789473684210526	-0.0710526315789474	-0.0615789473684211\\
-0.684210526315789	-0.684210526315789	-0.0615789473684211	-0.0615789473684211\\
-0.684210526315789	-0.578947368421053	-0.0521052631578948	-0.0615789473684211\\
-0.684210526315789	-0.473684210526316	-0.0426315789473684	-0.0615789473684211\\
-0.684210526315789	-0.368421052631579	-0.0331578947368421	-0.0615789473684211\\
-0.684210526315789	-0.263157894736842	-0.0236842105263158	-0.0615789473684211\\
-0.684210526315789	-0.157894736842105	-0.0142105263157895	-0.0615789473684211\\
-0.684210526315789	-0.0526315789473684	-0.00473684210526316	-0.0615789473684211\\
-0.684210526315789	0.0526315789473684	0.00473684210526316	-0.0615789473684211\\
-0.684210526315789	0.157894736842105	0.0142105263157895	-0.0615789473684211\\
-0.684210526315789	0.263157894736842	0.0236842105263158	-0.0615789473684211\\
-0.684210526315789	0.368421052631579	0.0331578947368421	-0.0615789473684211\\
-0.684210526315789	0.473684210526316	0.0426315789473684	-0.0615789473684211\\
-0.684210526315789	0.578947368421053	0.0521052631578948	-0.0615789473684211\\
-0.684210526315789	0.684210526315789	0.0615789473684211	-0.0615789473684211\\
-0.684210526315789	0.789473684210526	0.0710526315789474	-0.0615789473684211\\
-0.684210526315789	0.894736842105263	0.0805263157894737	-0.0615789473684211\\
-0.684210526315789	1	0.09	-0.0615789473684211\\
-0.578947368421053	-1	-0.09	-0.0521052631578948\\
-0.578947368421053	-0.894736842105263	-0.0805263157894737	-0.0521052631578948\\
-0.578947368421053	-0.789473684210526	-0.0710526315789474	-0.0521052631578948\\
-0.578947368421053	-0.684210526315789	-0.0615789473684211	-0.0521052631578948\\
-0.578947368421053	-0.578947368421053	-0.0521052631578948	-0.0521052631578948\\
-0.578947368421053	-0.473684210526316	-0.0426315789473684	-0.0521052631578948\\
-0.578947368421053	-0.368421052631579	-0.0331578947368421	-0.0521052631578948\\
-0.578947368421053	-0.263157894736842	-0.0236842105263158	-0.0521052631578948\\
-0.578947368421053	-0.157894736842105	-0.0142105263157895	-0.0521052631578948\\
-0.578947368421053	-0.0526315789473684	-0.00473684210526316	-0.0521052631578948\\
-0.578947368421053	0.0526315789473684	0.00473684210526316	-0.0521052631578948\\
-0.578947368421053	0.157894736842105	0.0142105263157895	-0.0521052631578948\\
-0.578947368421053	0.263157894736842	0.0236842105263158	-0.0521052631578948\\
-0.578947368421053	0.368421052631579	0.0331578947368421	-0.0521052631578948\\
-0.578947368421053	0.473684210526316	0.0426315789473684	-0.0521052631578948\\
-0.578947368421053	0.578947368421053	0.0521052631578948	-0.0521052631578948\\
-0.578947368421053	0.684210526315789	0.0615789473684211	-0.0521052631578948\\
-0.578947368421053	0.789473684210526	0.0710526315789474	-0.0521052631578948\\
-0.578947368421053	0.894736842105263	0.0805263157894737	-0.0521052631578948\\
-0.578947368421053	1	0.09	-0.0521052631578948\\
-0.473684210526316	-1	-0.09	-0.0426315789473684\\
-0.473684210526316	-0.894736842105263	-0.0805263157894737	-0.0426315789473684\\
-0.473684210526316	-0.789473684210526	-0.0710526315789474	-0.0426315789473684\\
-0.473684210526316	-0.684210526315789	-0.0615789473684211	-0.0426315789473684\\
-0.473684210526316	-0.578947368421053	-0.0521052631578948	-0.0426315789473684\\
-0.473684210526316	-0.473684210526316	-0.0426315789473684	-0.0426315789473684\\
-0.473684210526316	-0.368421052631579	-0.0331578947368421	-0.0426315789473684\\
-0.473684210526316	-0.263157894736842	-0.0236842105263158	-0.0426315789473684\\
-0.473684210526316	-0.157894736842105	-0.0142105263157895	-0.0426315789473684\\
-0.473684210526316	-0.0526315789473684	-0.00473684210526316	-0.0426315789473684\\
-0.473684210526316	0.0526315789473684	0.00473684210526316	-0.0426315789473684\\
-0.473684210526316	0.157894736842105	0.0142105263157895	-0.0426315789473684\\
-0.473684210526316	0.263157894736842	0.0236842105263158	-0.0426315789473684\\
-0.473684210526316	0.368421052631579	0.0331578947368421	-0.0426315789473684\\
-0.473684210526316	0.473684210526316	0.0426315789473684	-0.0426315789473684\\
-0.473684210526316	0.578947368421053	0.0521052631578948	-0.0426315789473684\\
-0.473684210526316	0.684210526315789	0.0615789473684211	-0.0426315789473684\\
-0.473684210526316	0.789473684210526	0.0710526315789474	-0.0426315789473684\\
-0.473684210526316	0.894736842105263	0.0805263157894737	-0.0426315789473684\\
-0.473684210526316	1	0.09	-0.0426315789473684\\
-0.368421052631579	-1	-0.09	-0.0331578947368421\\
-0.368421052631579	-0.894736842105263	-0.0805263157894737	-0.0331578947368421\\
-0.368421052631579	-0.789473684210526	-0.0710526315789474	-0.0331578947368421\\
-0.368421052631579	-0.684210526315789	-0.0615789473684211	-0.0331578947368421\\
-0.368421052631579	-0.578947368421053	-0.0521052631578948	-0.0331578947368421\\
-0.368421052631579	-0.473684210526316	-0.0426315789473684	-0.0331578947368421\\
-0.368421052631579	-0.368421052631579	-0.0331578947368421	-0.0331578947368421\\
-0.368421052631579	-0.263157894736842	-0.0236842105263158	-0.0331578947368421\\
-0.368421052631579	-0.157894736842105	-0.0142105263157895	-0.0331578947368421\\
-0.368421052631579	-0.0526315789473684	-0.00473684210526316	-0.0331578947368421\\
-0.368421052631579	0.0526315789473684	0.00473684210526316	-0.0331578947368421\\
-0.368421052631579	0.157894736842105	0.0142105263157895	-0.0331578947368421\\
-0.368421052631579	0.263157894736842	0.0236842105263158	-0.0331578947368421\\
-0.368421052631579	0.368421052631579	0.0331578947368421	-0.0331578947368421\\
-0.368421052631579	0.473684210526316	0.0426315789473684	-0.0331578947368421\\
-0.368421052631579	0.578947368421053	0.0521052631578948	-0.0331578947368421\\
-0.368421052631579	0.684210526315789	0.0615789473684211	-0.0331578947368421\\
-0.368421052631579	0.789473684210526	0.0710526315789474	-0.0331578947368421\\
-0.368421052631579	0.894736842105263	0.0805263157894737	-0.0331578947368421\\
-0.368421052631579	1	0.09	-0.0331578947368421\\
-0.263157894736842	-1	-0.09	-0.0236842105263158\\
-0.263157894736842	-0.894736842105263	-0.0805263157894737	-0.0236842105263158\\
-0.263157894736842	-0.789473684210526	-0.0710526315789474	-0.0236842105263158\\
-0.263157894736842	-0.684210526315789	-0.0615789473684211	-0.0236842105263158\\
-0.263157894736842	-0.578947368421053	-0.0521052631578948	-0.0236842105263158\\
-0.263157894736842	-0.473684210526316	-0.0426315789473684	-0.0236842105263158\\
-0.263157894736842	-0.368421052631579	-0.0331578947368421	-0.0236842105263158\\
-0.263157894736842	-0.263157894736842	-0.0236842105263158	-0.0236842105263158\\
-0.263157894736842	-0.157894736842105	-0.0142105263157895	-0.0236842105263158\\
-0.263157894736842	-0.0526315789473684	-0.00473684210526316	-0.0236842105263158\\
-0.263157894736842	0.0526315789473684	0.00473684210526316	-0.0236842105263158\\
-0.263157894736842	0.157894736842105	0.0142105263157895	-0.0236842105263158\\
-0.263157894736842	0.263157894736842	0.0236842105263158	-0.0236842105263158\\
-0.263157894736842	0.368421052631579	0.0331578947368421	-0.0236842105263158\\
-0.263157894736842	0.473684210526316	0.0426315789473684	-0.0236842105263158\\
-0.263157894736842	0.578947368421053	0.0521052631578948	-0.0236842105263158\\
-0.263157894736842	0.684210526315789	0.0615789473684211	-0.0236842105263158\\
-0.263157894736842	0.789473684210526	0.0710526315789474	-0.0236842105263158\\
-0.263157894736842	0.894736842105263	0.0805263157894737	-0.0236842105263158\\
-0.263157894736842	1	0.09	-0.0236842105263158\\
-0.157894736842105	-1	-0.09	-0.0142105263157895\\
-0.157894736842105	-0.894736842105263	-0.0805263157894737	-0.0142105263157895\\
-0.157894736842105	-0.789473684210526	-0.0710526315789474	-0.0142105263157895\\
-0.157894736842105	-0.684210526315789	-0.0615789473684211	-0.0142105263157895\\
-0.157894736842105	-0.578947368421053	-0.0521052631578948	-0.0142105263157895\\
-0.157894736842105	-0.473684210526316	-0.0426315789473684	-0.0142105263157895\\
-0.157894736842105	-0.368421052631579	-0.0331578947368421	-0.0142105263157895\\
-0.157894736842105	-0.263157894736842	-0.0236842105263158	-0.0142105263157895\\
-0.157894736842105	-0.157894736842105	-0.0142105263157895	-0.0142105263157895\\
-0.157894736842105	-0.0526315789473684	-0.00473684210526316	-0.0142105263157895\\
-0.157894736842105	0.0526315789473684	0.00473684210526316	-0.0142105263157895\\
-0.157894736842105	0.157894736842105	0.0142105263157895	-0.0142105263157895\\
-0.157894736842105	0.263157894736842	0.0236842105263158	-0.0142105263157895\\
-0.157894736842105	0.368421052631579	0.0331578947368421	-0.0142105263157895\\
-0.157894736842105	0.473684210526316	0.0426315789473684	-0.0142105263157895\\
-0.157894736842105	0.578947368421053	0.0521052631578948	-0.0142105263157895\\
-0.157894736842105	0.684210526315789	0.0615789473684211	-0.0142105263157895\\
-0.157894736842105	0.789473684210526	0.0710526315789474	-0.0142105263157895\\
-0.157894736842105	0.894736842105263	0.0805263157894737	-0.0142105263157895\\
-0.157894736842105	1	0.09	-0.0142105263157895\\
-0.0526315789473684	-1	-0.09	-0.00473684210526316\\
-0.0526315789473684	-0.894736842105263	-0.0805263157894737	-0.00473684210526316\\
-0.0526315789473684	-0.789473684210526	-0.0710526315789474	-0.00473684210526316\\
-0.0526315789473684	-0.684210526315789	-0.0615789473684211	-0.00473684210526316\\
-0.0526315789473684	-0.578947368421053	-0.0521052631578948	-0.00473684210526316\\
-0.0526315789473684	-0.473684210526316	-0.0426315789473684	-0.00473684210526316\\
-0.0526315789473684	-0.368421052631579	-0.0331578947368421	-0.00473684210526316\\
-0.0526315789473684	-0.263157894736842	-0.0236842105263158	-0.00473684210526316\\
-0.0526315789473684	-0.157894736842105	-0.0142105263157895	-0.00473684210526316\\
-0.0526315789473684	-0.0526315789473684	-0.00473684210526316	-0.00473684210526316\\
-0.0526315789473684	0.0526315789473684	0.00473684210526316	-0.00473684210526316\\
-0.0526315789473684	0.157894736842105	0.0142105263157895	-0.00473684210526316\\
-0.0526315789473684	0.263157894736842	0.0236842105263158	-0.00473684210526316\\
-0.0526315789473684	0.368421052631579	0.0331578947368421	-0.00473684210526316\\
-0.0526315789473684	0.473684210526316	0.0426315789473684	-0.00473684210526316\\
-0.0526315789473684	0.578947368421053	0.0521052631578948	-0.00473684210526316\\
-0.0526315789473684	0.684210526315789	0.0615789473684211	-0.00473684210526316\\
-0.0526315789473684	0.789473684210526	0.0710526315789474	-0.00473684210526316\\
-0.0526315789473684	0.894736842105263	0.0805263157894737	-0.00473684210526316\\
-0.0526315789473684	1	0.09	-0.00473684210526316\\
0.0526315789473684	-1	-0.09	0.00473684210526316\\
0.0526315789473684	-0.894736842105263	-0.0805263157894737	0.00473684210526316\\
0.0526315789473684	-0.789473684210526	-0.0710526315789474	0.00473684210526316\\
0.0526315789473684	-0.684210526315789	-0.0615789473684211	0.00473684210526316\\
0.0526315789473684	-0.578947368421053	-0.0521052631578948	0.00473684210526316\\
0.0526315789473684	-0.473684210526316	-0.0426315789473684	0.00473684210526316\\
0.0526315789473684	-0.368421052631579	-0.0331578947368421	0.00473684210526316\\
0.0526315789473684	-0.263157894736842	-0.0236842105263158	0.00473684210526316\\
0.0526315789473684	-0.157894736842105	-0.0142105263157895	0.00473684210526316\\
0.0526315789473684	-0.0526315789473684	-0.00473684210526316	0.00473684210526316\\
0.0526315789473684	0.0526315789473684	0.00473684210526316	0.00473684210526316\\
0.0526315789473684	0.157894736842105	0.0142105263157895	0.00473684210526316\\
0.0526315789473684	0.263157894736842	0.0236842105263158	0.00473684210526316\\
0.0526315789473684	0.368421052631579	0.0331578947368421	0.00473684210526316\\
0.0526315789473684	0.473684210526316	0.0426315789473684	0.00473684210526316\\
0.0526315789473684	0.578947368421053	0.0521052631578948	0.00473684210526316\\
0.0526315789473684	0.684210526315789	0.0615789473684211	0.00473684210526316\\
0.0526315789473684	0.789473684210526	0.0710526315789474	0.00473684210526316\\
0.0526315789473684	0.894736842105263	0.0805263157894737	0.00473684210526316\\
0.0526315789473684	1	0.09	0.00473684210526316\\
0.157894736842105	-1	-0.09	0.0142105263157895\\
0.157894736842105	-0.894736842105263	-0.0805263157894737	0.0142105263157895\\
0.157894736842105	-0.789473684210526	-0.0710526315789474	0.0142105263157895\\
0.157894736842105	-0.684210526315789	-0.0615789473684211	0.0142105263157895\\
0.157894736842105	-0.578947368421053	-0.0521052631578948	0.0142105263157895\\
0.157894736842105	-0.473684210526316	-0.0426315789473684	0.0142105263157895\\
0.157894736842105	-0.368421052631579	-0.0331578947368421	0.0142105263157895\\
0.157894736842105	-0.263157894736842	-0.0236842105263158	0.0142105263157895\\
0.157894736842105	-0.157894736842105	-0.0142105263157895	0.0142105263157895\\
0.157894736842105	-0.0526315789473684	-0.00473684210526316	0.0142105263157895\\
0.157894736842105	0.0526315789473684	0.00473684210526316	0.0142105263157895\\
0.157894736842105	0.157894736842105	0.0142105263157895	0.0142105263157895\\
0.157894736842105	0.263157894736842	0.0236842105263158	0.0142105263157895\\
0.157894736842105	0.368421052631579	0.0331578947368421	0.0142105263157895\\
0.157894736842105	0.473684210526316	0.0426315789473684	0.0142105263157895\\
0.157894736842105	0.578947368421053	0.0521052631578948	0.0142105263157895\\
0.157894736842105	0.684210526315789	0.0615789473684211	0.0142105263157895\\
0.157894736842105	0.789473684210526	0.0710526315789474	0.0142105263157895\\
0.157894736842105	0.894736842105263	0.0805263157894737	0.0142105263157895\\
0.157894736842105	1	0.09	0.0142105263157895\\
0.263157894736842	-1	-0.09	0.0236842105263158\\
0.263157894736842	-0.894736842105263	-0.0805263157894737	0.0236842105263158\\
0.263157894736842	-0.789473684210526	-0.0710526315789474	0.0236842105263158\\
0.263157894736842	-0.684210526315789	-0.0615789473684211	0.0236842105263158\\
0.263157894736842	-0.578947368421053	-0.0521052631578948	0.0236842105263158\\
0.263157894736842	-0.473684210526316	-0.0426315789473684	0.0236842105263158\\
0.263157894736842	-0.368421052631579	-0.0331578947368421	0.0236842105263158\\
0.263157894736842	-0.263157894736842	-0.0236842105263158	0.0236842105263158\\
0.263157894736842	-0.157894736842105	-0.0142105263157895	0.0236842105263158\\
0.263157894736842	-0.0526315789473684	-0.00473684210526316	0.0236842105263158\\
0.263157894736842	0.0526315789473684	0.00473684210526316	0.0236842105263158\\
0.263157894736842	0.157894736842105	0.0142105263157895	0.0236842105263158\\
0.263157894736842	0.263157894736842	0.0236842105263158	0.0236842105263158\\
0.263157894736842	0.368421052631579	0.0331578947368421	0.0236842105263158\\
0.263157894736842	0.473684210526316	0.0426315789473684	0.0236842105263158\\
0.263157894736842	0.578947368421053	0.0521052631578948	0.0236842105263158\\
0.263157894736842	0.684210526315789	0.0615789473684211	0.0236842105263158\\
0.263157894736842	0.789473684210526	0.0710526315789474	0.0236842105263158\\
0.263157894736842	0.894736842105263	0.0805263157894737	0.0236842105263158\\
0.263157894736842	1	0.09	0.0236842105263158\\
0.368421052631579	-1	-0.09	0.0331578947368421\\
0.368421052631579	-0.894736842105263	-0.0805263157894737	0.0331578947368421\\
0.368421052631579	-0.789473684210526	-0.0710526315789474	0.0331578947368421\\
0.368421052631579	-0.684210526315789	-0.0615789473684211	0.0331578947368421\\
0.368421052631579	-0.578947368421053	-0.0521052631578948	0.0331578947368421\\
0.368421052631579	-0.473684210526316	-0.0426315789473684	0.0331578947368421\\
0.368421052631579	-0.368421052631579	-0.0331578947368421	0.0331578947368421\\
0.368421052631579	-0.263157894736842	-0.0236842105263158	0.0331578947368421\\
0.368421052631579	-0.157894736842105	-0.0142105263157895	0.0331578947368421\\
0.368421052631579	-0.0526315789473684	-0.00473684210526316	0.0331578947368421\\
0.368421052631579	0.0526315789473684	0.00473684210526316	0.0331578947368421\\
0.368421052631579	0.157894736842105	0.0142105263157895	0.0331578947368421\\
0.368421052631579	0.263157894736842	0.0236842105263158	0.0331578947368421\\
0.368421052631579	0.368421052631579	0.0331578947368421	0.0331578947368421\\
0.368421052631579	0.473684210526316	0.0426315789473684	0.0331578947368421\\
0.368421052631579	0.578947368421053	0.0521052631578948	0.0331578947368421\\
0.368421052631579	0.684210526315789	0.0615789473684211	0.0331578947368421\\
0.368421052631579	0.789473684210526	0.0710526315789474	0.0331578947368421\\
0.368421052631579	0.894736842105263	0.0805263157894737	0.0331578947368421\\
0.368421052631579	1	0.09	0.0331578947368421\\
0.473684210526316	-1	-0.09	0.0426315789473684\\
0.473684210526316	-0.894736842105263	-0.0805263157894737	0.0426315789473684\\
0.473684210526316	-0.789473684210526	-0.0710526315789474	0.0426315789473684\\
0.473684210526316	-0.684210526315789	-0.0615789473684211	0.0426315789473684\\
0.473684210526316	-0.578947368421053	-0.0521052631578948	0.0426315789473684\\
0.473684210526316	-0.473684210526316	-0.0426315789473684	0.0426315789473684\\
0.473684210526316	-0.368421052631579	-0.0331578947368421	0.0426315789473684\\
0.473684210526316	-0.263157894736842	-0.0236842105263158	0.0426315789473684\\
0.473684210526316	-0.157894736842105	-0.0142105263157895	0.0426315789473684\\
0.473684210526316	-0.0526315789473684	-0.00473684210526316	0.0426315789473684\\
0.473684210526316	0.0526315789473684	0.00473684210526316	0.0426315789473684\\
0.473684210526316	0.157894736842105	0.0142105263157895	0.0426315789473684\\
0.473684210526316	0.263157894736842	0.0236842105263158	0.0426315789473684\\
0.473684210526316	0.368421052631579	0.0331578947368421	0.0426315789473684\\
0.473684210526316	0.473684210526316	0.0426315789473684	0.0426315789473684\\
0.473684210526316	0.578947368421053	0.0521052631578948	0.0426315789473684\\
0.473684210526316	0.684210526315789	0.0615789473684211	0.0426315789473684\\
0.473684210526316	0.789473684210526	0.0710526315789474	0.0426315789473684\\
0.473684210526316	0.894736842105263	0.0805263157894737	0.0426315789473684\\
0.473684210526316	1	0.09	0.0426315789473684\\
0.578947368421053	-1	-0.09	0.0521052631578948\\
0.578947368421053	-0.894736842105263	-0.0805263157894737	0.0521052631578948\\
0.578947368421053	-0.789473684210526	-0.0710526315789474	0.0521052631578948\\
0.578947368421053	-0.684210526315789	-0.0615789473684211	0.0521052631578948\\
0.578947368421053	-0.578947368421053	-0.0521052631578948	0.0521052631578948\\
0.578947368421053	-0.473684210526316	-0.0426315789473684	0.0521052631578948\\
0.578947368421053	-0.368421052631579	-0.0331578947368421	0.0521052631578948\\
0.578947368421053	-0.263157894736842	-0.0236842105263158	0.0521052631578948\\
0.578947368421053	-0.157894736842105	-0.0142105263157895	0.0521052631578948\\
0.578947368421053	-0.0526315789473684	-0.00473684210526316	0.0521052631578948\\
0.578947368421053	0.0526315789473684	0.00473684210526316	0.0521052631578948\\
0.578947368421053	0.157894736842105	0.0142105263157895	0.0521052631578948\\
0.578947368421053	0.263157894736842	0.0236842105263158	0.0521052631578948\\
0.578947368421053	0.368421052631579	0.0331578947368421	0.0521052631578948\\
0.578947368421053	0.473684210526316	0.0426315789473684	0.0521052631578948\\
0.578947368421053	0.578947368421053	0.0521052631578948	0.0521052631578948\\
0.578947368421053	0.684210526315789	0.0615789473684211	0.0521052631578948\\
0.578947368421053	0.789473684210526	0.0710526315789474	0.0521052631578948\\
0.578947368421053	0.894736842105263	0.0805263157894737	0.0521052631578948\\
0.578947368421053	1	0.09	0.0521052631578948\\
0.684210526315789	-1	-0.09	0.0615789473684211\\
0.684210526315789	-0.894736842105263	-0.0805263157894737	0.0615789473684211\\
0.684210526315789	-0.789473684210526	-0.0710526315789474	0.0615789473684211\\
0.684210526315789	-0.684210526315789	-0.0615789473684211	0.0615789473684211\\
0.684210526315789	-0.578947368421053	-0.0521052631578948	0.0615789473684211\\
0.684210526315789	-0.473684210526316	-0.0426315789473684	0.0615789473684211\\
0.684210526315789	-0.368421052631579	-0.0331578947368421	0.0615789473684211\\
0.684210526315789	-0.263157894736842	-0.0236842105263158	0.0615789473684211\\
0.684210526315789	-0.157894736842105	-0.0142105263157895	0.0615789473684211\\
0.684210526315789	-0.0526315789473684	-0.00473684210526316	0.0615789473684211\\
0.684210526315789	0.0526315789473684	0.00473684210526316	0.0615789473684211\\
0.684210526315789	0.157894736842105	0.0142105263157895	0.0615789473684211\\
0.684210526315789	0.263157894736842	0.0236842105263158	0.0615789473684211\\
0.684210526315789	0.368421052631579	0.0331578947368421	0.0615789473684211\\
0.684210526315789	0.473684210526316	0.0426315789473684	0.0615789473684211\\
0.684210526315789	0.578947368421053	0.0521052631578948	0.0615789473684211\\
0.684210526315789	0.684210526315789	0.0615789473684211	0.0615789473684211\\
0.684210526315789	0.789473684210526	0.0710526315789474	0.0615789473684211\\
0.684210526315789	0.894736842105263	0.0805263157894737	0.0615789473684211\\
0.684210526315789	1	0.09	0.0615789473684211\\
0.789473684210526	-1	-0.09	0.0710526315789474\\
0.789473684210526	-0.894736842105263	-0.0805263157894737	0.0710526315789474\\
0.789473684210526	-0.789473684210526	-0.0710526315789474	0.0710526315789474\\
0.789473684210526	-0.684210526315789	-0.0615789473684211	0.0710526315789474\\
0.789473684210526	-0.578947368421053	-0.0521052631578948	0.0710526315789474\\
0.789473684210526	-0.473684210526316	-0.0426315789473684	0.0710526315789474\\
0.789473684210526	-0.368421052631579	-0.0331578947368421	0.0710526315789474\\
0.789473684210526	-0.263157894736842	-0.0236842105263158	0.0710526315789474\\
0.789473684210526	-0.157894736842105	-0.0142105263157895	0.0710526315789474\\
0.789473684210526	-0.0526315789473684	-0.00473684210526316	0.0710526315789474\\
0.789473684210526	0.0526315789473684	0.00473684210526316	0.0710526315789474\\
0.789473684210526	0.157894736842105	0.0142105263157895	0.0710526315789474\\
0.789473684210526	0.263157894736842	0.0236842105263158	0.0710526315789474\\
0.789473684210526	0.368421052631579	0.0331578947368421	0.0710526315789474\\
0.789473684210526	0.473684210526316	0.0426315789473684	0.0710526315789474\\
0.789473684210526	0.578947368421053	0.0521052631578948	0.0710526315789474\\
0.789473684210526	0.684210526315789	0.0615789473684211	0.0710526315789474\\
0.789473684210526	0.789473684210526	0.0710526315789474	0.0710526315789474\\
0.789473684210526	0.894736842105263	0.0805263157894737	0.0710526315789474\\
0.789473684210526	1	0.09	0.0710526315789474\\
0.894736842105263	-1	-0.09	0.0805263157894737\\
0.894736842105263	-0.894736842105263	-0.0805263157894737	0.0805263157894737\\
0.894736842105263	-0.789473684210526	-0.0710526315789474	0.0805263157894737\\
0.894736842105263	-0.684210526315789	-0.0615789473684211	0.0805263157894737\\
0.894736842105263	-0.578947368421053	-0.0521052631578948	0.0805263157894737\\
0.894736842105263	-0.473684210526316	-0.0426315789473684	0.0805263157894737\\
0.894736842105263	-0.368421052631579	-0.0331578947368421	0.0805263157894737\\
0.894736842105263	-0.263157894736842	-0.0236842105263158	0.0805263157894737\\
0.894736842105263	-0.157894736842105	-0.0142105263157895	0.0805263157894737\\
0.894736842105263	-0.0526315789473684	-0.00473684210526316	0.0805263157894737\\
0.894736842105263	0.0526315789473684	0.00473684210526316	0.0805263157894737\\
0.894736842105263	0.157894736842105	0.0142105263157895	0.0805263157894737\\
0.894736842105263	0.263157894736842	0.0236842105263158	0.0805263157894737\\
0.894736842105263	0.368421052631579	0.0331578947368421	0.0805263157894737\\
0.894736842105263	0.473684210526316	0.0426315789473684	0.0805263157894737\\
0.894736842105263	0.578947368421053	0.0521052631578948	0.0805263157894737\\
0.894736842105263	0.684210526315789	0.0615789473684211	0.0805263157894737\\
0.894736842105263	0.789473684210526	0.0710526315789474	0.0805263157894737\\
0.894736842105263	0.894736842105263	0.0805263157894737	0.0805263157894737\\
0.894736842105263	1	0.09	0.0805263157894737\\
1	-1	-0.09	0.09\\
1	-0.894736842105263	-0.0805263157894737	0.09\\
1	-0.789473684210526	-0.0710526315789474	0.09\\
1	-0.684210526315789	-0.0615789473684211	0.09\\
1	-0.578947368421053	-0.0521052631578948	0.09\\
1	-0.473684210526316	-0.0426315789473684	0.09\\
1	-0.368421052631579	-0.0331578947368421	0.09\\
1	-0.263157894736842	-0.0236842105263158	0.09\\
1	-0.157894736842105	-0.0142105263157895	0.09\\
1	-0.0526315789473684	-0.00473684210526316	0.09\\
1	0.0526315789473684	0.00473684210526316	0.09\\
1	0.157894736842105	0.0142105263157895	0.09\\
1	0.263157894736842	0.0236842105263158	0.09\\
1	0.368421052631579	0.0331578947368421	0.09\\
1	0.473684210526316	0.0426315789473684	0.09\\
1	0.578947368421053	0.0521052631578948	0.09\\
1	0.684210526315789	0.0615789473684211	0.09\\
1	0.789473684210526	0.0710526315789474	0.09\\
1	0.894736842105263	0.0805263157894737	0.09\\
1	1	0.09	0.09\\
};
\end{axis}

\begin{axis}[%
width=2.367in,
height=2.367in,
at={(3.181in,0.319in)},
scale only axis,
xmin=-1,
xmax=1,
ymin=-1,
ymax=1,
axis background/.style={fill=white},
%title style={font=\bfseries},
title={$ \quatk\quad\qty(x\pdv{}{x} - y\pdv{}{y}) $},
axis lines = box,
]
\addplot [color=black!40, line width=0.4pt, forget plot]
  table[row sep=crcr]{%
-0.8	1\\
-0.8841	0.9048\\
-0.9771	0.8187\\
-1.0799	0.7408\\
-1.1935	0.6703\\
};
\addplot [color=black!40, line width=0.4pt, forget plot]
  table[row sep=crcr]{%
-0.7158	1\\
-0.7911	0.9048\\
-0.8743	0.8187\\
-0.9662	0.7408\\
-1.0678	0.6703\\
-1.1801	0.6065\\
};
\addplot [color=black!40, line width=0.4pt, forget plot]
  table[row sep=crcr]{%
-0.6316	1\\
-0.698	0.9048\\
-0.7714	0.8187\\
-0.8525	0.7408\\
-0.9422	0.6703\\
-1.0413	0.6065\\
-1.1508	0.5488\\
};
\addplot [color=black!40, line width=0.4pt, forget plot]
  table[row sep=crcr]{%
-0.5474	1\\
-0.6049	0.9048\\
-0.6686	0.8187\\
-0.7389	0.7408\\
-0.8166	0.6703\\
-0.9025	0.6065\\
-0.9974	0.5488\\
-1.1023	0.4966\\
};
\addplot [color=black!40, line width=0.4pt, forget plot]
  table[row sep=crcr]{%
-0.4632	1\\
-0.5119	0.9048\\
-0.5657	0.8187\\
-0.6252	0.7408\\
-0.691	0.6703\\
-0.7636	0.6065\\
-0.8439	0.5488\\
-0.9327	0.4966\\
-1.0308	0.4493\\
-1.1392	0.4066\\
};
\addplot [color=black!40, line width=0.4pt, forget plot]
  table[row sep=crcr]{%
-0.3789	1\\
-0.4188	0.9048\\
-0.4628	0.8187\\
-0.5115	0.7408\\
-0.5653	0.6703\\
-0.6248	0.6065\\
-0.6905	0.5488\\
-0.7631	0.4966\\
-0.8434	0.4493\\
-0.9321	0.4066\\
-1.0301	0.3679\\
-1.1384	0.3329\\
};
\addplot [color=black!40, line width=0.4pt, forget plot]
  table[row sep=crcr]{%
-0.2947	1\\
-0.3257	0.9048\\
-0.36	0.8187\\
-0.3979	0.7408\\
-0.4397	0.6703\\
-0.4859	0.6065\\
-0.537	0.5488\\
-0.5935	0.4966\\
-0.6559	0.4493\\
-0.7249	0.4066\\
-0.8012	0.3679\\
-0.8854	0.3329\\
-0.9786	0.3012\\
-1.0815	0.2725\\
-1.1952	0.2466\\
};
\addplot [color=black!40, line width=0.4pt, forget plot]
  table[row sep=crcr]{%
-0.2105	1\\
-0.2327	0.9048\\
-0.2571	0.8187\\
-0.2842	0.7408\\
-0.3141	0.6703\\
-0.3471	0.6065\\
-0.3836	0.5488\\
-0.4239	0.4966\\
-0.4685	0.4493\\
-0.5178	0.4066\\
-0.5723	0.3679\\
-0.6325	0.3329\\
-0.699	0.3012\\
-0.7725	0.2725\\
-0.8537	0.2466\\
-0.9435	0.2231\\
-1.0427	0.2019\\
-1.1524	0.1827\\
};
\addplot [color=black!40, line width=0.4pt, forget plot]
  table[row sep=crcr]{%
-0.1263	1\\
-0.1396	0.9048\\
-0.1543	0.8187\\
-0.1705	0.7408\\
-0.1884	0.6703\\
-0.2083	0.6065\\
-0.2302	0.5488\\
-0.2544	0.4966\\
-0.2811	0.4493\\
-0.3107	0.4066\\
-0.3434	0.3679\\
-0.3795	0.3329\\
-0.4194	0.3012\\
-0.4635	0.2725\\
-0.5122	0.2466\\
-0.5661	0.2231\\
-0.6256	0.2019\\
-0.6914	0.1827\\
-0.7642	0.1653\\
-0.8445	0.1496\\
-0.9334	0.1353\\
-1.0315	0.1225\\
-1.14	0.1108\\
};
\addplot [color=black!40, line width=0.4pt, forget plot]
  table[row sep=crcr]{%
-0.0421	1\\
-0.0465	0.9048\\
-0.0514	0.8187\\
-0.0568	0.7408\\
-0.0628	0.6703\\
-0.0694	0.6065\\
-0.0767	0.5488\\
-0.0848	0.4966\\
-0.0937	0.4493\\
-0.1036	0.4066\\
-0.1145	0.3679\\
-0.1265	0.3329\\
-0.1398	0.3012\\
-0.1545	0.2725\\
-0.1707	0.2466\\
-0.1887	0.2231\\
-0.2085	0.2019\\
-0.2305	0.1827\\
-0.2547	0.1653\\
-0.2815	0.1496\\
-0.3111	0.1353\\
-0.3438	0.1225\\
-0.38	0.1108\\
-0.42	0.1003\\
-0.4641	0.0907\\
-0.5129	0.0821\\
-0.5669	0.0743\\
-0.6265	0.0672\\
-0.6924	0.0608\\
-0.7652	0.055\\
-0.8457	0.0498\\
-0.9347	0.045\\
-1.0329	0.0408\\
-1.1416	0.0369\\
};
\addplot [color=black!40, line width=0.4pt, forget plot]
  table[row sep=crcr]{%
0.0421	1\\
0.0465	0.9048\\
0.0514	0.8187\\
0.0568	0.7408\\
0.0628	0.6703\\
0.0694	0.6065\\
0.0767	0.5488\\
0.0848	0.4966\\
0.0937	0.4493\\
0.1036	0.4066\\
0.1145	0.3679\\
0.1265	0.3329\\
0.1398	0.3012\\
0.1545	0.2725\\
0.1707	0.2466\\
0.1887	0.2231\\
0.2085	0.2019\\
0.2305	0.1827\\
0.2547	0.1653\\
0.2815	0.1496\\
0.3111	0.1353\\
0.3438	0.1225\\
0.38	0.1108\\
0.42	0.1003\\
0.4641	0.0907\\
0.5129	0.0821\\
0.5669	0.0743\\
0.6265	0.0672\\
0.6924	0.0608\\
0.7652	0.055\\
0.8457	0.0498\\
0.9347	0.045\\
1.0329	0.0408\\
1.1416	0.0369\\
};
\addplot [color=black!40, line width=0.4pt, forget plot]
  table[row sep=crcr]{%
0.1263	1\\
0.1396	0.9048\\
0.1543	0.8187\\
0.1705	0.7408\\
0.1884	0.6703\\
0.2083	0.6065\\
0.2302	0.5488\\
0.2544	0.4966\\
0.2811	0.4493\\
0.3107	0.4066\\
0.3434	0.3679\\
0.3795	0.3329\\
0.4194	0.3012\\
0.4635	0.2725\\
0.5122	0.2466\\
0.5661	0.2231\\
0.6256	0.2019\\
0.6914	0.1827\\
0.7642	0.1653\\
0.8445	0.1496\\
0.9334	0.1353\\
1.0315	0.1225\\
1.14	0.1108\\
};
\addplot [color=black!40, line width=0.4pt, forget plot]
  table[row sep=crcr]{%
0.2105	1\\
0.2327	0.9048\\
0.2571	0.8187\\
0.2842	0.7408\\
0.3141	0.6703\\
0.3471	0.6065\\
0.3836	0.5488\\
0.4239	0.4966\\
0.4685	0.4493\\
0.5178	0.4066\\
0.5723	0.3679\\
0.6325	0.3329\\
0.699	0.3012\\
0.7725	0.2725\\
0.8537	0.2466\\
0.9435	0.2231\\
1.0427	0.2019\\
1.1524	0.1827\\
};
\addplot [color=black!40, line width=0.4pt, forget plot]
  table[row sep=crcr]{%
0.2947	1\\
0.3257	0.9048\\
0.36	0.8187\\
0.3979	0.7408\\
0.4397	0.6703\\
0.4859	0.6065\\
0.537	0.5488\\
0.5935	0.4966\\
0.6559	0.4493\\
0.7249	0.4066\\
0.8012	0.3679\\
0.8854	0.3329\\
0.9786	0.3012\\
1.0815	0.2725\\
1.1952	0.2466\\
};
\addplot [color=black!40, line width=0.4pt, forget plot]
  table[row sep=crcr]{%
0.3789	1\\
0.4188	0.9048\\
0.4628	0.8187\\
0.5115	0.7408\\
0.5653	0.6703\\
0.6248	0.6065\\
0.6905	0.5488\\
0.7631	0.4966\\
0.8434	0.4493\\
0.9321	0.4066\\
1.0301	0.3679\\
1.1384	0.3329\\
};
\addplot [color=black!40, line width=0.4pt, forget plot]
  table[row sep=crcr]{%
0.4632	1\\
0.5119	0.9048\\
0.5657	0.8187\\
0.6252	0.7408\\
0.691	0.6703\\
0.7636	0.6065\\
0.8439	0.5488\\
0.9327	0.4966\\
1.0308	0.4493\\
1.1392	0.4066\\
};
\addplot [color=black!40, line width=0.4pt, forget plot]
  table[row sep=crcr]{%
0.5474	1\\
0.6049	0.9048\\
0.6686	0.8187\\
0.7389	0.7408\\
0.8166	0.6703\\
0.9025	0.6065\\
0.9974	0.5488\\
1.1023	0.4966\\
};
\addplot [color=black!40, line width=0.4pt, forget plot]
  table[row sep=crcr]{%
0.6316	1\\
0.698	0.9048\\
0.7714	0.8187\\
0.8525	0.7408\\
0.9422	0.6703\\
1.0413	0.6065\\
1.1508	0.5488\\
};
\addplot [color=black!40, line width=0.4pt, forget plot]
  table[row sep=crcr]{%
0.7158	1\\
0.7911	0.9048\\
0.8743	0.8187\\
0.9662	0.7408\\
1.0678	0.6703\\
1.1801	0.6065\\
};
\addplot [color=black!40, line width=0.4pt, forget plot]
  table[row sep=crcr]{%
0.8	1\\
0.8841	0.9048\\
0.9771	0.8187\\
1.0799	0.7408\\
1.1935	0.6703\\
};
\addplot [color=black!40, line width=0.4pt, forget plot]
  table[row sep=crcr]{%
-0.8	-1\\
-0.8841	-0.9048\\
-0.9771	-0.8187\\
-1.0799	-0.7408\\
-1.1935	-0.6703\\
};
\addplot [color=black!40, line width=0.4pt, forget plot]
  table[row sep=crcr]{%
-0.7158	-1\\
-0.7911	-0.9048\\
-0.8743	-0.8187\\
-0.9662	-0.7408\\
-1.0678	-0.6703\\
-1.1801	-0.6065\\
};
\addplot [color=black!40, line width=0.4pt, forget plot]
  table[row sep=crcr]{%
-0.6316	-1\\
-0.698	-0.9048\\
-0.7714	-0.8187\\
-0.8525	-0.7408\\
-0.9422	-0.6703\\
-1.0413	-0.6065\\
-1.1508	-0.5488\\
};
\addplot [color=black!40, line width=0.4pt, forget plot]
  table[row sep=crcr]{%
-0.5474	-1\\
-0.6049	-0.9048\\
-0.6686	-0.8187\\
-0.7389	-0.7408\\
-0.8166	-0.6703\\
-0.9025	-0.6065\\
-0.9974	-0.5488\\
-1.1023	-0.4966\\
};
\addplot [color=black!40, line width=0.4pt, forget plot]
  table[row sep=crcr]{%
-0.4632	-1\\
-0.5119	-0.9048\\
-0.5657	-0.8187\\
-0.6252	-0.7408\\
-0.691	-0.6703\\
-0.7636	-0.6065\\
-0.8439	-0.5488\\
-0.9327	-0.4966\\
-1.0308	-0.4493\\
-1.1392	-0.4066\\
};
\addplot [color=black!40, line width=0.4pt, forget plot]
  table[row sep=crcr]{%
-0.3789	-1\\
-0.4188	-0.9048\\
-0.4628	-0.8187\\
-0.5115	-0.7408\\
-0.5653	-0.6703\\
-0.6248	-0.6065\\
-0.6905	-0.5488\\
-0.7631	-0.4966\\
-0.8434	-0.4493\\
-0.9321	-0.4066\\
-1.0301	-0.3679\\
-1.1384	-0.3329\\
};
\addplot [color=black!40, line width=0.4pt, forget plot]
  table[row sep=crcr]{%
-0.2947	-1\\
-0.3257	-0.9048\\
-0.36	-0.8187\\
-0.3979	-0.7408\\
-0.4397	-0.6703\\
-0.4859	-0.6065\\
-0.537	-0.5488\\
-0.5935	-0.4966\\
-0.6559	-0.4493\\
-0.7249	-0.4066\\
-0.8012	-0.3679\\
-0.8854	-0.3329\\
-0.9786	-0.3012\\
-1.0815	-0.2725\\
-1.1952	-0.2466\\
};
\addplot [color=black!40, line width=0.4pt, forget plot]
  table[row sep=crcr]{%
-0.2105	-1\\
-0.2327	-0.9048\\
-0.2571	-0.8187\\
-0.2842	-0.7408\\
-0.3141	-0.6703\\
-0.3471	-0.6065\\
-0.3836	-0.5488\\
-0.4239	-0.4966\\
-0.4685	-0.4493\\
-0.5178	-0.4066\\
-0.5723	-0.3679\\
-0.6325	-0.3329\\
-0.699	-0.3012\\
-0.7725	-0.2725\\
-0.8537	-0.2466\\
-0.9435	-0.2231\\
-1.0427	-0.2019\\
-1.1524	-0.1827\\
};
\addplot [color=black!40, line width=0.4pt, forget plot]
  table[row sep=crcr]{%
-0.1263	-1\\
-0.1396	-0.9048\\
-0.1543	-0.8187\\
-0.1705	-0.7408\\
-0.1884	-0.6703\\
-0.2083	-0.6065\\
-0.2302	-0.5488\\
-0.2544	-0.4966\\
-0.2811	-0.4493\\
-0.3107	-0.4066\\
-0.3434	-0.3679\\
-0.3795	-0.3329\\
-0.4194	-0.3012\\
-0.4635	-0.2725\\
-0.5122	-0.2466\\
-0.5661	-0.2231\\
-0.6256	-0.2019\\
-0.6914	-0.1827\\
-0.7642	-0.1653\\
-0.8445	-0.1496\\
-0.9334	-0.1353\\
-1.0315	-0.1225\\
-1.14	-0.1108\\
};
\addplot [color=black!40, line width=0.4pt, forget plot]
  table[row sep=crcr]{%
-0.0421	-1\\
-0.0465	-0.9048\\
-0.0514	-0.8187\\
-0.0568	-0.7408\\
-0.0628	-0.6703\\
-0.0694	-0.6065\\
-0.0767	-0.5488\\
-0.0848	-0.4966\\
-0.0937	-0.4493\\
-0.1036	-0.4066\\
-0.1145	-0.3679\\
-0.1265	-0.3329\\
-0.1398	-0.3012\\
-0.1545	-0.2725\\
-0.1707	-0.2466\\
-0.1887	-0.2231\\
-0.2085	-0.2019\\
-0.2305	-0.1827\\
-0.2547	-0.1653\\
-0.2815	-0.1496\\
-0.3111	-0.1353\\
-0.3438	-0.1225\\
-0.38	-0.1108\\
-0.42	-0.1003\\
-0.4641	-0.0907\\
-0.5129	-0.0821\\
-0.5669	-0.0743\\
-0.6265	-0.0672\\
-0.6924	-0.0608\\
-0.7652	-0.055\\
-0.8457	-0.0498\\
-0.9347	-0.045\\
-1.0329	-0.0408\\
-1.1416	-0.0369\\
};
\addplot [color=black!40, line width=0.4pt, forget plot]
  table[row sep=crcr]{%
0.0421	-1\\
0.0465	-0.9048\\
0.0514	-0.8187\\
0.0568	-0.7408\\
0.0628	-0.6703\\
0.0694	-0.6065\\
0.0767	-0.5488\\
0.0848	-0.4966\\
0.0937	-0.4493\\
0.1036	-0.4066\\
0.1145	-0.3679\\
0.1265	-0.3329\\
0.1398	-0.3012\\
0.1545	-0.2725\\
0.1707	-0.2466\\
0.1887	-0.2231\\
0.2085	-0.2019\\
0.2305	-0.1827\\
0.2547	-0.1653\\
0.2815	-0.1496\\
0.3111	-0.1353\\
0.3438	-0.1225\\
0.38	-0.1108\\
0.42	-0.1003\\
0.4641	-0.0907\\
0.5129	-0.0821\\
0.5669	-0.0743\\
0.6265	-0.0672\\
0.6924	-0.0608\\
0.7652	-0.055\\
0.8457	-0.0498\\
0.9347	-0.045\\
1.0329	-0.0408\\
1.1416	-0.0369\\
};
\addplot [color=black!40, line width=0.4pt, forget plot]
  table[row sep=crcr]{%
0.1263	-1\\
0.1396	-0.9048\\
0.1543	-0.8187\\
0.1705	-0.7408\\
0.1884	-0.6703\\
0.2083	-0.6065\\
0.2302	-0.5488\\
0.2544	-0.4966\\
0.2811	-0.4493\\
0.3107	-0.4066\\
0.3434	-0.3679\\
0.3795	-0.3329\\
0.4194	-0.3012\\
0.4635	-0.2725\\
0.5122	-0.2466\\
0.5661	-0.2231\\
0.6256	-0.2019\\
0.6914	-0.1827\\
0.7642	-0.1653\\
0.8445	-0.1496\\
0.9334	-0.1353\\
1.0315	-0.1225\\
1.14	-0.1108\\
};
\addplot [color=black!40, line width=0.4pt, forget plot]
  table[row sep=crcr]{%
0.2105	-1\\
0.2327	-0.9048\\
0.2571	-0.8187\\
0.2842	-0.7408\\
0.3141	-0.6703\\
0.3471	-0.6065\\
0.3836	-0.5488\\
0.4239	-0.4966\\
0.4685	-0.4493\\
0.5178	-0.4066\\
0.5723	-0.3679\\
0.6325	-0.3329\\
0.699	-0.3012\\
0.7725	-0.2725\\
0.8537	-0.2466\\
0.9435	-0.2231\\
1.0427	-0.2019\\
1.1524	-0.1827\\
};
\addplot [color=black!40, line width=0.4pt, forget plot]
  table[row sep=crcr]{%
0.2947	-1\\
0.3257	-0.9048\\
0.36	-0.8187\\
0.3979	-0.7408\\
0.4397	-0.6703\\
0.4859	-0.6065\\
0.537	-0.5488\\
0.5935	-0.4966\\
0.6559	-0.4493\\
0.7249	-0.4066\\
0.8012	-0.3679\\
0.8854	-0.3329\\
0.9786	-0.3012\\
1.0815	-0.2725\\
1.1952	-0.2466\\
};
\addplot [color=black!40, line width=0.4pt, forget plot]
  table[row sep=crcr]{%
0.3789	-1\\
0.4188	-0.9048\\
0.4628	-0.8187\\
0.5115	-0.7408\\
0.5653	-0.6703\\
0.6248	-0.6065\\
0.6905	-0.5488\\
0.7631	-0.4966\\
0.8434	-0.4493\\
0.9321	-0.4066\\
1.0301	-0.3679\\
1.1384	-0.3329\\
};
\addplot [color=black!40, line width=0.4pt, forget plot]
  table[row sep=crcr]{%
0.4632	-1\\
0.5119	-0.9048\\
0.5657	-0.8187\\
0.6252	-0.7408\\
0.691	-0.6703\\
0.7636	-0.6065\\
0.8439	-0.5488\\
0.9327	-0.4966\\
1.0308	-0.4493\\
1.1392	-0.4066\\
};
\addplot [color=black!40, line width=0.4pt, forget plot]
  table[row sep=crcr]{%
0.5474	-1\\
0.6049	-0.9048\\
0.6686	-0.8187\\
0.7389	-0.7408\\
0.8166	-0.6703\\
0.9025	-0.6065\\
0.9974	-0.5488\\
1.1023	-0.4966\\
};
\addplot [color=black!40, line width=0.4pt, forget plot]
  table[row sep=crcr]{%
0.6316	-1\\
0.698	-0.9048\\
0.7714	-0.8187\\
0.8525	-0.7408\\
0.9422	-0.6703\\
1.0413	-0.6065\\
1.1508	-0.5488\\
};
\addplot [color=black!40, line width=0.4pt, forget plot]
  table[row sep=crcr]{%
0.7158	-1\\
0.7911	-0.9048\\
0.8743	-0.8187\\
0.9662	-0.7408\\
1.0678	-0.6703\\
1.1801	-0.6065\\
};
\addplot [color=black!40, line width=0.4pt, forget plot]
  table[row sep=crcr]{%
0.8	-1\\
0.8841	-0.9048\\
0.9771	-0.8187\\
1.0799	-0.7408\\
1.1935	-0.6703\\
};
\addplot[-stealth, color=accent1, point meta={sqrt((\thisrow{u})^2+(\thisrow{v})^2)}, point meta min=0, quiver={u=\thisrow{u}, v=\thisrow{v}, scale arrows = 1.45, every arrow/.append style={line width=1pt*\pgfplotspointmetatransformed/1000}}]
 table[row sep=crcr] {%
x	y	u	v\\
-1	-1	-0.09	0.09\\
-1	-0.894736842105263	-0.09	0.0805263157894737\\
-1	-0.789473684210526	-0.09	0.0710526315789474\\
-1	-0.684210526315789	-0.09	0.0615789473684211\\
-1	-0.578947368421053	-0.09	0.0521052631578948\\
-1	-0.473684210526316	-0.09	0.0426315789473684\\
-1	-0.368421052631579	-0.09	0.0331578947368421\\
-1	-0.263157894736842	-0.09	0.0236842105263158\\
-1	-0.157894736842105	-0.09	0.0142105263157895\\
-1	-0.0526315789473684	-0.09	0.00473684210526316\\
-1	0.0526315789473684	-0.09	-0.00473684210526316\\
-1	0.157894736842105	-0.09	-0.0142105263157895\\
-1	0.263157894736842	-0.09	-0.0236842105263158\\
-1	0.368421052631579	-0.09	-0.0331578947368421\\
-1	0.473684210526316	-0.09	-0.0426315789473684\\
-1	0.578947368421053	-0.09	-0.0521052631578948\\
-1	0.684210526315789	-0.09	-0.0615789473684211\\
-1	0.789473684210526	-0.09	-0.0710526315789474\\
-1	0.894736842105263	-0.09	-0.0805263157894737\\
-1	1	-0.09	-0.09\\
-0.894736842105263	-1	-0.0805263157894737	0.09\\
-0.894736842105263	-0.894736842105263	-0.0805263157894737	0.0805263157894737\\
-0.894736842105263	-0.789473684210526	-0.0805263157894737	0.0710526315789474\\
-0.894736842105263	-0.684210526315789	-0.0805263157894737	0.0615789473684211\\
-0.894736842105263	-0.578947368421053	-0.0805263157894737	0.0521052631578948\\
-0.894736842105263	-0.473684210526316	-0.0805263157894737	0.0426315789473684\\
-0.894736842105263	-0.368421052631579	-0.0805263157894737	0.0331578947368421\\
-0.894736842105263	-0.263157894736842	-0.0805263157894737	0.0236842105263158\\
-0.894736842105263	-0.157894736842105	-0.0805263157894737	0.0142105263157895\\
-0.894736842105263	-0.0526315789473684	-0.0805263157894737	0.00473684210526316\\
-0.894736842105263	0.0526315789473684	-0.0805263157894737	-0.00473684210526316\\
-0.894736842105263	0.157894736842105	-0.0805263157894737	-0.0142105263157895\\
-0.894736842105263	0.263157894736842	-0.0805263157894737	-0.0236842105263158\\
-0.894736842105263	0.368421052631579	-0.0805263157894737	-0.0331578947368421\\
-0.894736842105263	0.473684210526316	-0.0805263157894737	-0.0426315789473684\\
-0.894736842105263	0.578947368421053	-0.0805263157894737	-0.0521052631578948\\
-0.894736842105263	0.684210526315789	-0.0805263157894737	-0.0615789473684211\\
-0.894736842105263	0.789473684210526	-0.0805263157894737	-0.0710526315789474\\
-0.894736842105263	0.894736842105263	-0.0805263157894737	-0.0805263157894737\\
-0.894736842105263	1	-0.0805263157894737	-0.09\\
-0.789473684210526	-1	-0.0710526315789474	0.09\\
-0.789473684210526	-0.894736842105263	-0.0710526315789474	0.0805263157894737\\
-0.789473684210526	-0.789473684210526	-0.0710526315789474	0.0710526315789474\\
-0.789473684210526	-0.684210526315789	-0.0710526315789474	0.0615789473684211\\
-0.789473684210526	-0.578947368421053	-0.0710526315789474	0.0521052631578948\\
-0.789473684210526	-0.473684210526316	-0.0710526315789474	0.0426315789473684\\
-0.789473684210526	-0.368421052631579	-0.0710526315789474	0.0331578947368421\\
-0.789473684210526	-0.263157894736842	-0.0710526315789474	0.0236842105263158\\
-0.789473684210526	-0.157894736842105	-0.0710526315789474	0.0142105263157895\\
-0.789473684210526	-0.0526315789473684	-0.0710526315789474	0.00473684210526316\\
-0.789473684210526	0.0526315789473684	-0.0710526315789474	-0.00473684210526316\\
-0.789473684210526	0.157894736842105	-0.0710526315789474	-0.0142105263157895\\
-0.789473684210526	0.263157894736842	-0.0710526315789474	-0.0236842105263158\\
-0.789473684210526	0.368421052631579	-0.0710526315789474	-0.0331578947368421\\
-0.789473684210526	0.473684210526316	-0.0710526315789474	-0.0426315789473684\\
-0.789473684210526	0.578947368421053	-0.0710526315789474	-0.0521052631578948\\
-0.789473684210526	0.684210526315789	-0.0710526315789474	-0.0615789473684211\\
-0.789473684210526	0.789473684210526	-0.0710526315789474	-0.0710526315789474\\
-0.789473684210526	0.894736842105263	-0.0710526315789474	-0.0805263157894737\\
-0.789473684210526	1	-0.0710526315789474	-0.09\\
-0.684210526315789	-1	-0.0615789473684211	0.09\\
-0.684210526315789	-0.894736842105263	-0.0615789473684211	0.0805263157894737\\
-0.684210526315789	-0.789473684210526	-0.0615789473684211	0.0710526315789474\\
-0.684210526315789	-0.684210526315789	-0.0615789473684211	0.0615789473684211\\
-0.684210526315789	-0.578947368421053	-0.0615789473684211	0.0521052631578948\\
-0.684210526315789	-0.473684210526316	-0.0615789473684211	0.0426315789473684\\
-0.684210526315789	-0.368421052631579	-0.0615789473684211	0.0331578947368421\\
-0.684210526315789	-0.263157894736842	-0.0615789473684211	0.0236842105263158\\
-0.684210526315789	-0.157894736842105	-0.0615789473684211	0.0142105263157895\\
-0.684210526315789	-0.0526315789473684	-0.0615789473684211	0.00473684210526316\\
-0.684210526315789	0.0526315789473684	-0.0615789473684211	-0.00473684210526316\\
-0.684210526315789	0.157894736842105	-0.0615789473684211	-0.0142105263157895\\
-0.684210526315789	0.263157894736842	-0.0615789473684211	-0.0236842105263158\\
-0.684210526315789	0.368421052631579	-0.0615789473684211	-0.0331578947368421\\
-0.684210526315789	0.473684210526316	-0.0615789473684211	-0.0426315789473684\\
-0.684210526315789	0.578947368421053	-0.0615789473684211	-0.0521052631578948\\
-0.684210526315789	0.684210526315789	-0.0615789473684211	-0.0615789473684211\\
-0.684210526315789	0.789473684210526	-0.0615789473684211	-0.0710526315789474\\
-0.684210526315789	0.894736842105263	-0.0615789473684211	-0.0805263157894737\\
-0.684210526315789	1	-0.0615789473684211	-0.09\\
-0.578947368421053	-1	-0.0521052631578948	0.09\\
-0.578947368421053	-0.894736842105263	-0.0521052631578948	0.0805263157894737\\
-0.578947368421053	-0.789473684210526	-0.0521052631578948	0.0710526315789474\\
-0.578947368421053	-0.684210526315789	-0.0521052631578948	0.0615789473684211\\
-0.578947368421053	-0.578947368421053	-0.0521052631578948	0.0521052631578948\\
-0.578947368421053	-0.473684210526316	-0.0521052631578948	0.0426315789473684\\
-0.578947368421053	-0.368421052631579	-0.0521052631578948	0.0331578947368421\\
-0.578947368421053	-0.263157894736842	-0.0521052631578948	0.0236842105263158\\
-0.578947368421053	-0.157894736842105	-0.0521052631578948	0.0142105263157895\\
-0.578947368421053	-0.0526315789473684	-0.0521052631578948	0.00473684210526316\\
-0.578947368421053	0.0526315789473684	-0.0521052631578948	-0.00473684210526316\\
-0.578947368421053	0.157894736842105	-0.0521052631578948	-0.0142105263157895\\
-0.578947368421053	0.263157894736842	-0.0521052631578948	-0.0236842105263158\\
-0.578947368421053	0.368421052631579	-0.0521052631578948	-0.0331578947368421\\
-0.578947368421053	0.473684210526316	-0.0521052631578948	-0.0426315789473684\\
-0.578947368421053	0.578947368421053	-0.0521052631578948	-0.0521052631578948\\
-0.578947368421053	0.684210526315789	-0.0521052631578948	-0.0615789473684211\\
-0.578947368421053	0.789473684210526	-0.0521052631578948	-0.0710526315789474\\
-0.578947368421053	0.894736842105263	-0.0521052631578948	-0.0805263157894737\\
-0.578947368421053	1	-0.0521052631578948	-0.09\\
-0.473684210526316	-1	-0.0426315789473684	0.09\\
-0.473684210526316	-0.894736842105263	-0.0426315789473684	0.0805263157894737\\
-0.473684210526316	-0.789473684210526	-0.0426315789473684	0.0710526315789474\\
-0.473684210526316	-0.684210526315789	-0.0426315789473684	0.0615789473684211\\
-0.473684210526316	-0.578947368421053	-0.0426315789473684	0.0521052631578948\\
-0.473684210526316	-0.473684210526316	-0.0426315789473684	0.0426315789473684\\
-0.473684210526316	-0.368421052631579	-0.0426315789473684	0.0331578947368421\\
-0.473684210526316	-0.263157894736842	-0.0426315789473684	0.0236842105263158\\
-0.473684210526316	-0.157894736842105	-0.0426315789473684	0.0142105263157895\\
-0.473684210526316	-0.0526315789473684	-0.0426315789473684	0.00473684210526316\\
-0.473684210526316	0.0526315789473684	-0.0426315789473684	-0.00473684210526316\\
-0.473684210526316	0.157894736842105	-0.0426315789473684	-0.0142105263157895\\
-0.473684210526316	0.263157894736842	-0.0426315789473684	-0.0236842105263158\\
-0.473684210526316	0.368421052631579	-0.0426315789473684	-0.0331578947368421\\
-0.473684210526316	0.473684210526316	-0.0426315789473684	-0.0426315789473684\\
-0.473684210526316	0.578947368421053	-0.0426315789473684	-0.0521052631578948\\
-0.473684210526316	0.684210526315789	-0.0426315789473684	-0.0615789473684211\\
-0.473684210526316	0.789473684210526	-0.0426315789473684	-0.0710526315789474\\
-0.473684210526316	0.894736842105263	-0.0426315789473684	-0.0805263157894737\\
-0.473684210526316	1	-0.0426315789473684	-0.09\\
-0.368421052631579	-1	-0.0331578947368421	0.09\\
-0.368421052631579	-0.894736842105263	-0.0331578947368421	0.0805263157894737\\
-0.368421052631579	-0.789473684210526	-0.0331578947368421	0.0710526315789474\\
-0.368421052631579	-0.684210526315789	-0.0331578947368421	0.0615789473684211\\
-0.368421052631579	-0.578947368421053	-0.0331578947368421	0.0521052631578948\\
-0.368421052631579	-0.473684210526316	-0.0331578947368421	0.0426315789473684\\
-0.368421052631579	-0.368421052631579	-0.0331578947368421	0.0331578947368421\\
-0.368421052631579	-0.263157894736842	-0.0331578947368421	0.0236842105263158\\
-0.368421052631579	-0.157894736842105	-0.0331578947368421	0.0142105263157895\\
-0.368421052631579	-0.0526315789473684	-0.0331578947368421	0.00473684210526316\\
-0.368421052631579	0.0526315789473684	-0.0331578947368421	-0.00473684210526316\\
-0.368421052631579	0.157894736842105	-0.0331578947368421	-0.0142105263157895\\
-0.368421052631579	0.263157894736842	-0.0331578947368421	-0.0236842105263158\\
-0.368421052631579	0.368421052631579	-0.0331578947368421	-0.0331578947368421\\
-0.368421052631579	0.473684210526316	-0.0331578947368421	-0.0426315789473684\\
-0.368421052631579	0.578947368421053	-0.0331578947368421	-0.0521052631578948\\
-0.368421052631579	0.684210526315789	-0.0331578947368421	-0.0615789473684211\\
-0.368421052631579	0.789473684210526	-0.0331578947368421	-0.0710526315789474\\
-0.368421052631579	0.894736842105263	-0.0331578947368421	-0.0805263157894737\\
-0.368421052631579	1	-0.0331578947368421	-0.09\\
-0.263157894736842	-1	-0.0236842105263158	0.09\\
-0.263157894736842	-0.894736842105263	-0.0236842105263158	0.0805263157894737\\
-0.263157894736842	-0.789473684210526	-0.0236842105263158	0.0710526315789474\\
-0.263157894736842	-0.684210526315789	-0.0236842105263158	0.0615789473684211\\
-0.263157894736842	-0.578947368421053	-0.0236842105263158	0.0521052631578948\\
-0.263157894736842	-0.473684210526316	-0.0236842105263158	0.0426315789473684\\
-0.263157894736842	-0.368421052631579	-0.0236842105263158	0.0331578947368421\\
-0.263157894736842	-0.263157894736842	-0.0236842105263158	0.0236842105263158\\
-0.263157894736842	-0.157894736842105	-0.0236842105263158	0.0142105263157895\\
-0.263157894736842	-0.0526315789473684	-0.0236842105263158	0.00473684210526316\\
-0.263157894736842	0.0526315789473684	-0.0236842105263158	-0.00473684210526316\\
-0.263157894736842	0.157894736842105	-0.0236842105263158	-0.0142105263157895\\
-0.263157894736842	0.263157894736842	-0.0236842105263158	-0.0236842105263158\\
-0.263157894736842	0.368421052631579	-0.0236842105263158	-0.0331578947368421\\
-0.263157894736842	0.473684210526316	-0.0236842105263158	-0.0426315789473684\\
-0.263157894736842	0.578947368421053	-0.0236842105263158	-0.0521052631578948\\
-0.263157894736842	0.684210526315789	-0.0236842105263158	-0.0615789473684211\\
-0.263157894736842	0.789473684210526	-0.0236842105263158	-0.0710526315789474\\
-0.263157894736842	0.894736842105263	-0.0236842105263158	-0.0805263157894737\\
-0.263157894736842	1	-0.0236842105263158	-0.09\\
-0.157894736842105	-1	-0.0142105263157895	0.09\\
-0.157894736842105	-0.894736842105263	-0.0142105263157895	0.0805263157894737\\
-0.157894736842105	-0.789473684210526	-0.0142105263157895	0.0710526315789474\\
-0.157894736842105	-0.684210526315789	-0.0142105263157895	0.0615789473684211\\
-0.157894736842105	-0.578947368421053	-0.0142105263157895	0.0521052631578948\\
-0.157894736842105	-0.473684210526316	-0.0142105263157895	0.0426315789473684\\
-0.157894736842105	-0.368421052631579	-0.0142105263157895	0.0331578947368421\\
-0.157894736842105	-0.263157894736842	-0.0142105263157895	0.0236842105263158\\
-0.157894736842105	-0.157894736842105	-0.0142105263157895	0.0142105263157895\\
-0.157894736842105	-0.0526315789473684	-0.0142105263157895	0.00473684210526316\\
-0.157894736842105	0.0526315789473684	-0.0142105263157895	-0.00473684210526316\\
-0.157894736842105	0.157894736842105	-0.0142105263157895	-0.0142105263157895\\
-0.157894736842105	0.263157894736842	-0.0142105263157895	-0.0236842105263158\\
-0.157894736842105	0.368421052631579	-0.0142105263157895	-0.0331578947368421\\
-0.157894736842105	0.473684210526316	-0.0142105263157895	-0.0426315789473684\\
-0.157894736842105	0.578947368421053	-0.0142105263157895	-0.0521052631578948\\
-0.157894736842105	0.684210526315789	-0.0142105263157895	-0.0615789473684211\\
-0.157894736842105	0.789473684210526	-0.0142105263157895	-0.0710526315789474\\
-0.157894736842105	0.894736842105263	-0.0142105263157895	-0.0805263157894737\\
-0.157894736842105	1	-0.0142105263157895	-0.09\\
-0.0526315789473684	-1	-0.00473684210526316	0.09\\
-0.0526315789473684	-0.894736842105263	-0.00473684210526316	0.0805263157894737\\
-0.0526315789473684	-0.789473684210526	-0.00473684210526316	0.0710526315789474\\
-0.0526315789473684	-0.684210526315789	-0.00473684210526316	0.0615789473684211\\
-0.0526315789473684	-0.578947368421053	-0.00473684210526316	0.0521052631578948\\
-0.0526315789473684	-0.473684210526316	-0.00473684210526316	0.0426315789473684\\
-0.0526315789473684	-0.368421052631579	-0.00473684210526316	0.0331578947368421\\
-0.0526315789473684	-0.263157894736842	-0.00473684210526316	0.0236842105263158\\
-0.0526315789473684	-0.157894736842105	-0.00473684210526316	0.0142105263157895\\
-0.0526315789473684	-0.0526315789473684	-0.00473684210526316	0.00473684210526316\\
-0.0526315789473684	0.0526315789473684	-0.00473684210526316	-0.00473684210526316\\
-0.0526315789473684	0.157894736842105	-0.00473684210526316	-0.0142105263157895\\
-0.0526315789473684	0.263157894736842	-0.00473684210526316	-0.0236842105263158\\
-0.0526315789473684	0.368421052631579	-0.00473684210526316	-0.0331578947368421\\
-0.0526315789473684	0.473684210526316	-0.00473684210526316	-0.0426315789473684\\
-0.0526315789473684	0.578947368421053	-0.00473684210526316	-0.0521052631578948\\
-0.0526315789473684	0.684210526315789	-0.00473684210526316	-0.0615789473684211\\
-0.0526315789473684	0.789473684210526	-0.00473684210526316	-0.0710526315789474\\
-0.0526315789473684	0.894736842105263	-0.00473684210526316	-0.0805263157894737\\
-0.0526315789473684	1	-0.00473684210526316	-0.09\\
0.0526315789473684	-1	0.00473684210526316	0.09\\
0.0526315789473684	-0.894736842105263	0.00473684210526316	0.0805263157894737\\
0.0526315789473684	-0.789473684210526	0.00473684210526316	0.0710526315789474\\
0.0526315789473684	-0.684210526315789	0.00473684210526316	0.0615789473684211\\
0.0526315789473684	-0.578947368421053	0.00473684210526316	0.0521052631578948\\
0.0526315789473684	-0.473684210526316	0.00473684210526316	0.0426315789473684\\
0.0526315789473684	-0.368421052631579	0.00473684210526316	0.0331578947368421\\
0.0526315789473684	-0.263157894736842	0.00473684210526316	0.0236842105263158\\
0.0526315789473684	-0.157894736842105	0.00473684210526316	0.0142105263157895\\
0.0526315789473684	-0.0526315789473684	0.00473684210526316	0.00473684210526316\\
0.0526315789473684	0.0526315789473684	0.00473684210526316	-0.00473684210526316\\
0.0526315789473684	0.157894736842105	0.00473684210526316	-0.0142105263157895\\
0.0526315789473684	0.263157894736842	0.00473684210526316	-0.0236842105263158\\
0.0526315789473684	0.368421052631579	0.00473684210526316	-0.0331578947368421\\
0.0526315789473684	0.473684210526316	0.00473684210526316	-0.0426315789473684\\
0.0526315789473684	0.578947368421053	0.00473684210526316	-0.0521052631578948\\
0.0526315789473684	0.684210526315789	0.00473684210526316	-0.0615789473684211\\
0.0526315789473684	0.789473684210526	0.00473684210526316	-0.0710526315789474\\
0.0526315789473684	0.894736842105263	0.00473684210526316	-0.0805263157894737\\
0.0526315789473684	1	0.00473684210526316	-0.09\\
0.157894736842105	-1	0.0142105263157895	0.09\\
0.157894736842105	-0.894736842105263	0.0142105263157895	0.0805263157894737\\
0.157894736842105	-0.789473684210526	0.0142105263157895	0.0710526315789474\\
0.157894736842105	-0.684210526315789	0.0142105263157895	0.0615789473684211\\
0.157894736842105	-0.578947368421053	0.0142105263157895	0.0521052631578948\\
0.157894736842105	-0.473684210526316	0.0142105263157895	0.0426315789473684\\
0.157894736842105	-0.368421052631579	0.0142105263157895	0.0331578947368421\\
0.157894736842105	-0.263157894736842	0.0142105263157895	0.0236842105263158\\
0.157894736842105	-0.157894736842105	0.0142105263157895	0.0142105263157895\\
0.157894736842105	-0.0526315789473684	0.0142105263157895	0.00473684210526316\\
0.157894736842105	0.0526315789473684	0.0142105263157895	-0.00473684210526316\\
0.157894736842105	0.157894736842105	0.0142105263157895	-0.0142105263157895\\
0.157894736842105	0.263157894736842	0.0142105263157895	-0.0236842105263158\\
0.157894736842105	0.368421052631579	0.0142105263157895	-0.0331578947368421\\
0.157894736842105	0.473684210526316	0.0142105263157895	-0.0426315789473684\\
0.157894736842105	0.578947368421053	0.0142105263157895	-0.0521052631578948\\
0.157894736842105	0.684210526315789	0.0142105263157895	-0.0615789473684211\\
0.157894736842105	0.789473684210526	0.0142105263157895	-0.0710526315789474\\
0.157894736842105	0.894736842105263	0.0142105263157895	-0.0805263157894737\\
0.157894736842105	1	0.0142105263157895	-0.09\\
0.263157894736842	-1	0.0236842105263158	0.09\\
0.263157894736842	-0.894736842105263	0.0236842105263158	0.0805263157894737\\
0.263157894736842	-0.789473684210526	0.0236842105263158	0.0710526315789474\\
0.263157894736842	-0.684210526315789	0.0236842105263158	0.0615789473684211\\
0.263157894736842	-0.578947368421053	0.0236842105263158	0.0521052631578948\\
0.263157894736842	-0.473684210526316	0.0236842105263158	0.0426315789473684\\
0.263157894736842	-0.368421052631579	0.0236842105263158	0.0331578947368421\\
0.263157894736842	-0.263157894736842	0.0236842105263158	0.0236842105263158\\
0.263157894736842	-0.157894736842105	0.0236842105263158	0.0142105263157895\\
0.263157894736842	-0.0526315789473684	0.0236842105263158	0.00473684210526316\\
0.263157894736842	0.0526315789473684	0.0236842105263158	-0.00473684210526316\\
0.263157894736842	0.157894736842105	0.0236842105263158	-0.0142105263157895\\
0.263157894736842	0.263157894736842	0.0236842105263158	-0.0236842105263158\\
0.263157894736842	0.368421052631579	0.0236842105263158	-0.0331578947368421\\
0.263157894736842	0.473684210526316	0.0236842105263158	-0.0426315789473684\\
0.263157894736842	0.578947368421053	0.0236842105263158	-0.0521052631578948\\
0.263157894736842	0.684210526315789	0.0236842105263158	-0.0615789473684211\\
0.263157894736842	0.789473684210526	0.0236842105263158	-0.0710526315789474\\
0.263157894736842	0.894736842105263	0.0236842105263158	-0.0805263157894737\\
0.263157894736842	1	0.0236842105263158	-0.09\\
0.368421052631579	-1	0.0331578947368421	0.09\\
0.368421052631579	-0.894736842105263	0.0331578947368421	0.0805263157894737\\
0.368421052631579	-0.789473684210526	0.0331578947368421	0.0710526315789474\\
0.368421052631579	-0.684210526315789	0.0331578947368421	0.0615789473684211\\
0.368421052631579	-0.578947368421053	0.0331578947368421	0.0521052631578948\\
0.368421052631579	-0.473684210526316	0.0331578947368421	0.0426315789473684\\
0.368421052631579	-0.368421052631579	0.0331578947368421	0.0331578947368421\\
0.368421052631579	-0.263157894736842	0.0331578947368421	0.0236842105263158\\
0.368421052631579	-0.157894736842105	0.0331578947368421	0.0142105263157895\\
0.368421052631579	-0.0526315789473684	0.0331578947368421	0.00473684210526316\\
0.368421052631579	0.0526315789473684	0.0331578947368421	-0.00473684210526316\\
0.368421052631579	0.157894736842105	0.0331578947368421	-0.0142105263157895\\
0.368421052631579	0.263157894736842	0.0331578947368421	-0.0236842105263158\\
0.368421052631579	0.368421052631579	0.0331578947368421	-0.0331578947368421\\
0.368421052631579	0.473684210526316	0.0331578947368421	-0.0426315789473684\\
0.368421052631579	0.578947368421053	0.0331578947368421	-0.0521052631578948\\
0.368421052631579	0.684210526315789	0.0331578947368421	-0.0615789473684211\\
0.368421052631579	0.789473684210526	0.0331578947368421	-0.0710526315789474\\
0.368421052631579	0.894736842105263	0.0331578947368421	-0.0805263157894737\\
0.368421052631579	1	0.0331578947368421	-0.09\\
0.473684210526316	-1	0.0426315789473684	0.09\\
0.473684210526316	-0.894736842105263	0.0426315789473684	0.0805263157894737\\
0.473684210526316	-0.789473684210526	0.0426315789473684	0.0710526315789474\\
0.473684210526316	-0.684210526315789	0.0426315789473684	0.0615789473684211\\
0.473684210526316	-0.578947368421053	0.0426315789473684	0.0521052631578948\\
0.473684210526316	-0.473684210526316	0.0426315789473684	0.0426315789473684\\
0.473684210526316	-0.368421052631579	0.0426315789473684	0.0331578947368421\\
0.473684210526316	-0.263157894736842	0.0426315789473684	0.0236842105263158\\
0.473684210526316	-0.157894736842105	0.0426315789473684	0.0142105263157895\\
0.473684210526316	-0.0526315789473684	0.0426315789473684	0.00473684210526316\\
0.473684210526316	0.0526315789473684	0.0426315789473684	-0.00473684210526316\\
0.473684210526316	0.157894736842105	0.0426315789473684	-0.0142105263157895\\
0.473684210526316	0.263157894736842	0.0426315789473684	-0.0236842105263158\\
0.473684210526316	0.368421052631579	0.0426315789473684	-0.0331578947368421\\
0.473684210526316	0.473684210526316	0.0426315789473684	-0.0426315789473684\\
0.473684210526316	0.578947368421053	0.0426315789473684	-0.0521052631578948\\
0.473684210526316	0.684210526315789	0.0426315789473684	-0.0615789473684211\\
0.473684210526316	0.789473684210526	0.0426315789473684	-0.0710526315789474\\
0.473684210526316	0.894736842105263	0.0426315789473684	-0.0805263157894737\\
0.473684210526316	1	0.0426315789473684	-0.09\\
0.578947368421053	-1	0.0521052631578948	0.09\\
0.578947368421053	-0.894736842105263	0.0521052631578948	0.0805263157894737\\
0.578947368421053	-0.789473684210526	0.0521052631578948	0.0710526315789474\\
0.578947368421053	-0.684210526315789	0.0521052631578948	0.0615789473684211\\
0.578947368421053	-0.578947368421053	0.0521052631578948	0.0521052631578948\\
0.578947368421053	-0.473684210526316	0.0521052631578948	0.0426315789473684\\
0.578947368421053	-0.368421052631579	0.0521052631578948	0.0331578947368421\\
0.578947368421053	-0.263157894736842	0.0521052631578948	0.0236842105263158\\
0.578947368421053	-0.157894736842105	0.0521052631578948	0.0142105263157895\\
0.578947368421053	-0.0526315789473684	0.0521052631578948	0.00473684210526316\\
0.578947368421053	0.0526315789473684	0.0521052631578948	-0.00473684210526316\\
0.578947368421053	0.157894736842105	0.0521052631578948	-0.0142105263157895\\
0.578947368421053	0.263157894736842	0.0521052631578948	-0.0236842105263158\\
0.578947368421053	0.368421052631579	0.0521052631578948	-0.0331578947368421\\
0.578947368421053	0.473684210526316	0.0521052631578948	-0.0426315789473684\\
0.578947368421053	0.578947368421053	0.0521052631578948	-0.0521052631578948\\
0.578947368421053	0.684210526315789	0.0521052631578948	-0.0615789473684211\\
0.578947368421053	0.789473684210526	0.0521052631578948	-0.0710526315789474\\
0.578947368421053	0.894736842105263	0.0521052631578948	-0.0805263157894737\\
0.578947368421053	1	0.0521052631578948	-0.09\\
0.684210526315789	-1	0.0615789473684211	0.09\\
0.684210526315789	-0.894736842105263	0.0615789473684211	0.0805263157894737\\
0.684210526315789	-0.789473684210526	0.0615789473684211	0.0710526315789474\\
0.684210526315789	-0.684210526315789	0.0615789473684211	0.0615789473684211\\
0.684210526315789	-0.578947368421053	0.0615789473684211	0.0521052631578948\\
0.684210526315789	-0.473684210526316	0.0615789473684211	0.0426315789473684\\
0.684210526315789	-0.368421052631579	0.0615789473684211	0.0331578947368421\\
0.684210526315789	-0.263157894736842	0.0615789473684211	0.0236842105263158\\
0.684210526315789	-0.157894736842105	0.0615789473684211	0.0142105263157895\\
0.684210526315789	-0.0526315789473684	0.0615789473684211	0.00473684210526316\\
0.684210526315789	0.0526315789473684	0.0615789473684211	-0.00473684210526316\\
0.684210526315789	0.157894736842105	0.0615789473684211	-0.0142105263157895\\
0.684210526315789	0.263157894736842	0.0615789473684211	-0.0236842105263158\\
0.684210526315789	0.368421052631579	0.0615789473684211	-0.0331578947368421\\
0.684210526315789	0.473684210526316	0.0615789473684211	-0.0426315789473684\\
0.684210526315789	0.578947368421053	0.0615789473684211	-0.0521052631578948\\
0.684210526315789	0.684210526315789	0.0615789473684211	-0.0615789473684211\\
0.684210526315789	0.789473684210526	0.0615789473684211	-0.0710526315789474\\
0.684210526315789	0.894736842105263	0.0615789473684211	-0.0805263157894737\\
0.684210526315789	1	0.0615789473684211	-0.09\\
0.789473684210526	-1	0.0710526315789474	0.09\\
0.789473684210526	-0.894736842105263	0.0710526315789474	0.0805263157894737\\
0.789473684210526	-0.789473684210526	0.0710526315789474	0.0710526315789474\\
0.789473684210526	-0.684210526315789	0.0710526315789474	0.0615789473684211\\
0.789473684210526	-0.578947368421053	0.0710526315789474	0.0521052631578948\\
0.789473684210526	-0.473684210526316	0.0710526315789474	0.0426315789473684\\
0.789473684210526	-0.368421052631579	0.0710526315789474	0.0331578947368421\\
0.789473684210526	-0.263157894736842	0.0710526315789474	0.0236842105263158\\
0.789473684210526	-0.157894736842105	0.0710526315789474	0.0142105263157895\\
0.789473684210526	-0.0526315789473684	0.0710526315789474	0.00473684210526316\\
0.789473684210526	0.0526315789473684	0.0710526315789474	-0.00473684210526316\\
0.789473684210526	0.157894736842105	0.0710526315789474	-0.0142105263157895\\
0.789473684210526	0.263157894736842	0.0710526315789474	-0.0236842105263158\\
0.789473684210526	0.368421052631579	0.0710526315789474	-0.0331578947368421\\
0.789473684210526	0.473684210526316	0.0710526315789474	-0.0426315789473684\\
0.789473684210526	0.578947368421053	0.0710526315789474	-0.0521052631578948\\
0.789473684210526	0.684210526315789	0.0710526315789474	-0.0615789473684211\\
0.789473684210526	0.789473684210526	0.0710526315789474	-0.0710526315789474\\
0.789473684210526	0.894736842105263	0.0710526315789474	-0.0805263157894737\\
0.789473684210526	1	0.0710526315789474	-0.09\\
0.894736842105263	-1	0.0805263157894737	0.09\\
0.894736842105263	-0.894736842105263	0.0805263157894737	0.0805263157894737\\
0.894736842105263	-0.789473684210526	0.0805263157894737	0.0710526315789474\\
0.894736842105263	-0.684210526315789	0.0805263157894737	0.0615789473684211\\
0.894736842105263	-0.578947368421053	0.0805263157894737	0.0521052631578948\\
0.894736842105263	-0.473684210526316	0.0805263157894737	0.0426315789473684\\
0.894736842105263	-0.368421052631579	0.0805263157894737	0.0331578947368421\\
0.894736842105263	-0.263157894736842	0.0805263157894737	0.0236842105263158\\
0.894736842105263	-0.157894736842105	0.0805263157894737	0.0142105263157895\\
0.894736842105263	-0.0526315789473684	0.0805263157894737	0.00473684210526316\\
0.894736842105263	0.0526315789473684	0.0805263157894737	-0.00473684210526316\\
0.894736842105263	0.157894736842105	0.0805263157894737	-0.0142105263157895\\
0.894736842105263	0.263157894736842	0.0805263157894737	-0.0236842105263158\\
0.894736842105263	0.368421052631579	0.0805263157894737	-0.0331578947368421\\
0.894736842105263	0.473684210526316	0.0805263157894737	-0.0426315789473684\\
0.894736842105263	0.578947368421053	0.0805263157894737	-0.0521052631578948\\
0.894736842105263	0.684210526315789	0.0805263157894737	-0.0615789473684211\\
0.894736842105263	0.789473684210526	0.0805263157894737	-0.0710526315789474\\
0.894736842105263	0.894736842105263	0.0805263157894737	-0.0805263157894737\\
0.894736842105263	1	0.0805263157894737	-0.09\\
1	-1	0.09	0.09\\
1	-0.894736842105263	0.09	0.0805263157894737\\
1	-0.789473684210526	0.09	0.0710526315789474\\
1	-0.684210526315789	0.09	0.0615789473684211\\
1	-0.578947368421053	0.09	0.0521052631578948\\
1	-0.473684210526316	0.09	0.0426315789473684\\
1	-0.368421052631579	0.09	0.0331578947368421\\
1	-0.263157894736842	0.09	0.0236842105263158\\
1	-0.157894736842105	0.09	0.0142105263157895\\
1	-0.0526315789473684	0.09	0.00473684210526316\\
1	0.0526315789473684	0.09	-0.00473684210526316\\
1	0.157894736842105	0.09	-0.0142105263157895\\
1	0.263157894736842	0.09	-0.0236842105263158\\
1	0.368421052631579	0.09	-0.0331578947368421\\
1	0.473684210526316	0.09	-0.0426315789473684\\
1	0.578947368421053	0.09	-0.0521052631578948\\
1	0.684210526315789	0.09	-0.0615789473684211\\
1	0.789473684210526	0.09	-0.0710526315789474\\
1	0.894736842105263	0.09	-0.0805263157894737\\
1	1	0.09	-0.09\\
};
\end{axis}

\end{tikzpicture}%

    \end{center}
    \caption{Basis vector fields corresponding to the basis elements of the split-quaternions.}
    \label{fig:basis_vf}
\end{figure}

\section{Notes}
! orthogonal refers to `regular' orthogonal, Lorentz-orthogonal makes the distinction.

Motivation: $\vec{u}$ seems to be `aligned' with major direction of the elliptic trajectory in the Lorentz-orthogonal subspace, generated by the action of its cross-product. Show this formally by making use of the eigenvectors.

The basis vectors $ \qty{\vec{e}_2, \vec{e}_3}$, where $\vec{e}_2$ is the orthogonal projection of the vector $\vec{e}_1 = \uvec{u}$ on its Lorentz-orthogonal subspace, and $\vec{e}_3 \triangleq \lorcrossp{\vec{e}_1}{\vec{e}_2}$, form the real and imaginary parts of two of the eigenvectors of the matrix $\mat{U}_{\lorcrossp{}{}}$. 

Because the basis vectors $\vec{e}_2$ and $\vec{e}_3$ are also orthogonal in the Euclidean sense, the 

\begin{proof}
    Let $\uvec{u} = u_1\uvec{i} + u_2\uvec{j} + u_3\uvec{k}$. A normal vector to the Lorentz-orthogonal subspace is $
    \uvec{n} = u_1\uvec{i} - u_2\uvec{j} - u_3\uvec{k}$. Then, the basis vectors are
    \begin{equation}
        \begin{split}
            \vec{e}_2 &= 
            \uvec{u} - \frac{ \inner{\uvec{u}}{\uvec{n}} }{ \inner{\uvec{n}}{\uvec{n}} } \uvec{n} \\
            \vec{e}_3 &= \lorcrossp{\uvec{u}}{\vec{e}_2} = -\frac{ \inner{\uvec{u}}{\uvec{n}} }{ \inner{\uvec{n}}{\uvec{n}} } \qty(\lorcrossp{\uvec{u}}{\uvec{n}}),
        \end{split}
    \end{equation}
    because the Lorentz-cross product distributes over addition and $\lorcrossp{\uvec{u}}{\uvec{u}} = \vec{0}$. The proposition above claims that $\vec{e}_2 + \ii\vec{e}_3$ is an eigenvector of the matrix $\mat{U}_{\lorcrossp{}{}}$. Hence, it must be the case that $\mat{U}_{\lorcrossp{}{}}(\vec{e}_2 + \ii\vec{e}_3) = \lambda(\vec{e}_2 + \ii\vec{e}_3)$, where $\lambda$ is then an eigenvalue of the matrix. This can be verified by replacing the action of $\mat{U}_{\lorcrossp{}{}}$ with the cross product. Plugging in the definition and exploiting the linearity of the Lorentz cross-product, we obtain:
    \begin{equation*}
        \begin{split}
            \lorcrossp{\uvec{u}}{\qty(\vec{e}_2 + \ii\vec{e}_3)} 
            &= \lorcrossp{\uvec{u}}{\vec{e}_2} +
        \ii\qty(\lorcrossp{\uvec{u}}{\vec{e}_3}) \\
            &= \vec{e}_3 + \qty(\lorcrossp{\uvec{u}}{\vec{e}_3})\ii \\ 
            &=\vec{e}_3 +  \qty(\lorcrossp{\uvec{u}}{\qty(\lorcrossp{\uvec{u}}{\vec{e}_2})})\ii \\
            &=\vec{e}_3 -  \frac{ \inner{\uvec{u}}{\uvec{n}} }{ \inner{\uvec{n}}{\uvec{n}} }\qty(\lorcrossp{\uvec{u}}{\qty(\lorcrossp{\uvec{u}}{\uvec{n}})})\ii.  \\
        \end{split}
    \end{equation*}
The triple cross-product expansion, or `Lagrange formula', relates the regular cross product to the corresponding dot product:
    $$ \vec{a}\times\qty(\vec{b}\times\vec{c}) = \vec{b}\:\inner{\vec{c}}{\vec{a}} - \vec{c}\:\inner{\vec{a}}{\vec{b}}. $$
This well-known identity generalizes (easily verified) to the Lorentzian counterpart of the cross- and inner products:
    $$ 
        \lorcrossp{\vec{a}}{\qty(\lorcrossp{\vec{b}}{\vec{c}})} 
       = \vec{b}\:\lorinner{\vec{c}}{\vec{a}} - \vec{c}\:\lorinner{\vec{a}}{\vec{b}}. 
    $$
Using the Lagrange formula, the above expression becomes
    \begin{equation*}
        \begin{split}
            & \vec{e}_3 - \frac{ \inner{\uvec{u}}{\uvec{n}} }{ \inner{\uvec{n}}{\uvec{n}} }\qty(\uvec{u}\,\lorinner{\uvec{u}}{\uvec{n}} - \uvec{n}\lorinner{\uvec{u}}{\uvec{u}})\ii \\
            & =\, \vec{e}_3 - \qty(\uvec{u}\,\frac{ \lorinner{\uvec{u}}{\uvec{n}} \, \inner{\uvec{u}}{\uvec{n}} }{ \inner{\uvec{n}}{\uvec{n}} } - \uvec{n}\frac{ \inner{\uvec{u}}{\uvec{n}} }{ \inner{\uvec{n}}{\uvec{n}} })\ii \\
            & =\, \vec{e}_3 - \qty(\uvec{u} - \uvec{n}\frac{ \inner{\uvec{u}}{\uvec{n}} }{ \inner{\uvec{n}}{\uvec{n}} })\ii \\
            & =\, \vec{e}_3 - \vec{e}_2\ii. 
        \end{split}
    \end{equation*}
    The latter is the scalar multiple of the vector $\vec{e}_2 + \vec{e}_3$ by $-\ii$ - hence, this is indeed an eigenvector of the corresponding matrix.
\end{proof}
Because $\vec{e}_2$ and $\vec{e}_3$ are also orthogonal in the normal sense, they are aligned with the major axes of the elliptic trajectories generated by the cross product. Hence, they can be used to find a basis of the invariant subspace which makes the trajectories identical to those in the phase plane.


