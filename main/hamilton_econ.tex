\chapter{Hamiltonian economics}

\section{Symplectic formulation}

\begin{aside}{Lie derivatives \& Max' question}
    The Lie derivative of the tautological form $\alpha = p\dd{q}$ with respect to the
    Hamiltonian vector field
    $$ X_H = \pdv{H}{p}\pdv{}{q} - \pdv{H}{p}\pdv{}{p}$$
    is denoted by
    $$ \lied{X_H}{\alpha}.$$
    Using Cartan's magic formula ($ \lied{V}{\theta} = \dd{(\intpr{V}{\theta})} + \intpr{V}{\dd{\theta}}$), this expression
    can be written as
    \begin{equation*} 
        \begin{split}
            \lied{X_H}{p\dd{q}} &= \dd{(\intpr{X_H}{p\dd{q}})} + \intpr{X_H}{\dd{(p\dd{q})}} \\
                                &= \dd{(\intpr{X_H}{p\dd{q}})} - \intpr{X_H}{\omega} \\
                                &= \dd{(\intpr{X_H}{p\dd{q}})} - \dd{H} \\
                                &= \dd{\qty(\pdv{H}{p}p)} - \dd{H} \\
                                &= \dd{(\dot{q}p)} - \dd{H} \\
                                &= \dd{(\dot{q}p - H)} \\
                                &= \dd{L} \\
        \end{split}
    \end{equation*}

    Explicitly in components:
    \begin{equation*}
        \lied{X_H}{\alpha} = \qty[X_H^{\nu}(\partial_{\nu}\alpha_{\mu}) + (\partial_{\mu} X_H^{\nu})\alpha_{\nu}]\dd{x}^{\mu}
    \end{equation*}
    \begin{equation*}
        \begin{split}
            \lied{X_H}{\alpha} =\, &\qty[\pdv{H}{p}\qty(\pdv{}{q}p) + \qty(\pdv{}{q}\pdv{H}{p})p - \pdv{H}{q}\qty(\pdv{}{p}p) - \qty(\pdv{}{q}\pdv{H}{q})0]\dd{q} \\
                                + &\qty[\pdv{H}{p}\qty(\pdv{}{q}0) + \qty(\pdv{}{p}\pdv{H}{p})p - \pdv{H}{q}\qty(\pdv{}{q} 0) - \qty(\pdv{}{p}\pdv{H}{q})0]\dd{p} \\
                               =\,&\qty[ \qty(\pdv{}{q}\pdv{H}{p})p - \pdv{H}{q}]\dd{q} + \qty[\qty(\pdv{}{p}\pdv{H}{p})p]\dd{p}\\
        \end{split}
    \end{equation*}
    Compare this with the expression using the Cartan equation:
    \begin{equation*}
        \begin{split}
            \dd{\qty(\pdv{H}{p}p - H)} &= \pdv{}{q}\qty(\pdv{H}{p}p)\dd{q} + \pdv{}{p}\qty(\pdv{H}{p}p)\dd{p} - \pdv{H}{q}\dd{q} - \pdv{H}{p}\dd{p}\\
                                       &= \qty[p\pdv{}{q}\qty(\pdv{H}{p}) + \pdv{H}{p}\pdv{p}{q} - \pdv{H}{q}]\dd{q} + \qty[p\pdv{}{p}\qty(\pdv{H}{p}) + \pdv{H}{p}\pdv{p}{p} - \pdv{H}{p}]\dd{p}
        \end{split}
    \end{equation*}
    which coincides with the previous expression.

\end{aside}

\section{The Liouville theorem}
This section explores the different forms in which the Liouville theorem appears, both in classical mechanics and the general study of differential equations.

\subsection{About divergence}
In this text, divergence appears in general levels of generalization, from the standard vector calculus definition to one in terms of differential forms that applies to curved spaces as well.

CHECK Foundations of mechanics p. 130 for a rigorous treatment of divergence on manifolds.

\section{Variational formulation}
