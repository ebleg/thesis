\chapter{Bla bla}
\section{Split-quaternion geometry}
! orthogonal refers to 'regular' orthogonal, Lorentz-orthogonal makes the distinction.

Motivation: $\vec{u}$ seems to be 'aligned' with major direction of the elliptic trajectory in the Lorentz-orthogonal subspace, generated by the action of its cross-product. Show this formally by making use of the eigenvectors.

The basis vectors $ \qty{\vec{e}_2, \vec{e}_3}$, where $\vec{e}_2$ is the orthogonal projection of the vector $\vec{e}_1 = \uvec{u}$ on its Lorentz-orthogonal subspace, and $\vec{e}_3 \triangleq \lorcrossp{\vec{e}_1}{\vec{e}_2}$, form the real and imaginary parts of two of the eigenvectors of the matrix $\mat{U}_{\lorcrossp{}{}}$. 

Because the basis vectors $\vec{e}_2$ and $\vec{e}_3$ are also orthogonal in the Euclidean sense, the 

\begin{proof}
    Let $\uvec{u} = u_1\uvec{i} + u_2\uvec{j} + u_3\uvec{k}$. A normal vector to the Lorentz-orthogonal subspace is $
    \uvec{n} = u_1\uvec{i} - u_2\uvec{j} - u_3\uvec{k}$. Then, the basis vectors are
    \begin{equation}
        \begin{split}
            \vec{e}_2 &= 
            \uvec{u} - \frac{ \inner{\uvec{u}}{\uvec{n}} }{ \inner{\uvec{n}}{\uvec{n}} } \uvec{n} \\
            \vec{e}_3 &= \lorcrossp{\uvec{u}}{\vec{e}_2} = -\frac{ \inner{\uvec{u}}{\uvec{n}} }{ \inner{\uvec{n}}{\uvec{n}} } \qty(\lorcrossp{\uvec{u}}{\uvec{n}}),
        \end{split}
    \end{equation}
    because the Lorentz-cross product distributes over addition and $\lorcrossp{\uvec{u}}{\uvec{u}} = \vec{0}$. The proposition above claims that $\vec{e}_2 + \ii\vec{e}_3$ is an eigenvector of the matrix $\mat{U}_{\lorcrossp{}{}}$. Hence, it must be the case that $\mat{U}_{\lorcrossp{}{}}(\vec{e}_2 + \ii\vec{e}_3) = \lambda(\vec{e}_2 + \ii\vec{e}_3)$, where $\lambda$ is then an eigenvalue of the matrix. This can be verified by replacing the action of $\mat{U}_{\lorcrossp{}{}}$ with the cross product. Plugging in the definition and exploiting the linearity of the Lorentz cross-product, one obtains:
    \begin{equation*}
        \begin{split}
            \lorcrossp{\uvec{u}}{\qty(\vec{e}_2 + \ii\vec{e}_3)} 
            &= \lorcrossp{\uvec{u}}{\vec{e}_2} +
        \ii\qty(\lorcrossp{\uvec{u}}{\vec{e}_3}) \\
            &= \vec{e}_3 + \qty(\lorcrossp{\uvec{u}}{\vec{e}_3})\ii \\ 
            &=\vec{e}_3 +  \qty(\lorcrossp{\uvec{u}}{\qty(\lorcrossp{\uvec{u}}{\vec{e}_2})})\ii \\
            &=\vec{e}_3 -  \frac{ \inner{\uvec{u}}{\uvec{n}} }{ \inner{\uvec{n}}{\uvec{n}} }\qty(\lorcrossp{\uvec{u}}{\qty(\lorcrossp{\uvec{u}}{\uvec{n}})})\ii.  \\
        \end{split}
    \end{equation*}
The triple cross-product expansion, or `Lagrange formula', relates the regular cross product to the corresponding dot product:
    $$ \vec{a}\times\qty(\vec{b}\times\vec{c}) = \vec{b}\:\inner{\vec{c}}{\vec{a}} - \vec{c}\:\inner{\vec{a}}{\vec{b}}. $$
This well-known identity generalizes (easily verified) to the Lorentzian counterpart of the cross- and inner products:
    $$ 
        \lorcrossp{\vec{a}}{\qty(\lorcrossp{\vec{b}}{\vec{c}})} 
       = \vec{b}\:\lorinner{\vec{c}}{\vec{a}} - \vec{c}\:\lorinner{\vec{a}}{\vec{b}}. 
    $$
Using the Lagrange formula, the above expression becomes
    \begin{equation*}
        \begin{split}
            & \vec{e}_3 - \frac{ \inner{\uvec{u}}{\uvec{n}} }{ \inner{\uvec{n}}{\uvec{n}} }\qty(\uvec{u}\,\lorinner{\uvec{u}}{\uvec{n}} - \uvec{n}\lorinner{\uvec{u}}{\uvec{u}})\ii \\
            & =\, \vec{e}_3 - \qty(\uvec{u}\,\frac{ \lorinner{\uvec{u}}{\uvec{n}} \, \inner{\uvec{u}}{\uvec{n}} }{ \inner{\uvec{n}}{\uvec{n}} } - \uvec{n}\frac{ \inner{\uvec{u}}{\uvec{n}} }{ \inner{\uvec{n}}{\uvec{n}} })\ii \\
            & =\, \vec{e}_3 - \qty(\uvec{u} - \uvec{n}\frac{ \inner{\uvec{u}}{\uvec{n}} }{ \inner{\uvec{n}}{\uvec{n}} })\ii \\
            & =\, \vec{e}_3 - \vec{e}_2\ii. 
        \end{split}
    \end{equation*}
    The latter is the scalar multiple of the vector $\vec{e}_2 + \vec{e}_3$ by $-\ii$ - hence, this is indeed an eigenvector of the corresponding matrix.
\end{proof}
Because $\vec{e}_2$ and $\vec{e}_3$ are also orthogonal in the normal sense, they are aligned with the major axes of the elliptic trajectories generated by the cross product. Hence, they can be used to find a basis of the invariant subspace which makes the trajectories identical to those in the phase plane.


