% Abstract (does not appear in the Table of Contents)
\chapter*{Abstract}%

Conservative mechanical systems admit a symplectic structure. 
However, since real systems typically exhibit energy dissipation, this symplectic structure is often too restrictive for engineering purposes. 
Also in economic systems, dissipative phenomena are ubiquitous in the form of consumption and depreciation. In this thesis, we propose an extension of the symplectic structure that does intrinsically incorporate dissipation. 

The purpose of this thesis is to present the geometric structure in a way that makes it applicable for engineering applications, which is done in two steps.

First, by combining the symplectic structure of conservative mechanics and the contact-geometric description of thermodynamics, we construct a contact Hamiltonian system for the damped harmonic oscillator. 
Because the associated contact structure is based on thermodynamic principles, this system is readily modified to the harmonic oscillator with both a parallel and serial damper as well. 
In addition, we show how the widely adopted Caldirola-Kanai Hamiltonian for the damped harmonic oscillator emerges from the symplectification of the contact Hamiltonian system. 

The contact structure is then extended to a Jacobi structure in order to deal with general, multi-degree of freedom systems.
In contrast to the contact structure, the Jacobi structure encodes the pairing of canonical variables and the dissipation as two separate entries. We argue that this makes it possible to construct Hamiltonian systems for any mechanical systems, and illustrate the practicality of this formalism by applying it to a multi-degree of freedom system.

Second, we propose split-quaternions as an alternative to the traditional matrix representation of two-dimensional linear mechanical systems. 
We demonstrate how the properties of the dynamical system are directly reflected in its split-quaternion representation. 
As a result, the split-quaternion representation offers several advantages for practical applications, e.g., for the classification of fixed points or when computing the system solution. 
We use models of the hyperbolic plane to find a relation between the solution geometry of underdamped systems and their split-quaternion representation.
 
 
