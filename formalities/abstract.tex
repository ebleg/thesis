% Abstract (does not appear in the Table of Contents)
\chapter*{Abstract}%

Conservative mechanical systems admit a symplectic structure. However, since engineering systems typically exhibit energy dissipation, this symplectic structure is too restrictive. In this thesis, we propose a different geometric structure that allows for energy dissipation in mechanical systems and the analogous phenomena in economic engineering. We first provide a theoretical basis in the form of Jacobi manifolds and then propose a practical approach using split-quaternions to represent mechanical systems.

First, we draw inspiration from the role of contact geometry in thermodynamics to construct a contact Hamiltonian system for the damped harmonic oscillator. The contact structure has a direct physical interpretation, which makes it suitable for engineering applications. With the same methodology, we derive a contact structure for the damped harmonic oscillator with both a parallel and serial damper. In addition, we show how the widely adopted Caldirola-Kanai Hamiltonian for the damped harmonic oscillator emerges from the symplectification of the contact Hamiltonian system. 

For more general, multi-degree of freedom mechanical systems, a contact Hamiltonian system does not suffice. Therefore, we extend the formalism towards a Jacobi structure that encodes the pairing between the conjugate variables and the dissipation in the system as two separate entities. We argue that this Jacobi structure applies to any mechanical system and illustrate its practicality by applying it to a complex system.

Second, we propose split-quaternions as an alternative to the traditional matrix representation of two-dimensional linear mechanical systems. We demonstrate how the properties of the dynamical system are directly reflected in its split-quaternion representation. As a result, the split-quaternion representation offers several advantages for practical applications, e.g., for the classification of fixed points or when computing the system solution. Moreover, we use models of the hyperbolic plane to find a relation between the solution geometry of underdamped systems and their split-quaternion representation.
 
 
