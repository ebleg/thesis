% Abstract (does not appear in the Table of Contents)
\chapter*{Abstract}%

Traditionally, systems in classical mechanics have a symplectic structure. 
However, autonomous mechanical systems defined on symplectic manifolds necessarily conserve energy.
In this thesis, we propose a different geometric structure that allows for energy dissipation in mechanical systems and the analogous phenomena in economic engineering. 

First, we use thermodynamic theory to derive the contact structure and modify the existing approaches to this problem to assign them with a direct physical interpretation suitable for engineering applications. 
The damped harmonic oscillator with one or two dampers is used as the prototypical mechanical system. 
In addition, we show that the symplectification of the contact Hamiltonian system is directly equivalent to the widely adopted Caldirola-Kanai model for the damped oscillator.
However, we show that a contact structure does not offer enough flexibility for general, multi-degree of freedom mechanical systems.
A modified geometric structure, an instance of the more general class of Jacobi structures, is proposed to establish a framework that is applicable to any mechanical system and directly interpretable from an engineering point of view.

Second, we propose an alternative to the traditional matrix representation of two-dimensional linear dynamical systems. 
We show that the natural properties of split-quaternions translate directly to the qualitative features of the associated dynamical systems and conveniently facilitate the classification of fixed points. Moreover, the split-quaternion representation also offers computational advantages when solving the dynamical system.
We again employ the harmonic oscillator with two dampers as a representative system to relate the geometry of the solution trajectories directly to the split-quaternion representation of the system. In particular, we describe the geometry of underdamped mechanical systems in the Lorentzian 3-space using models of the hyperbolic plane.

 
