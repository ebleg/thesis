% --------------------------------------------------------------------------------
%% PACKAGES
% --------------------------------------------------------------------------------

\usepackage{amsmath}
\usepackage{amssymb}
\usepackage{amsthm}
\usepackage{physics}
\usepackage{siunitx}
\usepackage[official]{eurosym}
\usepackage{xspace}
\usepackage[mathscr]{euscript}
\usepackage{alphabeta}
\usepackage{xfrac}
\usepackage{nomencl}
%\usepackage{dutchcal}
\usepackage{MnSymbol}
\usepackage{cancel}
\usepackage{romanbar}
\usepackage{colonequals}
\usepackage{stmaryrd}
\usepackage[sfdefault=cmbr,OMLmathsans]{isomath}  % Math alphabets for tensors and vectors
\usepackage{nth}
\usepackage{interval}
\intervalconfig{soft open fences}

\setlength{\jot}{10pt} % affecting the line spacing in the split environment

% Units as fractions
\sisetup{per-mode = fraction}%

% --------------------------------------------------------------------------------
%% MATH ALPHABET MACROS
% --------------------------------------------------------------------------------

% Replace upper case Greek fonts with their italicized variant, 
% put them in up[letter] macro's instead
\let\epsilon\varepsilon
\let\theta\vartheta

%\DeclareSymbolFont{tensors}{OML}{cmbrm}{b}{it}
%\DeclareSymbolFontAlphabet{\mathtens}{tensors}

% Vectors, Tensors, Matrices
% Vector notation: bold italic
\renewcommand{\vec}[1]{\ensuremath{\mathbold{#1}}}
\newcommand{\uvec}[1]{\ensuremath{\hat{\mathbold{#1}}}}
\newcommand{\mat}[1]{\ensuremath{\mathbold{#1}}}
\newcommand{\tens}[1]{\ensuremath{\mathboldsans{#1}}}
% additionally, there is the \mathsans font which is not used currently

% --------------------------------------------------------------------------------
% MATH CONSTANT MACROS
% --------------------------------------------------------------------------------

\newcommand{\pic}{\ensuremath{\textnormal{\pi}}}   
\newcommand{\ec}{\ensuremath{\mathrm{e}}}          % Euler's constant
\newcommand{\ii}{\ensuremath{\mathrm{i}}}          % Imaginary unit
\newcommand{\jj}{\ensuremath{\mathrm{j}}}          % Split-complex unit
%\newcommand{\jjp}{\ensuremath{\mathrm{j}_{_+}}}    % Split-complex unit
%\newcommand{\jjm}{\ensuremath{\mathrm{j}_{_-}}}    % Split-complex unit

% --------------------------------------------------------------------------------
% NUMBER SET & SPACE MACROS
% --------------------------------------------------------------------------------

% Spaces and number systems
\renewcommand{\real}{\ensuremath{\mathbb{R}}\xspace}
\newcommand{\field}{\ensuremath{\mathbb{F}}\xspace}

\newcommand{\quaternions}{\ensuremath{\mathbb{H}}\xspace}
\newcommand{\spquaternions}{\ensuremath{\hat{\mathbb{H}}}\xspace}
\newcommand{\complex}{\ensuremath{\mathbb{C}}\xspace}
\newcommand{\integer}{\ensuremath{\mathbb{Z}}\xspace}
\newcommand{\sphere}[1]{\ensuremath{\mathbb{S}^{#1}}\xspace}
\newcommand{\torus}[1]{\ensuremath{\mathbb{T}^{#1}}\xspace}

%<symbol: \real^n> <expl: Real coordinate space of dimension $n$> <tags: numberset, geometry>

% Tangent, cotangent, contact bundles
\newcommand{\ctbundle}[1]{\ensuremath{\mathrm{T}^*#1}}
\newcommand{\ctzbundle}[1]{\ensuremath{\mathrm{\dot{T}}^*#1}}
\newcommand{\tbundle}[1]{\ensuremath{\mathrm{T}#1}}

\newcommand{\ctspace}[2]{\ensuremath{\mathrm{T}_{#1}^*#2}}
\newcommand{\tspace}[2]{\ensuremath{\mathrm{T}_{#1} #2}}
\newcommand{\ctzspace}[2]{\ensuremath{\mathrm{\dot{T}}^*_{#1} #2}}

%<symbol: \tbundle{M}> <expl: Tangent bundle of the manifold $M$> <tags: misc, math, geometry> <sortkey: a3>
%<symbol: \ctbundle{M}> <expl: Cotangent bundle of the manifold $M$> <tags: misc, math, geometry> <sortkey: a4>
%<symbol: \tspace{x}{M}> <expl: Tangent space to the manifold $M$ at the point $x$> <tags: misc, math, geometry> <sortkey: a1>
%<symbol: \ctspace{x}{M}> <expl: Cotangent space of the manifold $M$ at the point $x$> <tags: misc, math, geometry> <sortkey: a2>

% Projectivized cotangent bundle and space
\newcommand{\pctbundle}[1]{\ensuremath{\mathbb{P}\mathrm{T}^* #1}}
\newcommand{\pctspace}[2]{\ensuremath{\mathbb{P}\mathrm{T}^*_{#1} #2}}

\newcommand{\cbundle}[1]{\ensuremath{\mathrm{C}#1}}
%\newcommand{\chbundle}[1]{\ensuremath{C^*\!#1}}

% Bundle sections
\newcommand{\bsection}[1]{\ensuremath{\mathrm{\Gamma}}\qty(#1)}

% Vector fields
\newcommand{\vfields}[1]{\ensuremath{\mathfrak{X}}\qty(#1)}
\newcommand{\vsfields}[2]{\ensuremath{\mathfrak{X}_{#1}}\qty(#2)}
%<symbol: \vfields{M}> <expl: Set of vector fields (smooth sections of \tbundle{M}) on the manifold $M$> <tags: misc, math, geometry> <sortkey: a5>

% Smooth functions
\newcommand{\functions}[1]{\ensuremath{C^\infty}\!\qty(#1)}
%<symbol: \functions{M}> <expl: Set of smooth functions on the manifold $M$> <tags: misc, math, geometry> <sortkey: a6>

% Differential n-forms
\newcommand{\nforms}[2]{\ensuremath{\Omega^{#1}(#2)}}
%<symbol: \nforms{n}{M}> <expl: Set of $n$-forms on the manifold $M$> <tags: misc, math, geometry> <sortkey: a7>

% --------------------------------------------------------------------------------
% ANALYTICAL MECHANICS 
% --------------------------------------------------------------------------------

\newcommand{\lag}{\ensuremath{L}}
%\newcommand{\ham}{\ensuremath{\mathscr{H}}}
\newcommand{\ham}{\ensuremath{H}}

\newcommand{\bundle}[3]{\ensuremath{#1\xrightarrow[]{#2}#3}}
%<symbol: \bundle{E}{\pi}{B}> <expl: Bundle with total space $E$, projection map $\pi$ and base space $B$> <tags: misc, math, geometry> <sortkey: a0>

\newcommand{\ecokin}{\ensuremath{T^*}\xspace}
\newcommand{\ekin}{\ensuremath{T}\xspace}
\newcommand{\epot}{\ensuremath{U}\xspace}
\newcommand{\ecopot}{\ensuremath{U^*}\xspace}

% --------------------------------------------------------------------------------
% GROUP MACROS
% --------------------------------------------------------------------------------

% (Lie) groups
\newcommand{\moebiusgroup}{\ensuremath{\text{Möb}}\xspace}
\newcommand{\automorphgroup}[1]{\ensuremath{\mathrm{Aut}(#1)}\xspace}
\newcommand{\pglgroup}[2]{\ensuremath{\mathrm{PGL}(#1, #2)}\xspace}
\newcommand{\pslgroup}[2]{\ensuremath{\mathrm{PSL}(#1, #2)}\xspace}
\newcommand{\glgroup}[2]{\ensuremath{\mathrm{GL}(#1, #2)}\xspace}
\newcommand{\slgroup}[2]{\ensuremath{\mathrm{SL}(#1, #2)}\xspace}
\newcommand{\sogroup}[2]{\ensuremath{\mathrm{SO}(#1, #2)}\xspace}
\newcommand{\sugroup}[1]{\ensuremath{\mathrm{SU}(#1)}\xspace} % SU(2) etc
\newcommand{\restlorentzgroup}{\ensuremath{\mathrm{SO^{+}(1, 3, \real)}}}
\newcommand{\spgroup}[1]{\ensuremath{\mathrm{Sp}(#1)}}
\newcommand{\firstff}{\ensuremath{\text{\textsf{\textbf{\Romanbar{I}}}}}}
\newcommand{\secondff}{\ensuremath{\firstff\!\firstff}}
\newcommand{\mgroup}{\ensuremath{\real_{\times}}}
\newcommand{\digroup}[1]{\ensuremath{\mathrm{D}_#1}}

% (Lie) algebras
\newcommand{\glalg}[2]{\ensuremath{\mathfrak{gl}(#1, #2)}\xspace}
\newcommand{\slalg}[2]{\ensuremath{\mathfrak{sl}(#1, #2)}\xspace}
\newcommand{\spalg}[1]{\ensuremath{\mathfrak{sp}(#1)}}

% --------------------------------------------------------------------------------
%% OPERATORS & SYMBOLS MACROS
% --------------------------------------------------------------------------------

% Exterior derivative
\newcommand{\extdiff}[1]{\ensuremath{\vb{d}#1}\xspace}

% Poisson bracket
\newcommand{\poisson}[2]{\ensuremath{\qty{#1,\,#2}}\xspace}

% Jacobi bracket
%\newcommand{\jacobi}[2]{\ensuremath{\left\llbracket #1,\,#2\right\rrbracket}}
\newcommand{\jacobi}[2]{\ensuremath{\{#1,\,#2\}}}
\newcommand{\schouten}[2]{\ensuremath{\llbracket#1,\,#2\rrbracket}}

% Lie bracket
\newcommand{\liebr}[2]{\ensuremath{\qty[#1,\,#2]}\xspace}

% Inner product
\newcommand{\inner}[2]{\ensuremath{\langle #1,\, #2\rangle}}
\newcommand{\lorinner}[2]{\ensuremath{\langle #1,\, #2\rangle_\text{\tiny{L}}}}

% Cross product
\newcommand{\crossp}[2]{\ensuremath{#1\times#2}}
\newcommand{\lorcrossp}[2]{\ensuremath{#1\times_\text{\tiny{L}} #2}}

% Wedge product
\newcommand{\wedgep}[2]{\ensuremath{ #1\,\wedge\,#2 }}

% Lie derivative
\newcommand{\lied}[2]{\ensuremath{\mbox{\Large $\mathsterling$}_{#1}#2}}

% Fiber derivative
\newcommand{\fiberder}[1]{\ensuremath{\mathbb{F}#1}}


% Interior product
\newcommand{\intpr}[2]{\ensuremath{#1 \righthalfcup\,#2}}

\newcommand{\corresponds}{\ensuremath{\quad \leftrightsquigarrow \quad}}
\newcommand{\conj}[1]{\ensuremath{#1^*}}
\DeclareMathOperator{\adjugate}{adj}
\DeclareMathOperator{\support}{supp}


% Quaternion
%\newcommand{\quati}{\ensuremath{\text{\textbf{î}}}\xspace}
%\newcommand{\quatj}{\ensuremath{\text{\textbf{\^{\j}}}\xspace}}
%\newcommand{\quatk}{\ensuremath{\vb{\hat{k}}}\xspace}

% Serif
\newcommand{\quati}{\ensuremath{\text{\textrm{\textit{\textbf{î}}}}}\xspace}
\newcommand{\quatj}{\ensuremath{\text{\textrm{\textit{\textbf{\^{\j}}}}}}\xspace}
\newcommand{\quatk}{\ensuremath{\vec{\hat{k}}}\xspace}

\newcommand{\quat}[4]{\ensuremath{#1 + #2\quati + #3\quatj + #4\quatk}\xspace}
\newcommand{\quatvec}[3]{\ensuremath{#1\quati + #2\quatj + #3\quatk}\xspace}

\newcommand{\vecpart}[1]{\ensuremath{\mathrm{vec}\qty(#1)}\xspace}
\newcommand{\scapart}[1]{\ensuremath{\mathrm{sca}\qty(#1)}\xspace}


\newcommand{\ctran}[1]{\ensuremath{#1^{\dagger}}}
\newcommand{\tran}[1]{\ensuremath{#1^{\top}}}

\newcommand{\largefrac}[2]{\frac{\displaystyle #1}{\displaystyle #2}}
\newcommand{\toDual}[1]{\ensuremath{#1^\flat}}
\newcommand{\fromDual}[1]{\ensuremath{#1^\sharp}}

%\newcommand{\christ}{\ensuremath{\mathrm{\Gamma}}}
%\renewcommand{\var}{\ensuremath{\mathrm{\delta}}}
\renewcommand{\var}{\ensuremath{\textnormal{\delta}}}   

\newcommand{\raction}[1]{\ensuremath{\blacktriangleleft\,#1}}

\DeclareMathOperator{\sgn}{sgn}


% --------------------------------------------------------------------------------
%% DISTRIBUTIONS
% --------------------------------------------------------------------------------

\newcommand{\gaussian}[2]{\ensuremath{\mathscr{N}\qty(#1,\,#2)}}

% --------------------------------------------------------------------------------
%% SHORTCUTS
% --------------------------------------------------------------------------------

% All with capital
\newcommand{\Lck}{\ensuremath{L_\text{\tiny{\textsc{CK}}}}}
\newcommand{\Hck}{\ensuremath{H_\text{\tiny{\textsc{CK}}}}}
\newcommand{\Pcan}{\ensuremath{\rho}}

% --------------------------------------------------------------------------------
%% SETTINGS 
% --------------------------------------------------------------------------------

\theoremstyle{definition}
\newtheorem{definition}{Definition}
\newtheorem*{definition_inf}{Definition}

\theoremstyle{plain}
\newtheorem*{question}{Question}

\theoremstyle{plain}
\newtheorem*{remark}{Remark}

%\DeclareSIUnit\money{\$}
%\DeclareSIUnit\year{yr}

\renewcommand\qedsymbol{$\blacksquare$}

% Economic engineering units
\DeclareSIUnit\quantity{\#}
\DeclareSIUnit\money{\text{\euro}}
\DeclareSIUnit\year{yr}
