% --------------------------------------------------------------------------------
%% PACKAGES
% --------------------------------------------------------------------------------

\usepackage{amsmath}
\usepackage{amssymb}
\usepackage{amsthm}
\usepackage{physics}
\usepackage{siunitx}
\usepackage[official]{eurosym}
\usepackage{xspace}
\usepackage[mathscr]{euscript}
\usepackage{alphabeta}
\usepackage{xfrac}
\usepackage{nomencl}
%\usepackage{dutchcal}
\usepackage{MnSymbol}
\usepackage{romanbar}
\usepackage[sfdefault=cmbr,OMLmathsans]{isomath}  % Math alphabets for tensors and vectors
\usepackage{nth}

% --------------------------------------------------------------------------------
%% MATH ALPHABET MACROS
% --------------------------------------------------------------------------------

% Replace upper case Greek fonts with their italicized variant, 
% put them in up[letter] macro's instead
\let\epsilon\varepsilon

%\DeclareSymbolFont{tensors}{OML}{cmbrm}{b}{it}
%\DeclareSymbolFontAlphabet{\mathtens}{tensors}

% Vectors, Tensors, Matrices
% Vector notation: bold italic
\renewcommand{\vec}[1]{\ensuremath{\mathbold{#1}}}
\newcommand{\uvec}[1]{\ensuremath{\hat{\mathbold{#1}}}}
\newcommand{\mat}[1]{\ensuremath{\mathbold{#1}}}
\newcommand{\tens}[1]{\ensuremath{\mathboldsans{#1}}}
% additionally, there is the \mathsans font which is not used currently

% --------------------------------------------------------------------------------
% MATH CONSTANT MACROS
% --------------------------------------------------------------------------------

\newcommand{\pic}{\ensuremath{\textnormal{\pi}}}   
\newcommand{\ec}{\ensuremath{\mathrm{e}}}          % Euler's constant
\newcommand{\ii}{\ensuremath{\mathrm{i}}}          % Imaginary unit
\newcommand{\jj}{\ensuremath{\mathrm{j}}}          % Split-complex unit
%\newcommand{\jjp}{\ensuremath{\mathrm{j}_{_+}}}    % Split-complex unit
%\newcommand{\jjm}{\ensuremath{\mathrm{j}_{_-}}}    % Split-complex unit

% --------------------------------------------------------------------------------
% NUMBER SET & SPACE MACROS
% --------------------------------------------------------------------------------

% Spaces and number systems
\renewcommand{\real}{\ensuremath{\mathbb{R}}\xspace}
\newcommand{\field}{\ensuremath{\mathbb{F}}\xspace}

\newcommand{\quaternions}{\ensuremath{\mathbb{H}}\xspace}
\newcommand{\spquaternions}{\ensuremath{\hat{\mathbb{H}}}\xspace}
\newcommand{\complex}{\ensuremath{\mathbb{C}}\xspace}
\newcommand{\integer}{\ensuremath{\mathbb{Z}}\xspace}
\newcommand{\sphere}[1]{\ensuremath{\mathbb{S}^{#1}}\xspace}
\newcommand{\torus}[1]{\ensuremath{\mathbb{T}^{#1}}\xspace}

% Tangent, cotangent, contact bundles
\newcommand{\ctbundle}[1]{\ensuremath{T^*\!#1}}
\newcommand{\tbundle}[1]{\ensuremath{T#1}}

\newcommand{\ctspace}[2]{\ensuremath{T_#1^*\!#2}}
\newcommand{\tspace}[2]{\ensuremath{T_#1#2}}

\newcommand{\clbundle}[1]{\ensuremath{C\!#1}}
\newcommand{\chbundle}[1]{\ensuremath{C^*\!#1}}

% Bundle sections
\newcommand{\bsection}[1]{\ensuremath{\mathrm{\Gamma}}\qty(#1)}

% Vector fields
\newcommand{\vfields}[1]{\ensuremath{\mathscr{X}}\qty(#1)}

% Smooth functions
\newcommand{\functions}[1]{\ensuremath{C^\infty}\!\qty(#1)}


% --------------------------------------------------------------------------------
% ANALYTICAL MECHANICS 
% --------------------------------------------------------------------------------

\newcommand{\lag}{\ensuremath{L}}
%\newcommand{\ham}{\ensuremath{\mathscr{H}}}
\newcommand{\ham}{\ensuremath{H}}

\newcommand{\bundle}[3]{\ensuremath{#1\xrightarrow[]{#2}#3}}

\newcommand{\ecokin}{\ensuremath{T^*}\xspace}
\newcommand{\ekin}{\ensuremath{T}\xspace}
\newcommand{\epot}{\ensuremath{U}\xspace}
\newcommand{\ecopot}{\ensuremath{U^*}\xspace}
\newcommand{\gpos}{\ensuremath{\vec{q}}\xspace}
\newcommand{\gvel}{\ensuremath{\dot{\vec{q}}}\xspace}
\newcommand{\gmom}{\ensuremath{\vec{p}}\xspace}

% --------------------------------------------------------------------------------
% GROUP MACROS
% --------------------------------------------------------------------------------

\newcommand{\moebiusgroup}{\ensuremath{\text{Möb}}\xspace}
\newcommand{\automorphgroup}[1]{\ensuremath{\mathrm{Aut}(#1)}\xspace}
\newcommand{\pglgroup}[2]{\ensuremath{\mathrm{PGL}(#1, #2)}\xspace}
\newcommand{\pslgroup}[2]{\ensuremath{\mathrm{PSL}(#1, #2)}\xspace}
\newcommand{\glgroup}[2]{\ensuremath{\mathrm{GL}(#1, #2)}\xspace}
\newcommand{\slgroup}[2]{\ensuremath{\mathrm{SL}(#1, #2)}\xspace}
\newcommand{\sogroup}{\ensuremath{\mathrm{SO}(3, \real)}\xspace}
\newcommand{\sugroup}[1]{\ensuremath{\mathrm{SU}(#1)}\xspace} % SU(2) etc
\newcommand{\restlorentzgroup}{\ensuremath{\mathrm{SO^{+}(1, 3, \real)}}}
\newcommand{\spgroup}[2]{\ensuremath{\mathrm{Sp}(#1, #2)}}
\newcommand{\firstff}{\ensuremath{\text{\textsf{\textbf{\Romanbar{I}}}}}}
\newcommand{\secondff}{\ensuremath{\firstff\!\firstff}}

% --------------------------------------------------------------------------------
%% OPERATORS & SYMBOLS MACROS
% --------------------------------------------------------------------------------

% Exterior derivative
\newcommand{\extdiff}[1]{\ensuremath{\vb{d}#1}}

% Poisson bracket
\newcommand{\poisson}[2]{\ensuremath{\qty{#1,\,#2}}}

% Lie bracket
\newcommand{\liebr}[2]{\ensuremath{\qty{#1,\,#2}}}

% Inner product
\newcommand{\inner}[2]{\ensuremath{\langle #1,\, #2\rangle}}
\newcommand{\lorinner}[2]{\ensuremath{\langle #1,\, #2\rangle_\text{\tiny{L}}}}

% Cross product
\newcommand{\crossp}[2]{\ensuremath{#1\times#2}}
\newcommand{\lorcrossp}[2]{\ensuremath{#1\times_\text{\tiny{L}} #2}}

% Wedge product
\newcommand{\wedgep}[2]{\ensuremath{ #1\,\wedge\,#2 }}

% Lie derivative
\newcommand{\lied}[2]{\ensuremath{\mbox{\Large $\mathsterling$}_{#1}#2}}

% Fiber derivative
\newcommand{\fiberder}[1]{\ensuremath{\mathbb{F}#1}}


% Interior product
\newcommand{\intpr}[2]{\ensuremath{#1\!\righthalfcup#2}}

\newcommand{\corresponds}{\ensuremath{\quad \leftrightsquigarrow \quad}}
\newcommand{\conj}[1]{\ensuremath{\bar{#1}}}

% Quaternion
\newcommand{\quat}[4]{\ensuremath{#1 + #2\vb{i} + #3\vb{j} + #4\vb{k}}\xspace}

\newcommand{\ctran}[1]{\ensuremath{#1^{\dagger}}}
\newcommand{\tran}[1]{\ensuremath{#1^{\top}}}

\newcommand{\largefrac}[2]{\frac{\displaystyle #1}{\displaystyle #2}}
\newcommand{\lowerIndex}[1]{\ensuremath{#1^\flat}}
\newcommand{\raiseIndex}[1]{\ensuremath{#1^\sharp}}

%\newcommand{\christ}{\ensuremath{\mathrm{\Gamma}}}
%\renewcommand{\var}{\ensuremath{\mathrm{\delta}}}
\renewcommand{\var}{\ensuremath{\textnormal{\delta}}}   

\DeclareMathOperator{\sgn}{sgn}


% --------------------------------------------------------------------------------
%% DISTRIBUTIONS
% --------------------------------------------------------------------------------

\newcommand{\gaussian}[2]{\ensuremath{\mathscr{N}\qty(#1,\,#2)}}

% --------------------------------------------------------------------------------
%% SHORTCUTS
% --------------------------------------------------------------------------------

% All with capital
\newcommand{\Lck}{\ensuremath{L_\text{\tiny{\textsc{CK}}}}}
\newcommand{\Hck}{\ensuremath{H_\text{\tiny{\textsc{CK}}}}}
\newcommand{\Pcan}{\ensuremath{\rho}}

% --------------------------------------------------------------------------------
%% SETTINGS 
% --------------------------------------------------------------------------------

\theoremstyle{definition}
\newtheorem{definition}{Definition}
\newtheorem*{definition_inf}{Definition}

\theoremstyle{plain}
\newtheorem*{question}{Question}

\theoremstyle{remark}
\newtheorem*{remark}{Remark}

\DeclareSIUnit\money{\$}
\DeclareSIUnit\year{yr}

\renewcommand\qedsymbol{$\blacksquare$}
