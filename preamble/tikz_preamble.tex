% --------------------------------------------------------------------
%% PACKAGES
% --------------------------------------------------------------------

\usepackage{tikz}
\usepackage{pgfplots}
\usepackage{etoolbox}

\usetikzlibrary{patterns,decorations.pathmorphing,positioning}
\usetikzlibrary{shapes,arrows}
\usetikzlibrary{arrows.meta}
\usetikzlibrary{positioning}
\usetikzlibrary{intersections, pgfplots.fillbetween}
\usetikzlibrary{decorations.markings}
\usetikzlibrary{calc}
\usetikzlibrary{cd}
\usepgfplotslibrary{fillbetween}
\usetikzlibrary{external}

\pgfkeys{/pgf/number format/.cd,1000 sep={\,}}
\pgfplotsset{compat=newest}
\pgfplotsset{every tick label/.append style={font=\tiny}}
\pgfplotsset{%
    colormap={accentmap}{color=(accent1) color=(accent3) color=(accent4)}%
}%    

% To make tikzcd work with externalize
\AtBeginEnvironment{tikzcd}{\tikzexternaldisable}
\AtEndEnvironment{tikzcd}{\tikzexternalenable}

% --------------------------------------------------------------------
%% SETTINGS
% --------------------------------------------------------------------

% Make sure that TikZ cache files are named properly
\newcommand{\inputtikz}[1]{%
  \tikzsetnextfilename{#1}%
  \input{#1}%
}

% --------------------------------------------------------------------
%% TEMPLATES FOR POINCARE DIAGRAM
% --------------------------------------------------------------------

\tikzset
 {every pin/.style = {pin edge = {black, <-}},    % pins are arrows from label to point
  > = stealth,                            % arrow tips look like stealth bombers
  flow/.style =    % everything marked as "flow" will be decorated with an arrow
   {decoration = {markings, mark=at position #1 with {\arrow{>}}},
    postaction = {decorate}
   },
  flow/.default = 0.5,          % default position of the arrow is in the middle
  main/.style = {line width=1pt}                    % thick lines for main graph
 }

% \newtemplate[Scaling, default 0.18]{\NameOfTemplate}{Caption}{Code}
%
% Typesets Code and stores it in the box \NameOfTemplate.
% This way we avoid nested tikzpictures when inserting the templates into the
% main picture, since nesting is not guaranteed to work.
\newcommand\newtemplate[4][0.18]%
 {\newsavebox#2%
  \savebox#2%
   {\begin{tabular}{@{}c@{}}
      \begin{tikzpicture}[scale=#1]
          #4
      \end{tikzpicture}\\[-1ex]
      \templatecaption{#3}\\[-1ex]
    \end{tabular}%
   }%
 }
\newcommand\template[1]{\usebox{#1}}             % use the Code stored in box #1
\newcommand\templatecaption[1]{{\sffamily\scriptsize#1}}       % typeset caption
%\newcommand\Tr{\mathop{\mathrm{Tr}}}

\newtemplate\sink{sink}%
 {\foreach \sx in {+,-}                   % for right/left half do:
   {\draw[flow] (\sx4,0) -- (0,0);        %   draw half of horizontal axis
    \draw[flow] (0,\sx4) -- (0,0);        %   draw half of vertical axis
    \foreach \sy in {+,-}                 %   for upper/lower quadrant do:
      \foreach \a/\b in {2/1,3/0.44}      %     draw two half-parabolas
        \draw[flow,domain=\sx\a:0] plot (\x, {\sy\b*\x*\x});
   }
 }

\newtemplate\source{source}%
 {\foreach \sx in {+,-}                   % for right/left half do:
   {\draw[flow] (0,0) -- (\sx4,0);        %   draw half of horizontal axis
    \draw[flow] (0,0) -- (0,\sx4);        %   draw half of vertical axis
    \foreach \sy in {+,-}                 %   for upper/lower quadrant do:
      \foreach \a/\b in {2/1,3/0.44}      %     draw two half-parabolas
        \draw[flow,domain=0:\sx\a] plot (\x, {\sy\b*\x*\x});
   }
 }

\newtemplate\stablefp{stable line}%
 {\draw (-4,0) -- (4,0);                  % draw horizontal axis
  \foreach \sy in {+,-}                   % for upper/lower half do:
   {\draw[flow] (0,\sy4) -- (0,0);        %   draw half of vertical axis
    \foreach \x in {-3,-2,-1,1,2,3}       %   draw six vertical half-lines
      \draw[flow] (\x,\sy3) -- (\x,0);
   }
   \node[circle, fill=white, inner sep = 0mm] at (3.5, 3.5) {\circled{2}};
 }

\newtemplate\unstablefp{unstable line}%
 {\draw (-4,0) -- (4,0);                  % draw horizontal axis
  \foreach \sy in {+,-}                   % for upper/lower half do:
   {\draw[flow] (0,0) -- (0,\sy4);        %   draw half of vertical axis
    \foreach \x in {-3,-2,-1,1,2,3}       %   draw six vertical half-lines
      \draw[flow] (\x,0) -- (\x,\sy3);
   }
   \node[circle, fill=white, inner sep =0mm] at (3.5, 3.5) {\circled{2}};
 }

\newtemplate\spiralsink{spiral sink}%
 {\draw (-4,0) -- (4,0);                  % draw horizontal axis
  \draw (0,-4) -- (0,4);                  % draw vertical axis
  \draw [samples=100,smooth,domain=27:7]  % draw spiral
       plot ({\x r}:{0.005*\x*\x});       % Using "flow" here gives "Dimension
  \def\x{26}                              %        too large", so we draw a tiny
  \draw[->] ({\x r}:{0.005*\x*\x}) -- +(0.01,-0.01);%     tangent for the arrow.
 }

\newtemplate\spiralsource{spiral source}%
 {\draw (-4,0) -- (4,0);                  % draw horizontal axis
  \draw (0,-4) -- (0,4);                  % draw vertical axis
  \draw [samples=100,smooth,domain=10:28] % draw spiral
       plot ({-\x r}:{0.005*\x*\x});      % Using "flow" here gives "Dimension
  \def\x{27.5}                            %        too large", so we draw a tiny
  \draw[<-] ({-\x r}:{0.005*\x*\x}) -- +(0.01,-0.008);%   tangent for the arrow.
 }

\newtemplate[0.15]\centre{center} % British spelling since \center is in use
 {\draw (-4,0) -- (4,0);                  % draw horizontal axis
  \draw (0,-4) -- (0,4);                  % draw vertical axis
  \foreach \r in {1,2,3}                   % draw three circles
    \draw[flow=0.63] (\r,0) arc (0:-360:\r cm);
   \node[circle, fill=white, inner sep = 0mm, outer sep=0mm] at (3, 3) {\circled{6}};
 }

\newtemplate\saddle{saddle}%
 {\foreach \sx in {+,-}                   % for right/left half do:
   {\draw[flow] (\sx4,0) -- (0,0);        %   draw half of horizontal axis
    \draw[flow] (0,0) -- (0,\sx4);        %   draw half of vertical axis
    \foreach \sy in {+,-}                 %   for upper/lower quadrant do:
      \foreach \a/\b/\c/\d in {2.8/0.3/0.7/0.6, 3.9/0.4/1.3/1.1}
        \draw[flow] (\sx\a,\sy\b)         %     draw two bent lines
          .. controls (\sx\c,\sy\d) and (\sx\d,\sy\c)
          .. (\sx\b,\sy\a);
   }
 }

\newtemplate\degensink{degenerate sink}%
 {\draw (0,-4) -- (0,4);                  % draw vertical axis
  \foreach \s in {+,-}                    % for upper/lower half do:
   {\draw[flow] (\s4,0) -- (0,0);         %   draw half of horizontal axis
    \foreach \a/\b/\c/\d in {3.5/4/1.5/1, 2.5/2/1/0.8}
      \draw[flow] (\s-3.5,\s\a)           %   draw two bent lines
        .. controls (\s\b,\s\c) and (\s\b,\s\d)
        .. (0,0);
   }
   \node[circle, fill=white, inner sep = 0mm] at (3.5, 3.5) {\circled{5}};
 }

\newtemplate\degensource{degenerate source}%
 {\draw (0,-4) -- (0,4);                  % draw vertical axis
  \foreach \s in {+,-}                    % for upper/lower half do:
   {\draw[flow] (0,0) -- (\s4,0);         %   draw half of horizontal axis
    \foreach \a/\b/\c/\d in {3.5/4/1.5/1, 2.5/2/1/0.8}
      \draw[flow] (0,0)                   %   draw two bent lines
        .. controls (\s\b,\s\d) and (\s\b,\s\c)
        .. (\s-3.5,\s\a);
   }
   \node[circle, fill=white, inner sep = 0mm] at (3.5, 3.5) {\circled{5}};
 }

