\chapter{Geometry of Lagrangian Mechanics}
\label{app:symplectic_geometry}

Just like the cotangent bundle, the tangent bundle admits a canonical structure, which is called the \emph{vertical endomorphism}. Its construction is slightly more convoluted than the canonical symplectic structure of the cotangent bundle, but nevertheless essential for a proper geometric interpretation of Lagrangian mechanics. 

\section{The double tangent bundle} The \emph{double tangent bundle} is the tangent bundle to \(\tbundle{Q}\), denoted by \(\tbundle{\tbundle{Q}}\). This space has not one but two canonical vector bundle structures, defined by projection maps from \(\tbundle{\tbundle{Q}} \to \tbundle{Q}\). First, there is the trivial projection \(\pi_{\tbundle{Q}}\) thats `forgets' about the tangent elements to \(\tbundle{Q}\). Secondly, there is \( (\pi_{Q})_* \) the pushforward (tangent map) of the projection map \(\pi_Q: \tbundle{Q} \to Q\). \cite{Abraham1978}
\begin{center}
   \begin{tikzcd} 
                    & \tbundle{(\tbundle{Q})} \arrow[rd, "\pi_{\tbundle{Q}}"] \arrow[ld, swap, "(\pi_Q)_*"] & \\
        \tbundle{Q} \arrow[rd, swap, "\pi_Q"] &   & \tbundle{Q} \arrow[ld, "\pi_Q"] \\
                    & Q        &  
   \end{tikzcd}
\end{center}
Vectors on the tangent bundle \(\tbundle{Q}\) (they live in \(\tbundle{(\tbundle{Q})}\)) are called vertical if they vanish under the action of \( (\pi_Q)_* \). These vectors point entirely in the `direction' of the fiber: in the Lagrangian formalism, they reflect a pure change in velocity, and no change in the generalized position. The \emph{vertical lift} \(\Psi\) maps a vector on \(Q\) to a vertical vector on \(\tbundle{Q}\). \cite{Carinena1990}
\begin{equation}
    \begin{split}
        \Psi_{\vec{v}}: \tspace{q}{Q} \to & \tspace{\vec{v}}{\tspace{q}{Q}}: \\ 
        & \Psi_{\vec{v}}(\vec{w})\,f = \left. \dv{}{t} f(\vec{v} + t\vec{w})\right |_{t = 0}
        \qquad q \in Q,\:\: \vec{v},\vec{w}\in \tspace{q}{Q},\:\: f \in \functions{\tbundle{Q}}.
    \end{split}
\end{equation}
In components, the effect of the vertical lift is as follows:
\begin{equation}
     \Psi_{\vec{v}}: \quad \vec{w} = \left. w^i \pdv{}{q^i}\right|_q \quad \mapsto \quad  \Psi_{\vec{v}}(\vec{w}) = \left.w^i \pdv{}{v^i}\right|_{(q, \vec{v})}.
\end{equation}
The vertical lift can also lift entire sections of \(\tbundle{Q}\) by simply applying the vertical lift pointwise.

Using the concept of the vertical lift, we can define the \emph{vertical isomorphism} \(S\) from the double tangent bundle to itself, first by projecting with \((\pi_Q)_*\) and then lifting again:
\begin{equation}
     S: \tbundle{\tbundle{Q}} \to \tbundle{\tbundle{Q}}: \quad S(q, \vec{v})\,u = \qty(\Psi_{\vec{v}} \circ (\pi_Q)_*)\,u \qquad u \in \tspace{(q, \vec{v})}{\tbundle{Q}}.
\end{equation}
The action of \(S\) can also be stated in the form of the following diagram:
\begin{center}
   \begin{tikzcd} 
        \tbundle{\tbundle{Q}} \arrow[d, swap, "(\pi_Q)_*"] \arrow[r, "S"] & \tbundle{\tbundle{Q}} \\
        \tbundle{Q} \arrow[r, swap, "\mathrm{id}_{\tbundle{Q}}"] &  \tbundle{Q} \arrow[swap, u, "\Psi"]  \\
   \end{tikzcd}.
\end{center}
The action of the vertical endomorphism on the chart-induced basis is:
\begin{equation}
     S: \quad \left. \pdv{}{q^i}\right|_{(q, \vec{v})} \mapsto \left.\pdv{}{v^i}\right|_{(q, \vec{v})} \qquad  \left. \pdv{}{v^i}\right|_{(q, \vec{v})} \mapsto 0.
\end{equation}
The vertical isomorphism is therefore a tensor of valence (1, 1) --- it takes a vector and produces another. Locally, \(S\) can be expressed as:
\begin{equation}
     S = \pdv{}{v^i}\otimes \dd{q^i}.
\end{equation}
%<symbol: \otimes> <expl: Tensor product> <tags: math, misc>
with \(\otimes\) being the tensor product \cite{Carinena1990}.

The Lagrangian formalism only applies to second-order vector fields. A second-order vector field is a vector field \(X\) such that \((\pi_Q \circ X) = \mathrm{id}_{\tbundle{Q}}\); i.e. the following diagram commutes: \cite{Abraham1978}
\begin{center}
   \begin{tikzcd} 
                    & \tbundle{\tbundle{Q}} \arrow[ld, swap, "(\pi_Q)_*"] & \\
        \tbundle{Q} \arrow[rr, swap, "\mathrm{id}_{\tbundle{Q}}"] &   & \tbundle{Q} \arrow[swap, lu, "X"]  \\
   \end{tikzcd}.
\end{center}
The identity on \(\tbundle{Q}\) is \(\mathrm{id}_{\tbundle{Q}}: (q, \vec{v}) \mapsto (q, \vec{v})\). Therefore, for a vector field \(X\) to be second order, we should have that the component in \(\pdv{}{q^i}\) that is picked out by \((\pi_Q)_*\) should be equal to \(v^i\); for example
\begin{equation}
     X = \sum_{i=1}^n \qty[v^i \pdv{q^i} + F^i\pdv{v^i}].
\end{equation}
The corresponding differential equations are
\begin{equation}
     \dv{q^i}{t} = v^i \qquad \dv{v^i}{t} = F^i,
\end{equation}
which means that the second-order vector field coincides with the notion of a `second-order differential equation' in \(q^i\).

\section{The Euler-Lagrange equations} With the infrastructure set up in the preceding section, we can now define the precise geometric setting of Lagrangian mechanics. Given a Lagrangian function \(L \in \functions{\tbundle{Q}}\), define the \emph{Lagrange 1-form}\footnote
{\citet{Carinena1990} calls \(\theta\) the Euler-Poincaré 1-form.}
\begin{equation}
    \theta_L \coloneq \dd{L} \circ S = \sum_{j=1}^n \pdv{L}{v^j}\dd{q^j}.
\end{equation}
Observe that the Lagrange 1-form is also equal to the pullback of the Liouville form under the Legendre transformation: \(\theta_L = (\fiberder{L})^* \theta\). \cite{Abraham1978}
Secondly, we define the \emph{Lagrange 2-form} as: \cite{Abraham1978,Carinena1990}
\begin{equation}
    \omega_L \coloneq - \dd{\theta}_L = \pdv[2]{L}{v^i}{v^j}\wedgep{\dd{q^j}}{\dd{v^i}} + \pdv[2]{L}{q^i}{v^j}\wedgep{\dd{q^j}}{\dd{q^i}}.
\end{equation}
Because the exterior derivative and the pullback commute, the Lagrange 2-form is equal to the pullback of the symplectic 2-form under the Legendre transform. If the rank of the Hessian \( \pdv[2]{L}{v^i}{v^j}\) is full (and constant), then \(\omega_L\) is nondegenerate and therefore defines a symplectic structure on \(\tbundle{Q}\). However, observe that whether \(\omega_L\) is symplectic or not depends on the nature of the Lagrangian, while the symplectic structure in the Hamiltonian setting is canonically derived from the cotangent bundle itself --- there is no need for the Hamiltonian to be regular.

The final ingredient for the Euler-Lagrange equations is the \emph{energy function}
\begin{equation}
     E \coloneq Z(L) - L,
\end{equation}
where \(Z = \sum v^i \pdv{}{v^i}\) is the Liouville vector field on \(\tbundle{Q}\).

The \emph{Lagrangian vector field} \(X_L\) is then the unique vector field that satisfies the equation: \cite{Godbillon1969}
\begin{equation}
    \intpr{X_L}{\omega_L} = \dd{E},
    \label{eq:EL_nocomp}
\end{equation}
In components, the right hand side of this equation is:
\begin{equation}
    \begin{split}
        \dd{E} &= \sum_{i, j}\qty(\pdv[2]{L}{v^j}{q^i}v^j\dd{q^i} + \pdv[2]{L}{v^j}{v^i}v^j\dd{v^i} + \pdv{L}{v^j}\dd{v^j}) - \dd{L}, \\
        \dd{E} &= \sum_{i, j}\qty(\pdv[2]{L}{v^j}{q^i}v^j\dd{q^i} + \pdv[2]{L}{v^j}{v^i}v^j\dd{v^i} - \pdv{L}{q^j}\dd{q^j}).
    \end{split}
    \label{eq:dE}
\end{equation}
Let \(X_L = \sum_i \qty( A^i \pdv{}{q^i} + B^i\pdv{}{v^i}) \); the left hand side can then be written as follows:
\begin{equation}
    \intpr{X_L}{\omega_L} =  - \sum_{i,j} A^i \pdv[2]{L}{q^i}{v^j}\dd{q^j} 
                             + \sum_{i,j} A^j \pdv[2]{L}{q^i}{v^j}\dd{q^i} 
                             - \sum_{i, j} B^i \pdv[2]{L}{v^i}{v^j}\dd{q^j}
                             + \sum_{i, j} A^j \pdv[2]{L}{v^i}{v^j}\dd{v^i}.
\end{equation}
Comparing this expression with \cref{eq:dE}, it is immediately clear that
\begin{equation}
     A^j \pdv[2]{L}{v^i}{v^j} = v^j \pdv[2]{L}{v^i}{v^j}.
\end{equation}
We therefore have that \(A^j = v^j\), but \emph{only} if the Hessian of \(L\) with respect to the velocities is nonsingular. If this is indeed the case (i.e. \(L\) is regular), and the condition implies that the vector field \(X_L\) is second-order. We can use this knowledge to obtain a second condition (since the terms in \(\dd{q^i}\) cancel): 
\begin{equation}
     \sum_{i} B^i \pdv[2]{L}{v^i}{v^j} = \pdv{L}{q^j} - \sum_{i} v^i \pdv[2]{L}{q^i}{v^j}.
\end{equation}
The Hessian of \(L\) in the velocities \(M_{ij} = \pdv[2]{L}{v^i}{v^j}\) is also called the mass matrix of the system. We have already assumed that this matrix is invertible (i.e. \(L\) is regular). As such, we have that
\begin{equation}
     \sum_{i}\pdv[2]{L}{v^i}{v^j}\dv[2]{q^j}{t} + \sum_{i} \pdv[2]{L}{q^i}{v^j}\dv{q^i}{t} = \pdv{L}{q^j},
\end{equation}
or equivalently
\begin{equation}
     \dv{}{t}\qty(\pdv{L}{v^j}) - \pdv{L}{q^j} = 0,
\end{equation}
which is the traditional form of the Euler-Lagrange equations.

Provided that \(X_L\) is a second-order vector field, the equation \cref{eq:EL_nocomp} is equivalent to the following statement:
\begin{equation}
    \lied{X_L}{\theta_L} = \dd{L}.
\end{equation}
The equivalence is easily shown using the Cartan formula:
\begin{equation*}
    \begin{split}
        \lied{X_L}{\theta_L} &= \dd{L} \\
        \dd{\qty(\intpr{X_L}{\theta_L})} + \intpr{X_L}{\dd{\theta_L}} &= \dd{L} \\
        \dd{\qty(\intpr{X_L}{\theta_L})} - \intpr{X_L}{\omega_L} &= \dd{L}
    \end{split}
\end{equation*}
The fact that \(X_L\) is second-order implies that \(\intpr{X_L}{\theta_L} = Z(L)\). Therefore 
\begin{equation*}
    \begin{split}
        \dd{(Z(L))} - \intpr{X_L}{\omega_L} &= \dd{L} \\
        \intpr{X_L}{\omega_L} &= \dd{Z(L) - L} \\
        \intpr{X_L}{\omega_L} &= \dd{E}.
    \end{split}
\end{equation*}
Lagrangians are not unique: from \cref{eq:EL_nocomp} we can deduce that the addition of a closed 1-form (as a map from \(\tbundle{Q} \to \real\)) to the Lagrangian will not alter the Euler-Lagrange equations. The closed 1-forms on \(Q\)  therefore constitute the \emph{gauge group} of Lagrangian mechanics. An equivalent statement is that the Euler-Lagrange equations remain invariant if a total derivative is added to the Lagrangian function. \cite{Abraham1978}
