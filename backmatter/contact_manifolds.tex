\chapter{Contact geometry}
\label{app:contact_geometry}
This appendix provides a short introduction to the basic concepts of contact geometry that are relevant in this thesis, particularly \cref{chap:contact_mechanics}.

%Contact manifolds are odd-dimensional manifolds with the addition of a contact structure. This contact structure can be considered to be like a symplectic structure (which is necessarily even-dimensional) with the addition of one `special' dimension. The relation between contact structures and symplectic structures is crucial for the extension of Lagrangian and Hamiltonian mechanics to contact manifolds.
\section{Contact structures}
\label{sec:contact_structures}
A \emph{contact element} on a manifold $M$ is a point $m \in M$ combined with a tangent hyperplane $\xi_m \subset \tspace{m}{M}$ (a subspace of the tangent space  with codimension 1). The term `contact' refers to the intuitive notion that if two submanifolds `touch', they share a contact element: they are \emph{in contact} (which is a slightly weaker condition than tangency). \cite{Cannas2001} For example, contact elements to a two-dimensional manifold are simply lines through the origin in the tangent space, contact elements on a three-dimensional manifold are planes through the origin, etc.

A \emph{contact manifold} is a manifold $M$ (of dimension $2n+1$) with a \emph{contact structure}, which is a smooth field (or distribution) of contact elements on $M$. Locally, any contact element determines a 1-form $\alpha$ (up to multiplication by a nonzero scalar) whose kernel constitutes the tangent hyperplane distribution, i.e. 
\begin{equation}
    \xi_m = \ker \alpha_m
    \label{eq:contact_form}
\end{equation}
This $\alpha$ is called the (local) \emph{contact form}, and it acts like a `normal (co-)vector' to the hyperplane. For the field hyperplanes to be a constact structure, it must satisfy a nondegeneracy condition: it should be \emph{nonintegrable}. This can be expressed as the the Frobenius condition for nonintegrability: \cite{Cannas2001,Abraham1978,Arnold1989}
$$ \wedgep{\alpha}{(\dd{\alpha})^n} \neq 0, $$
where integrable distributions would have this expression vanish everywhere. Roughly equivalent statements are that (i) one cannot find foliations of $M$ such that the $\xi$ is everywhere tangent to it, or (ii) that $\dd{\alpha}\vert_\xi$ is a \emph{symplectic form}. In this treatment, all contact forms are assumed to be global, which is the case if the quotient $TM/\xi$ is a trivial line bundle, i.e. the orientation is preserved across the entire manifold \cite{Geiges2008}.

The \emph{Darboux theorem} for contact manifolds states that it is always possible to find coordinates $z, x_i, y_i$ such that locally, the contact form is equal to 
$$ \dd{z} - \sum y_i\dd{x_i}, $$
which is also called the standard or natural contact structure. The standard contact structure on $\real^3$ is illustrated in \cref{fig:standard_contact}.
\begin{figure}
    \centering
    % This file was created by matlab2tikz.
%
%The latest updates can be retrieved from
%  http://www.mathworks.com/matlabcentral/fileexchange/22022-matlab2tikz-matlab2tikz
%where you can also make suggestions and rate matlab2tikz.
%
\begin{tikzpicture}

\begin{axis}[%
    width=4.in,
    height=2.8in,
    at={(0.772in,0.457in)},
    scale only axis,
    plot box ratio=3 3 1,
    xmin=-1.2,
    grid,
    3d box = complete,
    xmax=1.2,
    tick align=outside,
    ymin=-1.2,
    ymax=1.2,
    zmin=-0.4,
    zmax=0.4,
    view={-39.1473836532351}{27.2327575315137},
    xlabel = $x$,
    ylabel = $y$,
    zlabel = $z$,
    axis background/.style={fill=white},
    %axis x line*=origin,
    %axis y line*=origin,
    %axis z line*=origin
    %axis lines = middle,
]

\addplot3[area legend, draw=black, fill=accent1, forget plot]
table[row sep=crcr] {%
x	y	z\\
0.936360389693211	0.91	-0.0636396103067893\\
0.936360389693211	1.09	-0.0636396103067893\\
1.06363961030679	1.09	0.0636396103067893\\
1.06363961030679	0.91	0.0636396103067893\\
}--cycle;

\addplot3[area legend, draw=black, fill=accent1, forget plot]
table[row sep=crcr] {%
x	y	z\\
0.929721807150127	0.71	-0.0562225542798982\\
0.929721807150127	0.89	-0.0562225542798982\\
1.07027819284987	0.89	0.0562225542798982\\
1.07027819284987	0.71	0.0562225542798982\\
}--cycle;

\addplot3[area legend, draw=black, fill=accent1, forget plot]
table[row sep=crcr] {%
x	y	z\\
0.922825636685871	0.51	-0.0463046179884774\\
0.922825636685871	0.69	-0.0463046179884774\\
1.07717436331413	0.69	0.0463046179884774\\
1.07717436331413	0.51	0.0463046179884774\\
}--cycle;

\addplot3[area legend, draw=black, fill=accent1, forget plot]
table[row sep=crcr] {%
x	y	z\\
0.916437097820327	0.31	-0.0334251608718693\\
0.916437097820327	0.49	-0.0334251608718693\\
1.08356290217967	0.49	0.0334251608718693\\
1.08356290217967	0.31	0.0334251608718693\\
}--cycle;

\addplot3[area legend, draw=black, fill=accent1, forget plot]
table[row sep=crcr] {%
x	y	z\\
0.911747739187817	0.11	-0.0176504521624366\\
0.911747739187817	0.29	-0.0176504521624366\\
1.08825226081218	0.29	0.0176504521624366\\
1.08825226081218	0.11	0.0176504521624366\\
}--cycle;

\addplot3[area legend, draw=black, fill=accent1, forget plot]
table[row sep=crcr] {%
x	y	z\\
0.91	-0.09	-0\\
0.91	0.09	-0\\
1.09	0.09	0\\
1.09	-0.09	0\\
}--cycle;

\addplot3[area legend, draw=black, fill=accent1, forget plot]
table[row sep=crcr] {%
x	y	z\\
0.911747739187817	-0.29	0.0176504521624366\\
0.911747739187817	-0.11	0.0176504521624366\\
1.08825226081218	-0.11	-0.0176504521624366\\
1.08825226081218	-0.29	-0.0176504521624366\\
}--cycle;

\addplot3[area legend, draw=black, fill=accent1, forget plot]
table[row sep=crcr] {%
x	y	z\\
0.916437097820327	-0.49	0.0334251608718693\\
0.916437097820327	-0.31	0.0334251608718693\\
1.08356290217967	-0.31	-0.0334251608718693\\
1.08356290217967	-0.49	-0.0334251608718693\\
}--cycle;

\addplot3[area legend, draw=black, fill=accent1, forget plot]
table[row sep=crcr] {%
x	y	z\\
0.922825636685871	-0.69	0.0463046179884774\\
0.922825636685871	-0.51	0.0463046179884774\\
1.07717436331413	-0.51	-0.0463046179884774\\
1.07717436331413	-0.69	-0.0463046179884774\\
}--cycle;

\addplot3[area legend, draw=black, fill=accent1, forget plot]
table[row sep=crcr] {%
x	y	z\\
0.929721807150127	-0.89	0.0562225542798982\\
0.929721807150127	-0.71	0.0562225542798982\\
1.07027819284987	-0.71	-0.0562225542798982\\
1.07027819284987	-0.89	-0.0562225542798982\\
}--cycle;

\addplot3[area legend, draw=black, fill=accent1, forget plot]
table[row sep=crcr] {%
x	y	z\\
0.936360389693211	-1.09	0.0636396103067893\\
0.936360389693211	-0.91	0.0636396103067893\\
1.06363961030679	-0.91	-0.0636396103067893\\
1.06363961030679	-1.09	-0.0636396103067893\\
}--cycle;

\addplot3[area legend, draw=black, fill=accent1, forget plot]
table[row sep=crcr] {%
x	y	z\\
0.736360389693211	0.91	-0.0636396103067893\\
0.736360389693211	1.09	-0.0636396103067893\\
0.863639610306789	1.09	0.0636396103067893\\
0.863639610306789	0.91	0.0636396103067893\\
}--cycle;

\addplot3[area legend, draw=black, fill=accent1, forget plot]
table[row sep=crcr] {%
x	y	z\\
0.729721807150127	0.71	-0.0562225542798982\\
0.729721807150127	0.89	-0.0562225542798982\\
0.870278192849873	0.89	0.0562225542798982\\
0.870278192849873	0.71	0.0562225542798982\\
}--cycle;

\addplot3[area legend, draw=black, fill=accent1, forget plot]
table[row sep=crcr] {%
x	y	z\\
0.722825636685871	0.51	-0.0463046179884774\\
0.722825636685871	0.69	-0.0463046179884774\\
0.877174363314129	0.69	0.0463046179884774\\
0.877174363314129	0.51	0.0463046179884774\\
}--cycle;

\addplot3[area legend, draw=black, fill=accent1, forget plot]
table[row sep=crcr] {%
x	y	z\\
0.716437097820327	0.31	-0.0334251608718693\\
0.716437097820327	0.49	-0.0334251608718693\\
0.883562902179673	0.49	0.0334251608718693\\
0.883562902179673	0.31	0.0334251608718693\\
}--cycle;

\addplot3[area legend, draw=black, fill=accent1, forget plot]
table[row sep=crcr] {%
x	y	z\\
0.711747739187817	0.11	-0.0176504521624366\\
0.711747739187817	0.29	-0.0176504521624366\\
0.888252260812183	0.29	0.0176504521624366\\
0.888252260812183	0.11	0.0176504521624366\\
}--cycle;

\addplot3[area legend, draw=black, fill=accent1, forget plot]
table[row sep=crcr] {%
x	y	z\\
0.71	-0.09	-0\\
0.71	0.09	-0\\
0.89	0.09	0\\
0.89	-0.09	0\\
}--cycle;

\addplot3[area legend, draw=black, fill=accent1, forget plot]
table[row sep=crcr] {%
x	y	z\\
0.711747739187817	-0.29	0.0176504521624366\\
0.711747739187817	-0.11	0.0176504521624366\\
0.888252260812183	-0.11	-0.0176504521624366\\
0.888252260812183	-0.29	-0.0176504521624366\\
}--cycle;

\addplot3[area legend, draw=black, fill=accent1, forget plot]
table[row sep=crcr] {%
x	y	z\\
0.716437097820327	-0.49	0.0334251608718693\\
0.716437097820327	-0.31	0.0334251608718693\\
0.883562902179673	-0.31	-0.0334251608718693\\
0.883562902179673	-0.49	-0.0334251608718693\\
}--cycle;

\addplot3[area legend, draw=black, fill=accent1, forget plot]
table[row sep=crcr] {%
x	y	z\\
0.722825636685871	-0.69	0.0463046179884774\\
0.722825636685871	-0.51	0.0463046179884774\\
0.877174363314129	-0.51	-0.0463046179884774\\
0.877174363314129	-0.69	-0.0463046179884774\\
}--cycle;

\addplot3[area legend, draw=black, fill=accent1, forget plot]
table[row sep=crcr] {%
x	y	z\\
0.729721807150127	-0.89	0.0562225542798982\\
0.729721807150127	-0.71	0.0562225542798982\\
0.870278192849873	-0.71	-0.0562225542798982\\
0.870278192849873	-0.89	-0.0562225542798982\\
}--cycle;

\addplot3[area legend, draw=black, fill=accent1, forget plot]
table[row sep=crcr] {%
x	y	z\\
0.736360389693211	-1.09	0.0636396103067893\\
0.736360389693211	-0.91	0.0636396103067893\\
0.863639610306789	-0.91	-0.0636396103067893\\
0.863639610306789	-1.09	-0.0636396103067893\\
}--cycle;

\addplot3[area legend, draw=black, fill=accent1, forget plot]
table[row sep=crcr] {%
x	y	z\\
0.536360389693211	0.91	-0.0636396103067893\\
0.536360389693211	1.09	-0.0636396103067893\\
0.663639610306789	1.09	0.0636396103067893\\
0.663639610306789	0.91	0.0636396103067893\\
}--cycle;

\addplot3[area legend, draw=black, fill=accent1, forget plot]
table[row sep=crcr] {%
x	y	z\\
0.529721807150127	0.71	-0.0562225542798982\\
0.529721807150127	0.89	-0.0562225542798982\\
0.670278192849873	0.89	0.0562225542798982\\
0.670278192849873	0.71	0.0562225542798982\\
}--cycle;

\addplot3[area legend, draw=black, fill=accent1, forget plot]
table[row sep=crcr] {%
x	y	z\\
0.522825636685871	0.51	-0.0463046179884774\\
0.522825636685871	0.69	-0.0463046179884774\\
0.677174363314129	0.69	0.0463046179884774\\
0.677174363314129	0.51	0.0463046179884774\\
}--cycle;

\addplot3[area legend, draw=black, fill=accent1, forget plot]
table[row sep=crcr] {%
x	y	z\\
0.516437097820327	0.31	-0.0334251608718693\\
0.516437097820327	0.49	-0.0334251608718693\\
0.683562902179673	0.49	0.0334251608718693\\
0.683562902179673	0.31	0.0334251608718693\\
}--cycle;

\addplot3[area legend, draw=black, fill=accent1, forget plot]
table[row sep=crcr] {%
x	y	z\\
0.511747739187817	0.11	-0.0176504521624366\\
0.511747739187817	0.29	-0.0176504521624366\\
0.688252260812183	0.29	0.0176504521624366\\
0.688252260812183	0.11	0.0176504521624366\\
}--cycle;

\addplot3[area legend, draw=black, fill=accent1, forget plot]
table[row sep=crcr] {%
x	y	z\\
0.51	-0.09	-0\\
0.51	0.09	-0\\
0.69	0.09	0\\
0.69	-0.09	0\\
}--cycle;

\addplot3[area legend, draw=black, fill=accent1, forget plot]
table[row sep=crcr] {%
x	y	z\\
0.511747739187817	-0.29	0.0176504521624366\\
0.511747739187817	-0.11	0.0176504521624366\\
0.688252260812183	-0.11	-0.0176504521624366\\
0.688252260812183	-0.29	-0.0176504521624366\\
}--cycle;

\addplot3[area legend, draw=black, fill=accent1, forget plot]
table[row sep=crcr] {%
x	y	z\\
0.516437097820327	-0.49	0.0334251608718693\\
0.516437097820327	-0.31	0.0334251608718693\\
0.683562902179673	-0.31	-0.0334251608718693\\
0.683562902179673	-0.49	-0.0334251608718693\\
}--cycle;

\addplot3[area legend, draw=black, fill=accent1, forget plot]
table[row sep=crcr] {%
x	y	z\\
0.522825636685871	-0.69	0.0463046179884774\\
0.522825636685871	-0.51	0.0463046179884774\\
0.677174363314129	-0.51	-0.0463046179884774\\
0.677174363314129	-0.69	-0.0463046179884774\\
}--cycle;

\addplot3[area legend, draw=black, fill=accent1, forget plot]
table[row sep=crcr] {%
x	y	z\\
0.529721807150127	-0.89	0.0562225542798982\\
0.529721807150127	-0.71	0.0562225542798982\\
0.670278192849873	-0.71	-0.0562225542798982\\
0.670278192849873	-0.89	-0.0562225542798982\\
}--cycle;

\addplot3[area legend, draw=black, fill=accent1, forget plot]
table[row sep=crcr] {%
x	y	z\\
0.536360389693211	-1.09	0.0636396103067893\\
0.536360389693211	-0.91	0.0636396103067893\\
0.663639610306789	-0.91	-0.0636396103067893\\
0.663639610306789	-1.09	-0.0636396103067893\\
}--cycle;

\addplot3[area legend, draw=black, fill=accent1, forget plot]
table[row sep=crcr] {%
x	y	z\\
0.336360389693211	0.91	-0.0636396103067893\\
0.336360389693211	1.09	-0.0636396103067893\\
0.463639610306789	1.09	0.0636396103067893\\
0.463639610306789	0.91	0.0636396103067893\\
}--cycle;

\addplot3[area legend, draw=black, fill=accent1, forget plot]
table[row sep=crcr] {%
x	y	z\\
0.329721807150127	0.71	-0.0562225542798982\\
0.329721807150127	0.89	-0.0562225542798982\\
0.470278192849873	0.89	0.0562225542798982\\
0.470278192849873	0.71	0.0562225542798982\\
}--cycle;

\addplot3[area legend, draw=black, fill=accent1, forget plot]
table[row sep=crcr] {%
x	y	z\\
0.322825636685871	0.51	-0.0463046179884774\\
0.322825636685871	0.69	-0.0463046179884774\\
0.477174363314129	0.69	0.0463046179884774\\
0.477174363314129	0.51	0.0463046179884774\\
}--cycle;

\addplot3[area legend, draw=black, fill=accent1, forget plot]
table[row sep=crcr] {%
x	y	z\\
0.316437097820327	0.31	-0.0334251608718693\\
0.316437097820327	0.49	-0.0334251608718693\\
0.483562902179673	0.49	0.0334251608718693\\
0.483562902179673	0.31	0.0334251608718693\\
}--cycle;

\addplot3[area legend, draw=black, fill=accent1, forget plot]
table[row sep=crcr] {%
x	y	z\\
0.311747739187817	0.11	-0.0176504521624366\\
0.311747739187817	0.29	-0.0176504521624366\\
0.488252260812183	0.29	0.0176504521624366\\
0.488252260812183	0.11	0.0176504521624366\\
}--cycle;

\addplot3[area legend, draw=black, fill=accent1, forget plot]
table[row sep=crcr] {%
x	y	z\\
0.31	-0.09	-0\\
0.31	0.09	-0\\
0.49	0.09	0\\
0.49	-0.09	0\\
}--cycle;

\addplot3[area legend, draw=black, fill=accent1, forget plot]
table[row sep=crcr] {%
x	y	z\\
0.311747739187817	-0.29	0.0176504521624366\\
0.311747739187817	-0.11	0.0176504521624366\\
0.488252260812183	-0.11	-0.0176504521624366\\
0.488252260812183	-0.29	-0.0176504521624366\\
}--cycle;

\addplot3[area legend, draw=black, fill=accent1, forget plot]
table[row sep=crcr] {%
x	y	z\\
0.316437097820327	-0.49	0.0334251608718693\\
0.316437097820327	-0.31	0.0334251608718693\\
0.483562902179673	-0.31	-0.0334251608718693\\
0.483562902179673	-0.49	-0.0334251608718693\\
}--cycle;

\addplot3[area legend, draw=black, fill=accent1, forget plot]
table[row sep=crcr] {%
x	y	z\\
0.322825636685871	-0.69	0.0463046179884774\\
0.322825636685871	-0.51	0.0463046179884774\\
0.477174363314129	-0.51	-0.0463046179884774\\
0.477174363314129	-0.69	-0.0463046179884774\\
}--cycle;

\addplot3[area legend, draw=black, fill=accent1, forget plot]
table[row sep=crcr] {%
x	y	z\\
0.329721807150127	-0.89	0.0562225542798982\\
0.329721807150127	-0.71	0.0562225542798982\\
0.470278192849873	-0.71	-0.0562225542798982\\
0.470278192849873	-0.89	-0.0562225542798982\\
}--cycle;

\addplot3[area legend, draw=black, fill=accent1, forget plot]
table[row sep=crcr] {%
x	y	z\\
0.336360389693211	-1.09	0.0636396103067893\\
0.336360389693211	-0.91	0.0636396103067893\\
0.463639610306789	-0.91	-0.0636396103067893\\
0.463639610306789	-1.09	-0.0636396103067893\\
}--cycle;

\addplot3[area legend, draw=black, fill=accent1, forget plot]
table[row sep=crcr] {%
x	y	z\\
0.136360389693211	0.91	-0.0636396103067893\\
0.136360389693211	1.09	-0.0636396103067893\\
0.263639610306789	1.09	0.0636396103067893\\
0.263639610306789	0.91	0.0636396103067893\\
}--cycle;

\addplot3[area legend, draw=black, fill=accent1, forget plot]
table[row sep=crcr] {%
x	y	z\\
0.129721807150127	0.71	-0.0562225542798982\\
0.129721807150127	0.89	-0.0562225542798982\\
0.270278192849873	0.89	0.0562225542798982\\
0.270278192849873	0.71	0.0562225542798982\\
}--cycle;

\addplot3[area legend, draw=black, fill=accent1, forget plot]
table[row sep=crcr] {%
x	y	z\\
0.122825636685871	0.51	-0.0463046179884774\\
0.122825636685871	0.69	-0.0463046179884774\\
0.277174363314129	0.69	0.0463046179884774\\
0.277174363314129	0.51	0.0463046179884774\\
}--cycle;

\addplot3[area legend, draw=black, fill=accent1, forget plot]
table[row sep=crcr] {%
x	y	z\\
0.116437097820327	0.31	-0.0334251608718693\\
0.116437097820327	0.49	-0.0334251608718693\\
0.283562902179673	0.49	0.0334251608718693\\
0.283562902179673	0.31	0.0334251608718693\\
}--cycle;

\addplot3[area legend, draw=black, fill=accent1, forget plot]
table[row sep=crcr] {%
x	y	z\\
0.111747739187817	0.11	-0.0176504521624366\\
0.111747739187817	0.29	-0.0176504521624366\\
0.288252260812183	0.29	0.0176504521624366\\
0.288252260812183	0.11	0.0176504521624366\\
}--cycle;

\addplot3[area legend, draw=black, fill=accent1, forget plot]
table[row sep=crcr] {%
x	y	z\\
0.11	-0.09	-0\\
0.11	0.09	-0\\
0.29	0.09	0\\
0.29	-0.09	0\\
}--cycle;

\addplot3[area legend, draw=black, fill=accent1, forget plot]
table[row sep=crcr] {%
x	y	z\\
0.111747739187817	-0.29	0.0176504521624366\\
0.111747739187817	-0.11	0.0176504521624366\\
0.288252260812183	-0.11	-0.0176504521624366\\
0.288252260812183	-0.29	-0.0176504521624366\\
}--cycle;

\addplot3[area legend, draw=black, fill=accent1, forget plot]
table[row sep=crcr] {%
x	y	z\\
0.116437097820327	-0.49	0.0334251608718693\\
0.116437097820327	-0.31	0.0334251608718693\\
0.283562902179673	-0.31	-0.0334251608718693\\
0.283562902179673	-0.49	-0.0334251608718693\\
}--cycle;

\addplot3[area legend, draw=black, fill=accent1, forget plot]
table[row sep=crcr] {%
x	y	z\\
0.122825636685871	-0.69	0.0463046179884774\\
0.122825636685871	-0.51	0.0463046179884774\\
0.277174363314129	-0.51	-0.0463046179884774\\
0.277174363314129	-0.69	-0.0463046179884774\\
}--cycle;

\addplot3[area legend, draw=black, fill=accent1, forget plot]
table[row sep=crcr] {%
x	y	z\\
0.129721807150127	-0.89	0.0562225542798982\\
0.129721807150127	-0.71	0.0562225542798982\\
0.270278192849873	-0.71	-0.0562225542798982\\
0.270278192849873	-0.89	-0.0562225542798982\\
}--cycle;

\addplot3[area legend, draw=black, fill=accent1, forget plot]
table[row sep=crcr] {%
x	y	z\\
0.136360389693211	-1.09	0.0636396103067893\\
0.136360389693211	-0.91	0.0636396103067893\\
0.263639610306789	-0.91	-0.0636396103067893\\
0.263639610306789	-1.09	-0.0636396103067893\\
}--cycle;

\addplot3[area legend, draw=black, fill=accent1, forget plot]
table[row sep=crcr] {%
x	y	z\\
-0.0636396103067893	0.91	-0.0636396103067893\\
-0.0636396103067893	1.09	-0.0636396103067893\\
0.0636396103067893	1.09	0.0636396103067893\\
0.0636396103067893	0.91	0.0636396103067893\\
}--cycle;

\addplot3[area legend, draw=black, fill=accent1, forget plot]
table[row sep=crcr] {%
x	y	z\\
-0.0702781928498727	0.71	-0.0562225542798982\\
-0.0702781928498727	0.89	-0.0562225542798982\\
0.0702781928498727	0.89	0.0562225542798982\\
0.0702781928498727	0.71	0.0562225542798982\\
}--cycle;

\addplot3[area legend, draw=black, fill=accent1, forget plot]
table[row sep=crcr] {%
x	y	z\\
-0.077174363314129	0.51	-0.0463046179884774\\
-0.077174363314129	0.69	-0.0463046179884774\\
0.077174363314129	0.69	0.0463046179884774\\
0.077174363314129	0.51	0.0463046179884774\\
}--cycle;

\addplot3[area legend, draw=black, fill=accent1, forget plot]
table[row sep=crcr] {%
x	y	z\\
-0.0835629021796733	0.31	-0.0334251608718693\\
-0.0835629021796733	0.49	-0.0334251608718693\\
0.0835629021796733	0.49	0.0334251608718693\\
0.0835629021796733	0.31	0.0334251608718693\\
}--cycle;

\addplot3[area legend, draw=black, fill=accent1, forget plot]
table[row sep=crcr] {%
x	y	z\\
-0.0882522608121828	0.11	-0.0176504521624366\\
-0.0882522608121828	0.29	-0.0176504521624366\\
0.0882522608121828	0.29	0.0176504521624366\\
0.0882522608121828	0.11	0.0176504521624366\\
}--cycle;

\addplot3[area legend, draw=black, fill=accent1, forget plot]
table[row sep=crcr] {%
x	y	z\\
-0.09	-0.09	-0\\
-0.09	0.09	-0\\
0.09	0.09	0\\
0.09	-0.09	0\\
}--cycle;

\addplot3[area legend, draw=black, fill=accent1, forget plot]
table[row sep=crcr] {%
x	y	z\\
-0.0882522608121828	-0.29	0.0176504521624366\\
-0.0882522608121828	-0.11	0.0176504521624366\\
0.0882522608121828	-0.11	-0.0176504521624366\\
0.0882522608121828	-0.29	-0.0176504521624366\\
}--cycle;

\addplot3[area legend, draw=black, fill=accent1, forget plot]
table[row sep=crcr] {%
x	y	z\\
-0.0835629021796733	-0.49	0.0334251608718693\\
-0.0835629021796733	-0.31	0.0334251608718693\\
0.0835629021796733	-0.31	-0.0334251608718693\\
0.0835629021796733	-0.49	-0.0334251608718693\\
}--cycle;

\addplot3[area legend, draw=black, fill=accent1, forget plot]
table[row sep=crcr] {%
x	y	z\\
-0.077174363314129	-0.69	0.0463046179884774\\
-0.077174363314129	-0.51	0.0463046179884774\\
0.077174363314129	-0.51	-0.0463046179884774\\
0.077174363314129	-0.69	-0.0463046179884774\\
}--cycle;

\addplot3[area legend, draw=black, fill=accent1, forget plot]
table[row sep=crcr] {%
x	y	z\\
-0.0702781928498727	-0.89	0.0562225542798982\\
-0.0702781928498727	-0.71	0.0562225542798982\\
0.0702781928498727	-0.71	-0.0562225542798982\\
0.0702781928498727	-0.89	-0.0562225542798982\\
}--cycle;

\addplot3[area legend, draw=black, fill=accent1, forget plot]
table[row sep=crcr] {%
x	y	z\\
-0.0636396103067893	-1.09	0.0636396103067893\\
-0.0636396103067893	-0.91	0.0636396103067893\\
0.0636396103067893	-0.91	-0.0636396103067893\\
0.0636396103067893	-1.09	-0.0636396103067893\\
}--cycle;

\addplot3[area legend, draw=black, fill=accent1, forget plot]
table[row sep=crcr] {%
x	y	z\\
-0.263639610306789	0.91	-0.0636396103067893\\
-0.263639610306789	1.09	-0.0636396103067893\\
-0.136360389693211	1.09	0.0636396103067893\\
-0.136360389693211	0.91	0.0636396103067893\\
}--cycle;

\addplot3[area legend, draw=black, fill=accent1, forget plot]
table[row sep=crcr] {%
x	y	z\\
-0.270278192849873	0.71	-0.0562225542798982\\
-0.270278192849873	0.89	-0.0562225542798982\\
-0.129721807150127	0.89	0.0562225542798982\\
-0.129721807150127	0.71	0.0562225542798982\\
}--cycle;

\addplot3[area legend, draw=black, fill=accent1, forget plot]
table[row sep=crcr] {%
x	y	z\\
-0.277174363314129	0.51	-0.0463046179884774\\
-0.277174363314129	0.69	-0.0463046179884774\\
-0.122825636685871	0.69	0.0463046179884774\\
-0.122825636685871	0.51	0.0463046179884774\\
}--cycle;

\addplot3[area legend, draw=black, fill=accent1, forget plot]
table[row sep=crcr] {%
x	y	z\\
-0.283562902179673	0.31	-0.0334251608718693\\
-0.283562902179673	0.49	-0.0334251608718693\\
-0.116437097820327	0.49	0.0334251608718693\\
-0.116437097820327	0.31	0.0334251608718693\\
}--cycle;

\addplot3[area legend, draw=black, fill=accent1, forget plot]
table[row sep=crcr] {%
x	y	z\\
-0.288252260812183	0.11	-0.0176504521624366\\
-0.288252260812183	0.29	-0.0176504521624366\\
-0.111747739187817	0.29	0.0176504521624366\\
-0.111747739187817	0.11	0.0176504521624366\\
}--cycle;

\addplot3[area legend, draw=black, fill=accent1, forget plot]
table[row sep=crcr] {%
x	y	z\\
-0.29	-0.09	-0\\
-0.29	0.09	-0\\
-0.11	0.09	0\\
-0.11	-0.09	0\\
}--cycle;

\addplot3[area legend, draw=black, fill=accent1, forget plot]
table[row sep=crcr] {%
x	y	z\\
-0.288252260812183	-0.29	0.0176504521624366\\
-0.288252260812183	-0.11	0.0176504521624366\\
-0.111747739187817	-0.11	-0.0176504521624366\\
-0.111747739187817	-0.29	-0.0176504521624366\\
}--cycle;

\addplot3[area legend, draw=black, fill=accent1, forget plot]
table[row sep=crcr] {%
x	y	z\\
-0.283562902179673	-0.49	0.0334251608718693\\
-0.283562902179673	-0.31	0.0334251608718693\\
-0.116437097820327	-0.31	-0.0334251608718693\\
-0.116437097820327	-0.49	-0.0334251608718693\\
}--cycle;

\addplot3[area legend, draw=black, fill=accent1, forget plot]
table[row sep=crcr] {%
x	y	z\\
-0.277174363314129	-0.69	0.0463046179884774\\
-0.277174363314129	-0.51	0.0463046179884774\\
-0.122825636685871	-0.51	-0.0463046179884774\\
-0.122825636685871	-0.69	-0.0463046179884774\\
}--cycle;

\addplot3[area legend, draw=black, fill=accent1, forget plot]
table[row sep=crcr] {%
x	y	z\\
-0.270278192849873	-0.89	0.0562225542798982\\
-0.270278192849873	-0.71	0.0562225542798982\\
-0.129721807150127	-0.71	-0.0562225542798982\\
-0.129721807150127	-0.89	-0.0562225542798982\\
}--cycle;

\addplot3[area legend, draw=black, fill=accent1, forget plot]
table[row sep=crcr] {%
x	y	z\\
-0.263639610306789	-1.09	0.0636396103067893\\
-0.263639610306789	-0.91	0.0636396103067893\\
-0.136360389693211	-0.91	-0.0636396103067893\\
-0.136360389693211	-1.09	-0.0636396103067893\\
}--cycle;

\addplot3[area legend, draw=black, fill=accent1, forget plot]
table[row sep=crcr] {%
x	y	z\\
-0.463639610306789	0.91	-0.0636396103067893\\
-0.463639610306789	1.09	-0.0636396103067893\\
-0.336360389693211	1.09	0.0636396103067893\\
-0.336360389693211	0.91	0.0636396103067893\\
}--cycle;

\addplot3[area legend, draw=black, fill=accent1, forget plot]
table[row sep=crcr] {%
x	y	z\\
-0.470278192849873	0.71	-0.0562225542798982\\
-0.470278192849873	0.89	-0.0562225542798982\\
-0.329721807150127	0.89	0.0562225542798982\\
-0.329721807150127	0.71	0.0562225542798982\\
}--cycle;

\addplot3[area legend, draw=black, fill=accent1, forget plot]
table[row sep=crcr] {%
x	y	z\\
-0.477174363314129	0.51	-0.0463046179884774\\
-0.477174363314129	0.69	-0.0463046179884774\\
-0.322825636685871	0.69	0.0463046179884774\\
-0.322825636685871	0.51	0.0463046179884774\\
}--cycle;

\addplot3[area legend, draw=black, fill=accent1, forget plot]
table[row sep=crcr] {%
x	y	z\\
-0.483562902179673	0.31	-0.0334251608718693\\
-0.483562902179673	0.49	-0.0334251608718693\\
-0.316437097820327	0.49	0.0334251608718693\\
-0.316437097820327	0.31	0.0334251608718693\\
}--cycle;

\addplot3[area legend, draw=black, fill=accent1, forget plot]
table[row sep=crcr] {%
x	y	z\\
-0.488252260812183	0.11	-0.0176504521624366\\
-0.488252260812183	0.29	-0.0176504521624366\\
-0.311747739187817	0.29	0.0176504521624366\\
-0.311747739187817	0.11	0.0176504521624366\\
}--cycle;

\addplot3[area legend, draw=black, fill=accent1, forget plot]
table[row sep=crcr] {%
x	y	z\\
-0.49	-0.09	-0\\
-0.49	0.09	-0\\
-0.31	0.09	0\\
-0.31	-0.09	0\\
}--cycle;

\addplot3[area legend, draw=black, fill=accent1, forget plot]
table[row sep=crcr] {%
x	y	z\\
-0.488252260812183	-0.29	0.0176504521624366\\
-0.488252260812183	-0.11	0.0176504521624366\\
-0.311747739187817	-0.11	-0.0176504521624366\\
-0.311747739187817	-0.29	-0.0176504521624366\\
}--cycle;

\addplot3[area legend, draw=black, fill=accent1, forget plot]
table[row sep=crcr] {%
x	y	z\\
-0.483562902179673	-0.49	0.0334251608718693\\
-0.483562902179673	-0.31	0.0334251608718693\\
-0.316437097820327	-0.31	-0.0334251608718693\\
-0.316437097820327	-0.49	-0.0334251608718693\\
}--cycle;

\addplot3[area legend, draw=black, fill=accent1, forget plot]
table[row sep=crcr] {%
x	y	z\\
-0.477174363314129	-0.69	0.0463046179884774\\
-0.477174363314129	-0.51	0.0463046179884774\\
-0.322825636685871	-0.51	-0.0463046179884774\\
-0.322825636685871	-0.69	-0.0463046179884774\\
}--cycle;

\addplot3[area legend, draw=black, fill=accent1, forget plot]
table[row sep=crcr] {%
x	y	z\\
-0.470278192849873	-0.89	0.0562225542798982\\
-0.470278192849873	-0.71	0.0562225542798982\\
-0.329721807150127	-0.71	-0.0562225542798982\\
-0.329721807150127	-0.89	-0.0562225542798982\\
}--cycle;

\addplot3[area legend, draw=black, fill=accent1, forget plot]
table[row sep=crcr] {%
x	y	z\\
-0.463639610306789	-1.09	0.0636396103067893\\
-0.463639610306789	-0.91	0.0636396103067893\\
-0.336360389693211	-0.91	-0.0636396103067893\\
-0.336360389693211	-1.09	-0.0636396103067893\\
}--cycle;

\addplot3[area legend, draw=black, fill=accent1, forget plot]
table[row sep=crcr] {%
x	y	z\\
-0.663639610306789	0.91	-0.0636396103067893\\
-0.663639610306789	1.09	-0.0636396103067893\\
-0.536360389693211	1.09	0.0636396103067893\\
-0.536360389693211	0.91	0.0636396103067893\\
}--cycle;

\addplot3[area legend, draw=black, fill=accent1, forget plot]
table[row sep=crcr] {%
x	y	z\\
-0.670278192849873	0.71	-0.0562225542798982\\
-0.670278192849873	0.89	-0.0562225542798982\\
-0.529721807150127	0.89	0.0562225542798982\\
-0.529721807150127	0.71	0.0562225542798982\\
}--cycle;

\addplot3[area legend, draw=black, fill=accent1, forget plot]
table[row sep=crcr] {%
x	y	z\\
-0.677174363314129	0.51	-0.0463046179884774\\
-0.677174363314129	0.69	-0.0463046179884774\\
-0.522825636685871	0.69	0.0463046179884774\\
-0.522825636685871	0.51	0.0463046179884774\\
}--cycle;

\addplot3[area legend, draw=black, fill=accent1, forget plot]
table[row sep=crcr] {%
x	y	z\\
-0.683562902179673	0.31	-0.0334251608718693\\
-0.683562902179673	0.49	-0.0334251608718693\\
-0.516437097820327	0.49	0.0334251608718693\\
-0.516437097820327	0.31	0.0334251608718693\\
}--cycle;

\addplot3[area legend, draw=black, fill=accent1, forget plot]
table[row sep=crcr] {%
x	y	z\\
-0.688252260812183	0.11	-0.0176504521624366\\
-0.688252260812183	0.29	-0.0176504521624366\\
-0.511747739187817	0.29	0.0176504521624366\\
-0.511747739187817	0.11	0.0176504521624366\\
}--cycle;

\addplot3[area legend, draw=black, fill=accent1, forget plot]
table[row sep=crcr] {%
x	y	z\\
-0.69	-0.09	-0\\
-0.69	0.09	-0\\
-0.51	0.09	0\\
-0.51	-0.09	0\\
}--cycle;

\addplot3[area legend, draw=black, fill=accent1, forget plot]
table[row sep=crcr] {%
x	y	z\\
-0.688252260812183	-0.29	0.0176504521624366\\
-0.688252260812183	-0.11	0.0176504521624366\\
-0.511747739187817	-0.11	-0.0176504521624366\\
-0.511747739187817	-0.29	-0.0176504521624366\\
}--cycle;

\addplot3[area legend, draw=black, fill=accent1, forget plot]
table[row sep=crcr] {%
x	y	z\\
-0.683562902179673	-0.49	0.0334251608718693\\
-0.683562902179673	-0.31	0.0334251608718693\\
-0.516437097820327	-0.31	-0.0334251608718693\\
-0.516437097820327	-0.49	-0.0334251608718693\\
}--cycle;

\addplot3[area legend, draw=black, fill=accent1, forget plot]
table[row sep=crcr] {%
x	y	z\\
-0.677174363314129	-0.69	0.0463046179884774\\
-0.677174363314129	-0.51	0.0463046179884774\\
-0.522825636685871	-0.51	-0.0463046179884774\\
-0.522825636685871	-0.69	-0.0463046179884774\\
}--cycle;

\addplot3[area legend, draw=black, fill=accent1, forget plot]
table[row sep=crcr] {%
x	y	z\\
-0.670278192849873	-0.89	0.0562225542798982\\
-0.670278192849873	-0.71	0.0562225542798982\\
-0.529721807150127	-0.71	-0.0562225542798982\\
-0.529721807150127	-0.89	-0.0562225542798982\\
}--cycle;

\addplot3[area legend, draw=black, fill=accent1, forget plot]
table[row sep=crcr] {%
x	y	z\\
-0.663639610306789	-1.09	0.0636396103067893\\
-0.663639610306789	-0.91	0.0636396103067893\\
-0.536360389693211	-0.91	-0.0636396103067893\\
-0.536360389693211	-1.09	-0.0636396103067893\\
}--cycle;

\addplot3[area legend, draw=black, fill=accent1, forget plot]
table[row sep=crcr] {%
x	y	z\\
-0.863639610306789	0.91	-0.0636396103067893\\
-0.863639610306789	1.09	-0.0636396103067893\\
-0.736360389693211	1.09	0.0636396103067893\\
-0.736360389693211	0.91	0.0636396103067893\\
}--cycle;

\addplot3[area legend, draw=black, fill=accent1, forget plot]
table[row sep=crcr] {%
x	y	z\\
-0.870278192849873	0.71	-0.0562225542798982\\
-0.870278192849873	0.89	-0.0562225542798982\\
-0.729721807150127	0.89	0.0562225542798982\\
-0.729721807150127	0.71	0.0562225542798982\\
}--cycle;

\addplot3[area legend, draw=black, fill=accent1, forget plot]
table[row sep=crcr] {%
x	y	z\\
-0.877174363314129	0.51	-0.0463046179884774\\
-0.877174363314129	0.69	-0.0463046179884774\\
-0.722825636685871	0.69	0.0463046179884774\\
-0.722825636685871	0.51	0.0463046179884774\\
}--cycle;

\addplot3[area legend, draw=black, fill=accent1, forget plot]
table[row sep=crcr] {%
x	y	z\\
-0.883562902179673	0.31	-0.0334251608718693\\
-0.883562902179673	0.49	-0.0334251608718693\\
-0.716437097820327	0.49	0.0334251608718693\\
-0.716437097820327	0.31	0.0334251608718693\\
}--cycle;

\addplot3[area legend, draw=black, fill=accent1, forget plot]
table[row sep=crcr] {%
x	y	z\\
-0.888252260812183	0.11	-0.0176504521624366\\
-0.888252260812183	0.29	-0.0176504521624366\\
-0.711747739187817	0.29	0.0176504521624366\\
-0.711747739187817	0.11	0.0176504521624366\\
}--cycle;

\addplot3[area legend, draw=black, fill=accent1, forget plot]
table[row sep=crcr] {%
x	y	z\\
-0.89	-0.09	-0\\
-0.89	0.09	-0\\
-0.71	0.09	0\\
-0.71	-0.09	0\\
}--cycle;

\addplot3[area legend, draw=black, fill=accent1, forget plot]
table[row sep=crcr] {%
x	y	z\\
-0.888252260812183	-0.29	0.0176504521624366\\
-0.888252260812183	-0.11	0.0176504521624366\\
-0.711747739187817	-0.11	-0.0176504521624366\\
-0.711747739187817	-0.29	-0.0176504521624366\\
}--cycle;

\addplot3[area legend, draw=black, fill=accent1, forget plot]
table[row sep=crcr] {%
x	y	z\\
-0.883562902179673	-0.49	0.0334251608718693\\
-0.883562902179673	-0.31	0.0334251608718693\\
-0.716437097820327	-0.31	-0.0334251608718693\\
-0.716437097820327	-0.49	-0.0334251608718693\\
}--cycle;

\addplot3[area legend, draw=black, fill=accent1, forget plot]
table[row sep=crcr] {%
x	y	z\\
-0.877174363314129	-0.69	0.0463046179884774\\
-0.877174363314129	-0.51	0.0463046179884774\\
-0.722825636685871	-0.51	-0.0463046179884774\\
-0.722825636685871	-0.69	-0.0463046179884774\\
}--cycle;

\addplot3[area legend, draw=black, fill=accent1, forget plot]
table[row sep=crcr] {%
x	y	z\\
-0.870278192849873	-0.89	0.0562225542798982\\
-0.870278192849873	-0.71	0.0562225542798982\\
-0.729721807150127	-0.71	-0.0562225542798982\\
-0.729721807150127	-0.89	-0.0562225542798982\\
}--cycle;

\addplot3[area legend, draw=black, fill=accent1, forget plot]
table[row sep=crcr] {%
x	y	z\\
-0.863639610306789	-1.09	0.0636396103067893\\
-0.863639610306789	-0.91	0.0636396103067893\\
-0.736360389693211	-0.91	-0.0636396103067893\\
-0.736360389693211	-1.09	-0.0636396103067893\\
}--cycle;

\addplot3[area legend, draw=black, fill=accent1, forget plot]
table[row sep=crcr] {%
x	y	z\\
-1.06363961030679	0.91	-0.0636396103067893\\
-1.06363961030679	1.09	-0.0636396103067893\\
-0.936360389693211	1.09	0.0636396103067893\\
-0.936360389693211	0.91	0.0636396103067893\\
}--cycle;

\addplot3[area legend, draw=black, fill=accent1, forget plot]
table[row sep=crcr] {%
x	y	z\\
-1.07027819284987	0.71	-0.0562225542798982\\
-1.07027819284987	0.89	-0.0562225542798982\\
-0.929721807150127	0.89	0.0562225542798982\\
-0.929721807150127	0.71	0.0562225542798982\\
}--cycle;

\addplot3[area legend, draw=black, fill=accent1, forget plot]
table[row sep=crcr] {%
x	y	z\\
-1.07717436331413	0.51	-0.0463046179884774\\
-1.07717436331413	0.69	-0.0463046179884774\\
-0.922825636685871	0.69	0.0463046179884774\\
-0.922825636685871	0.51	0.0463046179884774\\
}--cycle;

\addplot3[area legend, draw=black, fill=accent1, forget plot]
table[row sep=crcr] {%
x	y	z\\
-1.08356290217967	0.31	-0.0334251608718693\\
-1.08356290217967	0.49	-0.0334251608718693\\
-0.916437097820327	0.49	0.0334251608718693\\
-0.916437097820327	0.31	0.0334251608718693\\
}--cycle;

\addplot3[area legend, draw=black, fill=accent1, forget plot]
table[row sep=crcr] {%
x	y	z\\
-1.08825226081218	0.11	-0.0176504521624366\\
-1.08825226081218	0.29	-0.0176504521624366\\
-0.911747739187817	0.29	0.0176504521624366\\
-0.911747739187817	0.11	0.0176504521624366\\
}--cycle;

\addplot3[area legend, draw=black, fill=accent1, forget plot]
table[row sep=crcr] {%
x	y	z\\
-1.09	-0.09	-0\\
-1.09	0.09	-0\\
-0.91	0.09	0\\
-0.91	-0.09	0\\
}--cycle;

\addplot3[area legend, draw=black, fill=accent1, forget plot]
table[row sep=crcr] {%
x	y	z\\
-1.08825226081218	-0.29	0.0176504521624366\\
-1.08825226081218	-0.11	0.0176504521624366\\
-0.911747739187817	-0.11	-0.0176504521624366\\
-0.911747739187817	-0.29	-0.0176504521624366\\
}--cycle;

\addplot3[area legend, draw=black, fill=accent1, forget plot]
table[row sep=crcr] {%
x	y	z\\
-1.08356290217967	-0.49	0.0334251608718693\\
-1.08356290217967	-0.31	0.0334251608718693\\
-0.916437097820327	-0.31	-0.0334251608718693\\
-0.916437097820327	-0.49	-0.0334251608718693\\
}--cycle;

\addplot3[area legend, draw=black, fill=accent1, forget plot]
table[row sep=crcr] {%
x	y	z\\
-1.07717436331413	-0.69	0.0463046179884774\\
-1.07717436331413	-0.51	0.0463046179884774\\
-0.922825636685871	-0.51	-0.0463046179884774\\
-0.922825636685871	-0.69	-0.0463046179884774\\
}--cycle;

\addplot3[area legend, draw=black, fill=accent1, forget plot]
table[row sep=crcr] {%
x	y	z\\
-1.07027819284987	-0.89	0.0562225542798982\\
-1.07027819284987	-0.71	0.0562225542798982\\
-0.929721807150127	-0.71	-0.0562225542798982\\
-0.929721807150127	-0.89	-0.0562225542798982\\
}--cycle;

\addplot3[area legend, draw=black, fill=accent1, forget plot]
table[row sep=crcr] {%
x	y	z\\
-1.06363961030679	-1.09	0.0636396103067893\\
-1.06363961030679	-0.91	0.0636396103067893\\
-0.936360389693211	-0.91	-0.0636396103067893\\
-0.936360389693211	-1.09	-0.0636396103067893\\
}--cycle;
\end{axis}

\begin{axis}[%
width=5.938in,
height=3.854in,
at={(0in,0in)},
scale only axis,
xmin=0,
xmax=1,
ymin=0,
ymax=1,
axis line style={draw=none},
ticks=none,
axis x line*=bottom,
axis y line*=left
]
\end{axis}
\end{tikzpicture}%

    \caption{The standard contact structure on $\real^3$, given by the contact form $\dd{z} - y\dd{x}$; the hyperplanes tilt more in the increasing $y$-direction.}
    \label{fig:standard_contact}
\end{figure}
Finally, it is clear that the contact form singles out a `special direction' in the tangent space at every point of the manifold. This direction is given by the unique \emph{Reeb vector field},
\begin{equation}
    R_\alpha \in \vfields{M}:\quad \intpr{R_\alpha}{\dd{\alpha}}= 0 \quad \text{and} \quad \intpr{R_\alpha}{\alpha} = 1. 
    \label{eq:reeb_vf}
\end{equation}
The special direction identified by the Reeb vector field is referred to as the \emph{vertical} direction. Likewise, vector field components in the direction of the Reeb vector field are vertical. A vector field with no vertical component is called \emph{horizontal}.

\section{The manifold of contact elements}
\label{ssec:mfd_contact_elements}
A contact manifold is a manifold with a contact structure. One can, however, associate a \emph{canonical} $(2n-1)$-dimensional contact manifold to \emph{any} $n$-dimensional manifold $Q$, just like one can always find a canonical symplectic structure on $\ctbundle{Q}$. Roughly speaking, this attaches a fiber containing all possible contact elements to every point of the manifold $Q$. As it turns out, this `manifold of contact elements' has a natural contact structure.

The \emph{manifold of contact elements} of an $n$-dimensional manifold is \cite{Cannas2001}
$$ \cbundle{Q} = \qty{(q, \xi_q) \mid q \in Q \text{ and } \xi_q \text{ a hyperplane on } \tspace{q}{Q}}. $$
This manifold $\cbundle{Q}$ has dimension $2n - 1$. It is clear that $C$ has a natural bundle structure, i.e. $\bundle{C}{\pi}{Q}$ where the bundle projection `forgets' the contact element, that is
$$ \pi: \cbundle{Q} \to {Q}: (q, \xi_q) \mapsto q.$$
\begin{figure}[ht!]
    \centering
    \begin{tikzpicture}[h1]
    \draw[->] (0, 0) -- (0, 4) node[anchor=south] {$x_1$};
    \draw[->] (0, 0) -- (4, 0) node[anchor=west] {$x_0$};
    \node[circle, draw=black, fill=black, inner sep = 1pt, label=left:$q$] (q) at (1.5, 1.5) {};
    
    \draw[thick] (q) ++(-0.375, -0.5) --  ++(1.5,2);
    \draw (q) -- node[anchor=north] {1} ++(1, 0) -- node[anchor = west] {$\eta_1$} ++(0, 1.333);
    
    %\draw[dotted] (q) -- (1.5, 0);
    %\draw[dotted] (q) -- (0, 1.5);

\end{tikzpicture}

    \caption{A point in the manifold of contact elements on $Q = \real^2$. A coordinate system for $\cbundle{Q}$ consists of $(x_0, x_1)$ to indicate a point on $Q$, and projective coordinates $[\eta_0:\eta_1]$, which denote the contact element at that point. Without loss of generalization, one can choose $\eta_0 = 1$, and the remaining coordinate $\eta_1$ covers all but one points in the projective space. A potential confusion rests in this two-dimensional example, since both the `hyperplane` and the equivalence class of 1-forms are both lines in the tangent and cotangent space respectively. This is not the case for higher-dimensions, for which $n - 1 \neq 1$.}
    \label{fig:contact_element}
\end{figure}

There is a convenient way to characterize this manifold of contact elements, for it is isomorphic to the \emph{projectivization of the cotangent bundle} to $Q$, denoted by $\pctbundle{Q}$. This projectivization can be defined in terms of an equivalence relation between two nonzero elements in the cotangent bundle at every point in the manifold:
$$ \vec{\eta},\vec{\chi}\in \ctspace{q}{Q} \setminus \{\mathbf{0}\}:\quad (q, \vec{\eta}) \sim (q, \vec{\chi}) \Leftrightarrow \vec{\eta} = \lambda \vec{\chi},\quad \lambda \in \real_0, \text{ for all } q \in Q.$$
This equivalence relations identifies all the covectors in the cotangent space that are a nonzero multiple of each other. It is precisely this identification that takes care of the ambiguity in \cref{eq:contact_form}, in that any nonzero multiple of a 1-form has the same kernel, and therefore gives rise to the same contact structure. $\pctbundle{Q}$ is then the quotient set of $\ctbundle{Q}$ (without zero section) with respect to the equivalence relation $\sim$. Visually, the projectivization of an $n$-dimensional vector space is the space of all \emph{lines} through the origin in that vector space, which has dimension $n - 1$. It can be shown that this space is bundle-isomorphic to the manifold $\cbundle{Q}$. \cite{Cannas2001} 

As shown in \cref{fig:contact_element}, coordinates of the equivalence class of 1-forms are `projective coordinates', $[\eta_0\;:\;\eta_1:\ldots :\eta_{n-1}]$, where $\eta_i$ are coordinates for $\ctspace{q}{Q}$. The projective coordinates acknowledge the invariance under multiplication by a nonzero number. If one assumes $\eta_0$ to be nonzero, the tuple $(1, \eta_1, \ldots, \eta_n)$ provides coordinates that cover most of $\pctbundle{Q}$. 

Now, it remains to be explained why the `manifold of contact elements` is itself a contact manifold. Indeed, there is a canonical field of hyperplanes \emph{on} $\cbundle{Q}$, which lifts the hyperplane tangent to $Q$ to a hyperplane tangent to $\cbundle{Q}$ (this is akin to the `tautological' trick played in the symplectic structure of the cotangent bundle). The contact structure distinguishes the curves in $\cbundle{Q}$ that are lifted versions from curves in $Q$. This is illustrated in \cref{fig:contact_lift}. \cite{Burke1985}
Said otherwise, a tangent vector on $\cbundle{Q}$ lies in the hyperplane defined by the contact structure if its projection down on $Q$ lies in the hyperplane on $Q$ defined by the given point on the $\cbundle{Q}$. This contact structure is associated with the 1-form:
$$ \alpha = \dd{x}_0 + \sum_{i = 1}^{n-1} \eta_i\dd{x}_i,$$
given that the $\eta_0$ is the `special' coordinate wich is chosen to be 1.

\begin{figure}[h!]
    \centering
    % This file was created by matlab2tikz.
%
%The latest updates can be retrieved from
%  http://www.mathworks.com/matlabcentral/fileexchange/22022-matlab2tikz-matlab2tikz
%where you can also make suggestions and rate matlab2tikz.
%
\begin{tikzpicture}

\begin{axis}[%
width=1in,
height=1.5in,
at={(0in,0.2in)},
scale only axis,
xmin=0,
xmax=3,
xlabel style={font=\color{white!15!black}},
xlabel={$t$},
ymin=48.2,
ymax=50,
ylabel style={font=\color{white!15!black}},
ylabel={$q$},
axis background/.style={fill=white},
xmajorgrids,
ymajorgrids
]
\addplot [color=black, forget plot]
  table[row sep=crcr]{%
0	50\\
0.0303030303030303	49.9998163452709\\
0.0606060606060606	49.9992653810836\\
0.0909090909090909	49.998347107438\\
0.121212121212121	49.9970615243342\\
0.151515151515152	49.9954086317723\\
0.181818181818182	49.9933884297521\\
0.212121212121212	49.9910009182736\\
0.242424242424242	49.988246097337\\
0.272727272727273	49.9851239669421\\
0.303030303030303	49.9816345270891\\
0.333333333333333	49.9777777777778\\
0.363636363636364	49.9735537190083\\
0.393939393939394	49.9689623507805\\
0.424242424242424	49.9640036730946\\
0.454545454545455	49.9586776859504\\
0.484848484848485	49.952984389348\\
0.515151515151515	49.9469237832874\\
0.545454545454545	49.9404958677686\\
0.575757575757576	49.9337006427915\\
0.606060606060606	49.9265381083563\\
0.636363636363636	49.9190082644628\\
0.666666666666667	49.9111111111111\\
0.696969696969697	49.9028466483012\\
0.727272727272727	49.8942148760331\\
0.757575757575758	49.8852157943067\\
0.787878787878788	49.8758494031221\\
0.818181818181818	49.8661157024793\\
0.848484848484849	49.8560146923783\\
0.878787878787879	49.8455463728191\\
0.909090909090909	49.8347107438017\\
0.939393939393939	49.823507805326\\
0.96969696969697	49.8119375573921\\
1	49.8\\
1.03030303030303	49.7876951331497\\
1.06060606060606	49.7750229568411\\
1.09090909090909	49.7619834710744\\
1.12121212121212	49.7485766758494\\
1.15151515151515	49.7348025711662\\
1.18181818181818	49.7206611570248\\
1.21212121212121	49.7061524334252\\
1.24242424242424	49.6912764003673\\
1.27272727272727	49.6760330578512\\
1.3030303030303	49.660422405877\\
1.33333333333333	49.6444444444444\\
1.36363636363636	49.6280991735537\\
1.39393939393939	49.6113865932048\\
1.42424242424242	49.5943067033976\\
1.45454545454545	49.5768595041322\\
1.48484848484848	49.5590449954086\\
1.51515151515152	49.5408631772268\\
1.54545454545455	49.5223140495868\\
1.57575757575758	49.5033976124885\\
1.60606060606061	49.484113865932\\
1.63636363636364	49.4644628099174\\
1.66666666666667	49.4444444444444\\
1.6969696969697	49.4240587695133\\
1.72727272727273	49.403305785124\\
1.75757575757576	49.3821854912764\\
1.78787878787879	49.3606978879706\\
1.81818181818182	49.3388429752066\\
1.84848484848485	49.3166207529844\\
1.87878787878788	49.2940312213039\\
1.90909090909091	49.2710743801653\\
1.93939393939394	49.2477502295684\\
1.96969696969697	49.2240587695133\\
2	49.2\\
2.03030303030303	49.1755739210285\\
2.06060606060606	49.1507805325987\\
2.09090909090909	49.1256198347107\\
2.12121212121212	49.1000918273646\\
2.15151515151515	49.0741965105601\\
2.18181818181818	49.0479338842975\\
2.21212121212121	49.0213039485767\\
2.24242424242424	48.9943067033976\\
2.27272727272727	48.9669421487603\\
2.3030303030303	48.9392102846648\\
2.33333333333333	48.9111111111111\\
2.36363636363636	48.8826446280992\\
2.39393939393939	48.853810835629\\
2.42424242424242	48.8246097337006\\
2.45454545454545	48.795041322314\\
2.48484848484848	48.7651056014692\\
2.51515151515152	48.7348025711662\\
2.54545454545455	48.704132231405\\
2.57575757575758	48.6730945821855\\
2.60606060606061	48.6416896235078\\
2.63636363636364	48.6099173553719\\
2.66666666666667	48.5777777777778\\
2.6969696969697	48.5452708907254\\
2.72727272727273	48.5123966942149\\
2.75757575757576	48.4791551882461\\
2.78787878787879	48.4455463728191\\
2.81818181818182	48.4115702479339\\
2.84848484848485	48.3772268135904\\
2.87878787878788	48.3425160697888\\
2.90909090909091	48.3074380165289\\
2.93939393939394	48.2719926538108\\
2.96969696969697	48.2361799816345\\
3	48.2\\
};
\end{axis}

\begin{axis}[%
width=2.5in,
height=2in,
at={(1.8 in,0in)},
scale only axis,
plot box ratio=2.395 1.395 1,
xmin=-0.09,
xmax=3,
tick align=outside,
xlabel near ticks,
ylabel near ticks,
xlabel style={font=\color{white!15!black}},
xlabel={$t$},
ymin=48.2,
ymax=50,
ylabel style={rotate=0, font=\color{white!15!black}},
ylabel={$q$},
zmin=-1.2,
zmax=0.09,
zlabel style={font=\color{white!15!black}},
zlabel={$v$},
view={22.9492154880461}{21.3334429468537},
axis background/.style={fill=white},
xmajorgrids,
xminorgrids,
ymajorgrids,
yminorgrids,
zmajorgrids,
zminorgrids,
3d box = complete,
]
 \addplot3 [color=black, dashed]
 table[row sep=crcr] {%
0	50	0\\
0.0303030303030303	49.9998163452709	0\\
0.0606060606060606	49.9992653810836	0\\
0.0909090909090909	49.998347107438	0\\
0.121212121212121	49.9970615243342	0\\
0.151515151515152	49.9954086317723	0\\
0.181818181818182	49.9933884297521	0\\
0.212121212121212	49.9910009182736	0\\
0.242424242424242	49.988246097337	0\\
0.272727272727273	49.9851239669421	0\\
0.303030303030303	49.9816345270891	0\\
0.333333333333333	49.9777777777778	0\\
0.363636363636364	49.9735537190083	0\\
0.393939393939394	49.9689623507805	0\\
0.424242424242424	49.9640036730946	0\\
0.454545454545455	49.9586776859504	0\\
0.484848484848485	49.952984389348	0\\
0.515151515151515	49.9469237832874	0\\
0.545454545454545	49.9404958677686	0\\
0.575757575757576	49.9337006427915	0\\
0.606060606060606	49.9265381083563	0\\
0.636363636363636	49.9190082644628	0\\
0.666666666666667	49.9111111111111	0\\
0.696969696969697	49.9028466483012	0\\
0.727272727272727	49.8942148760331	0\\
0.757575757575758	49.8852157943067	0\\
0.787878787878788	49.8758494031221	0\\
0.818181818181818	49.8661157024793	0\\
0.848484848484849	49.8560146923783	0\\
0.878787878787879	49.8455463728191	0\\
0.909090909090909	49.8347107438017	0\\
0.939393939393939	49.823507805326	0\\
0.96969696969697	49.8119375573921	0\\
1	49.8	0\\
1.03030303030303	49.7876951331497	0\\
1.06060606060606	49.7750229568411	0\\
1.09090909090909	49.7619834710744	0\\
1.12121212121212	49.7485766758494	0\\
1.15151515151515	49.7348025711662	0\\
1.18181818181818	49.7206611570248	0\\
1.21212121212121	49.7061524334252	0\\
1.24242424242424	49.6912764003673	0\\
1.27272727272727	49.6760330578512	0\\
1.3030303030303	49.660422405877	0\\
1.33333333333333	49.6444444444444	0\\
1.36363636363636	49.6280991735537	0\\
1.39393939393939	49.6113865932048	0\\
1.42424242424242	49.5943067033976	0\\
1.45454545454545	49.5768595041322	0\\
1.48484848484848	49.5590449954086	0\\
1.51515151515152	49.5408631772268	0\\
1.54545454545455	49.5223140495868	0\\
1.57575757575758	49.5033976124885	0\\
1.60606060606061	49.484113865932	0\\
1.63636363636364	49.4644628099174	0\\
1.66666666666667	49.4444444444444	0\\
1.6969696969697	49.4240587695133	0\\
1.72727272727273	49.403305785124	0\\
1.75757575757576	49.3821854912764	0\\
1.78787878787879	49.3606978879706	0\\
1.81818181818182	49.3388429752066	0\\
1.84848484848485	49.3166207529844	0\\
1.87878787878788	49.2940312213039	0\\
1.90909090909091	49.2710743801653	0\\
1.93939393939394	49.2477502295684	0\\
1.96969696969697	49.2240587695133	0\\
2	49.2	0\\
2.03030303030303	49.1755739210285	0\\
2.06060606060606	49.1507805325987	0\\
2.09090909090909	49.1256198347107	0\\
2.12121212121212	49.1000918273646	0\\
2.15151515151515	49.0741965105601	0\\
2.18181818181818	49.0479338842975	0\\
2.21212121212121	49.0213039485767	0\\
2.24242424242424	48.9943067033976	0\\
2.27272727272727	48.9669421487603	0\\
2.3030303030303	48.9392102846648	0\\
2.33333333333333	48.9111111111111	0\\
2.36363636363636	48.8826446280992	0\\
2.39393939393939	48.853810835629	0\\
2.42424242424242	48.8246097337006	0\\
2.45454545454545	48.795041322314	0\\
2.48484848484848	48.7651056014692	0\\
2.51515151515152	48.7348025711662	0\\
2.54545454545455	48.704132231405	0\\
2.57575757575758	48.6730945821855	0\\
2.60606060606061	48.6416896235078	0\\
2.63636363636364	48.6099173553719	0\\
2.66666666666667	48.5777777777778	0\\
2.6969696969697	48.5452708907254	0\\
2.72727272727273	48.5123966942149	0\\
2.75757575757576	48.4791551882461	0\\
2.78787878787879	48.4455463728191	0\\
2.81818181818182	48.4115702479339	0\\
2.84848484848485	48.3772268135904	0\\
2.87878787878788	48.3425160697888	0\\
2.90909090909091	48.3074380165289	0\\
2.93939393939394	48.2719926538108	0\\
2.96969696969697	48.2361799816345	0\\
3	48.2	0\\
};
 
\addplot3[area legend, draw=black, fill=accent1, forget plot]
table[row sep=crcr] {%
x	y	z\\
-0.09	50	-0.09\\
-0.09	50	0.09\\
0.09	50	0.09\\
0.09	50	-0.09\\
}--cycle;

\addplot3[area legend, draw=black, fill=accent1, forget plot]
table[row sep=crcr] {%
x	y	z\\
0.213684262647269	49.9924643501658	-0.211212121212121\\
0.213684262647269	49.9924643501658	-0.0312121212121212\\
0.392376343413337	49.9708047040123	-0.0312121212121212\\
0.392376343413337	49.9708047040123	-0.211212121212121\\
}--cycle;

\addplot3[area legend, draw=black, fill=accent1, forget plot]
table[row sep=crcr] {%
x	y	z\\
0.518594096548623	49.9477421106622	-0.332424242424242\\
0.518594096548623	49.9477421106622	-0.152424242424242\\
0.69352711557259	49.9053341060504	-0.152424242424242\\
0.69352711557259	49.9053341060504	-0.332424242424242\\
}--cycle;

\addplot3[area legend, draw=black, fill=accent1, forget plot]
table[row sep=crcr] {%
x	y	z\\
0.82450950097695	49.8654676194795	-0.453636363636364\\
0.82450950097695	49.8654676194795	-0.273636363636364\\
0.993672317204868	49.8039538681238	-0.273636363636364\\
0.993672317204868	49.8039538681238	-0.453636363636364\\
}--cycle;

\addplot3[area legend, draw=black, fill=accent1, forget plot]
table[row sep=crcr] {%
x	y	z\\
1.13113794247355	49.7454170490119	-0.574848484848485\\
1.13113794247355	49.7454170490119	-0.394848484848485\\
1.29310448176887	49.6668878178384	-0.394848484848485\\
1.29310448176887	49.6668878178384	-0.574848484848485\\
}--cycle;

\addplot3[area legend, draw=black, fill=accent1, forget plot]
table[row sep=crcr] {%
x	y	z\\
1.43818371031165	49.5875103316752	-0.696060606060606\\
1.43818371031165	49.5875103316752	-0.516060606060606\\
1.59211931999138	49.4942160227784	-0.516060606060606\\
1.59211931999138	49.4942160227784	-0.696060606060606\\
}--cycle;

\addplot3[area legend, draw=black, fill=accent1, forget plot]
table[row sep=crcr] {%
x	y	z\\
1.74539557059453	49.3917784279974	-0.817272727272727\\
1.74539557059453	49.3917784279974	-0.637272727272727\\
1.89096806576911	49.2859075224159	-0.637272727272727\\
1.89096806576911	49.2859075224159	-0.817272727272727\\
}--cycle;

\addplot3[area legend, draw=black, fill=accent1, forget plot]
table[row sep=crcr] {%
x	y	z\\
2.05258629902857	49.1583197977021	-0.938484848484848\\
2.05258629902857	49.1583197977021	-0.758484848484849\\
2.18983794339567	49.041863857027	-0.758484848484849\\
2.18983794339567	49.041863857027	-0.938484848484848\\
}--cycle;

\addplot3[area legend, draw=black, fill=accent1, forget plot]
table[row sep=crcr] {%
x	y	z\\
2.35963138883183	48.8872628589473	-1.05969696969697\\
2.35963138883183	48.8872628589473	-0.87969696969697\\
2.48885345965302	48.761956608454	-0.87969696969697\\
2.48885345965302	48.761956608454	-1.05969696969697\\
}--cycle;

\addplot3[area legend, draw=black, fill=accent1, forget plot]
table[row sep=crcr] {%
x	y	z\\
2.66645751070617	48.5787405668329	-1.18090909090909\\
2.66645751070617	48.5787405668329	-1.00090909090909\\
2.78808794383929	48.4460528215968	-1.00090909090909\\
2.78808794383929	48.4460528215968	-1.18090909090909\\
}--cycle;

\addplot3 [color=black]
 table[row sep=crcr] {%
0	50	-0\\
0.0303030303030303	49.9998163452709	-0.0121212121212121\\
0.0606060606060606	49.9992653810836	-0.0242424242424242\\
0.0909090909090909	49.998347107438	-0.0363636363636364\\
0.121212121212121	49.9970615243342	-0.0484848484848485\\
0.151515151515152	49.9954086317723	-0.0606060606060606\\
0.181818181818182	49.9933884297521	-0.0727272727272727\\
0.212121212121212	49.9910009182736	-0.0848484848484849\\
0.242424242424242	49.988246097337	-0.096969696969697\\
0.272727272727273	49.9851239669421	-0.109090909090909\\
0.303030303030303	49.9816345270891	-0.121212121212121\\
0.333333333333333	49.9777777777778	-0.133333333333333\\
0.363636363636364	49.9735537190083	-0.145454545454545\\
0.393939393939394	49.9689623507805	-0.157575757575758\\
0.424242424242424	49.9640036730946	-0.16969696969697\\
0.454545454545455	49.9586776859504	-0.181818181818182\\
0.484848484848485	49.952984389348	-0.193939393939394\\
0.515151515151515	49.9469237832874	-0.206060606060606\\
0.545454545454545	49.9404958677686	-0.218181818181818\\
0.575757575757576	49.9337006427915	-0.23030303030303\\
0.606060606060606	49.9265381083563	-0.242424242424242\\
0.636363636363636	49.9190082644628	-0.254545454545455\\
0.666666666666667	49.9111111111111	-0.266666666666667\\
0.696969696969697	49.9028466483012	-0.278787878787879\\
0.727272727272727	49.8942148760331	-0.290909090909091\\
0.757575757575758	49.8852157943067	-0.303030303030303\\
0.787878787878788	49.8758494031221	-0.315151515151515\\
0.818181818181818	49.8661157024793	-0.327272727272727\\
0.848484848484849	49.8560146923783	-0.339393939393939\\
0.878787878787879	49.8455463728191	-0.351515151515152\\
0.909090909090909	49.8347107438017	-0.363636363636364\\
0.939393939393939	49.823507805326	-0.375757575757576\\
0.96969696969697	49.8119375573921	-0.387878787878788\\
1	49.8	-0.4\\
1.03030303030303	49.7876951331497	-0.412121212121212\\
1.06060606060606	49.7750229568411	-0.424242424242424\\
1.09090909090909	49.7619834710744	-0.436363636363636\\
1.12121212121212	49.7485766758494	-0.448484848484848\\
1.15151515151515	49.7348025711662	-0.460606060606061\\
1.18181818181818	49.7206611570248	-0.472727272727273\\
1.21212121212121	49.7061524334252	-0.484848484848485\\
1.24242424242424	49.6912764003673	-0.496969696969697\\
1.27272727272727	49.6760330578512	-0.509090909090909\\
1.3030303030303	49.660422405877	-0.521212121212121\\
1.33333333333333	49.6444444444444	-0.533333333333333\\
1.36363636363636	49.6280991735537	-0.545454545454545\\
1.39393939393939	49.6113865932048	-0.557575757575758\\
1.42424242424242	49.5943067033976	-0.56969696969697\\
1.45454545454545	49.5768595041322	-0.581818181818182\\
1.48484848484848	49.5590449954086	-0.593939393939394\\
1.51515151515152	49.5408631772268	-0.606060606060606\\
1.54545454545455	49.5223140495868	-0.618181818181818\\
1.57575757575758	49.5033976124885	-0.63030303030303\\
1.60606060606061	49.484113865932	-0.642424242424242\\
1.63636363636364	49.4644628099174	-0.654545454545455\\
1.66666666666667	49.4444444444444	-0.666666666666667\\
1.6969696969697	49.4240587695133	-0.678787878787879\\
1.72727272727273	49.403305785124	-0.690909090909091\\
1.75757575757576	49.3821854912764	-0.703030303030303\\
1.78787878787879	49.3606978879706	-0.715151515151515\\
1.81818181818182	49.3388429752066	-0.727272727272727\\
1.84848484848485	49.3166207529844	-0.739393939393939\\
1.87878787878788	49.2940312213039	-0.751515151515152\\
1.90909090909091	49.2710743801653	-0.763636363636364\\
1.93939393939394	49.2477502295684	-0.775757575757576\\
1.96969696969697	49.2240587695133	-0.787878787878788\\
2	49.2	-0.8\\
2.03030303030303	49.1755739210285	-0.812121212121212\\
2.06060606060606	49.1507805325987	-0.824242424242424\\
2.09090909090909	49.1256198347107	-0.836363636363636\\
2.12121212121212	49.1000918273646	-0.848484848484849\\
2.15151515151515	49.0741965105601	-0.860606060606061\\
2.18181818181818	49.0479338842975	-0.872727272727273\\
2.21212121212121	49.0213039485767	-0.884848484848485\\
2.24242424242424	48.9943067033976	-0.896969696969697\\
2.27272727272727	48.9669421487603	-0.909090909090909\\
2.3030303030303	48.9392102846648	-0.921212121212121\\
2.33333333333333	48.9111111111111	-0.933333333333333\\
2.36363636363636	48.8826446280992	-0.945454545454546\\
2.39393939393939	48.853810835629	-0.957575757575758\\
2.42424242424242	48.8246097337006	-0.96969696969697\\
2.45454545454545	48.795041322314	-0.981818181818182\\
2.48484848484848	48.7651056014692	-0.993939393939394\\
2.51515151515152	48.7348025711662	-1.00606060606061\\
2.54545454545455	48.704132231405	-1.01818181818182\\
2.57575757575758	48.6730945821855	-1.03030303030303\\
2.60606060606061	48.6416896235078	-1.04242424242424\\
2.63636363636364	48.6099173553719	-1.05454545454545\\
2.66666666666667	48.5777777777778	-1.06666666666667\\
2.6969696969697	48.5452708907254	-1.07878787878788\\
2.72727272727273	48.5123966942149	-1.09090909090909\\
2.75757575757576	48.4791551882461	-1.1030303030303\\
2.78787878787879	48.4455463728191	-1.11515151515152\\
2.81818181818182	48.4115702479339	-1.12727272727273\\
2.84848484848485	48.3772268135904	-1.13939393939394\\
2.87878787878788	48.3425160697888	-1.15151515151515\\
2.90909090909091	48.3074380165289	-1.16363636363636\\
2.93939393939394	48.2719926538108	-1.17575757575758\\
2.96969696969697	48.2361799816345	-1.18787878787879\\
3	48.2	-1.2\\
};
\end{axis}

\end{tikzpicture}%

    \caption{Intuitive picture of the canonical contact on the manifold of contact elements. In this case, let $ (t, q) \in Q$, and let $v$ be a coordinate for the contact (line) element. The standard contact form is then $\dd{q} - v\dd{t}$. On the left, the curve corresponding to a falling object is shown in $Q$. When this curve is `lifted' to $\cbundle{Q}$, the contact structure imposes that it be locally tangent to the contact structure, or that $v = \dv{q}{t}$. If the vertical direction is projected down onto the $(q-t)$-plane ($\cbundle(Q) \to Q$), the hyperplanes defined by the contact structure are line elements tangent to the trajectory, making $v$ the actual velocity of the curve.}
    \label{fig:contact_lift}
\end{figure}

\section{Contact Hamiltonian systems}
\todo{Introduction}

\subsection{Contact Hamiltonian vector fields}
\label{ssec:contact_ham_vfields}
Just like in the symplectic case, the contact Hamiltonian formalism defines an automorphism between a function on the contact manifold, $H \in \functions{M}$, and an associated `Hamiltonian' vector field $X_H \in \vfields{M}$. While the isomorphism is rather straightforward for symplectic manifolds, the contact counterpart is not so perspicuous: this is the prime reason behind the computational advantage of symplectification, as opposed to performing the calculations directly on the contact manifold.
\paragraph{Coordinate-free derivation} Given a contact manifold $(M, \xi)$ with contact form $\alpha$ (i.e. $\xi \in \ker \alpha$), the tangent bundle $M$ can be decomposed into two subbundles: \cite{Libermann1987,Cannas2001}
$$ \tbundle{M} = \ker \alpha \oplus \ker \dd{\alpha}, $$
where $\oplus$ denotes the Whitney sum. The first subbundle is referred to as the \emph{horizontal} bundle, the second as the \emph{vertical}
 bundle. The vertical subbundle is of rank 1 and its fiber is spanned by the Reeb vector field (cf. \cref{eq:reeb_vf}). As mentioned to in \cref{sec:contact_structures}, \emph{any} vector field $X \in \vfields{M}$ may be decomposed accordingly. This decomposition is unique and given by
\begin{equation}
    X = \underbrace{(\intpr{X}{\alpha})R_\alpha}_{X^\text{ver}} + \underbrace{\qty[X - (\intpr{X}{\alpha})R_\alpha]}_{X^\text{hor}}.
    \label{eq:contact_vf_decomp}
\end{equation}
Observe that indeed $X^\text{ver} \in \ker \dd{\alpha}$ and $X^\text{hor} \in \ker \alpha$. \cite{Cannas2001,DeLeon2020,Libermann1987}

We now wish to find the relation between the contact Hamiltonian $H \in \functions{M}$ and the associated Hamiltonian vector field $X_H \in \vfields{M}$. This one-to-one relation is uniquely determined by two conditions. Firstly, we impose that\footnote{This is the sign convention observed by \citet{Bravetti2017} en \citet{VanderSchaft2021}, as opposed to \citet{Libermann1987}.}
    $$ H \equiv -\intpr{X_H}{\alpha}. $$
This condition already provides us with the vertical component of the Hamiltonian vector field, namely
$$ X_H^\text{ver} = -H R_\alpha. $$

Secondly, the automorphism generated by the Hamiltonian vector field must be a \emph{contact automorphism}: it must preserve the contact structure. This condition is encoded in terms of the Lie derivative:\footnote
{Terminology differs somewhat in literature on this point: some authors, such as \citet{DeLeon2020} only refer to contactomorphisms as the special case where $g = 0$; while the more general case is called \emph{conformal} contactomorphisms.}
$$ X_H \text{ is an infinitesimal contact automorphism} \Leftrightarrow \lied{X_H}{\alpha} = s\alpha, $$
where $s \in \functions{M}$. The function $s$ is there because $s \alpha $ and $\alpha$ give rise to the same hyperplane distribution. 
Using Cartan's `magic' formula, the Lie derivative can be expanded as follows:
\begin{equation*}
    \begin{split}
        \lied{X_H}{\alpha} &= s \alpha\\
        \dd{\qty(\intpr{X_H}{\alpha})} + \intpr{X_H}{\dd{\alpha}} &= s\alpha \\
        -\dd{H} + \intpr{X_H}{\dd{\alpha}} &= s\alpha \\
    \end{split}
\end{equation*}
Contracting both sides with the Reeb vector field yields:
\begin{equation*}
    \begin{split}
        \intpr{R_\alpha}{\qty(-\dd{H} + \intpr{X_H}{\dd{\alpha}})} &= \intpr{R_\alpha}{\qty(s\alpha)} \\
        -\intpr{R_\alpha}{\dd{H}} + \intpr{R_\alpha}{\intpr{X_H}{\dd{\alpha}}} &= s\,\intpr{R_\alpha}{\alpha} \\
        -\intpr{R_\alpha}{\dd{H}} - \intpr{X_H}{\intpr{R_\alpha}{\dd{\alpha}}} &= s.
    \end{split}
\end{equation*}
Hence, we have $s = -R_\alpha(\dd{H})$. Because the vertical component of $X_H$ is spanned by the Reeb vector field, its contraction with $\dd{\alpha}$ vanishes. As a result, we can rewrite the previous expression in terms of the \emph{horizontal} component of $X_H$:
\begin{equation}
    \intpr{X_H}{\dd{\alpha}} = \intpr{X_H^\text{hor}}{\dd{\alpha}}=  \qty[\dd{H} - \qty(\intpr{R_\alpha}{\dd{H}})\alpha], 
    \label{eq:basic_form}
\end{equation}
We must therefore recover $X^\text{hor}_H$ from the above expression. Define the mapping  
$$ \toDual{\alpha}: \tbundle{M} \to \ctbundle{M}: X \mapsto  \intpr{X}{\dd{\alpha}},$$
when restricted to the space of horizontal vector fields, this mapping is an isomorphism onto the `semi-basic' forms\footnote{Semi-basic forms are forms that vanish when contracted with a vertical vector field. \cite{Libermann1987}}. Define the inverse mapping of $\toDual{\alpha}$ by $\fromDual{\alpha}$, such that
$$ X_H^\text{hor} = \fromDual{\alpha}\qty(\dd{H} - \qty(\intpr{R_\alpha}{\dd{H}})\,\alpha). $$
As such, the Hamiltonian vector field associated to the contact Hamiltonian $H$ is
\begin{equation}
    X_H = H R_\alpha + \fromDual{\alpha}(-\dd{H} + \qty(\intpr{R_\alpha}{\dd{H}})\,\alpha).
    \label{eq:contact_ham_vf_cofree}
\end{equation}

\paragraph{Coordinate expression} Given the contact manifold $(M, \xi)$ with contact form
$$ \dd{q_0} - \sum_{i = 1}^n p_i\dd{q_i}, $$
and define the contact Hamiltonian $H = H(q_0, q_1, \ldots, q_n, p_1, \ldots, p_n)$. 
The vertical component of the Hamiltonian vector field is straightforward (cf. \cref{eq:reeb_vf}): 
$$ X_H^\text{ver} = -H\pdv{}{q_0}. $$
For the horizontal component, first evaluate the right hand side of \cref{eq:basic_form} in coordinates:
$$ \intpr{X_H^\text{hor}}{\dd{\alpha}} =  \sum_{i = 1}^n \qty(\pdv{H}{q_i} + p_i \pdv{H}{q_0})\dd{q_i} + \pdv{H}{p_i}\dd{p_i}. $$
In terms of the basis vectors, the mapping $\toDual{\alpha}$ is
$$ \pdv{}{q_i} \mapsto \dd{p_i} \qquad \pdv{}{p_i} \mapsto -\dd{q_i} \qquad \pdv{}{q_0} \mapsto 0 \qquad i = 1, \ldots, n.$$
The inverse transformation is slightly ambiguous at first sight, since any $\pdv{}{q_0}$ cannot be recovered directly from the `forward' mapping. However, we know that $\fromDual{\alpha}$ must produce a horizontal vector field. Therefore, first perform the inverse mapping in the $q_i, p_i$-components to obtain
$$ - \sum_{i = 1}^n \qty(\pdv{H}{q_i} + p_i \pdv{H}{q_0})\pdv{}{p_i} + \sum_{i=1}^n \pdv{H}{p_i}\pdv{}{q_i}. $$
Contracting this expression with $\alpha$ produces $ - \sum_{i=1}^n p\pdv{H}{p_i} $. Hence, we can use this knowledge to find the actual horizontal component:
$$ X_H^\text{hor} = \sum_{i=1}^n p\pdv{H}{p_i} \pdv{}{q_0}- \sum_{i = 1}^n \qty(\pdv{H}{q_i} + p_i \pdv{H}{q_0})\pdv{}{p_i} + \sum_{i=1}^n \pdv{H}{p_i}\pdv{}{q_i}. $$
As such, the coordinate expression of \cref{eq:contact_ham_vf_cofree} is 
\begin{equation}
    X_H = \qty(\sum_{i=1}^n p\pdv{H}{p_i} - H) \pdv{}{q_0}- \sum_{i = 1}^n \qty(\pdv{H}{q_i} + p_i \pdv{H}{q_0})\pdv{}{p_i} + \sum_{i=1}^n \pdv{H}{p_i}\pdv{}{q_i}
\end{equation}
Furthermore, we have
$$ \lied{X_H} \alpha = - \pdv{H}{q_0} \alpha, $$ 
and 
$$ \lied{X_H} H = - H\pdv{H}{q_0}.  $$ 

\subsection{Jacobi brackets}
Just like the Poisson brackets define a Poisson algebra of the smooth functions on a symplectic manifold, there is a bracket operation on contact manifolds that serves (about) the same purpose. These brackets do not define a Poisson structure, but rather a \emph{Jacobi structure}, which is a more general notion that includes the Poisson structure as a particular instance. In this treatment we will only focus on the associated \emph{Jacobi bracket} for contact Hamiltonian systems. For more details regarding Jacobi manifolds, the reader is referred to \cite[chap. V]{Libermann1987} and \cite{DeLeon2020}.

For two smooth functions $f, g \in \functions{M}$, and $M$ a contact manifold with contact form $\alpha$, the \emph{Jacobi bracket} is defined as
\begin{equation}
    \jacobi{\:}{}: \functions{M} \times \functions{M} \to \functions{M}:\: \jacobi{f}{g} = -\intpr{\liebr{X_f}{X_g}}{\alpha},
\end{equation}
where $X_f, X_g \in \vfields{M}$ are the contact Hamiltonian vector fields of $f$ and $g$ respectively, and $\liebr{\cdot\,}{\cdot}$ is the Lie bracket (i.e. the commutator of vector fields). Equivalent expressions for the Jacobi bracket are: \cite{Libermann1987}
\begin{equation}
    \begin{split}
        \jacobi{f}{g} =& -\intpr{X_f}{\dd{g}} + g(\intpr{R_\alpha}{\dd{f}}) \\
                      =& \, \intpr{X_g}{\dd{f}} - f(\intpr{R_\alpha}{\dd{g}}) \\
                      =& -\dd{\alpha}(X_f, X_g) - f(\intpr{R_\alpha}{\dd{g}}) + g(\intpr{R_\alpha}{\dd{f}}).\\
    \end{split}
\end{equation}
From these expressions, it is also clear that the Jacobi bracket is antisymmetric, i.e. $\jacobi{f}{g} = - \jacobi{g}{f}$ and $\jacobi{f}{f} = 0$. As a time evolution operator (with respect to the Hamiltonian $H$), we have
$$ \dv{f}{t} = \jacobi{f}{H} + f(\intpr{R_\alpha}{\dd{H}}) = \jacobi{f}{H} - fs. $$

Using the same coordinates as in \cref{ssec:contact_ham_vfields}, the Jacobi bracket is equal to:
$$ 
    \jacobi{f}{g} = \qty(\sum_{i=1}^n p_i \pdv{g}{p_i} - g)\pdv{f}{q_0} - \qty(\sum_{i=1}^n p_i \pdv{f}{p_i} - f)\pdv{g}{q_0} 
    + \sum_{i=1}^n\qty(\pdv{f}{q_i}\pdv{g}{p_i} - \pdv{g}{q_i}\pdv{f}{p_i}).
$$
\todo{Check signs of Jacobi bracket, sign convention is again different from Libermann and Marle + mistake?}

